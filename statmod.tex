%!TEX root = root.tex
%

\section{Static Semantics for Modules}
\label{statmod-sec}
\subsection{Semantic Objects}
\label{statmod-sem-obj-sec}
The\index{34.1} simple 
objects for Modules static semantics are exactly as for the Core.
The compound objects are those for the Core,
augmented by those in Figure~\ref{module-objects}.


\begin{figure}[h]
%\vspace{2pt}
\adhocreplacementl{\thenostrsharing}{0mm}{
\begin{displaymath}
\begin{array}{rcl}
\M              & \in   & \StrNameSets = \Fin(\StrNames)\\
\N\ {\rm or}\ (\M,\T)
                & \in   & \NameSets = \StrNameSets\times\TyNameSets\\
\sig\ {\rm or}\ \longsig{}
                & \in   & \Sig =  \NameSets\times\Str \\
\funsig\ {\rm or}\ \longfunsig{}
                & \in   & \FunSig = \NameSets\times
                                         (\Str\times\Sig)\\
\G              & \in   & \SigEnv        =       \finfun{\SigId}{\Sig} \\
\F              & \in   & \FunEnv        =       \finfun{\FunId}{\FunSig} \\
\B\ {\rm or}\ \N,\F,\G,\E
                & \in   & \Basis = \NameSets\times
                                              \FunEnv\times\SigEnv\times\Env\\
\end{array}
\end{displaymath}}{\begin{displaymath}
\begin{array}{rcl}
\sig\ {\rm or}\ \newlongsig{}
                & \in   & \Sig =  \TyNameSets\times\Env \\
\funsig\ {\rm or}\ \newlongfunsig{}
                & \in   & \FunSig = \TyNameSets\times
                                         (\Env\times\Sig)\\
\G              & \in   & \SigEnv        =       \finfun{\SigId}{\Sig} \\
\F              & \in   & \FunEnv        =       \finfun{\FunId}{\FunSig} \\
\B\ {\rm or}\ \T,\F,\G,\E
                & \in   & \Basis = \TyNameSets\times
                                              \FunEnv\times\SigEnv\times\Env\\
\end{array}
\end{displaymath}}
\caption{Further Compound Semantic Objects}
\label{module-objects}
%\vspace{3pt}
\end{figure}
%
\replacement{\thenostrsharing}{
The prefix $(\N)$, in signatures and functor signatures, binds both type names
and structure names. We shall always consider a set $\N$ of names as
partitioned into a pair $(\M,\T)$ of sets of the two kinds of name.}{The 
prefix $(\T)$, in signatures and functor signatures, binds  type names.}
\deletion{\thenostrsharing}{It is sometimes convenient to work with an arbitrary semantic object $A$, or
assembly $A$ of such objects.
As with the function $\TyNamesFcn$,
$\StrNamesFcn(A)$ and $\NamesFcn(A)$ denote respectively the set of structure names
and the set of names occurring free in $A$.}
Certain operations require a change of bound names in semantic objects;
see for example \replacement{\thenostrsharing}{Section~\ref{realisation-sec}}{Section~\ref{tyrea.sec}}. When bound type names are
changed, we demand that all of their attributes (i.e. \deletion{\thenoimptypes}{imperative, }equality
and arity) are preserved.\index{34.2}

\deletion{\thenotypexplication}{For any structure $\S=\longS{}$ we call $m$ the {\sl structure name} or
{\sl name} of $\S$; also, the {\sl proper substructures} of $\S$ are
the members of $\Ran\SE$ and their proper substructures.  The 
{\sl substructures} of
$\S$ are $\S$ itself and its proper substructures.  The structures
{\sl occurring in}
an object or assembly $A$ are the structures and
substructures from which it is built.}

The operations of projection, injection and modification are as for the
Core. Moreover, we define $\of{\C}{\B}$ to be the context
$(\of{\T}{\B},\emptyset,\of{\E}{\B})$, i.e.~with an empty set of
explicit type variables.
Also,
we frequently need to modify a basis $\B$ by an environment $\E$
(or a structure environment $\SE$ say),
at the same time extending \replacement{\thenostrsharing}{$\of{\N}{\B}$}{$\of{\T}{\B}$} to include the type names 
\deletion{\thenostrsharing}{and
structure names }of $\E$ (or of $\SE$ say).
We therefore define $\B\oplus\SE$, for example, to mean
\replacement{\thenostrsharing}{$\B+(\NamesFcn\SE,\SE)$}{$\B+(\TyNamesFcn\SE,\SE)$}.
\index{34.3}
 
\insertion{\thenostrsharing}{
There is no separate kind of semantic object to represent structures: 
structure expressions elaborate to environments, just as structure-level
declarations do. Thus, notions which are commonly associated with structures
(for example the notion of matching a structure against a signature) are defined
in terms of environments.}

\deletion{\thenostrsharing}{
\subsection{Consistency}
\label{consistency-sec}
A\index{35.1} set of type structures is said to be {\sl consistent} if, for all
$(\theta_1,\CE_1)$ and $(\theta_2,\CE_2)$ in the set, if $\theta_1 = \theta_2$
then
\[\CE_1=\emptymap\ {\rm or}\ 
\CE_2=\emptymap\ {\rm or}\ \Dom\CE_1=\Dom\CE_2\]
A semantic object $A$ or assembly $A$ of objects is said to be
{\sl consistent} if (after changing bound names to make all nameset prefixes
in $A$ disjoint) 
for all $\S_1$ and
$\S_2$ occurring in $A$ and for every $\longstrid$ 
and every $\longtycon$
\begin{enumerate}
\item If $\of{\m}{\S_1}=\of{\m}{\S_2}$, and both
      $\S_1(\longstrid)$ and $\S_2(\longstrid)$ exist, then
      \[ \of{\m}{\S_1(\longstrid)}\ =\ \of{\m}{\S_2(\longstrid)}\]

\item If $\of{\m}{\S_1}=\of{\m}{\S_2}$, and both
      $\S_1(\longtycon)$ and $\S_2(\longtycon)$ exist, then
      \[ \of{\theta}{\S_1(\longtycon)}\ =\ \of{\theta}{\S_2(\longtycon)}\]

\item The set of all type structures in $A$ is consistent
\end{enumerate}

As an example, a functor signature 
$\longfunsig{}$ is
consistent if, assuming first that 
$\N\cap\N'=\emptyset$,
the assembly $A=\{\S,\S'\}$ is consistent.

We may loosely say that two 
structures $\S_1$ and $\S_2$
are consistent if
$\{\S_1,\S_2\}$ is consistent, but must remember that this is stronger than
the assertion that $\S_1$ is consistent and $\S_2$ is consistent.

Note that if $A$ is a consistent assembly and $A'\subset A$ then $A'$ is
also a consistent assembly.
}

\deletion{\thenostrsharing}{
\subsection{Well-formedness}
A signature\index{35.2} $\longsig{}$ is {\sl well-formed} 
if $\N\subseteq\NamesFcn\S$,
and also, whenever $(\m,\E)$ is a
substructure of $\S$ and $\m\notin\N$, then $\N\cap(\NamesFcn\E)=\emptyset$.
A functor signature $\longfunsig{}$ is {\sl well-formed} if
$\longsig{}$ and  $(\N')\S'$ are well-formed, and also, whenever
$(\m',\E')$ is a substructure of $\S'$ and $\m'\notin\N\cup\N'$,
then $(\N\cup\N')\cap(\NamesFcn\E')=\emptyset$.

An object or assembly $A$ is {\sl well-formed} if every type environment,
signature and functor signature occurring in $A$ is well-formed.}

\deletion{\thenostrsharing}{\subsection{Cycle-freedom}
An\index{35.3} object or assembly $A$ is {\sl cycle-free} if it contains no
cycle of structure names; that is, there is no sequence
\[\m_0,\cdots,\m_{k-1},\m_k=m_0\ \ (k>0)\]
of structure names such that, for each $i\ (0\leq i<k)$ some structure
with name $m_i$ occurring in $A$ has a proper substructure with name
$m_{i+1}$.
}

\deletion{\thenostrsharing}{
\subsection{Admissibility}
\label{admis-sec}
An\index{36.1} object or assembly $A$ is {\sl admissible} if it is
consistent, well-formed and cycle-free. 
Henceforth it is assumed
that
all objects mentioned are admissible.  
We also require that
\begin{enumerate}
\item In every sentence $A\ts\phrase\ra A'$  inferred by the rules
given in Section~\ref{statmod-rules-sec}, the assembly $\{A,A'\}$ is
admissible.  
\item In the special case of a sentence $\B\ts\sigexp\ra\S$,
we further require that the assembly consisting of all semantic
objects occurring in the entire inference of this sentence be
admissible. This  is important for the definition of principal
signatures in Section~\ref{prinsig-sec}.
\end{enumerate}
In our semantic definition we have not undertaken to
indicate how admissibility should be checked in an implementation.
}

\subsection{Type Realisation}
\label{tyrea.sec}
\replacement{\thenostrsharing}{
A {\sl type realisation}\index{36.2} is a map
$\tyrea:\TyNames\to\TypeFcn$
such that
$\t$ and $\tyrea(\t)$ have the same arity, and
if $t$ admits equality then so does $\tyrea(\t)$.

The {\sl support} $\Supp\tyrea$ of a type realisation $\tyrea$ is the set of
type names $\t$ for which $\tyrea(\t)\ne\t$.}{A 
{\sl (type) realisation}\index{36.2} is a map
$\rea:\TyNames\to\TypeFcn$
such that
$\t$ and $\rea(\t)$ have the same arity, and
if $t$ admits equality then so does $\rea(\t)$.

The {\sl support} $\Supp\rea$ of a type realisation $\rea$ is the set of
type names $\t$ for which $\rea(\t)\ne\t$.}
%
\deletion{\thenostrsharing}{\subsection{Realisation}
\label{realisation-sec}
A {\sl realisation}\index{36.3} is a function $\rea$ of names,
partitioned into a type realisation $\tyrea:\TyNames\to\TypeFcn$
and a function $\strrea : \StrNames\to\StrNames$.
The {\sl support} $\Supp\rea$
of a realisation $\rea$ is the set of
names $\n$ for which $\rea(\n)\ne\n$.}\replacement{\thenostrsharing}{The {\sl yield}
$\Yield\rea$ of a realisation $\rea$ is the set of
names which occur in some $\rea(\n)$ for which $\n\in\Supp\rea$.}{The
{\sl yield} $\Yield\rea$ of a realisation $\rea$ is the set of
type names which occur in some $\rea(\t)$ for which $\t\in\Supp\rea$.}

Realisations $\rea$ are extended to apply to all semantic objects; their
effect is to
replace each name \replacement{\thenostrsharing}{$n$ by $\rea(\n)$}{$\t$ by $\rea(\t)$}.  In applying $\rea$ to an object with
bound names, such as a signature \replacement{\thenostrsharing}{$\longsig{}$}{$\newlongsig{}$}, first bound names must be
changed so that, for each binding prefix \replacement{\thenostrsharing}{$(\N)$}{$(\T)$},
\replacement{\thenostrsharing}{
\[\N\cap(\Supp\rea\cup\Yield\rea)=\emptyset\ .\]}{\[\T\cap(\Supp\rea\cup\Yield\rea)=\emptyset\ .\]}
%
\deletion{\thenotypexplication}{
\subsection{Type Explication}
\label{type-explication-sec}
A\index{36.35} signature $(\N)\S$ is {\sl type-explicit\/} if,
whenever $\t\in\N$ and occurs free in $\S$, then some substructure of
$\S$ contains a type environment $\TE$ such that
$\TE(\tycon)=(\t,\CE)$ for some $\tycon$ and some $\CE$.} 
%
\subsection{Signature Instantiation}
\replacement{\thenostrsharing}{
A\index{36.4} structure $\S_2$ {\sl is an instance of} a signature
$\sig_1=\longsig{1}$,
written $\siginst{\sig_1}{}{\S_2}$, if there exists a realisation
$\rea$
such that $\rea(\S_1)=\S_2$ and $\Supp\rea\subseteq\N_1$.}{An\index{36.4} environment $\E_2$ {\sl is an instance of} a signature
$\sig_1=\newlongsig{1}$,
written $\siginst{\sig_1}{}{\E_2}$, if there exists a realisation
$\rea$
such that $\rea(\E_1)=\E_2$ and $\Supp\rea\subseteq\T_1$.}
\deletion{\thenotypexplication}{(Note that if $\sig_1$ is type-explicit then there is at most one
such $\rea$.)}\ 
\deletion{\thenostrsharing}{A signature
$\sig_2=\longsig{2}$ {\sl is an instance of}
$\sig_1 =\longsig{1}$,
written $\siginst{\sig_1}{}{\sig_2}$, if
$\siginst{\sig_1}{}{\S_2}$ and $\N_2\cap(\NamesFcn\sig_1)=\emptyset$.
It can be shown that $\siginst{\sig_1}{}{\sig_2}$ iff, for all $\S$,
whenever $\siginst{\sig_2}{}{\S}$ then $\siginst{\sig_1}{}{\S}$.}

\subsection{Functor Signature Instantiation}
\replacement{\thenostrsharing}{
A\index{36.5} pair $(\S,(\N')\S')$ is called a {\sl functor instance}.
Given $\funsig=\longfunsig{1}$,
a functor instance $(\S_2,(\N_2')\S_2')$ is an {\sl instance} of
$\funsig$,
written $\funsiginst{\funsig}{}{(\S_2,(\N_2')\S_2')}$,
if there exists a realisation $\rea$
such that
$\rea(\S_1,(\N_1')\S_1')=(\S_2,(\N_2')\S_2')$ and
$\Supp\rea\subseteq\N_1$.}{
A\index{36.5} pair $(\E,(\T')\E')$ is called a {\sl functor instance}.
Given $\funsig=\newlongfunsig{1}$,
a functor instance $(\E_2,(\T_2')\E_2')$ is an {\sl instance} of
$\funsig$,
written $\funsiginst{\funsig}{}{(\E_2,(\T_2')\E_2')}$,
if there exists a realisation $\rea$
such that
$\rea(\E_1,(\T_1')\E_1')=(\E_2,(\T_2')\E_2')$ and
$\Supp\rea\subseteq\T_1$.}
%
\subsection{Enrichment}
\label{enrichment-sec}
In\index{37.1} matching \replacement{\thenostrsharing}{a structure}{an environment} to a signature, the \replacement{\thenostrsharing}{structure}{environment} will be allowed both to
have more components, and to be more polymorphic, than (an instance of) the
signature.  Precisely, we  define enrichment of \deletion{\thenostrsharing}{structures, }environments and
type structures \replacement{\thenostrsharing}{by mutual recursion}{recursively} as follows.

\deletion{\thenostrsharing}{A structure $\S_1=(\m_1,\E_1)$
{\sl enriches} another structure
$\S_2=(\m_2,\E_2)$, written $\S_1\succ\S_2$, if
\begin{enumerate}
\item $\m_1=\m_2$
\item $\E_1\succ\E_2$
\end{enumerate}}
An environment \replacement{\theidstatus}{$\E_1=\longE{1}$}{$\E_1=\newlongE{1}$}
{\sl enriches} another environment \replacement{\theidstatus}{$\E_2=$ $\longE{2}$}{$\E_2=
( \SE_2,$\linebreak$\TE_2,\VE_2)$},
written $\E_1\succ\E_2$,
if
\begin{enumerate}
\item $\Dom\SE_1\supseteq\Dom\SE_2$, and $\SE_1(\strid)\succ\SE_2(\strid)$
                                               for all $\strid\in\Dom\SE_2$
\item $\Dom\TE_1\supseteq\Dom\TE_2$, and $\TE_1(\tycon)\succ\TE_2(\tycon)$
                                               for all $\tycon\in\Dom\TE_2$
\item \replacement{\theidstatus}{$\Dom\VE_1\supseteq\Dom\VE_2$, and $\VE_1(\id)\succ\VE_2(\id)$
                                               for all $\id\in\Dom\VE_2$}{$\Dom\VE_1\supseteq\Dom\VE_2$, and $\VE_1(\vid)\succ\VE_2(\vid)$
                                               for all $\vid\in\Dom\VE_2$,
where $(\sigma_1,\is_1)\succ(\sigma_2,\is_2)$ means $\sigma_1\succ\sigma_2$ and
$$\is_1 = \is_2\quad\hbox{or}\quad \is_2 = \isv$$}
\deletion{\theidstatus}{
\item $\Dom\EE_1\supseteq\Dom\EE_2$, and $\EE_1(\exn)=\EE_2(\exn)$
                                               for all $\exn\in\Dom\EE_2$}
\end{enumerate}
Finally, a type structure $(\theta_1,\adhocreplacementl{\thece}{6cm}{\CE}{\VE}_1)$
{\sl enriches} another type structure $(\theta_2,\adhocreplacementl{\thece}{-4cm}{\CE}{\VE}_2)$,
written $(\theta_1,\adhocreplacementl{\thece}{10mm}{\CE}{\VE}_1)\succ(\theta_2,\adhocreplacementl{\thece}{-12cm}{\CE}{\VE}_2)$,
if
\begin{enumerate}
\item $\theta_1=\theta_2$
\item Either $\adhocreplacementl{\thece}{3cm}{\CE}{\VE}_1=\adhocreplacementl{\thece}{-8cm}{\CE}{\VE}_2$ or $\adhocreplacementl{\thece}{-11cm}{\CE}{\VE}_2=\emptymap$
\end{enumerate}

\oldpagebreak
\subsection{Signature Matching}
\label{sigmatch-sec}
\replacement{\thenostrsharing}{
A\index{37.2} structure $\S$ {\sl matches} a signature $\sig_1$ if there exists
a structure $\S^-$ such that $\sig_1\geq\S^-\prec\S$. Thus matching
is a combination of instantiation and enrichment. There is at most
one such $\S^-$, given $\sig_1$ and $\S$.}{An\index{37.2} environment $\E$ {\sl matches} a signature $\sig_1$ if there exists
an environment $\E^-$ such that $\sig_1\geq\E^-\prec\E$. Thus matching
is a combination of instantiation and enrichment. There is at most
one such $\E^-$, given $\sig_1$ and $\E$.} \deletion{\thenotypexplication}{Moreover, writing $\sig_1=
\longsig{1}$, if $\sig_1\geq\S^-$ then there exists a realisation $\rea$
with $\Supp\rea\subseteq\N_1$ and $\rea(\S_1)=\S^-$.
We shall then say that $\S$ matches $\sig_1$ {\em via} $\rea$.
(Note that if $\sig_1$ is type-explicit 
then $\rea$ is uniquely determined by $\sig_1$ and $\S$.)}

\deletion{\thenostrsharing}{A\index{37.2.5} signature $\sig_2$ {\em matches} a signature $\sig_1$
if for all structures $\S$, if $\S$ matches $\sig_2$ then $\S$
matches $\sig_1$. It can be shown that $\sig_2=\longsig{2}$ matches
$\sig_1=\longsig{1}$ if and only if there exists a realisation
$\rea$ with $\Supp\rea\subseteq\N_1$ and $\rea(\S_1)\prec\S_2$
and $\N_2\cap\NamesFcn\sig_1=\emptyset$.}

\deletion{\thenostrsharing}{\subsection{Principal Signatures}
\label{prinsig-sec}
The definitions in this section concern the elaboration of signature
expressions; more precisely they concern inferences of sentences of the
form $\B\ts\sigexp\ra\S$, where $\S$ is a structure and $\B$ is a basis.
Recall, from Section~\ref{admis-sec}, that the assembly of all semantic
objects in such an inference must be admissible.

For any basis $\B$ and any structure $\S$, 
we say that $\B$ {\sl covers} $\S$
if for every substructure $(m,E)$ of $\S$ such that
$m\in\of{\N}{\B}$:
\begin{enumerate}
\item
For every structure identifier $\strid\in\Dom\E$,
$\B$ contains a substructure $(m,\E')$ with $m$
free and $\strid\in\Dom\E'$
\item
For every type constructor $\tycon\in\Dom\E$,
$\B$ contains a substructure $(m,\E')$ with $m$ free
and $\tycon\in\Dom\E'$
\end{enumerate}
(This condition is not a consequence of consistency of $\{\B,\S\}$; 
informally, it states that if $\S$ shares a substructure with $\B$,
then $\S$ mentions no more components of the substructure than
$\B$ does.)



We\index{38.1} say that a signature
$\longsig{}$ is {\sl principal for $\sigexp$ in $\B$} if, choosing $\N$
so that $(\of{\N}{\B})\cap\N=\emptyset$,
\begin{enumerate}
\item $\B$ covers $\S$ 
\item $\B\vdash\sigexp\ra\S$
\item Whenever $\B\vdash\sigexp\ra\S'$, then $\sigord{\longsig{}}{}{\S'}$
\end{enumerate}
We claim that if $\sigexp$ elaborates in $\B$ to some structure covered
by $\B$, then it possesses a principal signature in $\B$.

Analogous to the definition given for type environments in
Section~\ref{typeenv-wf-sec}, we say that a semantic object $A$
{\sl respects equality} if every type environment occurring in 
$A$ respects equality. 
%
%
%Further, let $T$ be the set of type names
%$\t$ such that $(\t,\CE)$ occurs in $A$ for some
%$\CE\neq\emptymap$.  Then $A$ is said to {\sl maximise equality}
%if (a) $A$ respects equality, and also (b) if any larger subset of
%$T$ were to admit equality (without any change in the equality
%attribute of any type names not in $T$) then $A$ would cease to
%respect equality.
%
\oldpagebreak
Now\index{38.5} let us assume that $\sigexp$ possesses a principal signature
$\sig_0=\longsig{0}$ in $B$. We wish to
define, in terms of $\sig_0$, another signature $\sig$ which provides more
information about the equality attributes of structures which will
match $\sig_0$. To this end, let $\T_0$ be the set of type names $\t\in\N_0$
which do not admit equality, and such that $(\t,\CE)$ occurs in $\S_0$
for some $\CE\neq\emptymap$.  Then we say $\sig$ is 
{\sl equality-principal for $\sigexp$ in $\B$} if
\begin{enumerate}
\item
$\sig$ respects equality
\item
$\sig$ is obtained from $\sig_0$ just by making as many
members of $\T_0$ admit equality as possible, subject to 1.~above
\end{enumerate}
It is easy to show that, if any such $\sig$ exists, it is determined
uniquely by $\sig_0$; moreover, $\sig$ exists if $\sig_0$ itself
respects equality.
\bigskip}


%
%\clearpage

%                   Inference Rules
%
\subsection{Inference Rules}
\label{statmod-rules-sec}
As\index{39.1} for the Core, the rules of the Modules static semantics allow
sentences of the form
\[ A\ts\phrase\ra A'\]
to be inferred, where in this case $A$ is either a basis, a context or
an environment and $A'$ is a semantic object.  The convention for options
is as in the Core semantics. 

Although not assumed in our definitions, it is intended that every basis
\replacement{\thenostrsharing}{$\B=\N,\F,\G,\E$}{$\B=\T,\F,\G,\E$} in which a $\topdec$ is elaborated has the property
that 
\replacement{\thenostrsharing}{$\NamesFcn\F\ \cup\NamesFcn\G\cup\NamesFcn\E\subseteq\N$}{$\TyNamesFcn\F
\ \cup\TyNamesFcn\G\cup\TyNamesFcn\E\subseteq\T$}. 
\replacement{\thenostrsharing}{
This is not
the case for bases in which signature expressions and specifications are
elaborated, but the following Theorem can be proved:}{
The following Theorem can be proved:}
\begin{quote}
Let S be an inferred sentence $\B\ts\topdec\ra\B'$ in which $\B$ satisfies
the above condition. Then $\B'$ also satisfies the condition.

\replacement{\thenostrsharing}{
Moreover, if S$'$ is a sentence of the form
$\B''\ts\phrase\ra A$ occurring in a proof of S, where $\phrase$ is
either a structure expression or a structure-level declaration, then $\B''$
also satisfies the condition.}{
Moreover, if S$'$ is a sentence of the form
$\B''\ts\phrase\ra A$ occurring in a proof of S, where $\phrase$ is
any Modules phrase, then $\B''$ also satisfies the condition.}

\replacement{\thenostrsharing}{Finally, if $\T,\U,\E\ts\phrase\ra A$ occurs
in a proof of S, where $\phrase$ is a phrase of the Core, then
$\TyNamesFcn\E\subseteq\T$.}{Finally, if $\T,\U,\E\ts\phrase\ra A$ occurs
in a proof of S, where $\phrase$ is a phrase of Modules or of the Core, then
$\TyNamesFcn\E\subseteq\T$.}
\end{quote}



%               SEMANTICS
%
%                       Structure Expressions
%
\replacement{\thenostrsharing}{
\rulesec{Structure Expressions}{\B\ts\strexp\ra \S}}
{\rulesec{Structure Expressions}{\B\ts\strexp\ra \E}}
\replacement{\thenostrsharing}{\begin{equation}        % generative strexp
\label{generative-strexp-rule}
\frac{\B\ts\strdec\ra\E\qquad\m\notin(\of{\N}{\B})\cup\NamesFcn\E}
     {\B\ts\encstrexp\ra(\m,\E)}\index{39.2}
\end{equation}}{\begin{equation}        % generative strexp
\label{generative-strexp-rule}
\frac{\B\ts\strdec\ra\E}
     {\B\ts\encstrexp\ra \E }\index{39.2}
\end{equation}}
\replacement{\thenostrsharing}{
\begin{equation}        % longstrid
%\label{longstrid-strexp-rule}
\frac{\B(\longstrid)=\S}
     {\B\ts\longstrid\ra\S}
\end{equation}}{\begin{equation}        % longstrid
%\label{longstrid-strexp-rule}
\frac{\B(\longstrid)=\E}
     {\B\ts\longstrid\ra\E}
\end{equation}}

\insertion{\thenostrsharing}{
\begin{equation}
\label{transparent-constraint-rule}
\frac{B\ts\strexp\ra\E\quad\B\ts\sigexp\ra\Sigma\quad\Sigma\geq\E'\prec\E}
     {\B\ts\transpconstraint\ra\E'}
\end{equation}
}

\insertion{\thenostrsharing}{
\begin{equation}
\label{opaque-constraint-rule}
\frac{\begin{array}{c}
   B\ts\strexp\ra\E\quad\B\ts\sigexp\ra(\T')\E'\\
   (\T')\E'\geq\E''\prec\E\quad \T' \cap(\of{\T}{\B}) = \emptyset
      \end{array}}
     {\B\ts\opaqueconstraint\ra\E'}
\end{equation}
}

\vspace{6pt}
\replacement{\thenostrsharing}{
\begin{equation}                % functor application
\label{functor-application-rule}
\frac{ \begin{array}{c}
        \B\ts\strexp\ra\S\\
        \funsiginst{\B(\funid)}{}{(\S'',(\N')\S')}\ ,
                                                    \ \S\succ\S''\\
        (\of{\N}{\B})\cap\N'=\emptyset
       \end{array}
     }
     {\B\ts\funappstr\ra\S'}
\end{equation}}{\begin{equation}                % functor application
\label{functor-application-rule}
\frac{ \begin{array}{c}
        \B\ts\strexp\ra\E\\
        \funsiginst{\B(\funid)}{}{(\E'',(\T')\E')}\ ,
                                                    \ \E\succ\E''\\
        (\TyNamesFcn \E\; \cup\; \of{\T}{\B})\cap\T'=\emptyset
       \end{array}
     }
     {\B\ts\funappstr\ra\E'}
\end{equation}}

\vspace{6pt}
\replacement{\thenostrsharing}{
\begin{equation}        % let strexp
\label{letstrexp-rule}
\frac{\B\ts\strdec\ra\E\qquad\B\oplus\E\ts\strexp\ra\S}
     {\B\ts\letstrexp\ra\S}
\end{equation}}{\begin{equation}        % let strexp
\label{letstrexp-rule}
\frac{\B\ts\strdec\ra\E_1\qquad\B\oplus\E_1\ts\strexp\ra\E_2}
     {\B\ts\letstrexp\ra\E_2}
\end{equation}}
\comments
\begin{description}
%
\oldpagebreak
\item{(\ref{functor-application-rule})}
   The side condition 
\replacement{\thenostrsharing}{$ (\of{\N}{\B})\cap\N'=\emptyset$}{$(\TyNamesFcn\E \cup \of{\T}{\B})\cap\T'=\emptyset$}
  can always
be satisfied by renaming bound names in \replacement{\thenostrsharing}{$(\N')S'$}{$(\T')E'$}; it ensures that the
generated \replacement{\thenostrsharing}{structures}{datatypes} receive new names.\index{40.1}

\replacement{\thenostrsharing}{Let $\B(\funid)=(N)(\S_f,(N')\S_f')$.}{Let $\B(\funid)=(\T)(\E_f,(T')\E_f')$.} 
\replacement{\thenotypexplication}{Assuming that $(\N)\S_f$ is
type-explicit, the realisation $\rea$ for which
$\rea(\S_f,(N')\S_f')=(\S'',(\N')\S')$ is uniquely determined by $\S$,
since $\S\succ\S''$ can only hold if the type names and structure
names in $\S$ and $\S''$ agree.  Recall that enrichment $\succ$ allows
more components and more polymorphism, while instantiation $\geq$ does
not.\par}{Let $\rea$ be a realisation such that\linebreak
$\rea(\E_f,(T')\E_f')=(\E'',(\T')\E')$.}
\replacement{\thenostrsharing}{
Sharing between argument and result specified in the declaration of
the functor $\funid$ is represented by the occurrence of the same name
in both $\S_f$ and $\S_f'$, and this repeated occurrence is preserved
by $\rea$, yielding sharing between the argument structure $\S$ and
the result structure $\S'$ of this functor application.}{
Sharing between argument and result specified in the declaration of
the functor $\funid$ is represented by the occurrence of the same name
in both $\E_f$ and $\E_f'$, and this repeated occurrence is preserved
by $\rea$, yielding sharing between the argument structure $\E$ and
the result structure $\E'$ of this functor application.}
%
\item{(\ref{letstrexp-rule})}
   The use of $\oplus$, here and elsewhere, ensures that \deletion{\thenostrsharing}{structure
and }type names generated by
the first sub-phrase
are distinct from names generated by the second
sub-phrase.
\end{description}

%                              declarations
\rulesec{Structure-level Declarations}{\B\ts\strdec\ra\E}               
\replacement{\thenostrsharing}{
\begin{equation}                % core declaration
\label{dec-rule}
\frac{ \of{\C}{\B}\ts\dec\ra\E
       \quad\E\ {\rm principal\ for\ \dec\ in\ } (\of{\C}{\B})
}
     { \B\ts\dec\ra\E }\index{40.2}
\end{equation}}{\begin{equation}                % core declaration
\label{dec-rule}
\frac{ \of{\C}{\B}\ts\dec\ra\E
}
     { \B\ts\dec\ra\E }\index{40.2}
\end{equation}}

\vspace{6pt}
\begin{equation}                % structure declaration
%\label{structure-decl-rule}
\frac{ \B\ts\strbind\ra\SE }
     { \B\ts\singstrdec\ra\SE\ \In\ \Env }
\end{equation}

\vspace{6pt}
\begin{equation}                % local structure-level declaration
%\label{local structure-level declaration}
\frac{ \B\ts\strdec_1\ra\E_1\qquad
       \B\oplus\E_1\ts\strdec_2\ra\E_2 }
     { \B\ts\localstrdec\ra\E_2 }
\end{equation}

\vspace{6pt}
\begin{equation}                % empty declaration
%\label{empty-strdec-rule}
\frac{}
     {\B\ts\emptystrdec\ra \emptymap{\rm\ in}\ \Env}
\end{equation}

\vspace{6pt}
\begin{equation}                % sequential declaration
%\label{sequential-strdec-rule}
\frac{ \B\ts\strdec_1\ra\E_1\qquad
       \B\oplus\E_1\ts\strdec_2\ra\E_2 }
     { \B\ts\seqstrdec\ra\plusmap{\E_1}{\E_2} }
\end{equation}
\deletion{\thenostrsharing}{
\comments
\begin{description}
\item{(\ref{dec-rule})}
The side condition ensures that all type schemes in $\E$ are as
general as possible.
% and that no imperative type variables occur
%free in $\E$.
%from version 1:
%   The side condition ensures that all type schemes in $\E$ are as
%general as possible and that all new type names in $\E$ admit
%equality, if possible.
\end{description}}
\oldpagebreak
\rulesec{Structure Bindings}{\B\ts\strbind\ra\SE}
\begin{equation}                % structure binding
\label{structure-binding-rule}
\frac{ 
       \B\ts\strexp\ra\E\quad
       \langle\plusmap{\B}{\TyNamesFcn\E}\ts
                                      \strbind\ra\SE\rangle
     }
     { \B\ts\barestrbindera\ra\{\strid\mapsto\E\}
       \ \langle +\ \SE\rangle}\index{41.1}
\end{equation}%
%
%                   Signature Rules
%
\rulesec{Signature Expressions}{\B\ts\sigexp\ra\E}
\begin{equation}                % encapsulation sigexp
\label{encapsulating-sigexp-rule}
\frac{\B\ts\spec\ra\E }
     {\B\ts\encsigexp\ra  \E }\index{41.2}
\end{equation}

\begin{equation}                % signature identifier
\label{signature-identifier-rule}
\frac{  \B(\sigid) = (\T)\E \quad \T\cap (\of{\T}{\B}) = \emptyset}
     { \B\ts\sigid\ra\E }
\end{equation}

\begin{equation}
\label{wheretype-rule}
\frac{
  \begin{array}{c}
     \B\ts\sigexp\ra \E\quad \tyvarseq = \alphak\quad \of{\C}{\B}\ts\ty\ra \tau\\
     \E(\longtycon) = (\t, \VE)\quad t\notin\of{\T}{\B} \quad \ADD{t\in\TyNamesk} \\
     \rea = \{\t\mapsto \Lambda\alphak.\tau\}\quad
     \hbox{$\Lambda\alphak.\tau$ admits equality, if $\t$ does\quad $\rea(\E)$ well-formed}
  \end{array}
 }
 {\B\ts\wheretypesigexp\ra\rea(\E)}
\end{equation}%
\comments
\begin{description}
\deletion{\thenostrsharing}{
\item{(\ref{encapsulating-sigexp-rule})}
   In contrast to rule~\ref{generative-strexp-rule}, $m$ is not here 
required to be new. 
The name $m$ may be chosen to achieve the sharing required
in rule~\ref{strshareq-rule}, or to achieve the enrichment side conditions
of rule~\ref{structure-binding-rule} or \ref{funbind-rule}. 
The choice of $m$ must result in an admissible object.}
\item{(\ref{signature-identifier-rule})}
   \replacement{\thenostrsharing}{The instance $\S$ of $\B(\sigid)$ is not determined by this rule,
but -- as in rule~\ref{encapsulating-sigexp-rule} -- the instance
may  be chosen to achieve sharing properties or enrichment
conditions.}{The bound names of $\B(\sigid)$ can always be renamed to satisfy $\T\cap(\of{\T}{\B}) = \emptyset$,
if necessary.}
\end{description}

\rulesec{}{\B\ts\sigexp\ra\sig}
\replacement{\thenostrsharing}{\begin{equation}                % any sigexp
\label{topmost-sigexp-rule}
\frac{\begin{array}{c}
\B\ts\sigexp\ra\S\quad\mbox{$(\N)\S$ equality-principal for $\sigexp$ in $\B$}\\
\mbox{$(\N)\S$ type-explicit}
      \end{array}}
     {\B\ts\sigexp\ra (\N)\S}\index{41.25}
\end{equation}}{\begin{equation}                % any sigexp
\label{topmost-sigexp-rule}
\frac{
\B\ts\sigexp\ra\E\quad\T= \TyNamesFcn\E\setminus(\of{\T}{\B})
}
     {\B\ts\sigexp\ra (\T)\E}\index{41.25}
\end{equation}}%
\noindent\comment
A signature expression $\sigexp$ which is an immediate constituent of
\deletion{\thenostrsharing}{a structure binding, } a signature binding\replacement{\thenostrsharing}{, a 
functor binding or a
functor signature}{, a signature constraint, or a
functor binding }is elaborated to \replacement{\thenostrsharing}{an equality-principal and type-explicit }{a }signature,
 see rules~\replacement{\thenostrsharing}{\ref{structure-binding-rule}, }{
\ref{transparent-constraint-rule}, \ref{opaque-constraint-rule}, }\ref{sigbind-rule}\deletion{\thenostrsharing}{, 
\ref{funsigexp-rule}} and \ref{funbind-rule}.  \deletion{\thenostrsharing}{By contrast, signature 
expressions occurring in structure descriptions are elaborated to
structures using the liberal rules
\ref{encapsulating-sigexp-rule} and \ref{signature-identifier-rule}, 
see rule~\ref{strdesc-rule}, so that names can be chosen to achieve
sharing, when necessary.}
\oldpagebreak

\rulesec{Signature Declarations}{\B\ts\sigdec\ra\G}
\begin{equation}        % single signature declaration
\label{single-sigdec-rule}
\frac{ \B\ts\sigbind\ra\G }
     { \B\ts\singsigdec\ra\G }\index{41.3}
\end{equation}
\deletion{\thenostrsharing}{
\begin{equation}        % empty signature declaration
%\label{empty-sigdec-rule}
\frac{}
     { \B\ts\emptysigdec\ra\emptymap }
\end{equation}

\begin{equation}        % sequential signature declaration
\label{sequence-sigdec-rule}
\frac{ \B\ts\sigdec_1\ra\G_1 \qquad \plusmap{\B}{\G_1}\ts\sigdec_2\ra\G_2 }
     { \B\ts\seqsigdec\ra\plusmap{\G_1}{\G_2} }
\end{equation}}
\deletion{\thenostrsharing}{\comments
\begin{description}
%
\item{(\ref{single-sigdec-rule})}
The first closure restriction of Section~\ref{closure-restr-sec}
can be  enforced by replacing the $\B$ in the premise by $\B_0+\of{\G}{\B}$.
\item{(\ref{sequence-sigdec-rule})}
   A signature declaration does not create any new structures
or types; hence the use of $+$ instead of $\oplus$.
\end{description}
}

\rulesec{Signature Bindings}{\B\ts\sigbind\ra\G}
\begin{equation}        % signature binding
\label{sigbind-rule}
\frac{ \B\ts\sigexp\ra\sig
        \qquad\langle\B\ts\sigbind\ra\G\rangle }
     { \B\ts\sigbinder\ra\{\sigid\mapsto\sig\}
       \ \langle +\ \G\rangle }\index{42.1}
\end{equation}%
%
                     % Specifications
\rulesec{Specifications}{\B\ts\spec\ra\E}
\begin{equation}        % value specification
\label{valspec-rule}
\frac{ \of{\C}{\B}\ts\valdesc\ra\VE }
     { \B\ts\valspec\ra\cl{}{\VE}\ \In\ \Env }\index{42.2}
\end{equation}

\replacement{\thenostrsharing}{
\begin{equation}        % type specification
\label{typespec-rule}
\frac{ \of{\C}{\B}\ts\typdesc\ra\TE }
     { \B\ts\typespec\ra\TE\ \In\ \Env }
\end{equation}}{\begin{equation}        % type specification
\label{typespec-rule}
\frac{ 
         \of{\C}{\B}\ts\typdesc\ra\TE \quad
         \forall(\t,\VE)\in\Ran\TE,\hbox{\ $t$ does not admit equality}
     }
     { \B\ts\typespec\ra\TE\ \In\ \Env }
\end{equation}}

\replacement{\thenostrsharing}{
\begin{equation}        % eqtype specification
\label{eqtypspec-rule}
\frac{ \of{\C}{\B}\ts\typdesc\ra\TE \qquad
       \forall(\theta,\CE)\in \Ran\TE,\ \theta {\rm\ admits\ equality} }
     { \B\ts\eqtypespec\ra\TE\ \In\ \Env }
\end{equation}}{\begin{equation}        % eqtype specification
\label{eqtypspec-rule}
\frac{ \of{\C}{\B}\ts\typdesc\ra\TE \qquad
       \forall(\t,\VE)\in \Ran\TE,\ \t {\rm\ admits\ equality} }
     { \B\ts\eqtypespec\ra\TE\ \In\ \Env }
\end{equation}}

\replacement{\thenostrsharing}{
\begin{equation}        % data specification
\label{datatypespec-rule}
\frac{ \plusmap{\of{\C}{\B}}{\TE}\ts\datdesc\ra\VE,\TE }
     { \B\ts\datatypespec\ra(\VE,\TE)\ \In\ \Env }
\end{equation}}{
\begin{equation}        % data specification
\label{datatypespec-rule}
\frac{ \begin{array}{c}
       \of{\C}{\B}\oplus\TE\ts\datdesc\ra\VE,\TE 
       \quad \forall(\t,\VE')\in\Ran\TE, \t\notin\of{\T}{\B}\\
       \hbox{$\TE$ maximises equality}
       \end{array}}
     { \B\ts\datatypespec\ra(\VE,\TE)\ \In\ \Env }
\end{equation}}

\insertion{\thedatatyperepl}{\begin{equation}
\label{datatypereplspec-rule}
\frac{ \B(\longtycon) = (\typefcn,\VE)\qquad
       \TE = \{\tycon\mapsto(\typefcn,\VE)\}
     }
     {\B\ts\datatypereplspec\ra (\VE,\TE)\ \In\ \Env}
\end{equation}}

\replacement{\theidstatus}{
\begin{equation}        % exception specification
\label{exceptionspec-rule}
\frac{ \of{\C}{\B}\ts\exndesc\ra\EE\quad\VE=\EE }
     { \B\ts\exceptionspec\ra(\VE,\EE)\ \In\ \Env }
\end{equation}}{\begin{equation}        % exception specification
\label{exceptionspec-rule}
\frac{ \of{\C}{\B}\ts\exndesc\ra\VE }
     { \B\ts\exceptionspec\ra\VE\ \In\ \Env }
\end{equation}}

\begin{equation}        % structure specification
%\label{structurespec-rule}
\frac{ \B\ts\strdesc\ra\SE }
     { \B\ts\structurespec\ra\SE\ \In\ \Env }
\end{equation}
\oldpagebreak

\deletion{\thetypabbr}{
\begin{equation}        % sharing specification
%\label{sharingspec-rule}
\frac{ \B\ts\shareq\ra\emptymap }
     { \B\ts\sharingspec\ra\emptymap\ \In\ \Env }\index{42.3}
\end{equation}}

\deletion{\thenolocalspec}{\begin{equation}        % local specification
%\label{localspec-rule}
\frac{ \B\ts\spec_1\ra\E_1 \qquad \plusmap{\B}{\E_1}\ts\spec_2\ra\E_2 }
     { \B\ts\localspec\ra\E_2 }
\end{equation}}

\deletion{\thenoopenspec}{\begin{equation}        % open specification
%\label{openspec-rule}
\frac{ \B(\longstrid_1)=(\m_1,\E_1)\quad\cdots\quad
       \B(\longstrid_n)=(\m_n,\E_n) }
     { \B\ts\openspec\ra\E_1 + \cdots +\E_n }
\end{equation}}

\replacement{\thesingleincludespec}{\begin{equation}        % include signature specification
\label{inclspec-rule}
\frac{ \sigord{\B(\sigid_1)}{}{(\m_1,\E_1)} \quad\cdots\quad
       \sigord{\B(\sigid_n)}{}{(\m_n,\E_n)} }
     { \B\ts\inclspec\ra\E_1 + \cdots +\E_n }
\end{equation}}{\begin{equation}        % include signature specification
\label{inclspec-rule}
\frac{  \B\ts\sigexp\ra\E}
     { \B\ts\singleinclspec\ra\E }
\end{equation}}

\begin{equation}        % empty specification
%\label{emptyspec-rule}
\frac{}
     { \B\ts\emptyspec\ra\emptymap{\rm\ in}\ \Env }
\end{equation}

\replacement{\thenostrsharing}{
\begin{equation}        % sequential specification
%\label{seqspec-rule}
\frac{ \B\ts\spec_1\ra\E_1 \qquad \plusmap{\B}{\E_1}\ts\spec_2\ra\E_2 }
     { \B\ts\seqspec\ra\plusmap{\E_1}{\E_2} }
\end{equation}}{\begin{equation}        % sequential specification
\label{seqspec-rule}
\frac{ \B\ts\spec_1\ra\E_1 \qquad \B\oplus\E_1\ts\spec_2\ra\E_2\qquad\Dom(\E_1)\cap\Dom(\E_2) = \emptyset }
     { \B\ts\seqspec\ra\plusmap{\E_1}{\E_2} }
\end{equation}}

\begin{equation}
\label{sharspec-rule}
\frac{\begin{array}{c}
        \B\ts \spec \ra \E\quad
          \hbox{$\E(\longtycon_i) = (\t_i,\VE_i)$ \ADD{and $\t_i\in\TyNamesk$}, $i = 1..n$}\\
        t\in\{\t_1,\ldots,\t_n\}\quad\hbox{$\t$ admits equality, if some $\t_i$ does}\\
        \{\t_1,\ldots,\t_n\}\cap\of{\T}{\B} = \emptyset\quad
           \rea = \{\t_1\mapsto\t,\ldots,\t_n\mapsto \t\}
      \end{array}
     }
     {\B\ts \newsharingspec\ra \rea(\E)}
\end{equation}%

\pagebreak
\noindent\comments
\begin{description}
\item{(\ref{valspec-rule})}
   $\VE$ is determined by $\B$ and $\valdesc$.
\item{(\ref{typespec-rule})--(\ref{datatypespec-rule})}
   \replacement{\thenostrsharing}{The type functions in $\TE$ may be chosen to achieve the sharing hypothesis
of rule~\ref{typshareq-rule} or the enrichment conditions of 
rules~\ref{structure-binding-rule} and~\ref{funbind-rule}. In particular, the type
names in $\TE$ in rule~\ref{datatypespec-rule} need not be new.
Also, in rule~\ref{typespec-rule} the type functions in $\TE$ may admit
equality.}{The type
names in $\TE$ are new.}
%
\item{(\ref{exceptionspec-rule})}
   \replacement{\theidstatus}{$\EE$}{$\VE$} is determined by $\B$ and $\exndesc$ and contains monotypes only.
\deletion{\thenostrsharing}{\item{(\ref{inclspec-rule})}
   The names $\m_i$ in the instances may be chosen to achieve sharing or
enrichment conditions.\index{43.0}}
\insertion{\thenostrsharing}{\item{(\ref{seqspec-rule})}
   Note that no sequential specification is allowed to specify the
   same identifier twice.}
\end{description} 

                  % Descriptions
\rulesec{Value Descriptions}{\C\ts\valdesc\ra\VE}
\replacement{\theidstatus}{
\begin{equation}         % value description
%\label{valdesc-rule}
\frac{ \C\ts\ty\ra\tau\qquad
       \langle\C\ts\valdesc\ra\VE\rangle }
     { \C\ts\valdescription\ra\{\var\mapsto\tau\}
       \ \langle +\ \VE\rangle }\index{43.1}
\end{equation}}{\begin{equation}         % value description
%\label{valdesc-rule}
\frac{ \C\ts\ty\ra\tau\qquad
       \langle\C\ts\valdesc\ra\VE\rangle }
     { \C\ts\valviddescription\ra\{\vid\mapsto(\tau,\isv)\}
       \ \langle +\ \VE\rangle }\index{43.1}
\end{equation}}

\rulesec{Type Descriptions}{\C\ts\typdesc\ra\TE}
\begin{equation}         % type description
\label{typdesc-rule}
\frac{ \begin{array}{c}
         \tyvarseq = \alphak\quad\t\notin\of{\T}{\C}\quad\arity\t=k \\
       \langle \C\ts\typdesc\ra\TE\qquad\t\notin\TyNamesFcn\TE\rangle
      \end{array}}
     { \C\ts\typdescription\ra\{\tycon\mapsto(\t,\emptymap)\}
       \ \langle +\ \TE\rangle }\index{43.2}
\end{equation}%
\comment Note that \deletion{\thenostrsharing}{any $\theta$ of arity $k$ may be chosen but that}
the \replacement{\thece}{constructor}{value} environment in the resulting type structure must be
empty. \replacement{\thefixtypos}{For example, \mbox{\ml{datatype s=c type t sharing s=t}}\  }{For example, \mbox{\ml{datatype s=C type t}} \mbox{\ml{sharing type t=s}}\  }
is a legal specification, but the type structure bound to \ml{t}
does not bind any value constructors.
\oldpagebreak

\rulesec{Datatype Descriptions}{\C\ts\datdesc\ra\VE,\TE}
\replacement{\thenostrsharing}{\begin{equation}         % datatype description
\label{datdesc-rule}
\frac{ \tyvarseq = \alphak\qquad\C,\alphakt\ts\condesc\ra\CE
       \qquad\langle\C\ts\datdesc\ra\VE,\TE\rangle }
     { \begin{array}{cl}
       \C\ts\datdescription\ra\\
       \qquad\qquad\cl{}{\CE}\langle +\ \VE\rangle,\
       \{\tycon\mapsto(t,\cl{}{\CE})\}\ \langle +\ \TE\rangle
       \end{array}
     }\index{43.3}
\end{equation}}{\begin{equation}         % datatype description
\label{datdesc-rule}
\frac{ \begin{array}{c}
        \tyvarseq = \alphak\qquad\C,\alphakt\ts\condesc\ra\VE
           \quad\arity \t  =  k\\
        \langle\C\ts\datdesc'\ra\VE',\TE'\qquad \forall(\t',\VE'')\in\Ran\TE',\,\t\neq\t'\rangle 
       \end{array}}
     { \begin{array}{l}
       \C\ts\newdatdescription\ra\\
       \qquad\cl{}{\VE} \langle +\ \VE'\rangle,\
       \{\tycon\mapsto(t,\cl{}{\VE})\}\ \langle +\ \TE'\rangle
       \end{array}
     }\index{43.3}
\end{equation}}

\replacement{\thece}{\rulesec{Constructor Descriptions}{\C,\tau\ts\condesc\ra\CE}}{\rulesec{Constructor Descriptions}{\C,\tau\ts\condesc\ra\VE}}
\replacement{\theidstatus}{\begin{equation}         % constructor description
%\label{condesc-rule}
\frac{\langle\C\ts\ty\ra\tau'\rangle\qquad
      \langle\langle\C,\tau\ts\condesc\ra\CE\rangle\rangle }
     {\begin{array}{l}
      \C,\tau\ts\longcondescription\ra\\
      \qquad\{\con\mapsto\tau\}\
     \langle +\ \{\con\mapsto\tau'\to\tau\}\ \rangle\
      \langle\langle +\ \CE\rangle\rangle
      \end{array}
     }\index{43.35}
\end{equation}}{\begin{equation}         % constructor description
%\label{condesc-rule}
\frac{\langle\C\ts\ty\ra\tau'\rangle\qquad
      \langle\langle\C,\tau\ts\condesc\ra\VE\rangle\rangle }
     {\begin{array}{l}
      \C,\tau\ts\longconviddescription\ra\\
      \qquad\{\vid\mapsto(\tau,\isc)\}\
     \langle +\ \{\vid\mapsto(\tau'\to\tau,\isc)\}\ \rangle\
      \langle\langle +\ \VE\rangle\rangle
      \end{array}
     }\index{43.35}
\end{equation}}

\replacement{\theidstatus}{
\rulesec{Exception Descriptions}{\C\ts\exndesc\ra\EE}}{\rulesec{Exception 
Descriptions}{\C\ts\exndesc\ra\VE}}
\replacement{\theidstatus}{
\begin{equation}         % exception description
\label{exndesc-rule}
\frac{ \langle\C\ts\ty\ra\tau\qquad\TyVarsFcn(\tau)=\emptyset\rangle\qquad
       \langle\langle\C\ts\exndesc\ra\EE\rangle\rangle }
     { \begin{array}{l}
        \C\ts\exndescriptiona\ra\\
        \quad\quad\{\exn\mapsto\EXCN\}\ \langle +\ \{\exn\mapsto\tau\rightarrow\EXCN\}\rangle\ \langle\langle +\ \EE\rangle\rangle 
       \end{array}
     }\index{43.4}
\end{equation}}{\begin{equation}         % exception description
\label{exndesc-rule}
\frac{ \langle\C\ts\ty\ra\tau\qquad\TyVarsFcn(\tau)=\emptyset\rangle\qquad
       \langle\langle\C\ts\exndesc\ra\VE\rangle\rangle }
     { \begin{array}{l}
        \C\ts\exnviddescriptiona\ra\\
        \quad\quad\{\vid\mapsto(\EXCN,\ise)\}\ \langle +\ \{\vid\mapsto(\tau\rightarrow\EXCN,\ise)\}\rangle\ \langle\langle +\ \VE\rangle\rangle 
       \end{array}
     }\index{43.4}
\end{equation}}


\rulesec{Structure Descriptions}{\B\ts\strdesc\ra\SE}
\replacement{\thenostrsharing}{
\begin{equation}
\label{strdesc-rule}
\frac{ \B\ts\sigexp\ra\S\qquad\langle\B\ts\strdesc\ra\SE\rangle }
     { \B\ts\strdescription\ra\{\strid\mapsto\S\}\ \langle +\ \SE\rangle }\index{43.5}
\end{equation}}{\begin{equation}
\label{strdesc-rule}
\frac{ \B\ts\sigexp\ra\E\qquad\langle\B + \TyNamesFcn\E\ts\strdesc\ra\SE\rangle }
     { \B\ts\strdescription\ra\{\strid\mapsto\E\}\ \langle +\ \SE\rangle }\index{43.5}
\end{equation}}


\deletion{\thenostrsharing}{
\rulesec{Sharing Equations}{\B\ts\shareq\ra\emptymap}
\begin{equation}          % structure sharing equation
\label{strshareq-rule}
\frac{ \of{\m}{\B(\longstrid_1)}=\cdots =\of{\m}{\B(\longstrid_n)} }
     { \B\ts\strshareq\ra\E,\emptymap }\index{44.1}
\end{equation}
\vspace{6pt}
\begin{equation}          % type sharing equation
\label{typshareq-rule}
\frac{ \of{\typefcn}{\B(\longtycon_1)}=\cdots=\of{\typefcn}{\B(\longtycon_n)} }
     { \B\ts\typshareq\ra\emptymap }
\end{equation}

\vspace{6pt}
\begin{equation}          % multiple sharing equation
%\label{multshareq-rule}
\frac{ \B\ts\shareq_1\ra\emptymap\qquad\B\ts\shareq_2\ra\emptymap }
     { \B\ts\multshareq\ra\emptymap }
\end{equation}%

%
\comments
\begin{description}
\item{(\ref{strshareq-rule})}
   By the definition of consistency the premise is weaker than\linebreak
$\B(\longstrid_1) = \cdots = \B(\longstrid_n)$.
Two different structures with the same name may be thought of
as representing different views. The requirement that $\B$ is 
consistent forces different views to be consistent.
\end{description}
%
\oldpagebreak
\begin{description}
\item{(\ref{typshareq-rule})}
   By\index{44.1.5} 
the definition of consistency the premise is weaker than\linebreak
$\B(\longtycon_1) = \cdots = \B(\longtycon_n)$.
A type structure with empty constructor environment may have the
same type name as one with a non-empty constructor environment;
the former could arise from a type description, and the latter
from a datatype description. 
However, the requirement that $\B$ is
consistent will prevent two type structures with constructor
environments which have different 
non-empty domains from sharing the same type name.
end{description}
}
%
%
%                       Type abbreviation rules
%
%
%
\deletion{\thenofuncspec}{
%
%                       Functor Specification rules
%
\rulesec{Functor Specifications}{\B\ts\funspec\ra\F}
\begin{equation}        % single functor specification
\label{singfunspec-rule}
\frac{ \B\ts\fundesc\ra\F }
     { \B\ts\singfunspec\ra\F }\index{44.2}
\end{equation}

\vspace{6pt}
\begin{equation}        % empty functor specification
%\label{emptyfunspec-rule}
\frac{}
     { \B\ts\emptyfunspec\ra\emptymap }
\end{equation}

\vspace{6pt}
\begin{equation}        % sequential functor specification
%\label{seqfunspec-rule}
\frac{ \B\ts\funspec_1\ra\F_1\qquad
       \B+\F_1\ts\funspec_2\ra\F_2 }
     { \B\ts\seqfunspec\ra\plusmap{\F_1}{\F_2} }
\end{equation}
\comments
\begin{description}
\item{(\ref{singfunspec-rule})}
The second closure restriction of Section~\ref{closure-restr-sec}
can be enforced by replacing the $\B$ in the premise by $\B_0+\of{\G}{\B}$.
\end{description}
\rulesec{Functor Descriptions}{\B\ts\fundesc\ra\F}
\begin{equation}        % functor description
%\label{fundesc-rule}
\frac{ \B\ts\funsigexp\ra\funsig\qquad
       \langle\B\ts\fundesc\ra\F\rangle}
     { \B\ts\longfundesc\ra\{\funid\mapsto\funsig\}
       \langle +\ \F\rangle}\index{44.3}
\end{equation}

\rulesec{Functor Signature Expressions}{\B\ts\funsigexp\ra\funsig}
\begin{equation}        % functor signature
\label{funsigexp-rule}
%version 1:
%\frac{
%      \begin{array}{c}
%      \B\ts\sigexp\ra\S\qquad\longsig{}{\rm\ principal\ in\ }\B\\
%      \B\oplus\{\strid\mapsto\S\} \ts\sigexp'\ra\S'\\
%      \N' = \NamesFcn\S'\setminus((\of{\N}{\B})\cup\N) 
%      \end{array}
%     }
%     {\B\ts\longfunsigexpa\ra(\N)(\S,(\N')\S')}\index{44.4}
%version2: \frac{\begin{array}{rl}
%      \B\ts\sigexp\ra\S&\mbox{$(N)S$ principal in $\B$}\\
%      \B\oplus\{\strid\mapsto\S\}\ts\sigexp'\ra\S'&
%      \mbox{$(N')S'$ principal in $\B\oplus\{\strid\mapsto\S\}$}
%      \end{array}}
%     {\B\ts\longfunsigexpa\ra(N)(S,(N')S')}\index{44.4}
%\end{equation}
\frac{\B\ts\sigexp\ra(\N)\S\qquad
      \B\oplus\{\strid\mapsto\S\}\ts\sigexp'\ra(\N')\S'}
     {\B\ts\longfunsigexpa\ra(N)(S,(N')S')}\index{44.4}
\end{equation}%
\comment
The signatures $(\N)\S$ and $(\N')\S'$ are equality-principal 
and type-explicit, see rule~\ref{topmost-sigexp-rule}.
} %deletion
%                       Functor and Program rules

\rulesec{Functor Declarations}{\B\ts\fundec\ra\F}
\begin{equation}        % single functor declaration
\label{singfundec-rule}
\frac{ \B\ts\funbind\ra\F }
     { \B\ts\singfundec\ra\F }\index{45.1}
\end{equation}

\deletion{\thenostrsharing}{
\vspace{6pt}
\begin{equation}        % empty functor declaration
%\label{emptyfundec-rule}
\frac{}
     { \B\ts\emptyfundec\ra\emptymap }
\end{equation}

\vspace{6pt}
\oldpagebreak

\begin{equation}        % sequential functor declaration
%\label{seqfundec-rule}
\frac{ \B\ts\fundec_1\ra\F_1\qquad
       \B+\F_1\ts\fundec_2\ra\F_2 }
     { \B\ts\seqfundec\ra\plusmap{\F_1}{\F_2} }\index{45.1.5}
\end{equation}}
\deletion{\thenoclosurerestriction}{\comments
\begin{description}
\item{(\ref{singfundec-rule})}
The third closure restriction of Section~\ref{closure-restr-sec}
can be enforced by replacing the $\B$ in the premise 
by $\B_0+(\of{\G}{\B})+(\of{\F}{\B})$.
\end{description}}
\rulesec{Functor Bindings}{\B\ts\funbind\ra\F}
\replacement{\thenostrsharing}{\begin{equation}        % functor binding
\label{funbind-rule}
\frac{
      \begin{array}{c}
      \B\ts\sigexp\ra(\N)\S\qquad
      \B\oplus\{\strid\mapsto\S\} \ts\strexp\ra\S' \\
       \langle
      \B\oplus\{\strid\mapsto\S\} \ts\sigexp'\ra\sig',\ \sig'\geq\S''\prec\S'
       \rangle\\
      \N' = \NamesFcn\S'\setminus((\of{\N}{\B})\cup\N) \\
       \langle\langle\B\ts\funbind\ra\F\rangle\rangle
      \end{array}
     }
     {
      \begin{array}{c}
       \B\ts\funstrbinder\ \optfunbind\ra\\
       \qquad\qquad \qquad
              \{\funid\mapsto(\N)(\S,(\N')\S'\langle'\rangle)\}
              \ \langle\langle +\ \F\rangle\rangle
      \end{array}
     }\index{45.2}
\end{equation}}{\begin{equation}        % functor binding
\label{funbind-rule}
\frac{
      \begin{array}{c}
      \B\ts\sigexp\ra(\T)\E\qquad
      \B\oplus\{\strid\mapsto\E\} \ts\strexp\ra\E' 
      \\
      \T\cap(\of{\T}{\B}) = \emptyset\quad \T' = \TyNamesFcn\E'\setminus((\of{\T}{\B})\cup\T) \\
       \langle\B\ts\funbind\ra\F\rangle
      \end{array}
     }
     {
      \begin{array}{c}
       \B\ts\barefunstrbinder\ \langle\boxml{and \funbind}\rangle\ra\\
       \qquad\qquad \qquad
              \{\funid\mapsto(\T)(\E,(\T')\E')\}
              \ \langle +\ \F\rangle
      \end{array}
     }\index{45.2}
\end{equation}
}\comment \deletion{\thenostrsharing}{The  requirement that $(\N)\S$ be equality-principal,
implicit in the first premise, forces $(\N)\S$ to be
as general as possible given the sharing constraints in $\sigexp$.
The requirement that $(\N)\S$ be type-explicit ensures that there is
at most one realisation via which an actual argument can match
$(\N)\S$.}Since $\oplus$ is used, any \deletion{\thenostrsharing}{structure name $\m$ and}type name $\t$ in
\replacement{\thenostrsharing}{$\S$ }{$\E$ }acts like a constant in the functor body; in particular,
it ensures that further names generated during elaboration of the
body are distinct from \deletion{\thenostrsharing}{$\m$ and }$\t$. 
\replacement{\thenostrsharing}{The set $\N'$ is
chosen such that every  name free
in $(\N)\S$ or $(\N)(\S,(\N')\S')$ is free in $\B$.}{The set $\T'$ is
chosen such that every  name free
in $(\T)\E$ or $(\T)(\E,(\T')\E')$ is free in $\B$.}
\rulesec{Top-level Declarations}{\B\ts\topdec\ra\B'}
%\rulesec{Programs}{\B\ts\program\ra\B'}
\replacement{\thenostrsharing}{\begin{equation}	% structure-level declaration
\label{strdectopdec-rule}
\frac{\B\ts\strdec\ra\E \quad\imptyvars\E=\emptyset}
     {\B\ts\strdec\ra
      (\NamesFcn\E,\E)\ \In\ \Basis
     }\index{45.3}
\end{equation}}{\begin{equation}        % structure-level declaration
\label{strdectopdec-rule}
\frac{\begin{array}{c}
        \B\ts\strdec\ra\E \quad \langle \B\oplus \E\ts \topdec\ra \B'\rangle\\
        B'' = (\TyNamesFcn\E,\E)\In\ \Basis\; \langle + \B'\rangle\quad\TyVarFcn\B''=\emptyset
      \end{array}}
     {\B\ts\strdecintopdec\ra\B''}
\end{equation}}

\vspace{6pt}
\replacement{\thenostrsharing}{\begin{equation}	% signature declaration
%\label{sigdectopdec-rule}
\frac{\B\ts\sigdec\ra\G \quad\imptyvars\G=\emptyset}
     {\B\ts\sigdec\ra
      (\NamesFcn\G,\G)\ \In\ \Basis
     }\index{46.0}
\end{equation}}{\begin{equation}        % signature declaration
\frac{\begin{array}{c}
        \B\ts\sigdec\ra \G\quad \langle\B\oplus\G\ts\topdec\ra\B'\rangle\\
        \B'' = (\TyNamesFcn\G,G)\ \In\ \Basis\;\langle + \B'\rangle
      \end{array}}
     {\B\ts\sigdecintopdec\ra \B''
     }\index{46.0}
\end{equation}}

\vspace{6pt}
\replacement{\thenostrsharing}{\begin{equation}	% functor declaration
\label{fundectopdec-rule}
\frac{\B\ts\fundec\ra\F \quad\imptyvars\F=\emptyset}
     {\B\ts\fundec\ra
      (\NamesFcn\F,\F)\ \In\ \Basis
     }
\end{equation}}{\begin{equation}        % functor declaration
\label{fundectopdec-rule}
\frac{\begin{array}{c}
          \B\ts\fundec\ra\F\quad\langle\B\oplus\F\ts\topdec\ra\B'\rangle\\
          B'' = (\TyNamesFcn\F,\F)\ \In\ \Basis\; \langle+\B'\rangle\quad \TyVarsFcn\B''=\emptyset
      \end{array}}
     {\B\ts\fundecintopdec\ra\B''}
\end{equation}}
  
\comments
\replacement{\thenostrsharing}{
\begin{description}
\item{(\ref{strdectopdec-rule})--(\ref{fundectopdec-rule})} The side
conditions ensure that no free imperative 
type variables enter the 
basis.\index{46.01}
\end{description}}{
\begin{description}
\item{(\ref{strdectopdec-rule})--(\ref{fundectopdec-rule})} 
No free type variables enter the  basis: if $\B\ts\topdec\ra\B'$
then $\TyVarsFcn(\B') = \emptyset$.\index{46.01}
\end{description}}
\oldpagebreak

\deletion{\thenofuncspec}{\subsection{Functor Signature Matching}
\label{fun-sig-match-sec}
As\index{46} pointed out in Section~\ref{mod-gram-sec} on the 
grammar for Modules, there is no phrase class whose elaboration 
requires matching one functor signature to another functor signature.
But a precise definition of this matching is needed, since a 
functor $g$ may only be separately compiled in the presence of 
specification of any functor $f$ to which $g$ refers, and then a 
real functor $f$ must match this specification.
In the case, then, that $f$ has been specified by a functor signature
\[\funsig_1\ =\ \longfunsig{1}\]
and that later $f$ is declared with functor signature
\[\funsig_2\ =\ \longfunsig{2}\]
the following matching rule will be employed:

A functor signature
$\funsig_2\ =\ \longfunsig{2}$ {\sl matches} another functor signature,
$\funsig_1\ =\ \longfunsig{1}$, if there exists a realisation $\rea$ 
such that
\begin{enumerate}
\item $\longsig{1}$ matches $\longsig{2}$ via $\rea$, and
\item $\rea((\N_2')\S_2')$ matches $(\N_1')\S_1'$.
\end{enumerate}
The first condition ensures that the real functor signature $\funsig_2$
for $f$ requires the argument $\strexp$ of any application $\f(\strexp)$
to have no more sharing, and no more richness, than was predicted by
the specified signature $\funsig_1$.
The second condition ensures that the real functor signature $\funsig_2$,
instantiated to $(\rea\S_2,\rea((\N_2')\S_2'))$, provides in the result of
the application $\f(\strexp)$
no less sharing, and no less richness, than was predicted by
the specified signature $\funsig_1$.

%We claim that any phrase -- e.g. the declaration of the functor $g$ above --
%which elaborates successfully in a basis $\B$ with $\B(f)=\funsig_1$ will
%also elaborate successfully in the basis $\B+\{f\mapsto\funsig_2\}$.  This
%claim justifies our definition of functor matching.
% -- this claim is false because of open.

} % deletion
