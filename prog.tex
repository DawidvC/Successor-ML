%!TEX root = root.tex
%

\section{Programs}
\label{prog-sec}
The phrase class Program of programs is defined as follows
\[\program ::= \longprog\]

Hitherto,\index{64.1} the semantic rules have not exposed the interactive
nature of the language.
During an ML session the user can type in a phrase, more
precisely a phrase of the form $\topdec$ as defined in Figure~\ref{prog-syn},
page~\pageref{prog-syn}. Upon the following semicolon,
the machine will then attempt to parse, elaborate and
evaluate the phrase returning either a result or, if any
of the phases fail, an error message.
The outcome is significant for what the user subsequently types,
so we need to answer questions such as: if the elaboration
of a top-level declaration succeeds, but its evaluation
fails, then does the result of the elaboration get
recorded in the static basis?

In practice, ML implementations may provide a directive as
a form of top-level declaration  for including
programs from files rather than directly from the terminal. 
In case a file consists of a sequence of top-level declarations 
(separated by semicolons) and the machine detects an error in one of these,
it is probably sensible to abort the execution of the directive.
Rather than introducing a distinction between,
say, batch programs and interactive programs, we shall tacitly regard
all programs as interactive, and leave to implementers to
clarify how the inclusion of files, if provided,  affects the 
updating of the static and dynamic basis.
Moreover, we shall focus on elaboration and evaluation and leave the
handling of parse errors to implementers (since it naturally depends
on the kind of parser being employed).  Hence, in this section the 
{\sl execution} of a program means the combined elaboration
and evaluation of the program.

So far, for simplicity, we have used the same notation $\B$
to stand for both a static and a dynamic basis, and this has
been possible because we have never needed to discuss static
and dynamic semantics at the same time.
In giving the semantics of programs, however, let us rename as 
\mbox{StaticBasis} the class  Basis defined in the static
semantics of modules, Section~\ref{statmod-sem-obj-sec},
and let us use $\Bstat$ to range over StaticBasis. Similarly, let
us rename as  \mbox{DynamicBasis} the class Basis defined in the
dynamic semantics of modules, Section~\ref{dynmod-comp-obj-sec},
and let us use $\Bdyn$ to range over \mbox{DynamicBasis}.
We now define
\[\B\ {\rm or} \ (\Bstat,\Bdyn)\in{\rm Basis}={\rm Static Basis}
\times {\rm DynamicBasis}.\]
Further, we shall use $\tsstat$ for elaboration as defined in 
Section~\ref{statmod-sec}, and $\tsdyn$ for evaluation
as defined in Section~\ref{dynmod-sec}.
Then $\ts$ will be reserved for the execution of
programs, which thus is expressed by a sentence of the
form
\[\s,\B\ts\program\ra\B',\s'\]
This may be read as follows: 
starting in basis $\B$ with state $\s$ the execution of
$\program$ results in a basis $\B'$ and a state $\s'$.

It\index{64.2} must be understood  that executing a program never results in an
exception. If the evaluation of a $\topdec$ yields an exception 
(for instance because of a $\RAISE$ expression\deletion{\thelibrary}{ or external intervention}) then
the result of executing the program ``$\topdec\ \mbox{\ml{;}}$'' is the
original basis together with the state which is in force when the exception is
generated. In particular, the exception convention\index{65.1} of 
Section~\ref{dyncor-inf-rules-sec}
is not applicable to the ensuing rules.

We represent the non-elaboration of a top-level declaration by
 $\ldots\tsstat\topdec\not\ra$. \deletion{\thelibrary}{(This covers also the case in which a 
user interrupts the elaboration.)}

\rulesec{Programs}{\s,\B\ts\program\ra\B',s'}
\begin{equation}            % failing elaboration
\label{fail-elab-prog-rule}
\frac{\of{\Bstat}{\B}\tsstat\topdec\not\ra\qquad
      \langle\s,\B\ts\program\ra\B',s'\rangle}
     {\s,\B\ts\longprog\ra\B\langle'\rangle,s\langle'\rangle}\index{65.2}
\end{equation}

\begin{equation}            % evaluation raising exception
\label{dyn-exc-prog-rule}
\frac{\begin{array}{lr}
      \of{\Bstat}{\B}\tsstat\topdec\ra\Bstat^{(1)} & \\
      \s,\of{\Bdyn}{\B}\tsdyn\topdec\ra\p,\s' &
                  \langle\s',\B\ts\program\ra\B',\s''\rangle
      \end{array}}
     {\s,\B\ts\longprog\ra\B\langle'\rangle,\s'\langle'\rangle}
\end{equation}

\begin{equation}            % success
\label{success-prog-rule}
\frac{\begin{array}{ll}
       \of{\Bstat}{\B}\tsstat\topdec\ra\Bstat^{(1)} & \\
       \s,\of{\Bdyn}{\B}\tsdyn\topdec\ra\Bdyn^{(1)},s'&
                   \B'=\B\oplus(\Bstat^{(1)},\Bdyn^{(1)})\\
       &           \langle\s',\B'\ts\program\ra\B'',s''\rangle
      \end{array}
      }
      {\s,\B\ts\longprog\ra\B'\langle'\rangle,\s'\langle'\rangle}
\end{equation}
\comments
\begin{description}
\item{(\ref{fail-elab-prog-rule})}
  A failing elaboration has no effect whatever\ADD{, except possibly for
  fixity directives contained in the \topdec}.
\item{(\ref{dyn-exc-prog-rule})}
  An evaluation which yields an exception nullifies
the change in the static basis, but does not
nullify side-effects on the state which may have occurred
before the exception was raised.
\end{description}
\subsection*{Core language Programs}
A\index{65.4} program is called a {\sl core language program\/} if it can be
parsed in the reduced grammar defined as follows:
\begin{enumerate}
\item Replace the definition of top-level declarations by
\[\topdec\ ::=\ \strdec\]
\item Replace the definition of structure-level declarations by
\[\strdec\ ::=\ \dec\]
\deletion{\thelibrary}{
\item Omit\index{65.5} the {\OPEN} declaration from the syntax class of
declarations $\dec$
\item Restrict the long identifier classes to identifiers, i.e.
omit qualified identifiers.}
\end{enumerate}
\deletion{\thelibrary}{
This means that  several components of a basis, for example the
signature and functor environments, are irrelevant to the execution
of a core language program.}