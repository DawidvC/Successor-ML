% this is file `root', the root of the whole semantics
%\documentclass[12pt,twoside]{article}
%\usepackage{a4}
\documentclass[12pt]{article}

\usepackage[textwidth=6.5in,textheight=8.5in,includehead,includefoot]{geometry}
\usepackage{url}

\title{The Definition of Successor ML\thanks{
  This document is derived from
  \emph{The Definition of Standard ML} (Milner, Tofte, Harper, and MacQueen,
  MIT Press, 1997), and
  \emph{HaMLet S: To Become Or Not To Become Successor ML} (Rossberg 2008).}}
\author{Robert Harper and David MacQueen(ed.)}
\date{\today}
%\inputonly{mac,syncor}
%!TEX root = root.tex
%

\newcommand{\ml}[1]{{\tt #1}}
\newcommand{\boxml}[1]{\hbox{\tt #1}}
\def\ignore#1{}
\renewcommand{\cdots}{\mbox{$\cdot\!\cdot\!\cdot$}}
\newcommand{\twoldots}{\mathinner{\ldotp\ldotp}}

        % Core Language
\newcommand{\ttlbrace}{\mbox{\tt\char'173}}
\newcommand{\ttrbrace}{\mbox{\tt\char'175}}
\newcommand{\lttbrace}{\mbox{\tt\char'173}}
\newcommand{\rttbrace}{\mbox{\tt\char'175}}
\newcommand{\ttprime}{\mbox{\tt\char'047}}
\newcommand{\tttilde}{\mbox{\tt\char'176}}
\newcommand\uconst{\mbox{\tt\char'134u$xxxx$}}
\newcommand{\ABSTYPE}{{\tt abstype}}
\newcommand{\AND }{{\tt and}}
\newcommand{\ANDALSO }{{\tt andalso}}
\newcommand{\AS}{{\tt as}}
\newcommand{\CASE}{{\tt case}}
\newcommand{\DO}{{\tt do}}
\newcommand{\DATATYPE}{{\tt datatype}}
\newcommand{\ELSE}{{\tt else}}
\newcommand{\END}{{\tt end}}
\newcommand{\EXCEPTION}{{\tt exception}}
\newcommand{\FUN}{{\tt fun}}
\newcommand{\FN}{{\tt fn}}
\newcommand{\HANDLE}{{\tt handle}}
\newcommand{\IF}{{\tt if}}
\newcommand{\IN}{{\tt in}}
\newcommand{\INFIX}{{\tt infix}}
\newcommand{\INFIXR}{{\tt infixr}}
\newcommand{\LET}{{\tt let}}
\newcommand{\LOCAL}{{\tt local}}
\newcommand{\NONFIX}{{\tt nonfix}}
\newcommand{\OF}{{\tt of}}
\newcommand{\OP}{{\tt op}}
\newcommand{\ORELSE}{{\tt orelse}}
\newcommand{\RAISE}{{\tt raise}}
\newcommand{\REC}{{\tt rec}}
\newcommand{\THEN}{{\tt then}}
\newcommand{\TYPE}{{\tt type}}
\newcommand{\VAL}{{\tt val}}
\newcommand{\WITH}{{\tt with}}
\newcommand{\WHERETYPE}{{\tt wheretype}}
\newcommand{\WHERE}{{\tt where}}
\newcommand{\WHILE}{{\tt while}}

        % Modules
\newcommand{\EQTYPE}{\ml{eqtype}}
\newcommand{\FUNCTOR}{\ml{functor}}
\newcommand{\INCLUDE}{\ml{include}}
\newcommand{\SIG}{\ml{sig}}
\newcommand{\OPEN}{\ml{open}}
\newcommand{\SHARING}{\ml{sharing}}
\newcommand{\SIGNATURE}{\ml{signature}}
\newcommand{\STRUCT}{\ml{struct}}
\newcommand{\STRUCTURE}{\ml{structure}}
\newcommand{\WITHTYPE}{{\tt withtype}}
\newcommand{\ABSTRACT}{\boxml{:>}}
          % Identifier class names
\newcommand{\VId}{{\rm VId}}
\newcommand{\Var}{{\rm Var}}
\newcommand{\Con}{{\rm Con}}
\newcommand{\SCon}{{\rm SCon}}
\newcommand{\Exn}{{\rm ExCon}}
\newcommand{\TyVar}{{\rm TyVar}}
\newcommand{\ImpTyVar}{{\rm ImpTyVar}}
\newcommand{\AppTyVar}{{\rm AppTyVar}}
\newcommand{\TyCon}{{\rm TyCon}}
\newcommand{\Lab}{{\rm Lab}}
\newcommand{\StrId}{{\rm StrId}}
\newcommand{\FunId}{{\rm FunId}}
\newcommand{\SigId}{{\rm SigId}}

          % Identifier class variables
\newcommand{\id}{{\it id}}
\newcommand{\vid}{{\it vid}}
\newcommand{\var}{{\it var}}
\newcommand{\con}{{\it con}}
\newcommand{\scon}{{\it scon}}
\newcommand{\exn}{{\it excon}}
\newcommand{\tyvar}{{\it tyvar}}
\newcommand{\atyvar}{\mbox{\tt \ttprime a}}
\newcommand{\btyvar}{\mbox{\tt \ttprime b}}
\newcommand{\ctyvar}{\mbox{\tt \ttprime c}}
\newcommand{\aityvar}{\mbox{\tt \ttprime\_a}}
\newcommand{\aetyvar}{\mbox{\tt \ttprime\ttprime a}}
\newcommand{\tycon}{{\it tycon}}
\newcommand{\lab}{{\it lab}}
\newcommand{\strid}{{\it strid}}

\newcommand{\longvid}{{\it longvid}}
\newcommand{\longvar}{{\it longvar}}
\newcommand{\longcon}{{\it longcon}}
\newcommand{\longexn}{{\it longexcon}}
\newcommand{\longtycon}{{\it longtycon}}
\newcommand{\longstrid}{{\it longstrid}}

          % General classes
\newcommand{\phrase}{{\it phrase}}

          % Core Syntax Classes
\newcommand{\apexp}{{\it appexp}}
\newcommand{\atpat}{{\it atpat}}
\newcommand{\appat}{{\it apppat}}
\newcommand{\inpat}{{\it infpat}}
\newcommand{\atexp}{{\it atexp}}
\newcommand{\bind}{{\it bind}}
\newcommand{\constrs}{{\it conbind}}
\newcommand{\ConBind}{{\rm ConBind}}
\newcommand{\datbind}{{\it datbind\/}}
\newcommand{\dec}{{\it dec}}
\newcommand{\dir}{{\it dir}}
\newcommand{\exnbind}{{\it exbind}}
\renewcommand{\exp}{{\it exp}}
\newcommand{\nexp}{{\it nexp}}
\newcommand{\conexp}{{\it conexp}}
\newcommand{\nexprow}{{\it nexprow}}
\newcommand{\fvalbind}{{\it fvalbind}}
\newcommand{\fmatch}{{\it fmatch}}  % SuccML
\newcommand{\fmrule}{{\it fmrule}}  % SuccML
\newcommand{\fpat}{{\it fpat}}  % SuccML
\newcommand{\handler}{{\it handler}}
\newcommand{\hanrule}{{\it hrule}}
\newcommand{\inexp}{{\it infexp}}
\newcommand{\labexps}{{\it exprow}}
\newcommand{\labpats}{{\it patrow}}
\newcommand{\labtys}{{\it tyrow}}
\newcommand{\match}{{\it match}}
\newcommand{\pat}{{\it pat}}
\newcommand{\mrule}{{\it mrule}}
\newcommand{\ty}{{\it ty}}
\newcommand{\tyseq}{{\it tyseq}}
\newcommand{\tyvarseq}{{\it tyvarseq}}
\newcommand{\typbind}{{\it typbind}}
\newcommand{\valbind}{{\it valbind}}

 % ranged over by the variables




       % Modules syntax classes

\newcommand{\datdesc}{{\it datdesc}}
\newcommand{\condesc}{{\it condesc}}
\newcommand{\exndesc}{{\it exdesc}}
\newcommand{\funbind}{{\it funbind}}
\newcommand{\fundec}{{\it fundec}}
\newcommand{\fundesc}{{\it fundesc}}
\newcommand{\funid}{{\it funid}}
\newcommand{\funsigexp}{{\it funsigexp}}
\newcommand{\funspec}{{\it funspec}}
\newcommand{\funtyp}{{\it funtyp}}
\newcommand{\longstridk}{\strid_1.\cdots.\strid_k}
\newcommand{\longtyconk}{\strid_1.\cdots.\strid_k.\tycon}
\newcommand{\topdec}{{\it topdec}}
\newcommand{\program}{{\it program}}
\newcommand{\sigid}{{\it sigid}}
\newcommand{\shareq}{{\it shareq}}
\newcommand{\sigbind}{{\it sigbind}}
\newcommand{\sigdec}{{\it sigdec}}
\newcommand{\sigexp}{{\it sigexp}}
\newcommand{\spec}{{\it spec}}
\newcommand{\strbind}{{\it strbind}}
\newcommand{\strdec}{{\it strdec}}
\newcommand{\strexp}{{\it strexp}}
\newcommand{\strdesc}{{\it strdesc}}
\newcommand{\typdesc}{{\it typdesc}}
\newcommand{\valdesc}{{\it valdesc}}
\newcommand{\typabbr}{{\it typabbr}}
           % Core Language Phrases

           % Expressions

\newcommand{\recexp}{\mbox{$\langle\labexps\rangle$}}
\newcommand{\longlabexps}{\mbox{\lab\ \ml{=} \exp\
                          $\langle$ \ml{,} \labexps$\rangle$}}
\newcommand{\parexp}{\mbox{\ml{(} \exp\ \ml{)}}}
\newcommand{\appexp}
           {\mbox{\exp\ \atexp}}
\newcommand{\infexp}
           {\mbox{$\exp_1\ \id\ \exp_2$}}
\newcommand{\vidinfexp}
           {\mbox{$\exp_1\ \vid\ \exp_2$}}
\newcommand{\opp}
           {\mbox{$\langle\OP\rangle$}}
\newcommand{\typedexp}
           {\mbox{\exp\ \ml{:} \ty}}
\newcommand{\handlexp}
           {\mbox{\exp\ \HANDLE\ \match}}
\newcommand{\raisexp}
           {\mbox{\RAISE\ \exp}}
\newcommand{\letexp}
           {\mbox{\LET\ \dec\ \IN\ \exp\ \END}}
\newcommand{\fnexp}
           {\mbox{\FN\ \match}}
\newcommand{\extrecexp}{\mbox{\ml{...}\ \ml{=}\ \exp}} % SuccML

           % Matches and Handlers
\newcommand{\longmatch}
           {\mbox{\mrule\ $\langle$ \ml{|} \match$\rangle$}}
\newcommand{\longmatcha}
           {\mbox{\mrule\ \ml{|} \match}}
\newcommand{\longmrule}
           {\mbox{\pat\ \ml{=>} \exp}}
\newcommand{\longhandler}
           {\mbox{\hanrule\ $\langle$ \ml{||} \handler$\rangle$}}
\newcommand{\longhandlera}
           {\mbox{\hanrule\ \ml{||} \handler}}
\newcommand{\longhanrule}
           {\mbox{\longexn\ \WITH\ \match}}
\newcommand{\lasthanrule}
           {\mbox{? \ml{=>} \exp}}

          % Declarations
\newcommand{\valdec}
           {\mbox{\VAL\ \valbind}}
\newcommand{\explicitvaldec}
           {\mbox{\VAL\ \ADD{$\langle\REC\rangle$}\ \tyvarseq\ \valbind}}
\newcommand{\valdecS}
           {\mbox{$\VAL_{\U}$\ \valbind}}
\newcommand{\typedec}
           {\mbox{\TYPE\ \typbind}}
\newcommand{\datatypedec}
           {\mbox{\DATATYPE\ \datbind}}
\newcommand{\datatypedeca}
           {\mbox{\DATATYPE\ \datbind\ $\langle\WITHTYPE\ \typbind\rangle$}}
\newcommand{\datatyperepldec}
           {\mbox{$\DATATYPE\, \tycon\,\EQ\;\DATATYPE\,\longtycon$}}
\newcommand{\datatyperepldecb}
           {\parbox{70mm}{\strut$\DATATYPE\, \tycon\,\EQ\,$\linebreak \hbox{}\hskip5mm$\DATATYPE\,\longtycon$\strut}}
%\newcommand{\datatyperepldeca}
%           {\mbox{$\DATATYPE\, \tycon\,\EQ\,\DATATYPE\,\longtycon$}}
\newcommand{\abstypedec}
           {\mbox{\ABSTYPE\ \datbind\ \WITH\ \dec\ \END}}
\newcommand{\abstypedeca}
           {\mbox{\ABSTYPE\ \datbind\ $\langle\WITHTYPE\ \typbind\rangle$}}
\newcommand{\exceptiondec}
           {\mbox{\EXCEPTION\ \exnbind}}
\newcommand{\localdec}
           {\mbox{\LOCAL\ $\dec_1\ \IN\ \dec_2$\ \END}}
\newcommand{\emptydec}
           {\mbox{\qquad}}
\newcommand{\seqdec}
           {\mbox{$\dec_1\ \langle\boxml{;}\rangle\ \dec_2$}}
\newcommand{\longinfix}
           {\mbox{$\INFIX\ \langle d\rangle\ \id_1\ \cdots\ \id_n$}}
\newcommand{\newlonginfix}
           {\mbox{$\INFIX\ \langle d\rangle\ \vid_1\ \cdots\ \vid_n$}}
\newcommand{\longinfixr}
           {\mbox{$\INFIXR\ \langle d\rangle\ \id_1\ \cdots\ \id_n$}}
\newcommand{\newlonginfixr}
           {\mbox{$\INFIXR\ \langle d\rangle\ \vid_1\ \cdots\ \vid_n$}}
\newcommand{\longnonfix}
           {\mbox{$\NONFIX\ \id_1\ \cdots\ \id_n$}}
\newcommand{\newlongnonfix}
           {\mbox{$\NONFIX\ \vid_1\ \cdots\ \vid_n$}}

          % Bindings
\newcommand{\longvalbind}
           {\mbox{\pat\ \ml{=} \exp\ $\langle\AND\ \valbind\rangle$}}
\newcommand{\recvalbind}
           {\mbox{\REC\ \valbind}}
\newcommand{\longtypbind}
           {\mbox{\tyvarseq\ \tycon\ \ml{=} \ty
            \ $\langle\AND\ \typbind\rangle$}}
\newcommand{\longdatbind}
           {\mbox{\tyvarseq\ \tycon\ \ml{=} \constrs
                  \ $\langle\AND\ \datbind\rangle$}}
\newcommand{\newlongdatbind}
           {\mbox{\tyvarseq\ \tycon\ \ml{=} \constrs
                  \ $\langle\AND\ \datbind'\rangle$}}
\newcommand{\longconstrs}
           {\mbox{$\con\ \langle\OF\ \ty\rangle\
                   \langle$ \ml{|} \constrs$\rangle$}}
\newcommand{\longvidconstrs}
           {\mbox{$\vid\ \langle\OF\ \ty\rangle\
                   \langle$\ml{|} \constrs$\rangle$}}
\newcommand{\longerconstrs}
           {\mbox{$\con\ \langle\OF\ \ty\rangle\
                   \langle\langle$\ml{|} \constrs$\rangle\rangle$}}
\newcommand{\longervidconstrs}
           {\mbox{$\vid\ \langle\OF\ \ty\rangle\
                   \langle\langle$\ml{|} \constrs$\rangle\rangle$}}
\newcommand{\generativeexnbind}
           {\mbox{$\langle\OP\rangle\exn\ \langle\OF\ \ty\rangle\
                   \langle\AND\ \exnbind\rangle$}}
\newcommand{\generativeexnvidbind}
           {\mbox{$\langle\OP\rangle\vid\ \langle\OF\ \ty\rangle\
                   \langle\AND\ \exnbind\rangle$}}
\newcommand{\eqexnbind}
           {\mbox{$\langle\OP\rangle$\exn\ \ml{=} 
                  $\langle\OP\rangle\longexn\ 
                  \langle\AND\ \exnbind\rangle$}}
\newcommand{\eqexnvidbind}
           {\mbox{$\langle\OP\rangle$\vid\ \ml{=} 
                  $\langle\OP\rangle\longvid\ 
                  \langle\AND\ \exnbind\rangle$}}
\newcommand{\longexnbinda}
           {\mbox{\exn\ $\langle\OF\ \ty\rangle$\ 
                  $\langle\langle\AND\ \exnbind\rangle\rangle$}}
\newcommand{\longvidexnbinda}
           {\mbox{\vid\ $\langle\OF\ \ty\rangle$\ 
                  $\langle\langle\AND\ \exnbind\rangle\rangle$}}
\newcommand{\longexnbindaa}
           {\mbox{\exn\
                  $\langle\AND\ \exnbind\rangle$}}
\newcommand{\longvidexnbindaa}
           {\mbox{\vid\
                  $\langle\AND\ \exnbind\rangle$}}
\newcommand{\longexnbindb}
           {\mbox{\exn\ \ml{=}\ \longexn\ $\langle\AND\ \exnbind\rangle$}}
\newcommand{\longvidexnbindb}
           {\mbox{\vid\ \ml{=}\ \longvid\ $\langle\AND\ \exnbind\rangle$}}
% from version 1:
%\newcommand{\longexnbind}
%           {\mbox{\exn\ $\langle$\ml{:} $\ty\rangle
%                  \langle$\ml{=} $\longexn\rangle
%                  \ \langle\AND\ \exnbind\rangle$}}
%\newcommand{\longexnbinda}
%           {\mbox{\exn\ $\langle$\ml{:} $\ty\rangle
%                  \ \langle\langle\AND\ \exnbind\rangle\rangle$}}
%\newcommand{\longexnbindaa}
%           {\mbox{\exn\
%                  $\langle\AND\ \exnbind\rangle$}}
%\newcommand{\longexnbindb}
%           {\mbox{\exn\ $\langle$\ml{:} $\ty\rangle
%                  \ $\ml{=} $\longexn
%                  \ \langle\langle\AND\ \exnbind\rangle\rangle$}}
%\newcommand{\longexnbindbb}
%           {\mbox{\exn\ \ml{=} \longexn\
%                  $\langle\AND\ \exnbind\rangle$}}

          % Patterns
%\newcommand{\wildpat}{\mbox{\ml{\_}}}
\newcommand{\wildpat}{\mbox{\ml{\char`\_}}}
\newcommand{\recpat}{\mbox{$\langle\labpats\rangle$}}
%\newcommand{\wildrec}{\mbox{\ml{...}}}
\newcommand{\wildrec}{\mbox{\ml{...}\ADD{\ \ml{=} \pat}}} % SuccML
\newcommand{\longlabpats}{\mbox{\lab\ \ml{=} \pat\
                          $\langle$ \ml{,} \labpats$\rangle$}}
\newcommand{\parpat}{\mbox{\ml{(} \pat\ \ml{)}}}
\newcommand{\conpat}
           {\mbox{\longcon\ \atpat}}
\newcommand{\vidpat}
           {\mbox{\longvid\ \atpat}}
\newcommand{\exconpat}
           {\mbox{\longexn\ \atpat}}
\newcommand{\infpat}
           {\mbox{$\pat_1\ \con\ \pat_2$}}
\newcommand{\vidinfpat}
           {\mbox{$\pat_1\ \vid\ \pat_2$}}
\newcommand{\infexpat}
           {\mbox{$\pat_1\ \exn\ \pat_2$}}
\newcommand{\typedpat}
           {\mbox{\pat\ \ml{:} \ty}}
\newcommand{\layeredpat}
           {\mbox{\var$\langle$\ml{:} \ty$\rangle$ \AS\ \pat}}
\newcommand{\layeredvidpat}
           {\mbox{\vid$\langle$\ml{:} \ty$\rangle$ \AS\ \pat}}
\newcommand{\layeredpata}
           {\mbox{\var\ \AS\ \pat}}
\newcommand{\layeredvidpata}
           {\mbox{\vid\ \AS\ \pat}}
\newcommand{\aspat}{\mbox{$\pat_1$\ \ml{as}\ $\pat_2$}} % SuccML
\newcommand{\orpat}{\mbox{$\pat_1$\ \ml{|}\ $\pat_2$}} % SuccML
\newcommand{\nestedpat}{\mbox{$\pat_1$\ \ml{with}\ $\pat_2$\ \ml{=}\ \exp}} % SuccML

          % Types
\newcommand{\rectype}{\mbox{$\langle\labtys\rangle$}}
\newcommand{\longlabtys}{\mbox{\lab\ \ml{:} \ty\
                          $\langle$ \ml{,} \labtys$\rangle$}}
\newcommand{\extrecty}{\mbox{\ml{...}\ \ml{:}\ \ty}} % SuccML
\newcommand{\constype}
           {\mbox{\tyseq\ \longtycon}}
\newcommand{\funtype}
%           {\mbox{\ty\ \ml{->} \ty$'$}}
           {\mbox{$\hbox{\ty}\;\boxml{->}\;\hbox{\it ty\/}'$}}
\newcommand{\partype}{\mbox{\ml{(} \ty\ \ml{)}}}
\newcommand{\longtyseq}{\mbox{\ml{(} $\ty_1,\cdots\,\ty_k$ \ml{)}}}
\newcommand{\longtyvarseq}{\mbox{\ml{(} $\tyvar_1,\cdots,\tyvar_k$ \ml{)}}}

        % Modules Phrases

\newcommand{\emptyphrase}{\qquad}


        % structure-level declarations

\newcommand{\singstrdec}{\mbox{$\STRUCTURE\ \strbind $}}
\newcommand{\localstrdec}{\mbox{$\LOCAL\ \strdec_1\ \IN\ \strdec_2\ \END $}}
\newcommand{\openstrdec}{\mbox{$\OPEN\ \longstrid_1\ \cdots\ \longstrid_n $}}
\newcommand{\emptystrdec}{\emptyphrase}
\newcommand{\seqstrdec}{\mbox{$\strdec_1\ \langle$\ml{;}$\rangle\ \strdec_2 $}}


        % structure bindings

\newcommand{\strbinder}
           {\mbox{$\strid\ \langle$\ml{:}$\ \sigexp\rangle$
            \ml{=} $\strexp\ \langle\langle\AND\ \strbind\rangle\rangle$}}
\newcommand{\derivedstrbinder}
           {\boxml{$\strid$:$\sigexp$=$\strexp\,\langle\boxml{and $\strbind$}\rangle$}}
\newcommand{\equivalentstrbinder}
           {\boxml{$\strid\boxml{=}\strexp\boxml{:}\sigexp\,\langle\boxml{and $\strbind$}\rangle$}}
\newcommand{\derivedabststrbinder}
           {\boxml{$\strid$\ABSTRACT$\sigexp$=$\strexp\,\langle\boxml{and $\strbind$}\rangle$}}
\newcommand{\equivalentabststrbinder}
           {\boxml{$\strid\boxml{=}\strexp\ABSTRACT\sigexp\,\langle\boxml{and $\strbind$}\rangle$}}
\newcommand{\barestrbinder}
           {\mbox{$\strid\ 
            \ml{=} $\strexp\ \langle\langle\AND\ \strbind\rangle\rangle$}}
\newcommand{\strbindera}
           {\mbox{$\strid\ \langle$\ml{:}$\ \sigexp\rangle$
            \ml{=} $\strexp\ \langle\AND\ \strbind\rangle$}}
\newcommand{\barestrbindera}
           {\boxml{$\strid$ =  $\strexp$ $\langle\AND\ \strbind\rangle$}}


        % structure expressions

\newcommand{\encstrexp}{\mbox{\STRUCT\ \strdec\ \END}}
\newcommand{\funappdec}{\mbox{\funid\ \ml{(}\ \strdec\ \ml{)} }}
\newcommand{\funappstr}{\mbox{\funid\ \ml{(}\ \strexp\ \ml{)} }}
\newcommand{\letstrexp}{\mbox{\LET\ \strdec\ \IN\ \strexp\ \END}}
\newcommand{\transpconstraint}{\boxml{$\strexp$:$\sigexp$}}
\newcommand{\opaqueconstraint}{\boxml{$\strexp$\ABSTRACT$\sigexp$}}

        % specifications

\newcommand{\valspec}{\mbox{\VAL\ \valdesc}}
\newcommand{\typespec}{\mbox{\TYPE\ \typdesc}}
\newcommand{\eqtypespec}{\mbox{\EQTYPE\ \typdesc}}
\newcommand{\datatypespec}{\mbox{\DATATYPE\ \datdesc}}
\newcommand{\datatypereplspec}{\datatyperepldec}
\newcommand{\datatypereplspeca}{\datatyperepldeca}
\newcommand{\datatypereplspecb}{\datatyperepldecb}
\newcommand{\exceptionspec}{\mbox{\EXCEPTION\ \exndesc}}
\newcommand{\structurespec}{\mbox{\STRUCTURE\ \strdesc}}
\newcommand{\sharingspec}{\mbox{\SHARING\ \shareq}}
\newcommand{\newsharingspec}{\boxml{$\spec$ sharing type $\longtycon_1$ = $\cdots$ = $\longtycon_n$}}
\newcommand{\newsharingspecb}{\parbox[t]{4cm}{\boxml{$\spec$ sharing type}\hfill\break \boxml{\qquad$\longtycon_1$ = $\cdots$ = $\longtycon_n$}}}
\newcommand{\wheretypespec}{\mbox{\spec\  \WHERETYPE\ \typabbr}}
\newcommand{\localspec}{\mbox{$\LOCAL\ \spec_1\ \IN\ \spec_2\ \END$}}
\newcommand{\openspec}{\mbox{$\OPEN\ \longstrid_1\ \cdots\ \longstrid_n $}}
\newcommand{\emptyspec}{\emptyphrase}
\newcommand{\seqspec}{\mbox{$\spec_1\ \langle$\ml{;}$\rangle\ \spec_2$}}
\newcommand{\inclspec}{\mbox{$\INCLUDE\ \sigid_1\ \cdots\ \sigid_n $}}
\newcommand{\singleinclspec}{\mbox{$\INCLUDE\ \sigexp$}}


        % descriptions

\newcommand{\valdescription}
           {\mbox{\var\ \ml{:} $\ty\ \langle\AND\ \valdesc\rangle$}}
\newcommand{\valviddescription}
           {\mbox{\vid\ \ml{:} $\ty\ \langle\AND\ \valdesc\rangle$}}
\newcommand{\typdescription}
           {\mbox{\tyvarseq\ \tycon\ $\langle\AND\ \typdesc\rangle$}}

\newcommand{\datdescription}
           {\mbox{\tyvarseq\ \tycon\ \ml{=} \condesc
             \ $\langle\AND\ \datdesc\rangle$}}

\newcommand{\newdatdescription}
           {\mbox{\tyvarseq\ \tycon\ \ml{=} \condesc
             \ $\langle\AND\ \datdesc'\rangle$}}

\newcommand{\datdescriptiona}
           {\mbox{\tyvarseq\ \tycon\ \ml{=} \condesc
             \ $\langle\AND\ \datdesc'\rangle$}}

\newcommand{\condescription}
           {\mbox{$\con\ \langle\OF\ \ty\rangle\
                   \langle$ \ml{|} \condesc$\rangle$}}
\newcommand{\conviddescription}
           {\mbox{$\vid\ \langle\OF\ \ty\rangle\
                   \langle$ \ml{|} \condesc$\rangle$}}
\newcommand{\shortcondesc}
           {\mbox{$\con\ 
                   \langle$ \ml{|} \condesc$\rangle$}}
\newcommand{\shortconviddesc}
           {\mbox{$\vid\ 
                   \langle$ \ml{|} \condesc$\rangle$}}

\newcommand{\longcondescription}
           {\mbox{$\con\ \langle\OF\ \ty\rangle\
                   \langle\langle$ \ml{|} \condesc$\rangle\rangle$}}

\newcommand{\longconviddescription}
           {\mbox{$\vid\ \langle\OF\ \ty\rangle\
                   \langle\langle$ \ml{|} \condesc$\rangle\rangle$}}

\newcommand{\exndescription}
           {\mbox{\exn\ $\langle\OF\ \ty\rangle$
            \ $\langle\AND\ \exndesc\rangle$}}
\newcommand{\exnviddescription}
           {\mbox{\vid\ $\langle\OF\ \ty\rangle$
            \ $\langle\AND\ \exndesc\rangle$}}

\newcommand{\exndescriptiona}
           {\mbox{\exn\ $\langle\OF\ \ty\rangle$
            \ $\langle\langle\AND\ \exndesc\rangle\rangle$}}

\newcommand{\exnviddescriptiona}
           {\mbox{\vid\ $\langle\OF\ \ty\rangle$
            \ $\langle\langle\AND\ \exndesc\rangle\rangle$}}

\newcommand{\strdescription}
           {\mbox{\strid\ \ml{:} \sigexp
            \ $\langle\AND\ \strdesc\rangle$}}

        % sharing equations

\newcommand{\strshareq}{\mbox{$\longstrid_1$ \ml{=} $\cdots$
                                             \ml{=} $\longstrid_n$ }}
\newcommand{\typshareq}{\mbox{$\TYPE\ \longtycon_1$ \ml{=} $\cdots$
                                                    \ml{=} $\longtycon_n$ }}
\newcommand{\multshareq}{\mbox{$\shareq_1\ \AND\ \shareq_2$}}

\newcommand{\longtypabbr}{\mbox{\tt $\tyvarseq\;\longtycon$ = $\ty\;\langle\boxml{and}\,\typabbr\,\rangle$}}


        % signature expressions

\newcommand{\encsigexp}{\mbox{\SIG\ \spec\ \END}}
\newcommand{\wheretypesigexp}{\boxml{$\sigexp$ where type $\tyvarseq$ $\longtycon$ =  $\ty$}}
%\newcommand{\wheretypesigexpb}{\parbox[t]{5cm}{\boxml{$\sigexp$ where type }
%                        \boxml{\qquad$\tyvarseq$ $\longtycon$ =  $\ty$}}}

        % signature declarations

\newcommand{\singsigdec}{\mbox{$\SIGNATURE\ \sigbind $}}
\newcommand{\emptysigdec}{\emptyphrase}
\newcommand{\seqsigdec}{\mbox{$\sigdec_1\ \langle$\ml{;}$\rangle\ \sigdec_2 $}}


        % signature bindings

\newcommand{\sigbinder}
           {\mbox{\sigid\ \ml{=} \sigexp
            \ $\langle\AND\ \sigbind\rangle$}}


        % functor declarations

\newcommand{\singfundec}{\mbox{$\FUNCTOR\ \funbind $}}
\newcommand{\emptyfundec}{\emptyphrase}
\newcommand{\seqfundec}{\mbox{$\fundec_1\ \langle$\ml{;}$\rangle\ \fundec_2 $}}


        % functor bindings

\newcommand{\funbinder}
           {\mbox{\funid\ \ml{(} \spec\ \ml{)}
                          $\langle$\ml{:}$\ \sigexp\rangle$\ \ml{=} $\strexp$
                          $\langle\langle\AND\ \funbind\rangle\rangle$}}
\newcommand{\funbindera}
           {\mbox{\funid\ \ml{(} \spec\ \ml{)}
                          $\langle$\ml{:}$\ \sigexp\rangle$\ \ml{=} $\strexp$}}
\newcommand{\funstrbinder}
           {\mbox{\funid\ \ml{(}\ \strid\ \ml{:}\ \sigexp\ \ml{)}
                        $\langle$\ml{:}$\ \sigexp'\rangle$\ \ml{=} $\strexp$}}
\newcommand{\barefunstrbinder}
           {\mbox{\funid\ \ml{(}\ \strid\ \ml{:}\ \sigexp\ \ml{)}
                        \ \ml{=} $\strexp$}}
\newcommand{\optfunbind}
           {\mbox{$\langle\langle\AND\ \funbind\rangle\rangle$}}
\newcommand{\optfunbinda}
           {\mbox{$\langle\AND\ \funbind\rangle$}}
\newcommand{\funstrbindera}
           {\mbox{\funid\ \ml{(}\ \strid\ \ml{)}\ \ml{=} \strexp
            \ $\langle\AND\ \funbind\rangle$}}


            % topdec 

\newcommand{\seqtopdec}{\boxml{$\topdec$ $\topdec$}}
\newcommand{\strdecintopdec}{\boxml{$\hbox to12mm{\strdec}\;\langle\topdec\rangle$}}
\newcommand{\fundecintopdec}{\boxml{$\hbox to12mm{\fundec}\;\langle\topdec\rangle$}}
\newcommand{\sigdecintopdec}{\boxml{$\hbox to12mm{\sigdec}\;\langle\topdec\rangle$}}

        %functor signature expressions

\newcommand{\longfunsigexp}
           {\mbox{\ml{(}\ \spec\ \ml{)}
                          \ml{:}\ \sigexp}}
\newcommand{\longfunsigexpa}
           {\mbox{\ml{(}\ \strid\ \ml{:}\ \sigexp\ \ml{)}
                          \ml{:}\ \sigexp$'$}}


        % functor specifications

\newcommand{\singfunspec}{\mbox{\FUNCTOR\ \fundesc}}
\newcommand{\emptyfunspec}{\emptyphrase}
\newcommand{\seqfunspec}
           {\mbox{$\funspec_1\ \langle$\ml{;}$\rangle\ \funspec_2$}}
 

        % functor descriptions

\newcommand{\longfundesc}
           {\mbox{\funid\ \funsigexp\ $\langle\AND\ \fundesc\rangle$}}


        % programs
\newcommand{\longprog}{\mbox{\topdec\ \ml{;}\ $\langle\program\rangle$}}
\newcommand{\seqprog}
 {\mbox{$\program_1\ \langle$\ml{;}$\rangle\ \program_2$}}
% **************** END SYNTAX *************************
\newcommand{\ML}{{\rm ML}}


%       finite sets and maps (assume math mode)
%
\newcommand{\Fin}{\mathop{\rm Fin}\nolimits}
\newcommand{\Dom}{\mathop{\rm Dom}\nolimits}
\newcommand{\Ran}{\mathop{\rm Ran}\nolimits}
\newcommand{\finfun}[2]{#1\stackrel{{\rm fin}}{\to}#2}
\newcommand{\emptymap}{\{\}}
\newcommand{\kmap}[2]{\{#1_1\mapsto#2_1,\cdots,#1_k\mapsto#2_k\}}
\newcommand{\plusmap}[2]{#1 + #2}
\newcommand{\unify}{\mathop{\rm Unify}}
\newcommand{\restrict}[2]{#1 \setminus #2} % SuccML
%
\newcommand{\TyVarSet}{{\rm TyVarSet}}
%
%
%       Names     (assume math mode)
\newcommand{\TyNames}{{\rm TyName}}
\newcommand{\TyNameSets}{{\rm TyNameSet}}
\newcommand{\StrNames}{{\rm StrName}}
\newcommand{\StrNameSets}{{\rm StrNameSet}}
\newcommand{\IdStatus}{\hbox{\rm IdStatus}}
\def\isc{\boxml{c}}
\def\ise{\boxml{e}}
\def\isv{\boxml{v}}
\def\is{\hbox{\it is}}
\newcommand{\NameSets}{{\rm NameSet}}
\newcommand{\TyNamesk}{\TyNames^{(k)}}
\newcommand{\Addr}{{\rm Addr}}
\newcommand{\Exc}{{\rm ExName}}
\newcommand{\BasVal}{{\rm BasVal}}
\newcommand{\SVal}{{\rm SVal}}
\newcommand{\BasExc}{{\rm BasExName}}
\newcommand{\BasTyp}{{\rm BasTyp}}
\newcommand{\CONT}{{\tt !}}
\newcommand{\ASS}{{\tt :=}}
\newcommand{\FAIL}{{\rm FAIL}}
\newcommand{\Fail}{{\rm FAIL}}
\newcommand{\fail}{{\rm FAIL}}
\newcommand{\APPLY}{{\rm APPLY}}

\newcommand{\A}{a}
\newcommand{\e}{\mbox{\it en}}
\newcommand{\sv}{\mbox{\it sv}}
\newcommand{\exval}{e}
\newcommand{\excs}{\mbox{\it ens}}
\newcommand{\exns}{\mbox{\it excons}} % used in the dynamic sem. of mod.
\newcommand{\f}{f}
\newcommand{\m}{m}
\newcommand{\mem}{\mbox{\it mem}}
\newcommand{\M}{M}
\newcommand{\n}{n}
\newcommand{\N}{N}
\newcommand{\p}{p}
\def\r{r}
\newcommand{\res}{\mbox{\it res}}
\newcommand{\s}{s}
\renewcommand{\t}{t}
\newcommand{\T}{T}
\newcommand{\U}{U}
\newcommand{\V}{v}
\newcommand{\vars}{\mbox{\it vars}}
\newcommand{\vids}{\mbox{\it vids}}
\newcommand{\X}{X}

          % Compound Objects (Core Language)
\newcommand{\TyEnv}{{\rm TyEnv}}
\newcommand{\TE}{\mbox{$T\!E$}}
\newcommand{\TyStr}{{\rm TyStr}}

\newcommand{\ConEnv}{{\rm ConEnv}}
\newcommand{\CE}{\mbox{$C\!E$}}

\newcommand{\VarEnv}{{\rm VarEnv}}
\newcommand{\ValEnv}{{\rm ValEnv}}
\newcommand{\VE}{\mbox{$V\!E$}}

\newcommand{\ExnEnv}{{\rm ExConEnv}}
\newcommand{\EE}{\mbox{$E\!E$}}

\newcommand{\IntEnv}{{\rm IntEnv}}
\newcommand{\IE}{\mbox{$I\!E$}}

\newcommand{\StrInt}{{\rm StrInt}}
\newcommand{\SI}{\mbox{$S\!I$}}

\newcommand{\TyInt}{{\rm TyInt}}
\newcommand{\TI}{\mbox{$T\!I$}}

%\newcommand{\IdInt}{{\rm IdInt}}
%\newcommand{\II}{\mbox{$I\!I$}}

\newcommand{\ValInt}{{\rm ValInt}}
\newcommand{\VI}{\mbox{$V\!I$}}


\newcommand{\Env}{{\rm Env}}
\newcommand{\E}{E}
\newcommand{\longE}[1]{(\SE_{#1},\TE_{#1},\VE_{#1},\EE_{#1})}
\newcommand{\newlongE}[1]{(\SE_{#1},\TE_{#1},\VE_{#1})}

\newcommand{\StrEnv}{{\rm StrEnv}}
\newcommand{\SE}{\mbox{$S\!E$}}

\newcommand{\Str}{{\rm Str}}
\renewcommand{\S}{S}
\newcommand{\longS}[1]{(\m_{#1},(\SE_{#1},\TE_{#1},\VE_{#1},\EE_{#1}))}

\newcommand{\Int}{{\rm Int}}
\newcommand{\I}{I}

\newcommand{\Context}{\rm Context}
\newcommand{\C}{C}


\newcommand{\Record}{{\rm Record}}
\newcommand{\ExVal}{{\rm ExVal}}
\newcommand{\Fun}{{\rm Fun}}  % used?
\newcommand{\Pack}{{\rm Pack}}
\newcommand{\Closure}{{\rm Closure}}
\newcommand{\FcnClosure}{{\rm FcnClosure}}
\newcommand{\FunctorClosure}{{\rm FunctorClosure}}
\newcommand{\Match}{{\rm Match}}
\newcommand{\State}{{\rm State}}
\newcommand{\StrExp}{{\rm StrExp}}
\newcommand{\Mem}{{\rm Mem}}
\newcommand{\ExcSet}{{\rm ExNameSet}}
\newcommand{\Val}{{\rm Val}}

\newcommand{\arity}{\mathop{\rm arity}\nolimits}
\def\k{\mbox{$k$}}
\newcommand{\longtauk}{(\tau_1,\cdots,\tau_k)}
\newcommand{\tauk}{\mbox{$\tau^{(k)}$}}
\newcommand{\longalphak}{(\alpha_1,\cdots,\alpha_k)}
\newcommand{\alphak}{\mbox{$\alpha^{(k)}$}}
\newcommand{\alphakt}{\mbox{$\alpha^{(k)}t$}}
%
\newcommand{\thetak}{\mbox{$\theta^{(k)}$}}
\newcommand{\tk}{\mbox{$\t^{(k)}$}}
\newcommand{\typefcn}{\theta}
\newcommand{\typefcnk}{\Lambda\alphak.\tau}
\newcommand{\Type}{{\rm Type}}
\newcommand{\ConsType}{{\rm ConsType}}
\newcommand{\RecType}{{\rm RecType}}
\newcommand{\RowType}{{\rm RowType}}
\newcommand{\FunType}{{\rm FunType}}
\newcommand{\constypek}{\t(\tau_1,\cdots,\tau_k)}
\newcommand{\TypeScheme}{{\rm TypeScheme}}
\newcommand{\tych}{\sigma}
\newcommand{\longtych}{\forall\alphak.\tau}
\newcommand{\Abs}{\mathop{\rm Abs}\nolimits}
\newcommand{\Inter}{\mathop{\rm Inter}\nolimits}
\newcommand{\Rec}{\mathop{\rm Rec}\nolimits}
\newcommand{\unroll}{\mathop{\hbox{Unroll}_{\VE}}\nolimits}
\newcommand{\TypeFcn}{{\rm TypeFcn}}
\newcommand{\TyConFcn}{\mathop{\rm TyCon}\nolimits} % used?
\newcommand{\TyVarFcn}{\mathop{\rm tyvars}\nolimits}
\newcommand{\imptyvars}{\mathop{\rm imptyvars}\nolimits}
\newcommand{\apptyvars}{\mathop{\rm apptyvars}\nolimits}
\newcommand{\scontype}{\mathop{\rm type}\nolimits}
\newcommand{\sconval}{\mathop{\rm val}\nolimits}
%
%         Compound Objects (Modules)
%
%
\newcommand{\Sig}{{\rm Sig}}
\newcommand{\sig}{\Sigma}
\newcommand{\longsig}[1]{(\N_{#1})\S_{#1}}
\newcommand{\newlongsig}[1]{(\T_{#1})\E_{#1}}
\newcommand{\FunSig}{{\rm FunSig}}
\newcommand{\funsig}{\Phi}
\newcommand{\longfunsig}[1]{(\N_{#1})(\S_{#1},(\N_{#1}')\S_{#1}')}
\newcommand{\newlongfunsig}[1]{(\T_{#1})(\E_{#1},(\T_{#1}')\E_{#1}')}
\newcommand{\FunEnv}{{\rm FunEnv}}
\newcommand{\F}{F}

\newcommand{\SigEnv}{{\rm SigEnv}}
\newcommand{\G}{G}

\newcommand{\Basis}{{\rm Basis}}
\newcommand{\B}{B}
\newcommand{\Bstat}{B_{\rm STAT}}
\newcommand{\Bdyn}{B_{\rm DYN}}

\newcommand{\IntBasis}{{\rm IntBasis}}
\newcommand{\IB}{\mbox{$I\!B$}}


%         Selection of Components
%
\newcommand{\of}[2]{#1 \mathbin{\rm of} #2}
\newcommand{\In}{\mbox{\rm in}}
%
%         Names in Structures
\newcommand{\StrNamesFcn}{\mathop{\rm strnames}\nolimits}
\newcommand{\TyNamesFcn}{\mathop{\rm tynames}\nolimits}
\newcommand{\TyVarsFcn}{\mathop{\rm tyvars}\nolimits}
\newcommand{\NamesFcn}{\mathop{\rm names}\nolimits}
%
%         Type Schemes (assume math mode)
%
\newcommand{\cl}[2]{{\rm Clos}_{#1}#2}
%
%         Realisations (assume math mode)
%
\newcommand{\tyrea}{\varphi_{\rm Ty}}
\newcommand{\strrea}{\varphi_{\rm Str}}
\newcommand{\rea}{\varphi}
\newcommand{\longrea}{(\tyrea,\strrea)}
%
%             Support (assume math mode)
%
\newcommand{\Supp}{\mathop{\rm Supp}\nolimits}
\newcommand{\Yield}{\mathop{\rm Yield}\nolimits}
%
%           Instantiation (assume math mode)
%
\newcommand{\siginst}[3]{#1 {\geq_{#2}} #3}
\newcommand{\sigord}[3]{#1 {\geq_{#2}} #3}
\newcommand{\funsiginst}[3]{#1 {\geq_{#2}} #3}
%
%            Inference Rules
%
%
\newcommand{\ts}{\vdash}
\newcommand{\tsdyn}{\vdash_{\rm DYN}}
\newcommand{\tsstat}{\vdash_{\rm STAT}}

\newcommand{\ra}{\Rightarrow}

%            Initial Static Basis

%              Type names
\newcommand{\BOOL}{\mbox{\tt bool}}
\newcommand{\INT}{\mbox{\tt int}}
\newcommand{\INt}{\mbox{\rm Int}}
\newcommand{\REAL}{\mbox{\tt real}}
\newcommand{\RREAL}{\mbox{\tt Real}}
\newcommand{\Real}{\mbox{\rm Real}}
\newcommand{\NUM}{\mbox{\tt num}}
\newcommand{\Num}{\mbox{\rm Num}}
\newcommand{\NUMTEXT}{\mbox{\tt numtxt}}
\newcommand{\NumTxt}{\mbox{\rm NumTxt}}
\newcommand{\WORDINT}{\mbox{\tt wordint}}
\newcommand{\WordInt}{\mbox{\rm WordInt}}
\newcommand{\RealInt}{\mbox{\rm RealInt}}
\newcommand{\REALINT}{\mbox{\tt realint}}
\newcommand{\EXCN}{\mbox{\tt exn}}
\newcommand{\String}{\mbox{\rm String}}
\newcommand{\STRING}{\mbox{\tt string}}
\newcommand{\UNISTRING}{\mbox{\tt unistring}}
\newcommand{\WORD}{\mbox{\tt word}}
\newcommand{\Word}{\mbox{\rm Word}}
\newcommand{\WORDEIGHT}{\mbox{\tt word8}}
\newcommand{\Char}{\mbox{\rm Char}}
\newcommand{\CHAR}{\mbox{\tt char}}
\newcommand{\UNICHAR}{\mbox{\tt unichar}}
\newcommand{\LIST}{\mbox{\tt list}}
\newcommand{\INSTREAM}{\mbox{\tt instream}}
\newcommand{\OUTSTREAM}{\mbox{\tt outstream}}
\newcommand{\ARRAY}{\mbox{\tt array}}
\newcommand{\VECTOR}{\mbox{\tt vector}}

%              Constructors
\newcommand{\FALSE}{\mbox{\tt false}}
\newcommand{\TRUE}{\mbox{\tt true}}
\newcommand{\NIL}{\mbox{\tt nil}}
\newcommand{\REF}{\mbox{\tt ref}}
\newcommand{\IT}{\mbox{\tt it}}
\newcommand{\UNIT}{\mbox{\tt unit}}

%              Basic Values BasVal
%\newcommand{\MAP}{\mbox{\tt map}}
%\newcommand{\REV}{\mbox{\tt rev}}
%\newcommand{\NOT}{\mbox{\tt not}}
%\newcommand{\NEG}{\mbox{\verb-~-}}
\newcommand{\NEG}{\tttilde}
%\newcommand{\ABS}{\mbox{\tt abs}}
%\newcommand{\FLOOR}{\mbox{\tt floor}}
%\newcommand{\REAL}{\mbox{\tt real}}
%\newcommand{\SQRT}{\mbox{\tt sqrt}}
%\newcommand{\SIN}{\mbox{\tt sin}}
%\newcommand{\COS}{\mbox{\tt cos}}
%\newcommand{\ARCTAN}{\mbox{\tt arctan}}
%\newcommand{\EXP}{\mbox{\tt exp}}
%\newcommand{\LN}{\mbox{\tt ln}}
%\newcommand{\SIZE}{\mbox{\tt size}}
%\newcommand{\CHR}{\mbox{\tt chr}}
%\newcommand{\ORD}{\mbox{\tt ord}}
%\newcommand{\EXPLODE}{\mbox{\tt explode}}
%\newcommand{\IMPLODE}{\mbox{\tt implode}}
%\newcommand{\REALDIV}{\mbox{\tt /}}
%\newcommand{\DIV}{\mbox{\tt div}}
%\newcommand{\MOD}{\mbox{\tt mod}}
%\newcommand{\TIMES}{\mbox{\tt  *}}
%\newcommand{\PLUS}{\mbox{\tt +}}
%\newcommand{\MINUS}{\mbox{\tt -}}
%\newcommand{\APPEND}{\mbox{\verb-@-}}
\newcommand{\EQ}{\mbox{\texttt{=}}}
%\newcommand{\NEQ}{\mbox{\verb-<>-}}
%\newcommand{\LESS}{\mbox{\verb-<-}}
%\newcommand{\GREATER}{\mbox{\verb->-}}
%\newcommand{\LEQ}{\mbox{\verb-<=-}}
%\newcommand{\GEQ}{\mbox{\verb->=-}}
%\newcommand{\COMP}{\mbox{\tt o}}

%\newcommand{\STDIN}{\mbox{\tt std\_in}}
%\newcommand{\OPENIN}{\mbox{\tt open\_in}}
%\newcommand{\INPUT}{\mbox{\tt input}}
%\newcommand{\LOOKAHEAD}{\mbox{\tt lookahead}}
%\newcommand{\CLOSEIN}{\mbox{\tt close\_in}}
%\newcommand{\ENDSTREAM}{\mbox{\tt end\_of\_stream}}
%\newcommand{\STDOUT}{\mbox{\tt std\_out}}
%\newcommand{\OPENOUT}{\mbox{\tt open\_out}}
%\newcommand{\OUTPUT}{\mbox{\tt output}}
%\newcommand{\CLOSEOUT}{\mbox{\tt close\_out}}
%\newcommand{\IOFAILURE}{\mbox{\tt io\_failure}}
\newcommand{\comment}{{\it Comment:\ }}
\newcommand{\comments}{{\it Comments:\ }}
\newcommand{\rulesec}[2]{\subsection*{{\bf#1}\hfill\fbox{$#2$}}}
%
%\font\msxm=msxm10
%\textfont10=\msxm
%\mathchardef\restrict="0A16
%
% $$f\restrict_A$$                   % Example of use
%alignment
%\halign{\indent$#$&\quad$#$&\quad$#$\hfil&\quad\parbox[t]{6cm}{\strut#\strut}\cr
\newcounter{changeno}
\newcounter{fixtypos}% 1 stands for true (i.e. fix typos)
\setcounter{fixtypos}{1} 
\newcounter{noimptypes}% 1 stands for true (i.e. replace imperative types by value restriction)
\setcounter{noimptypes}{1} 
\newcounter{typabbr}% 1 stands for true (i.e. introduce type abbreviations in signatures)
\setcounter{typabbr}{1} 
\newcounter{nostrsharing}% 1 stands for true (i.e. fix typos)
\setcounter{nostrsharing}{1}
\newcounter{newpreface}% 1 stands for true (i.e. new prefix)
\setcounter{newpreface}{1} 
\newcounter{noinfixinspec}% 1 stands for true (i.e. fix typos)
\setcounter{noinfixinspec}{1}
\newcounter{nofuncspec}% 1 stands for true (i.e. fix typos)
\setcounter{nofuncspec}{1}
\newcounter{noeqtypespec}% 1 stands for true (i.e. fix typos)
\setcounter{noeqtypespec}{1}
\newcounter{nolocalspec}% 1 stands for true (i.e. fix typos)
\setcounter{nolocalspec}{1}
\newcounter{library}% 1 stands for true (i.e. changes due to new library)
\setcounter{library}{1}
\newcounter{ce}% 1 stands for true (i.e. replace CE by VE)
\setcounter{ce}{1}
\newcounter{noopenspec}% 1 stands for true (i.e. fix typos)
\setcounter{noopenspec}{1}
\newcounter{noclosurerestriction}% 1 stands for true (i.e. fix typos)
\setcounter{noclosurerestriction}{1}
\newcounter{notypexplication}% 1 stands for true (i.e. fix typos)
\setcounter{notypexplication}{1}
\newcounter{singleincludespec}
\setcounter{singleincludespec}{1}
\newcounter{idstatus} % 1 stands for true (i.e., fix identifier status problems 
%                     % by eliminating exception environments)
\setcounter{idstatus}{1}
\newcounter{infixassociativity}
\setcounter{infixassociativity}{1} %1 stands for true (i.e., new rules for associativity of infix operators of same precedence)
\newcounter{explicittyvars}
\setcounter{explicittyvars}{1} %1 stands for true (i.e., allow explicity binding of type variables at VAL)
\newcounter{constructors}
\setcounter{constructors}{1} %1 stands for true (i.e., add missing rules for value constructors in the dynamic semantics)
\newcounter{comments}
\setcounter{comments}{1} %1 stands for true (i.e., more explanation concerning comments (syncor.tex)
\newcounter{commentary}
\setcounter{commentary}{1} %1 stands for true (i.e., fix mistakes listed in Commentary
\newcounter{safelet}
\setcounter{safelet}{1} %1 stands for true (i.e., fix bug concerning unsafe let)
\newcounter{principalenv}
\setcounter{principalenv}{1} %1 stands for true (i.e., firm of definition and
%claim concerning principal environments)
\newcounter{scon}
\setcounter{scon}{1} %1 stands for true (i.e., new special constants)
\newcounter{datatyperepl}
\setcounter{datatyperepl}{1} %1 stands for true (i.e., datatype replication adopted)
\newcounter{marginalnote} % 1 stands for true (i.e. put comments in margin)
\setcounter{marginalnote}{0} 
%
%   note: #1 condition variable   #2 note to put in changes document (not
%          in Definition itself)
\newcommand{\note}[2]{\ifnum#1=1\addtocounter{changeno}{1}\ifnum\themarginalnote=1\marginpar{{\rm\small note(\thechangeno)\index{note\thechangeno}}}\fi{}\else\fi} 
%
% replacement: #1: condition variable #2 old text, #3: new text
%              condition variable true: replace old by new
\newcommand{\replacement}[3]{\ifnum#1=1\addtocounter{changeno}{1}\ifnum\themarginalnote=1\marginpar{\parbox{15mm}{{\rm\small repl (\thechangeno)\index{repl\thechangeno}}}}\else\fi{#3}\else{#2}\fi} 
%
% addhocreplacementl:    #1 condition variable #2 back-up width (left), #3: old text #4 new text
%                      condition variable true: replace old by new
\newcommand{\adhocreplacementl}[4]{\ifnum#1=1\addtocounter{changeno}{1}%\index{repl\thechangeno}
\ifnum\themarginalnote=1\llap{{\rm\small repl (\thechangeno)\index{repl\thechangeno}}\hskip#2}\else\fi{#4}\else{#3}\fi} 
\newcommand{\adhocreplacementtexl}[4]{\ifnum#1=1\addtocounter{changeno}{1}\index{repl\thechangeno}
\ifnum\themarginalnote=1\llap{{\rm\small repl (\thechangeno)}\hskip#2}\else\fi#4\else#3\fi} 
%
%  insertion:    #1  condition variable  #2 new text
%                condition variable true: insert new text
\newcommand{\insertion}[2]{\ifnum#1=1\addtocounter{changeno}{1}\ifnum\themarginalnote=1\marginpar{{\rm\small ins(\thechangeno)\index{ins\thechangeno}}}\fi{#2}\else\fi} 
%
%  adhocinsertion:    #1  condition variable  #2 back up width (left) #3 new text
%                condition variable true: insert new text
\newcommand{\adhocinsertion}[3]{\ifnum#1=1\addtocounter{changeno}{1}\ifnum\themarginalnote=1\llap{{\rm\small ins (\thechangeno)\index{ins\thechangeno}}\hskip#2}\fi#3\else\fi} 
%
%  deletion:     #1  condition variable  #2 text to be deleted, if condition variable is true
%
\newcommand{\deletion}[2]{\ifnum#1=1\addtocounter{changeno}{1}\ifnum\themarginalnote=1\marginpar{{\rm\small del(\thechangeno)\index{del\thechangeno}}}\fi\else{#2}\fi}
%\newcommand{\deletion}[2]{}
%
%
%  adhocdeletion:    #1  condition variable  #2 back up width (left) #3 old text
%                condition variable true: insert new text
\newcommand{\adhocdeletion}[3]{\ifnum#1=1\addtocounter{changeno}{1}\ifnum\themarginalnote=1\llap{{\rm\small del (\thechangeno)\index{del\thechangeno}}\hskip#2}\fi\else{#3}\fi} 
%
%
\newcommand{\replacementPage}[3]{\addtocounter{changeno}{1}\noindent{\bf (\thechangeno)\ Page #1: replace} \begin{quote}#2\end{quote} \begin{center}\bf by\end{center} \begin{quote} #3 \end{quote}}
%
\newcommand{\replacementPagea}[3]{\addtocounter{changeno}{1}\noindent{\bf (\thechangeno)\ Page #1: replace}\par {#2} \begin{center} \bf by\end{center}  {#3}}
%
\newcommand{\insertionPage}[2]{\addtocounter{changeno}{1}\noindent{\bf (\thechangeno)\ Page #1: insert} \begin{quote}#2\end{quote}}
%
\newcommand{\deletionPage}[2]{\addtocounter{changeno}{1}\noindent{\bf (\thechangeno)\ Page #1: delete} \begin{quote}#2\end{quote}}
\newcommand{\notePage}[2]{\addtocounter{changeno}{1}\noindent{\bf (\thechangeno)\ Page #1: note} \begin{quote}#2\end{quote}}

\newcommand{\oldpagebreak}{}
\def\blankPage{\thispagestyle{empty}
\ 
\clearpage
\thispagestyle{headings}}
 % macros

%%
%% support for SuccML changes
%%
\usepackage[deletedmarkup=xout,authormarkup=none]{changes}
\newcommand{\fixcolor}{blue}  % corrections to definition
\newcommand{\addcolor}{magenta}  % new SuccML features
\newcommand{\delcolor}{gray}
\definechangesauthor[color=\fixcolor]{FixSML}
\definechangesauthor[color=\addcolor]{SuccML}
\newcommand\redsout{\bgroup\markoverwith{\textcolor{red}{\rule[0.5ex]{2pt}{0.6pt}}}\ULon}
\setdeletedmarkup{{\color{\delcolor}\redsout{#1}}}
\newcommand{\FIX}[1]{\added[id=FixSML]{#1}}
\newcommand{\FIXCUT}[1]{\deleted[id=FixSML]{#1}}
\newcommand{\FIXREPL}[2]{\replaced[id=FixSML]{#1}{#2}}
\newcommand{\ADD}[1]{\added[id=SuccML]{#1}}
\newcommand{\CUT}[1]{\deleted[id=SuccML]{#1}}
\newcommand{\REPL}[2]{\replaced[id=SuccML]{#1}{#2}}
%%
%% commands to support inserting equations without screwing up the original numbers
%%
\newcounter{saveeqn}
\newcommand{\BeginNewEqns}{%
  \setcounter{saveeqn}{\value{equation}}%
  \setcounter{equation}{0}%
  \renewcommand{\theequation}{\mbox{\textcolor{\addcolor}{\arabic{saveeqn}\alph{equation}}}}}
\newcommand{\EndNewEqns}{%
  \setcounter{equation}{\value{saveeqn}}%
  \renewcommand{\theequation}{\mbox{\arabic{equation}}}}
\newcommand{\SameEqn}{%
  \addtocounter{equation}{-1}%
  \renewcommand{\theequation}{\mbox{\CUT{\arabic{equation}}}}}
\newcommand{\NextEqn}{%
  \renewcommand{\theequation}{\mbox{\arabic{equation}}}}
%%
 
\usepackage{makeidx}
% Disable index printing
% \makeindex

%
%\inputonly{dynmod,overloading}
%for agfa: \voffset -12mm
%4 Sept. 89, for lw16:
%\advance\hoffset by -8mm
%AT DIKU:
%\addtolength{\textwidth}{-13mm}
%\addtolength{\textheight}{0.4mm}
%\nofiles

%\usepackage{hyperref}

\begin{document}

\pagestyle{empty}
\begin{flushleft}
{\Large\bf The Definition of \REPL{Successor}{Standard} ML}
\end{flushleft}
\vfill
\begin{flushleft}
%\normalsize\bf The MIT Press\\
%Cambridge, Massachusetts\\
%London, England
\end{flushleft}
\vspace*{1cm}
\clearpage
{This page is blank}
\clearpage
\pagestyle{headings}
\thispagestyle{empty}
\setcounter{page}{5}
\renewcommand{\thepage}{\roman{page}}
\tableofcontents
\clearpage
%\listofchanges  %% SuccML
%\clearpage %% SuccML
\pagestyle{empty}
\ \clearpage
\pagestyle{myheadings}
\pagestyle{empty}
%!TEX root = root.tex
%
%

\section*{Preface}
\markboth{}{}
A precise description of a programming language is a prerequisite for its 
implementation and for its use.  The description can take many forms, 
each suited to a different purpose.  A common form is a reference manual,
which is usually a careful narrative description of the meaning of each
construction in the language, often backed up with a formal presentation
of the grammar (for example, in Backus-Naur form). 
This gives the programmer enough 
understanding for many of his purposes.  But it is ill-suited for use
by an implementer, or by someone who wants to formulate laws for
equivalence of programs, or by a programmer who wants to design programs with
mathematical rigour. 

This document is a formal description of both the {\sl grammar} and the
{\sl meaning} of a language which is both designed for large projects and
widely used.  As such, it aims to serve the whole community of people
seriously concerned with the language.  At a time when it is increasingly
understood that programs must withstand rigorous analysis, particularly for 
systems where safety is critical, a rigorous language presentation is even
important for negotiators and contractors; for a robust program
written in an insecure language is like a house built upon sand.


Most people have not looked at a rigorous language presentation before.
To help them particularly, but also to put the present work in perspective
for those more theoretically prepared,  it will be useful here to say something
about three things: the nature of Standard ML, the task of language definition 
in general, and the form of the present Definition.
We also briefly describe the recent revisions to the Definition.
\vspace*{-8pt}

\subsubsection*{Standard ML}
\vspace*{-4pt}
Standard ML is a functional programming language, in the sense that the
full power of mathematical functions is present.  But it grew in response
to a particular programming task, for which it was equipped also
with full imperative power, and a sophisticated exception mechanism.  
It has an advanced form of parametric modules, aimed at organised
development of large programs.  Finally it is strongly typed, and it was
the first language to provide a particular form of polymorphic type which 
makes the strong typing remarkably flexible.  This combination of
ingredients has not made it unduly large, but their novelty has been
a fascinating challenge to semantic method (of which we say more below).
 
ML has evolved over \replacement{\thenewpreface}{fourteen}{twenty} years as a fusion of many ideas 
from many people. This evolution is
described in some detail in Appendix~\ref{story-app} of the book, where also we
acknowledge all those who have contributed to it, both in design
and in implementation.

`ML' stands for 
{\sl meta language}; this is the term logicians use for a language in which
other (formal or informal) languages are discussed and analysed.  
Originally ML was conceived as a medium for finding and performing 
proofs in a logical language.   Conducting rigorous argument as dialogue 
between person and machine has been a \replacement{\thenostrsharing}{strong research interest at Edinburgh 
and elsewhere, throughout these fourteen years.}{growing research topic 
throughout these twenty years.} The difficulties are enormous, 
and make stern demands
upon the programming language which is used for this dialogue.   Those who are
not familiar with computer-assisted reasoning may be surprised that a 
programming language, which was designed for this rather esoteric activity, 
should ever lay claim to being {\sl generally} useful.
On reflection, they should not be surprised.  LISP is a prime example of
a language invented for esoteric purposes and becoming widely used.  LISP
was invented for use in artificial intelligence (AI); the important thing
about AI here is not that it is esoteric,  but that
it is difficult and varied; so much so, that anything which works well for
it must work well for many other applications too.

The same can be said about the initial purpose of ML, but with a different
emphasis.  Rigorous proofs are complex things, which need varied
and sophisticated presentation -- particularly on the screen in interactive
mode. Furthermore the proof methods, or
strategies, involved are some of the most complex algorithms which we know.   
This all applies equally to AI, but one demand is made more strongly 
by proof than perhaps by any other application: the demand for rigour.  

This demand established the character of ML.  In order to be sure that,
when the user and the computer claim to have together performed a rigorous
argument, their claim is justified, it was seen 
that the language must be strongly typed. On the other hand, to be
useful in a difficult  application, the type system had to be rather
flexible, and permit the machine to guide the user rather than impose
a burden upon him.   A reasonable solution was found, in which the machine
helps the user significantly  by
inferring his types for him. Thereby the machine also confers complete
reliability on his programs, in this sense: 
If a program claims that a certain result 
follows from the rules of reasoning which the user has supplied, then the 
claim may be fully trusted.

The principle of inferring useful structural information about programs
is also represented, at the level of program modules, by the inference of 
{\sl signatures}.
Signatures describe the interfaces between modules, and are vital for robust
large-scale programs.  When the user combines modules, the signature
discipline prevents him from mismatching their interfaces.  By programming
with interfaces and parametric modules, it becomes possible to focus on the
structure of a large system, and to compile parts of it in isolation from
one another -- even when the system is incomplete.

This emphasis on types and signatures has had a profound effect on the 
language Definition. Over half this document is devoted to inferring types 
and signatures for programs.  But the method used is exactly the same as for 
inferring what {\sl values} a program delivers; indeed, a type or signature is 
the result of a kind of abstract evaluation of a program phrase. 

In designing ML, 
the interplay among three activities -- language design, definition and 
implementation -- was extremely close. This was particularly true for the
newest part, the parametric modules. This part of the language grew from an 
initial proposal by David MacQueen, itself highly developed; but both 
formal definition and implementation had a strong influence on the detailed
design.  In general, those who took part in the three activities cannot now 
imagine how they could have been properly done separately.  

\subsubsection*{Language Definition}
Every programming language presents  its own conceptual view of 
computation.  This view is usually indicated by the names used for the phrase 
classes of the language, or by its keywords: terms like package, module, 
structure, exception, channel, type, procedure, reference, sharing, \ldots. 
These terms also have their abstract counterparts,  which may be called
{\sl semantic objects}; these are what people really have in mind when they use
the language, or discuss it, or think in it.  Also, it is these objects,
not the syntax, which represent the particular conceptual view of each
language; they are the character of the language.  Therefore a definition
of the language must be in terms of these objects.  

As is commonly done in programming language semantics, we shall loosely
talk of these semantic objects as {\sl meanings}. Of course, it is 
perfectly possible to understand the semantic theory of a language, and
yet be unable to understand the meaning of a particular program, in the
sense of its {\sl intention} or {\sl purpose}.  The aim of a language
definition is not to formalise everything which could possibly be called the
meaning of a program, but to establish a theory of semantic objects  
upon which the understanding of particular programs may rest.

The job of a language-definer is twofold.  First -- as we have already 
suggested -- he must create a world of meanings appropriate for the language, 
and must find a way of saying
what these meanings precisely are.  Here, he meets a problem; notation
of {\sl some} kind must be used to denote and describe these meanings --
but not a {\sl programming language} notation, unless he is passing the
buck and defining one programming language in terms of another. Given
a concern for rigour, mathematical notation is an obvious choice.  Moreover, 
it is not enough just to
write down mathematical definitions. The world of meanings only becomes
meaningful if the objects possess nice properties, which make them tractable.
So the language-definer really has to develop a small {\sl theory} of his 
meanings, in the same way that a mathematician develops a theory.  
Typically, after initially defining some objects, the mathematician goes on to 
verify properties which indicate that they are objects worth studying. 
It is this part, a kind of scene-setting, which the language-definer shares 
with the 
mathematician.  Of course he can take many objects and their theories 
directly from mathematics,  such as functions, relations,
trees, sequences, \ldots. But he must also give some special theory for the
objects which make his language particular, as we do for types, structures and 
signatures in this book; otherwise his language definition may be 
formal but will give no insight.

The second part of the definer's job is to define {\sl evaluation} precisely.
This means that he must define at least {\sl what} meaning, $M$, results
from evaluating any phrase $P$ of his language (though he need not explain
exactly {\sl how} the meaning results; that is he need not give the full 
detail of every computation).  This part of his job must be formal
to some extent, if only because the phrases $P$ of his language are indeed
formal objects.  But there is another reason for formality.  The task is
complex and error-prone, and therefore demands a high level of explicit
organisation (which is, largely, the meaning of `formality'); moreover,
it will be used to specify an equally complex, error-prone and formal
construction: an implementation.

We shall now explain the keystone of our semantic method.  First, we need a
slight but important refinement. A phrase $P$ is never evaluated {\sl in
vacuo} to a meaning $M$, but always {\sl against a background}; this 
background -- call it $B$ -- is itself a semantic object, 
being a distillation of the meanings
preserved from evaluation of earlier phrases (typically variable declarations,
procedure declarations, etc.).  In fact evaluation is background-dependent
-- $M$ depends upon $B$ as well as upon $P$.   

The keystone of the method, then, is a certain kind of assertion about
evaluation; it takes the form
\[ B\vdash P\Rightarrow M\]
and may be pronounced: `Against the background $B$, the phrase $P$ evaluates
to the meaning $M$'.  {\sl The formal purpose of this Definition is no more, 
and no less, than to decree exactly which assertions of this form are true.}
This could be achieved in many ways. We have chosen to do it in a structured
way, as others have, by giving rules which allow assertions about a 
{\sl compound} phrase $P$ to be inferred from assertions about its
{\sl constituent} phrases $P_1,\ldots,P_n$.

We have written the Definition in a form suggested by the previous remarks.
That is, we have defined our semantic objects in mathematical
notation which is completely independent of Standard ML, and we have
developed just enough of their theory to give sense to our rules of
evaluation. 

   Following another suggestion above, we have factored our task
by describing {\sl abstract} evaluation -- the inference and checking of
types and signatures (which can be done at compile-time) -- completely 
separately from {\sl concrete} evaluation.  
It really is a factorisation, because a {\sl full} value in all its glory --
you can think of it as a concrete object with a type
attached -- never has to be presented.\note{\thecommentary}{Paragraph moved}

\deletion{\thenewpreface}{
\subsubsection*{The form of the Definition
\footnote{
   The Definition has evolved through a sequence of three previous versions,
   circulated as Technical Reports.  For those who have followed the
   sequence, we should point out that the treatment of {\sl equality types}
   and of {\sl admissibility} has been slightly modified in this publication 
   to meet the claim for principal signatures.  The changes are mainly
   in Sections~4.9, 5.5 and 5.13 and in the inference rules~19, 20,
   29 and 65.}}}

\subsubsection*{The Revision of Standard ML}
{\it The Definition of Standard ML} was published in 1990.  
Since then the implementation technology of
the language has advanced enormously, and its users have 
multiplied.  The language and its Definition 
have therefore incited close scrutiny, evaluation, much approval, 
sometimes strong criticism.\note{\thenostrsharing}{Section added.}

The originators of the language have sifted this response, and
found that there are inadequacies in the original language and its 
formal Definition.  They are of three kinds: missing features which
many users want;  complex and little-used features which most
users can do without; and mistakes of definition.  What is
remarkable is that these inadequacies are rather few, and that they
are rather uncontroversial.  

This new version of the Definition addresses the three kinds
of inadequacy respectively by additions, subtractions and corrections.
But we have only made such amendments when one or more aspects of
SML -- the language itself, its usage, its implementation, its formal 
Definition -- have thus become simpler, without complicating the
other aspects.  It is worth noting that even the additions meet this 
criterion; for example we have introduced type abbreviations
in signatures to simplify the use of the language, but the way we
have done it has even simplified the Definition too.   In fact,
after our changes the formal Definition has fewer rules.

In this exercise we have consulted the major implementers and several
users, and have found broad agreement.  In the 1990 Definition it was
predicted that further versions of the Definition would be produced as
the language develops, with the intention to minimise the number of
versions.  This is the first revised version, and we foresee no
others. The changes that have been made to the 1990~Definition are
enumerated in Appendix~\ref{whatisnew-app}.

The resulting document is, we hope, valuable as the essential point of 
reference for Standard ML.  If it is to play this role well, it must
be supplemented by other literature.  \replacement{\thenewpreface}{Some}{Many}
expository books have already been 
written, and this Definition will be useful as a 
background reference for their readers. 
We \replacement{\thenewpreface}{have also become}{became}
convinced, while
writing the \insertion{\thenewpreface}{1990} Definition, that we could 
not discuss many questions without 
making it far too long.  Such questions are: Why were certain design choices
made?  What are their implications for programming?  
Was there a good alternative meaning for some constructs, or was our
hand forced?  What different forms of phrase are equivalent? What is the proof
of certain claims?  Many of these
questions \replacement{\thenewpreface}{will not be}{are not} 
answered by pedagogic texts either. 
\replacement{\thenewpreface}{So we are writing
a Commentary\cite{mt91} on the Definition which will assist people in reading it, and
which will serve as a bridge between the Definition and other texts.}
{We therefore wrote a 
Commentary on the 1990
Definition to
assist people in reading it, and to
serve as a bridge between the Definition and other texts. Though in
part outdated by the present revision, the Commentary still largely fulfils its 
purpose.}

\insertion{\thenewpreface}{There exist several textbooks on programming
with Standard ML\cite{paulson96,mcp93,Ullman94,stansifer92}. The second
edition of Paulson's book\cite{paulson96} conforms with the present revision.}


We wish to thank Dave Berry, Lars Birkedal, Martin Elsman, Stefan Kahrs
and John Reppy for many detailed comments and suggestions which have assisted
the revision.

\begin{flushright}
Robin Milner\quad Mads Tofte\quad Robert Harper\quad David MacQueen\\[1cm]
November 1996
\end{flushright}

%\newpage
\deletion{\thenewpreface}{\begin{flushright} Edinburgh\\August 1989 \end{flushright}}

\subsection*{Successor ML}
{\it The Definition of Standard ML (Revised)} was published in 1997~\cite{sml97-definition}, and
{\it The Standard ML Basis Library} was published in 2004~\cite{sml-basis-lib}.
Since that time, while SML implementations have matured, the language that they
implement has remained static.
Successor ML is a collection of proposed changes and extensions to SML that both
address problems in the Definition and improve and grow the language in natural ways.
This document merges the formal description of Successor ML features developed
by Andreas Rossberg~\cite{hamlet-s} into the Standard ML Revised Definition.
It is hoped that the resulting document will serve as basis for the future development
of Standard ML.

We use the following conventions in highlighting the changes.
Old material that is no longer relevant is \CUT{grayed and struck out}; fixes to the
definition that do not represent new features or significant changes are
\FIX{rendered in blue text}; and new features are \ADD{rendered in magenta}.

\begin{flushright}
John Reppy\\[1cm]
June 2015
\end{flushright}


\clearpage
\pagestyle{empty}
\ \clearpage
\pagestyle{headings}
\setcounter{page}{1}
\renewcommand{\thepage}{\arabic{page}}
\clearpage{}
\thispagestyle{empty}
%!TEX root = root.tex
%
\section{Introduction}

This document formally defines \REPL{Successor }{Standard} ML.
\ADD{It is derived from the 1997 \emph{Definition of Standard ML} by adding the
changes suggested by Andreas Rossberg in the HaMLet S documentation.}

To understand the method of definition, at least in broad terms, it helps to
consider how an implementation of ML is naturally
organised.  ML is an interactive
language,\index{4.1} 
and a {\sl program}\index{4.2}  consists of a sequence of {\sl top-level
declarations};\index{4.3} the execution
of each declaration modifies the top-level environment, which we call
a {\sl basis}, and reports the modification to the user.

In the execution of a declaration there are three phases:
{\sl parsing}, {\sl elaboration}, and {\sl evaluation}.  
Parsing
determines the grammatical form of a declaration.  Elaboration, the
{\sl static} phase, determines whether it is well-typed and
well-formed in other ways, and records relevant type or form information
in the basis. Finally evaluation, the {\sl dynamic} phase, determines the
value of the declaration and records relevant value information in the
basis.  Corresponding to these phases, our formal definition divides
into three parts:  grammatical rules, elaboration rules, and evaluation
rules.  Furthermore, the basis is divided into the {\sl static} 
basis and the {\sl dynamic} basis; for example, a variable which has been
declared is associated with a type in the static basis and with a value in
the dynamic basis.

In an implementation, the basis need not be so divided.  But for the
purpose of formal definition, it eases presentation and understanding to
keep the static and dynamic parts of the basis separate.
This is further justified by programming experience.  A large proportion
of errors in ML programs are discovered during elaboration, and identified
as errors of type or form, so it follows that it is useful to perform
the elaboration phase separately.  In fact, elaboration without
evaluation is \replacement{\thenewpreface}{just}{part of} what is normally called {\sl compilation};  
once
a declaration (or larger entity) is compiled one wishes to evaluate it --
repeatedly -- without re-elaboration, from which it follows that it is
useful to perform the evaluation phase separately.

A further factoring of the formal definition is possible,
because of the structure of the language.  ML consists of a lower level
called the {\sl Core language} (or {\sl Core} for short), a middle level
concerned  with programming-in-the-large called {\sl Modules},
and a very small upper level called {\sl Programs}.
With the three phases described above, there is therefore
a possibility of nine components in the
complete language definition. We have allotted one section to each
of these components, except that we have combined the parsing,
elaboration and evaluation of Programs in one section. The
scheme for the ensuing seven sections is therefore as follows:

\vspace*{3mm}
\begin{tabular}{rccc}
                & {\em Core} & {\em Modules} & {\em Programs} \\
\cline{2-4}
{\em Syntax}    &\multicolumn{1}{|c}{Section 2}
                             &\multicolumn{1}{|c}{Section 3}
                                             &\multicolumn{1}{|c|}{ }\\
\cline{2-3}
{\em Static Semantics}
                &\multicolumn{1}{|c}{Section 4}
                             &\multicolumn{1}{|c}{Section 5}
                                             &\multicolumn{1}{|c|}{Section 8}\\
\cline{2-3}
{\em Dynamic Semantics}
                &\multicolumn{1}{|c}{Section 6}
                             &\multicolumn{1}{|c}{Section 7}
                                             &\multicolumn{1}{|c|}{}\\
\cline{2-4}
\end{tabular}
\vspace*{3mm}

%The Core is
%a complete language in its own right, and its embedding in the full
%language is simple;  therefore each of the three parts of the formal
%description is further divided into two -- one for the Core, and one for
%Modules.

The Core provides many phrase classes, for programming convenience.
But about half of these classes are derived forms, whose meaning can be
given by translation into the other half which we call the
{\sl Bare} language.   
Thus each of the three parts for the Core treats only the bare language;
the derived forms are treated in  Appendix~\ref{derived-forms-app}.
This appendix also contains a few derived forms for Modules.
A full grammar for the language is presented in
Appendix~\ref{core-gram-app}.

In\index{5.0} Appendices~\ref{init-stat-bas-app} and~\ref{init-dyn-bas-app} 
the {\sl initial basis} is detailed.  This basis,
divided into its static and dynamic parts, contains the static and
dynamic meanings of \replacement{\thelibrary}{all predefined identifiers.}{a
small set of predefined identifiers. A richer basis is defined
in a separate document\cite{sml-basis-lib}.}

The semantics is presented in a form  known as Natural
Semantics.\index{5.1}  It consists of a set of rules allowing 
{\sl sentences} of the form
\[ A \vdash phrase \Rightarrow A' \]
to be inferred, where $A$ is often a basis (static or dynamic) and $A'$ a
semantic object
-- often a type in the static semantics and a value in the dynamic
semantics. One should read such a sentence as follows: ``against
the background provided by
$A$, the phrase $phrase$ elaborates -- or evaluates -- to the object
$A'$''.
Although the rules themselves are formal the semantic
objects, particularly the static ones, are the subject of a mathematical
theory which is presented in a succinct form in the relevant sections.\deletion{\thenostrsharing}{
This theory, particularly the theory of types and signatures, will
benefit from a more pedagogic treatment in other publications; the
treatment here is
probably the minimum required to understand the meaning of the rules.}

The robustness of the semantics depends upon theorems.  
Usually these have been proven, but the proof is not included.
\deletion{\thenostrsharing}{In two cases, however, they are presented as ``claims'' rather than
theorems; these are the claim of principal environments in 
Section~\ref{principal-env-sec}, and the claim of principal signatures
in Section~\ref{prinsig-sec}. We need further confirmation of our
detailed proofs of these claims, before asserting them as theorems.}





\clearpage{}
\thispagestyle{empty}
%!TEX root = root.tex
%
\section{Syntax of the Core}
\label{syn-core-sec}

\subsection{Reserved Words}
The\index{6.1}
following are the {\sl reserved words} used in the Core.   They
may not (except~~{\tt =}~) be used as identifiers.
%In this document the alphabetic reserved words are
%always shown in typewriter font.

\vspace*{-6pt}
\begin{verbatim}
     abstype   and   andalso   as   case   datatype  do    else
     end    exception    fn    fun    handle    if   in   infix
     infixr   let     local    nonfix   of   op   open   orelse
     raise   rec   then   type   val   with   withtype    while
     (  )   [  ]   {  }  ,  :  ;  ...    _   |  =   =>   ->   #
\end{verbatim}
\vspace*{-6pt}

\subsection{Special constants}
\label{cr:speccon}
\REPL{A {\sl positive integer constant (in decimal notation)} is
a non-empty sequence of decimal digits
$\boxml{0},\twoldots,\boxml{9}$ and the underscore (\wildpat) that
neither starts nor ends with an underscore.
An\index{6.2} {\sl integer constant} is an optional negation symbol
(\tttilde) followed by a positive integer constant. }{%
An\index{6.2} {\sl integer constant (in
decimal notation)} is an optional negation symbol (\tttilde)
followed by a non-empty sequence of decimal digits
$\boxml{0},\twoldots,\boxml{9}$.}
An {\sl integer constant (in
hexadecimal notation)} is an optional negation symbol
followed by \boxml{0x} followed by a non-empty sequence of
hexadecimal digits $\boxml{0},\twoldots,\boxml{9}$
and $\boxml{a},\twoldots,\boxml{f}$\ADD{, and the underscore that does not
end with an underscore}.
($\boxml{A},\twoldots,\boxml{F}$
may be used as alternatives for $\boxml{a},\twoldots,\boxml{f}$.)
\ADD{
An {\sl integer constant (in binary notation)} is an optional negation
symbol followed by a non-empty sequence of binary digits \boxml{0}, \boxml{1},
and the underscore that does not end with an underscore.
}

A {\sl word constant (in decimal notation)} is \boxml{0w}  followed
by a non-empty sequence of decimal digits\ADD{ and the underscore that does not
end with an underscore}.
A {\sl word constant
(in hexadecimal notation)} is \boxml{0wx} followed by a non-empty
sequence of hexadecimal digits\ADD{ and the underscore that does not
end with an underscore}.
\ADD{A word constant (in binary notation) is \boxml{0wb} followed by a non-empty
sequence of binary digits \boxml{0}, \boxml{1}, and the underscore not ending with
an underscore.}

A {\sl real constant} is an integer constant in decimal notation,
possibly followed by a point ({\tt .}) and \REPL{a positive integer constant in decimal notation}{ one or
more decimal digits}, possibly followed by an exponent symbol ~({\tt E} or {\tt e})~ and an integer
constant in decimal notation; at least one of the optional parts must occur, hence no integer
constant is a real constant.
Examples:
\begin{center}
  {\tt 0.7}~~~{\tt 3.32E5}~~~\verb(3E~7(~~~\ADD{\tt 3.141\wildpat592\wildpat653}
\end{center}%
Non-examples:
\begin{center}
  {\tt 23}~~~{\tt .3}~~~{\tt 4.E5}~~~{\tt 1E2.0}~~~\ADD{\tt 1\wildpat.5}~~~\ADD{\tt 3.\wildpat{}678}
~~~\ADD{\tt 1.\wildpat{}E2}
\end{center}%

We assume an underlying alphabet of $N$ characters ($N \geq 256$), numbered
$0$ to $N-1$, which agrees with the ASCII character set on the characters
numbered 0 to 127. The interval $[0, N-1]$ is called the {\sl ordinal range} of
the alphabet.
A {\sl string constant} is a sequence, between quotes ({\tt "}), of zero or
more printable characters (i.e., numbered 33--126), spaces or escape
sequences.
Each escape sequence starts with the
escape character ~\verb+\+~, and stands for a character sequence. The
escape sequences are:
\bigskip

\note{\thescon}{Inserted additional escape sequences in figure concerning
string constants and unicodes}          %   hack: can't deal with verb in arg of replacement
\halign{\indent#\hfil&\quad\parbox[t]{12cm}{\strut#\strut}\cr
\verb+\a+   & A single character interpreted by the system as alert (ASCII~7)\cr
\verb+\b+   & Backspace (ASCII 8)\cr
\verb+\t+   & Horizontal tab (ASCII 9)\cr
\verb+\n+   & Linefeed, also known as newline (ASCII 10)\cr
\verb+\v+   & Vertical tab (ASCII 11)\cr
\verb+\f+   & Form feed (ASCII 12)\cr
\verb+\r+   & Carriage return (ASCII 13)\cr
\verb+\^+$c$  & The control character $c$, where $c$ may
                be any character with number 64--95. The number
                of ~{\tt\char'134\char'136}$c$~ is 64 less than the
                number of $c$.\cr
\verb+\+$ddd$ & The single character with number $ddd$ (3 decimal digits
denoting an integer in the ordinal range of the alphabet).\cr
\uconst & The single character with number $xxxx$ (4 hexadecimal digits
denoting an integer in the ordinal range of the alphabet).\cr
\ADD{\Uconst} & \ADD{The single character with number $xxxxxxxx$ (8 hexadecimal digits
denoting an integer in the ordinal range of the alphabet).}\cr
\verb+\"+   & {\tt "}\cr
\verb+\\+   & {\tt\char'134}\cr
\verb+\+$f\cdot\cdot f$\verb+\+
            & This sequence is ignored,
              where $f\cdot\cdot f$ stands for a sequence
             of one or more formatting characters.\cr
}
\medskip

The {\sl formatting characters}\index{6.3} are a subset of the non-printable
            characters including at least space, tab, newline, form feed\FIX{, vertical tab, and carriage return}.
The last form allows long strings to be written on more than one line, by
            writing ~\verb+\+~ at the end of one line and at the start of the
            next.\nopagebreak

\insertion{\thescon}{A {\sl character constant} is a sequence of the form
{\tt\#}$s$, where $s$ is a string constant denoting a string of size one character.

Libraries may provide multiple numeric types and multiple string types.
To each string type
corresponds an alphabet with ordinal range $[0, N-1]$
for some $N\geq 256$; each alphabet must agree with the ASCII character set on
the characters numbered 0 to 127. When multiple alphabets are supported,
all characters of a given string constant are interpreted over the same
alphabet. For each special constant, overloading
resolution is used for determining the type of the constant
(see Appendix~\ref{overload.sec}).

%   All the escape sequences in a given string constant are interpreted
%   over the same underlying alphabet (either 8-bit characters or 16-bit
%   characters), and for each string constant this alphabet is determined
%   by overloading resolution (see Appendix~\ref{overload.sec}).
%   It is a compile-time error
% if the
%constant contains an escape sequence of the form $\uconst$ where
%$xxxx$ denotes an integer outside the ordinal range of the alphabet so
%determined. For example, within a sequence of 8-bit characters, the
%two leftmost hexadecimal digits of $\uconst$ must be {\tt 0} (zero).
}

We denote by {\SCon} the class of {\sl special constants}, i.e., the integer,
real, \insertion{\thescon}{word, character} and string constants; we shall use {\scon}
to range over \SCon.\index{6.4}

\subsection{Comments}
\REPL{
A\index{7.1} {\sl comment} is either {\sl line comment} or a {\sl block comment}.
A line comment is any character sequence between the comment delimiter \boxml{(*)}
and the following end of line.
A block comment is any character sequence within comment brackets ~{\tt (* *)}~
in which other comments are properly nested.
No space is allowed between the characters that make up a comment bracket
\ml{(*)}, \ml{(*} or \ml{*)}.
An unmatched \boxml{(*} should be detected by the compiler.
}{%
A\index{7.1} {\sl comment}
is any character sequence within comment brackets ~{\tt (* *)}~
in which
comment brackets are properly nested.
No space is allowed between
            the two characters which make up
            a comment bracket \ml{(*} or \ml{*)}.  An unmatched
            \boxml{(*} should be detected by the compiler.
}
%
\subsection{Identifiers}
\label{cyn-core-identifiers-sec}
The classes of {\sl identifiers}\index{7.2} for the Core are shown in
Figure~\ref{identifiers}.
% \begin{figure}[b]
% \vspace{4pt}
% \makeatletter{}
% \tabskip\@centering
% \halign to\textwidth
% {#\hfil\tabskip1em&(#)\hfil\tabskip1em&#\hfil\tabskip\@centering\cr
% \Var    & value variables       & long\cr
% \Con    & value constructors    & long\cr
% \Exn    & exception constructors& long\cr
% \TyVar  & type variables        & \cr
% \TyCon  & type constructors     & long\cr
% \Lab    & record labels         & \cr
% \StrId  & structure identifiers & long\cr
% }
% \makeatother
% \caption{Identifiers}
% \label{identifiers}
% \vspace*{-3mm}
% \end{figure}
\note{\theidstatus}{Figure 1 replaced by new figure (Var, Con and ExCon
merged into VId)}
\begin{figure}[b]
\vspace{4pt}
\makeatletter{}
\tabskip\@centering
\halign to\textwidth
{#\hfil\tabskip1em&(#)\hfil\tabskip1em&#\hfil\tabskip\@centering\cr
\VId    & value identifiers     & long\cr
\TyVar  & type variables        & \cr
\TyCon  & type constructors     & long\cr
\Lab    & record labels         & \cr
\StrId  & structure identifiers & long\cr
}
\makeatother
\caption{Identifiers}
\label{identifiers}
\vspace*{-3mm}
\end{figure}
We use $\vid$, $\tyvar$ to range over
$\VId$, TyVar etc.  For each class
X marked ``long'' there is a class longX of {\sl long identifiers}; if
$x$ ranges over X then {\it longx} ranges over longX.  The syntax of
these long identifiers is given by the following:
\vspace*{-6pt}
\begin{quote}
\begin{tabular}{rcll} {\it longx} & $::=$ & $x$ & identifier\\
& &$\strid_1.\cdots.\strid_n.x$ & qualified identifier ($n\geq 1$)
\end{tabular}
\end{quote}
\vspace*{-6pt}
The qualified identifiers constitute a link between the Core and the
Modules. Throughout this document, the term ``identifier,'' occurring
without an adjective, refers to non-qualified identifiers only.
%version 2: For each class X marked
%``long'' there is also a class
%\[ {\rm LongX} = \StrId^\ast \times {\rm X} \]
%If $x$ ranges over X, then {\it longx}, or
%$\strid_1.\cdots.\strid_k.x$, $k\geq 0$, ranges over LongX.
%These long identifiers constitute the only link between the Core
%and the language constructs for Modules; by omitting them, and the $\OPEN$
%declaration,
%we obtain the Core as a complete programming language in
%its own right. (The corresponding adjustment to the Core static and
%dynamic semantics is simply to omit structure environments $\SE$.).

An identifier is either {\sl alphanumeric}: any sequence of
letters, digits, primes ({\tt '}) and underbars (\wildpat) starting
with a letter or prime, or {\sl symbolic}: any non-empty sequence of the
following {\sl symbols}\index{7.3}
\vspace*{-6pt}
\begin{center}
%\verb(!  %  &  $  +  -  /  :  <  =  >  ?  (@\verb(  \  ~  `  ^  |  *(
\verb(!  %  &  $  #  +  -  /  :  <  =  >  ?  @  \  ~  `  ^  |  *(
\end{center}
\vspace*{-6pt}
In either case, however, reserved words are excluded.   This means that for
example ~\verb+#+~ and ~{\tt |}~ are not identifiers, but  ~\verb+##+~ and
~{\tt |=|}~ are identifiers.
The only exception to this rule is that the symbol ~{\tt =}~, which is
a reserved word, is also allowed as an identifier to stand for
the equality predicate.
The identifier ~{\tt =}~ may not be re-bound;
this precludes any syntactic ambiguity.

A type variable $\tyvar$\index{7.4}\label{etyvar-lab} may be any
alphanumeric identifier starting with a prime; the subclass EtyVar of
TyVar, the {\sl equality} type variables, consists of those \ADD{which}
start with two or more primes.
%poly
The classes {\VId},
{\TyCon} and  {\Lab} are represented by identifiers
not starting with a prime.
However,\index{7.5} {\tt *} is excluded from {\TyCon},
to avoid confusion with the derived form of tuple type (see
Figure~\ref{typ-gram}). The class Lab\index{8.2} is extended to
include the {\em numeric} labels ~{\tt 1}~~{\tt 2}~~{\tt 3}~ $\cdots$,
i.e. any numeral not starting with~{\tt 0}. \insertion{\theidstatus}{The
identifier class $\hbox{\StrId}$ is represented by alphanumeric
identifiers not starting with a prime.}

TyVar is therefore disjoint from the other four classes.
Otherwise, the syntax class of an occurrence of
identifier $\id$ in a Core phrase (ignoring derived forms,
Section~\ref{cor-der-form-sec}) is determined thus:
\begin{enumerate}
  \item Immediately before ``.'' -- i.e. in a long identifier -- or in an
        $\OPEN$ declaration, $\id$ is a structure
        identifier.  The following rules assume that all occurrences of
        structure identifiers have been removed.
  \item At the start of a component in a record type, record pattern or record
        expression,  $\id$ is a record label.
  \item Elsewhere in types $\id$ is a type constructor.
%  \item Elsewhere $\id$ is an exception name if it occurs immediately after
%        $\RAISE$, at the start of a handler rule $\hanrule$, or within an
%        $\EXCEPTION$ declaration or specification.
  \item Elsewhere, $\id$ is a value identifier.
\end{enumerate}%

By means of the above rules a compiler can determine the class to which each
identifier occurrence belongs; for the remainder of this document we shall
therefore assume that the classes are all disjoint.

\subsection{Lexical analysis}
Each\index{8.4} item of lexical analysis is either a reserved word, a numeric label, a
special constant or a long identifier.
Comments and formatting characters
separate items (except within string constants; see Section~\ref{cr:speccon})
and are otherwise
ignored.   At each stage the longest next item is taken.

\subsection{Infixed operators}
\label{infop.sec}
An\index{8.5} identifier may be given {\sl infix status} by the
~$\INFIX$~ or ~$\INFIXR$~ directive, which may occur as a
declaration; this status only pertains to its use as a
$\vid$ within the scope (see below) of the
directive, and in these uses it is called an {\sl infixed operator}.
(Note that qualified identifiers never have infix status.)  If
$\vid$
has infix status, then ``$\exp_1\ \vid\ \exp_2$''
(resp.\ ``$\pat_1\ \vid\ \pat_2$'') may occur -- in parentheses if necessary -- wherever
the application ``$\vid$\verb+{+{\tt 1=}$\exp_1$\verb+,+{\tt
2=}$\exp_2$\verb+}+'' or its derived form
``$\vid$\verb+(+$\exp_1$\verb+,+$\exp_2$\verb+)+'' (resp
``$\vid$\verb+(+$\pat_1$\verb+,+$\pat_2$\verb+)+'') would otherwise
occur.  On the other hand, an occurrence of any long identifier (qualified
or not) prefixed by {\OP} is treated as non-infixed. The only required
use of {\OP} is in prefixing a non-infixed occurrence of an
identifier $\vid$ \REPL{that}{which} has infix status\FIX{ in an expression or pattern}; elsewhere {\OP}, where
permitted, has no effect.\index{9.1}
Infix status is cancelled by the ~$\NONFIX$~
directive.  We refer to the three directives collectively as {\sl fixity directives}.

The form of the fixity directives is as follows ($n\geq 1$):
\[ \newlonginfix \]
\[ \newlonginfixr \]
\[ \newlongnonfix \] where $\langle d\rangle$ is an optional decimal digit $d$ indicating
binding precedence. A higher value of $d$ indicates tighter binding;
the default is {\tt 0}. ~$\INFIX$~ and ~$\INFIXR$~ dictate left and right
associativity respectively. In an expression of the form  $\exp_1\, \vid_1\, \exp_2\, \vid_2\, \exp_3$, where
      $\vid_1$ and $\vid_2$ are infixed operators with the same precedence,
      either both must associate to the left or both must
associate to the right.
For example, suppose that {\tt <<} and {\tt >>} have equal precedence,
but associate to the left and right respectively; then
\medskip

\tabskip4cm
\halign to\hsize{\indent\hfil{\tt #}\tabskip1em&\hfil#\hfil\ &\ {\tt #}\hfil\cr
x << y << z&parses as&(x << y) << z\cr
x >> y >> z&parses as&x >> (y >> z)\cr
x << y >> z&is illegal\cr
x >> y << z&is illegal\cr}
\medskip
The precedence of infixed operators relative
to other expression and pattern constructions is given in
Appendix~\ref{core-gram-app}.

The {\sl scope}\index{9.2} of a fixity directive $\dir$ is the ensuing program text,
except that if $\dir$ occurs in a declaration $\dec$ in either of the phrases
\[ \LET\ \dec\ \IN\ \cdots\ \END \]
\[ \LOCAL\ \dec\ \IN\ \cdots\ \END \]
then the scope of $\dir$ does not extend beyond the phrase. Further scope
limitations are imposed for Modules\insertion{\theidstatus}{ (see
Section~\ref{infixopmod.sec})}.

These directives and ~$\OP$~ are omitted from the semantic rules, since they
affect only parsing.

\subsection{Derived Forms}
\label{cor-der-form-sec}
There\index{9.3} are many standard syntactic forms in ML whose meaning can be expressed
in terms of a smaller number of syntactic forms, called the {\sl bare} language.
These derived forms, and their equivalent forms in the bare language, are
given in
Appendix~\ref{derived-forms-app}.

%With one exception, these derived forms use no new lexical items.  The
%exception is that the symbol \verb+#+ prefixed to an identifier of the
%class Lab constitutes a
%lexical item;  \verb+#+{\it lab} denotes a selector function on records, cf. page~\pageref{der-exp}.


\subsection{Grammar}

The phrase classes for the Core are shown in Figure~\ref{cor-phr}.
We use the variable $\atexp$ to range over AtExp, etc.
The grammatical rules for the Core are shown in Figures~\ref{pat-syn} and~\ref{exp-syn}.

%\clearpage
\begin{figure}[t]
\vspace{4pt}
\makeatletter{}
\tabskip\@centering
\halign to\textwidth
{#\hfil\tabskip1em&#\hfil\tabskip\@centering\cr
AtExp   & atomic expressions \cr
ExpRow  & expression rows \cr
Exp     & expressions \cr
Match   & matches \cr
Mrule   & match rules \cr
\noalign{\vspace{2mm}}
%\cr
Dec     & declarations \cr
ValBind & value bindings \cr
TypBind & type bindings \cr
DatBind & datatype bindings \cr
ConBind & constructor bindings \cr
%version 1: Constrs & datatype constructions \cr
ExBind  & exception bindings \cr
\noalign{\vspace{2mm}}
%\cr
AtPat   & atomic patterns \cr
PatRow  & pattern rows \cr
Pat     & patterns \cr
\noalign{\vspace{2mm}}
%\cr
Ty      & type expressions \cr
TyRow   & type-expression rows \cr
}
\makeatother
\caption{Core Phrase Classes}
\label{cor-phr}
\end{figure}

The following\index{10.1} conventions are adopted in presenting the grammatical rules,
and in their interpretation:
\begin{itemize}
  \item The brackets\index{10.2} ~$\langle\ \rangle$~ enclose optional phrases.
  \item For any syntax class X (over which $x$ ranges)
we define the syntax class Xseq (over which {\it xseq} ranges) as follows:
    \begin{quote}
    \begin{tabular}{rcll}
       {\it xseq} & $::=$ & $x$ & (singleton sequence)\\
                  &       &     & (empty sequence)\\
                  &       & \ml{(}$x_1$\ml{,}$\cdots$\ml{,}$x_n$\ml{)}
                                & (sequence,~$n\geq 1$) \\
    \end{tabular}
    \end{quote}
(Note that the ``$\cdots$'' used here, meaning syntactic iteration, must not be
confused with ``$\mbox{\ml{...}}$'' which is a reserved word of the language.)
  \item Alternative forms for each phrase class are in order of decreasing
        precedence; this resolves ambiguity in parsing, as explained
        in Appendix~\ref{core-gram-app}.\index{10.3}
  \item L (resp. R)\index{10.4} means left (resp. right) association.

\item The syntax of types binds more tightly than that of expressions.

\item Each\index{10.6} iterated construct (e.g. $\match$, $\cdots$)
extends as far
right as possible; thus, parentheses may be needed around an expression which
terminates with a match, e.g. ``$\FN\ \match$'', if this occurs within a
larger
match.
\end{itemize}

\begin{figure}[h]
%\vspace{4pt}
\makeatletter{}
\tabskip\@centering
\halign to\textwidth
{#\hfil\tabskip1em&\hfil$#$\hfil&#\hfil&#\hfil\tabskip\@centering\cr
  \atpat& ::=   & \wildpat      & wildcard\cr
        &       & \scon         & special constant\cr
        &       & \opp\longvid  & value identifier\cr
        &       & \verb+{ +\recpat\verb+ }+
                                & record\cr
        &       & \parpat       & \cr
\noalign{\vspace{6pt}}
\labpats& ::=   & \wildrec      & \REPL{ellipses }{wildcard}\cr
        &       & \longlabpats  & pattern row\cr
\noalign{\vspace{6pt}}
  \pat  & ::=   & \atpat        & atomic\cr
        &  	& \opp\vidpat   & constructed pattern\cr
        &       & \vidinfpat	& infixed value construction\cr
        &       & \typedpat     & typed \ADD{(L)}\cr
        &       & \CUT{\opp\layeredvidpat} & \CUT{layered}\cr
        &	& \ADD{\aspat}	& \ADD{conjunction (R)}\cr
        &	& \ADD{\orpat}	& \ADD{disjunction (R)}\cr
        &	& \ADD{\nestedpat} & \ADD{nested match}\cr
\noalign{\vspace{6pt}}
  \ty   & ::=   & \tyvar        & type variable\cr
        &       & \verb+{ +\rectype\verb+ }+
                                & record type expression\cr
        &       & \constype     & type construction\cr
        &       & \funtype      & function type expression (R)\cr
        &       & \partype      & \cr
\noalign{\vspace{6pt}}
\labtys & ::=   & \longlabtys   & type-expression row\cr
        &       & \ADD{\extrecty} & \ADD{ellipses}\cr
\noalign{\vspace{6pt}}
}
\makeatother
\vspace{-2mm}
\caption{Grammar: Patterns and Type expressions\index{12.2}\index{13.1}}
\label{pat-syn}
\end{figure}
%\nopagebreak[4]

\begin{figure}[t]
\vspace{4pt}
\makeatletter{}
\tabskip\@centering
\halign to\textwidth
{#\hfil\tabskip1em&\hfil$#$\hfil&#\hfil&#\hfil\tabskip\@centering\cr
  \atexp& ::=   & \scon         & special constant\cr\adhocdeletion{\theidstatus}{1cm}{
        &       & \opp\longvar  & value variable\cr
        &       & \opp\longcon  & value constructor\cr
        &       & \opp\longexn  & exception constructor\cr}\adhocinsertion{\theidstatus}{-10cm}{
        &       & \opp\longvid  & value identifier\cr}
        &       & \verb+{ +\recexp\verb+ }+
                                & record\cr
        &       & \letexp       & local declaration\cr
        &       & \parexp       & \cr
\noalign{\vspace{6pt}}
\labexps& ::=   & \longlabexps  & expression row\cr
        &       & \ADD{\extrecexp} & \ADD{ellipses}\cr
\noalign{\vspace{6pt}}
  \exp  & ::=   & \atexp        & atomic\cr
        &       & \appexp       & application (L)\cr
        &       & \adhocreplacementl{\theidstatus}{1cm}{\infexp}{\vidinfexp}       & infixed application\cr
        &       & \typedexp     & typed (L)\cr
        &       & \handlexp     & handle exception\cr
        &       & \raisexp      & raise exception\cr
        &       & \fnexp        & function\cr
\noalign{\vspace{6pt}}
\match  & ::=   & \longmatch    & \cr
\noalign{\vspace{6pt}}
\mrule  & ::=   & \longmrule    & \cr
\noalign{\vspace{6pt}}
  \dec  & ::=   & \explicitvaldec & value declaration\cr
        &       & \typedec      & type declaration\cr
        &       & \datatypedec  & datatype declaration\cr
        &       & \datatyperepldec & datatype replication\cr
        &       & \CUT{\abstypedec} & \CUT{abstype declaration}\cr
        &       & \exceptiondec & exception declaration\cr
        &       & \localdec     & local declaration\cr
        &       & \openstrdec   & open declaration ($n\geq 1$) \cr
        &       & \emptydec     & empty declaration\cr
        &       & \seqdec       & sequential declaration \ADD{(L)} \cr
        &       & \newlonginfix & infix (L) directive\cr
        &       & \newlonginfixr & infix (R) directive\cr
        &       & \newlongnonfix & nonfix directive\cr
\noalign{\vspace{6pt}}
\valbind& ::=   & \longvalbind   & \cr
        &       & \CUT{\recvalbind}   & \cr
\noalign{\vspace{6pt}}
\typbind& ::=   & \longtypbind  & \cr
\noalign{\vspace{6pt}}
\datbind& ::=   & \longdatbind  & \cr
\noalign{\vspace{6pt}}
\constrs& ::=   & \opp\longvidconstrs & \cr
\noalign{\vspace{6pt}}
\exnbind& ::=   & \generativeexnvidbind    & \cr
        &       & \eqexnvidbind   & \cr
\noalign{\vspace{6pt}}
}
\makeatother
\vspace{-2mm}
\caption{Grammar: Expressions, Matches, Declarations and Bindings\index{11}\index{12.1}}
\label{exp-syn}
\end{figure}
%\clearpage % 1 August

\subsection{Syntactic Restrictions}\index{13.2}
\label{synres.sec}
\begin{itemize}
\item \CUT{No expression row,
      pattern row or type-expression row may bind the same $\lab$ twice.}\footnote{
        \ADD{This restriction is enforced by the static semantics of the core.}}
\item No binding $\valbind$, $\typbind$, $\datbind$ or $\exnbind$ may bind
      the same identifier twice; this applies also to value identifiers within
      a $\datbind$.
      \ADD{Identifiers appearing in both branches of a disjunctive pattern are bound
      only once.}
\item No $\tyvarseq$ may contain the same $\tyvar$ twice.
\item For each value binding \pat\ \ml{=} \exp\ \REPL{in a value declaration with $\REC$, }{within $\REC$,}
      $\exp$ must be of the form \fnexp. The derived form
      of function-value binding given in Appendix~\ref{derived-forms-app},
      page~\pageref{der-dec}, necessarily obeys this restriction.
\item No $\datbind$, $\valbind$ or $\exnbind$ may bind $\TRUE$, $\FALSE$, $\NIL$, \boxml{::} or $\REF$.
      No $\datbind$ or $\exnbind$ may bind {\tt it}.
\item No real constant may occur in a pattern.
\item In a value declaration  $\VAL\, \tyvarseq\, \valbind$, if $\valbind$
      contains another value declaration  $\VAL\, \tyvarseq'\, \valbind'$
      then $\tyvarseq$ and $\tyvarseq'$ must be disjoint.  In other
      words, no type variable may be scoped by two value
      declarations of which one occurs inside the other.  This
      restriction applies after $\tyvarseq$ and $\tyvarseq'$ have been
      extended to include implicitly scoped type variables, as
      explained in Section~\ref{scope-sec}.
\item[\textcolor{\fixcolor}{$\bullet$}]
      \FIX{Any $\tyvar$ occurring on the right-hand side of a $\typbind$ or $\datbind$ of the form
      ``$\tyvarseq$ $\tycon$ \ml{=} $\ldots$'' must occur in $\tyvarseq$.}
\item[\textcolor{\addcolor}{$\bullet$}]
      \ADD{The pattern $\pat_1$ in a nested match ``$\nestedpat$'' may not itself be a nested
      match, unless enclosed by parentheses.}
\item[\textcolor{\addcolor}{$\bullet$}]
      \ADD{The pattern $\pat$ in a $\valbind$ may not be a nested match, unless enclosed by parentheses.}
\end{itemize}

\clearpage{}
\thispagestyle{empty}
%!TEX root = root.tex
%
\section{Syntax of Modules}
\label{syn-mod-sec}
For Modules there are further reserved words, identifier classes and derived
forms. There are no further special constants; 
comments and lexical analysis are as for the Core.
\replacement{\thenostrsharing}{The derived forms for modules concern functors and  appear in 
Appendix~\ref{derived-forms-app}.}{The derived forms for modules  appear in 
Appendix~\ref{derived-forms-app}.}

%For Modules there are further keywords and identifier classes, but no
%further special constants and at present no further derived forms.  
%Comments and lexical analysis are as for the Core.

\subsection{Reserved Words}
The\index{14.1} following are the additional reserved words used in Modules.\note{\thenostrsharing}{Inserted new keywords  {\tt where} and {\tt :>}}
% \begin{verbatim}
%              eqtype  functor    include   sharing
%              sig     signature  struct  structure
% \end{verbatim}
\begin{verbatim}
             eqtype    functor   include   sharing   sig
             signature   struct   structure   where   :>
\end{verbatim}
\subsection{Identifiers}
\label{syn-mod-identifiers-sec}
The additional identifier classes for Modules are SigId (signature identifiers)
and FunId (functor identifiers). Functor and signature identifiers
must be alphanumeric, not starting with a prime.  The class of each identifier occurrence
is determined by the grammatical rules which follow.  
Henceforth, therefore,
we consider all identifier classes to be disjoint.

\subsection{Infixed operators}
\label{infixopmod.sec}
In addition to the scope rules for fixity directives given for the Core syntax,
there is a further scope limitation:
if $\dir$ occurs in a structure-level declaration $\strdec$ in any of the 
phrases
\[ \LET\ \strdec\ \IN\ \cdots\ \END \]
\[ \LOCAL\ \strdec\ \IN\ \cdots\ \END \]
\[ \STRUCT\ \strdec\ \END \]
then the scope of $\dir$ does not extend beyond the phrase.

One effect of this limitation is that fixity is local to a basic
structure expression --- in particular, to such an expression occurring
as a functor body.

\subsection{Grammar for Modules}
\label{mod-gram-sec}
The\index{14.2} phrase classes for Modules are shown in Figure~\ref{mod-phr}.
We use the variable $\strexp$ to range over StrExp, etc.
The conventions adopted in presenting the grammatical 
%\pagebreak
\begin{figure}[t]
\vspace{4pt}
\makeatletter{}
\tabskip\@centering
\halign to\textwidth
{#\hfil\tabskip1em&#\hfil\tabskip\@centering\cr
StrExp & structure expressions \cr
StrDec & structure-level declarations \cr
StrBind & structure bindings \cr
\cr
SigExp & signature expressions \cr
SigDec & signature declarations \cr
SigBind & signature bindings \cr
\cr
Spec & specifications \cr
ValDesc & value descriptions\cr
TypDesc & type descriptions\cr
DatDesc & datatype descriptions\cr
ConDesc & constructor descriptions\cr
ExDesc & exception descriptions\cr
StrDesc & structure descriptions\cr
%\ adhocinsertion{\thetypabbr}{2mm}{WhereBind & where bindings\cr}
\adhocdeletion{\thetypabbr}{2mm}{SharEq & sharing equations\cr}
\cr
FunDec & functor declarations\cr
FunBind & functor bindings\cr\adhocdeletion{\thenofuncspec}{3mm}{FunSigExp & functor signature expressions\cr
FunSpec & functor specifications\cr
FunDesc & functor descriptions\cr}TopDec  & top-level declarations\cr
}
\makeatother
\caption{Modules Phrase Classes\index{15.1}}
\label{mod-phr}
\end{figure}
rules for Modules
are the same as for the Core.
The grammatical rules are shown in Figures~\ref{str-syn},
\ref{spec-syn} and \ref{prog-syn}.

\deletion{\thenofuncspec}{It should be noted that functor specifications (FunSpec) cannot
occur in programs;
neither can the associated functor descriptions (FunDesc)
and functor signature expressions (FunSigExp).  The purpose of a $\funspec$
is to specify the static attributes (i.e. functor signature) of one
or more functors. This will be useful, in fact essential, for
separate compilation of functors. If, for example, a functor $g$
refers to another functor $f$ then --- in order to compile $g$ in
the absence of the declaration of $f$ --- at least the specification
of $f$ (i.e. its functor signature) must be available. At present there is no
special grammatical form for  a separately compilable ``chunk'' of text
-- which we may like to call call a {\sl module} -- containing a $\fundec$
together with a $\funspec$ specifying its global references. However, below in
the semantics for Modules it is defined when a
declared functor matches a functor signature specified for it. This determines
exactly those functor environments (containing declared functors
such as $f$) into which the separately compiled ``chunk''
containing the declaration of $g$ may be loaded.}
%\newpage
\begin{figure}[h]
\vspace{4pt}
\makeatletter{}
\tabskip\@centering
\halign to\textwidth
{#\hfil\tabskip1em&\hfil$#$\hfil&#\hfil&#\hfil\tabskip\@centering\cr
\strexp & ::=   & \encstrexp    & \adhocreplacementl{\thenostrsharing}{-4cm}{generative}{basic}\cr
        &       & \longstrid    & structure identifier\cr
\adhocinsertion{\thenostrsharing}{0mm}{ && \transpconstraint & transparent constraint\cr}
\adhocinsertion{\thenostrsharing}{0mm}{ && \opaqueconstraint & opaque constraint\cr}
        &       & \funappstr    & functor application\cr
        &       & \letstrexp    & local declaration\cr
\noalign{\vspace{6pt}}
%
\strdec & ::=   & \dec                          & declaration \cr
        &       & \singstrdec                   & structure \cr
        &       & \localstrdec                  & local \cr
        &       & \emptystrdec                  & empty \cr
        &       & \seqstrdec                    & sequential\cr
\noalign{\vspace{6pt}}
\strbind & ::=   & \adhocreplacementl{\thenostrsharing}{3cm}{\strbindera}{\barestrbindera} \cr
\noalign{\vspace{6pt}}
\sigexp & ::=   & \encsigexp            & \adhocreplacementl{\thenostrsharing}{-4cm}{generative}{basic}\cr
        &       & \sigid                & signature identifier\cr
\adhocinsertion{\thenostrsharing}{0mm}{&&\rlap{$\sigexp$}\hskip13mm \boxml{where type} & type realisation\cr
        &       & \hskip13mm \boxml{$\tyvarseq$ $\longtycon$ =  $\ty$}\cr}
\noalign{\vspace{6pt}}
\sigdec & ::=   & \singsigdec           & \adhocdeletion{\thenostrsharing}{0mm}{single}\cr\noalign{\vspace{6pt}}\adhocdeletion{\thenostrsharing}{0mm}{        &       & \emptysigdec          & empty\cr
        &       & \seqsigdec            & sequential\cr}%\noalign{\vspace{6pt}}
\sigbind & ::=   & \sigbinder \cr
\noalign{\vspace{6pt}}
}
\makeatother
\vspace{6pt}
\caption{Grammar: Structure and Signature Expressions\index{16.1}}
\label{str-syn}
\end{figure}

\subsection{Syntactic Restrictions}
\label{synresmod.sec}
\begin{itemize}
\item No\index{16.2} binding $\strbind$, $\sigbind$, or $\funbind$ may bind the
      same identifier twice.
\item[\textcolor{\fixcolor}{$\bullet$}]
	\FIX{A declaration \dec\ appearing in a \strdec\ may not be a sequential or local declaration.}
\item[\textcolor{\fixcolor}{$\bullet$}]
	\FIX{In a sequential specification ``$\seqspec$,'' $\spec_2$ may not contain a sharing specification.}
\item No description $\valdesc$, $\typdesc$, $\datdesc$, 
      $\exndesc$
      or $\strdesc$ may describe the same identifier
      twice; this applies also to value identifiers within a $\datdesc$.
\item No ${\it tyvarseq}$ may contain the same ${\it tyvar}$ twice.
\item 
	Any $\tyvar$ occurring on the right side of a $\datdesc$ of the form
 	$\tyvarseq \;\tycon\;\boxml{=}$ $\cdots$ must occur
	in the $\tyvarseq$; similarly, in signature expressions of the
	form\hfill\break $\sigexp\ \boxml{where type}\, \tyvarseq\,\longtycon\,$
	$\boxml{=}\,\ty$, any $\tyvar$ occurring in $\ty$ must occur in $\tyvarseq$.
\item	No $\datdesc$, $\valdesc$ 
	or $\exndesc$ may describe 
	$\TRUE$, $\FALSE$, $\NIL$, \boxml{::} or $\REF$.
	No $\datdesc$ or $\exndesc$ may describe {\tt it}.
\item[\textcolor{\fixcolor}{$\bullet$}]
	\FIX{No $\topdec$ may contain, as an initial segment, a $\strdec$ followed by a semicolon.
	Furthermore, the \strdec\ may not be a sequential declaration ``\seqdec.''}
\end{itemize}
%\clearpage %containing figure 'Grammar: Specifications'
\begin{figure}[t]
\vspace{4pt}
\makeatletter{}
\tabskip\@centering
\halign to\textwidth
{#\hfil\tabskip1em&\hfil$#$\hfil&#\hfil&#\hfil\tabskip\@centering\cr
\spec	& ::=	& \valspec		& value\cr
	&	& \typespec		& type\cr
	&	& \eqtypespec		& eqtype\cr 
	&	& \datatypespec		& datatype\cr
\adhocinsertion{\thedatatyperepl}{2cm}{&	& \datatypereplspec		& replication\cr}
	&       & \exceptionspec        & exception\cr
        &	& \structurespec	& structure\cr\adhocdeletion{\thetypabbr}{15mm}{&	& \sharingspec	        & sharing
	&	& \localspec    	& local\cr
        &       & \openspec             & open ($n\geq 1$)\cr }\adhocreplacementtexl{\thesingleincludespec}{0mm}{        &       & \inclspec           
  & include ($n\geq 1$)\cr}{        &       & \singleinclspec             & include \cr} 
        &       & \emptyspec            & empty \cr
        &       & \seqspec              & sequential \FIX{(L)}\cr
\adhocinsertion{\thetypabbr}{0mm}{&     & \boxml{$\spec$ sharing type}      &  sharing\cr &&\boxml{\qquad$\longtycon_1$ = $\cdots$ = $\longtycon_n$}& $(n\geq 2)$\cr}
\noalign{\vspace{6pt}}
\valdesc & ::=   & \adhocreplacementl{\theidstatus}{3cm}{\valdescription}{\valviddescription} \cr
\noalign{\vspace{6pt}}
\typdesc & ::=   & \typdescription \cr
\noalign{\vspace{6pt}}
\datdesc & ::=   & \datdescription \cr
\noalign{\vspace{6pt}}
\condesc & ::=   & \adhocreplacementl{\theidstatus}{3cm}{\condescription}{\conviddescription} \cr
\noalign{\vspace{6pt}}
\exndesc & ::=   & \adhocreplacementl{\theidstatus}{3cm}{\exndescription}{\exnviddescription} \cr
\noalign{\vspace{6pt}}
\strdesc & ::=   & \strdescription \cr
\noalign{\vspace{6pt}}
\adhocdeletion{\thenostrsharing}{0mm}{\shareq & ::=   & \strshareq            & structure sharing\cr
        &       &                       & \qquad ($n\geq 2$) \cr
        &       & \typshareq            & type sharing \cr
        &       &                       & \qquad ($n\geq 2$) \cr
        &       & \multshareq           & multiple\cr\noalign{\vspace{6pt}}
\noalign{\vspace{6pt}}}
\cr}
\makeatother
\vspace{3pt}
\caption{Grammar: Specifications\index{17}}
\label{spec-syn}
\end{figure}
%\clearpage %starting with Figure 'Grammar: Functors and Top-level declarations'

\note{\thenofuncspec}{Figure 8: Functor signature expressions and
functor specifications have been removed. Sequential and empty
functor declarations and signature declarations have been removed.
The grammar for $\topdec$ now allows sequencing (without semicolon) 
of structure-level declarations, signature declarations and functor
declarations.}
%
%   1990 Definition: 
%
%\begin{figure}[h]
%\vspace{4pt}
%\makeatletter{}
%\tabskip\@centering
%\halign to\textwidth
%{#\hfil\tabskip1em&\hfil$#$\hfil&#\hfil&#\hfil\tabskip\@centering\cr
%\fundec & ::=  & \singfundec           & single\cr
%        &      & \emptyfundec          & empty\cr
%        &      & \seqfundec            & sequence\cr\cr
%\noalign{\vspace{6pt}}
%\funbind & ::= & \funstrbinder        & functor binding \cr
%         &     & \qquad\qquad\qquad\qquad\optfunbinda \cr}
%\vspace*{6pt}
%\halign to\textwidth
%{#\hfil\tabskip1em&\hfil$#$\hfil&#\hfil&#\hfil\tabskip\@centering\cr
%\funsigexp & ::= & \longfunsigexpa       & functor signature expression\cr
%\noalign{\vspace{6pt}}
%\funspec & ::= & \singfunspec          & functor specification\cr
%        &     & \emptyfunspec           & empty\cr
%        &     & \seqfunspec             & sequence\cr
%\noalign{\vspace{6pt}}
%\fundesc & ::= & \longfundesc\cr
%\noalign{\vspace{6pt}}
%\topdec  & ::= & \strdec               & structure-level declaration\cr
%         &     & \sigdec               & signature declaration\cr
%         &     & \fundec               & functor declaration\cr}
%\noalign{\vspace{12pt}\hskip1cm\parbox{14cm}{{\sl Note:}\/  No $\topdec$ may
%contain, as an initial segment, a shorter top-level declaration followed by a semicolon.}
%}
%}
%\makeatother
%\vspace{6pt}
%\caption{Grammar: Functors and Top-level Declarations\index{18.1}}
%\label{prog-syn}
%\end{figure}


\begin{figure}[h]
\vspace{4pt}
\makeatletter{}
\tabskip\@centering
\halign to\textwidth
{#\hfil\tabskip1em&\hfil$#$\hfil&#\hfil&\quad#\hfil\tabskip\@centering\cr
\fundec & ::=  & \singfundec           & \cr
\noalign{\vspace{6pt}}
\funbind & ::= & \barefunstrbinder     & functor binding \cr
         &     & \qquad\optfunbinda \cr
\noalign{\vspace*{6pt}}
\topdec  & ::= & \strdecintopdec               & structure-level declaration\cr
         &     & \sigdecintopdec               & signature declaration\cr
         &     & \fundecintopdec               & functor declaration\cr
\noalign{\vspace{12pt}\hskip1cm\parbox{14cm}{\CUT{{\sl Restriction:}\/  No $\topdec$ may
contain, as an initial segment, a $\strdec$ followed by a semicolon.}}
}
}
\makeatother
\vspace{6pt}
\caption{Grammar: Functors and Top-level Declarations\index{18.1}}
\label{prog-syn}
\end{figure}


\deletion{\thenoclosurerestriction}{
\subsection{Closure Restrictions}
\label{closure-restr-sec}
The\index{18.2} semantics presented in later sections requires no restriction on
reference to non-local identifiers. For example, it allows a signature 
expression to refer to external signature identifiers and
(via ~$\SHARING$~ or ~$\OPEN$~) to external structure identifiers; it also
allows a functor to refer to external identifiers of any kind.

However, implementers who want to provide a simple facility for
separate compilation may want to impose the following restrictions
(ignoring references to identifiers bound in the initial basis
$\B_0$, which may occur anywhere):

%However, in the present version of the language,
%apart from references to identifiers bound in the initial basis $B_0$
%(which may occur anywhere), it is required that signatures only refer
%non-locally to signature identifiers and that functors only
%refer non-locally to functor and signature identifiers.
%These restrictions ease separate
%compilation; however, they may be relaxed in a future version of the language.
%
%More precisely, the restrictions are as follows (ignoring reference to
%identifiers bound in $B_0$):
\begin{enumerate}
\item In any signature binding ~$\sigid\ \mbox{{\tt =}}\ \sigexp$~,
the only non-local
references in $\sigexp$ are to signature identifiers.
\item In any functor description ~$\funid\ \longfunsigexpa$~,
the only non-local
references in $\sigexp$ and $\sigexp'$ are to signature identifiers,
except that $\sigexp'$ may refer to $\strid$ and its components.
\item In any functor binding ~$\funstrbinder$~, the only non-local
references in $\sigexp$, $\sigexp'$ and $\strexp$ are to functor and signature
identifiers,
except that both $\sigexp'$ and $\strexp$ may refer to $\strid$ and
its components.
\end{enumerate}
In the last two cases the final qualification allows, for example, sharing
constraints to be specified between functor argument and result.
(For a completely precise definition of these closure restrictions,
see the comments to rules \ref{single-sigdec-rule} 
(page~\pageref{single-sigdec-rule}), 
\ref{singfunspec-rule} (page~\pageref{singfunspec-rule})
and \ref{singfundec-rule} (page~\pageref{singfundec-rule})
in the static semantics of modules, Section~\ref{statmod-sec}.)

The\index{19.1} 
significance of these restrictions is that they may ease separate
compilation; this may be seen as follows. If one takes a {\sl module}
to be a sequence of signature declarations, functor specifications
and functor declarations satisfying the above restrictions then the
elaboration of a module can be made to depend on the initial
static basis alone (in particular, it will not rely on
structures outside the module). Moreover, the elaboration 
of a module cannot create new free structure or type names, so 
name consistency (as defined in Section~\ref{consistency-sec}, 
page \pageref{consistency-sec}) is automatically preserved
across separately compiled modules. On the other hand,
imposing these restrictions may force the programmer to write
many more sharing equations than is needed if functors
and signature expressions can refer to free structures.
}



\clearpage{}
\thispagestyle{empty}
%!TEX root = root.tex
%

\section{Static Semantics for the Core}
\label{statcor.sec}
Our\index{20.1} first task in presenting the semantics -- whether for Core or Modules,
static or dynamic -- is to define the objects concerned. In addition
to the class of {\em syntactic} objects, which we have already defined, 
there are classes of so-called {\em semantic} objects used to describe
the meaning of the syntactic objects. Some classes contain {\em simple}
semantic objects; such objects are usually identifiers or names of some
kind. Other classes contain {\em compound} semantic objects, such as
types or environments, which are constructed from component objects.

\subsection{Simple Objects}
%\ replacement{\thenostrsharing}{All semantic objects in the static semantics of the entire 
%language are built from identifiers and two further kinds of simple objects: 
%type constructor names and structure names.}{All semantic objects in 
%the static semantics of the entire 
%language are built from identifiers and one further kind of simple objects: 
%type constructor names.}
\replacement{\theidstatus}{All semantic objects in the static semantics of the entire 
language are built from identifiers and two further kinds of simple objects: 
type constructor names and structure names.}{All semantic objects in 
the static semantics of the entire 
language are built from identifiers and two further kinds of simple objects: 
type constructor names and identifier status descriptors.}
Type constructor names are the values taken by type constructors; we shall
usually refer to them briefly as type names, but they are to be clearly
distinguished from type variables and type constructors. 
\deletion{\thenostrsharing}{Structure names play an active role only in
the Modules semantics; they enter the Core semantics only because
they appear in structure environments, which (in turn) are needed in the Core
semantics only to determine the values of long identifiers.} The simple object
classes, and the variables ranging over them, are shown in
Figure~\ref{simple-objects}. We have included $\TyVar$ in the table to
make visible the use of $\alpha$ in the semantics to range over $\TyVar$.\index{20.2}

%\vspace{-7mm}
%\vspace{-8mm}
\begin{figure}[h]
\vspace{2pt}
% \ adhocreplacementl{\thenostrsharing}{1cm}{
% \begin{displaymath}
% \begin{array}{rclr}
% \alpha\ {\rm or}\ \tyvar & \in   & \TyVar       & \mbox{type variables}\\
% \t               & \in   & \TyNames     & \mbox{type names}\\ 
% \m              & \in   & \StrNames     & \mbox{structure names}
% \end{array}
% \end{displaymath}}{\begin{displaymath}
% \begin{array}{rclr}
% \alpha\ {\rm or}\ \tyvar & \in   & \TyVar       & \mbox{type variables}\\
% \t               & \in   & \TyNames     & \mbox{type names}
% \end{array}
% \end{displaymath}}
\adhocreplacementl{\theidstatus}{1cm}{
\begin{displaymath}
\begin{array}{rclr}
\alpha\ {\rm or}\ \tyvar & \in   & \TyVar       & \mbox{type variables}\\
\t               & \in   & \TyNames     & \mbox{type names}\\ 
\m              & \in   & \StrNames     & \mbox{structure names}
\end{array}
\end{displaymath}}{\begin{displaymath}
\begin{array}{rcll}
\alpha\ {\rm or}\ \tyvar & \in   & \TyVar       & \mbox{type variables}\\
\t               & \in   & \TyNames     & \mbox{type names}\\
\is              & \in   & \IdStatus = \{\isc,\ise,\isv\}    & \mbox{identifier status descriptors}
\end{array}
\end{displaymath}}
\caption{Simple Semantic Objects}
\label{simple-objects}
%\vspace{3pt}
\end{figure}

Each\index{20.3} $\alpha \in\TyVar$ possesses a boolean {\sl equality} attribute,
which determines whether or not it {\sl admits equality}, i.e. whether
it is a member of EtyVar (defined on page~\pageref{etyvar-lab}).
%-- in which case we
%also say that it is an {\sl equality} type variable. 
%poly 

Each $\t\in\TyNames$ has
an arity $k\geq 0$, and also possesses an equality attribute.
We denote the class of type names with arity $k$ by $\TyNamesk$.

With\index{20.35} each special constant {\scon} we associate a type
name $\scontype(\scon)$ which is either \replacement{\thescon}{{\INT}, {\REAL} 
 or {\STRING}}{{\INT}, {\REAL}, {\WORD}, {\CHAR}
 or {\STRING}}
as indicated by Section~\ref{cr:speccon}.
\insertion{\thelibrary}{(However, see Appendix~\ref{overload.sec} 
concerning types of overloaded special constants.)}

\subsection{Compound Objects}
When\index{20.4} $A$ and $B$ are sets $\Fin A$ denotes the set of finite subsets of $A$,
and $\finfun{A}{B}$ denotes the set of {\sl finite maps} (partial functions
with finite domain) from $A$ to $B$.
The domain\index{21.1} and range of a finite map, $f$, are denoted $\Dom f$ and
$\Ran f$.
A finite map will often be written explicitly in the form $\kmap{a}{b},
\ k\geq 0$;
in particular the empty map is $\emptymap$.
We shall use the form $\{x\mapsto e\  ;\  \phi\}$ -- a form of set
comprehension -- to stand for the finite map $f$ whose domain
is the set of values $x$ which satisfy the condition $\phi$, and
whose value on this domain is given by $f(x)=e$.

When $f$ and $g$ are finite maps the map $\plusmap{f}{g}$, called
$f$ {\sl modified} by $g$, is the finite map with domain
$\Dom f \cup \Dom g$ and values
\[(\plusmap{f}{g})(a) = \mbox{if $a\in\Dom g$ then $g(a)$ else $f(a)$.}
\]%
\ADD{The restriction of a map $f$ by a set $S$, written $\restrict{f}{S}$ is defined
to be}
\[
  \ADD{\restrict{f}{S} = \{x \mapsto f(s); x\in\restrict{\Dom f}{S}\}}
\]%

The compound objects for the static semantics of the Core Language are
shown in Figure~\ref{compound-objects}.
We take $\cup$ to mean disjoint union over
semantic object classes. We also understand all the defined object
classes to be disjoint.

\begin{figure}[h]
%\vspace{2pt}
\begin{displaymath}
\begin{array}{rcl}
        \tau    &\in    &\Type = \TyVar\cup\adhocreplacementl{\theidstatus}{6cm}{\RecType}{\RowType}\cup\FunType
                                 \cup\ConsType\\
 \longtauk\ {\rm or}\ \tauk
                & \in   & \Type^k\\
 \longalphak\ {\rm or}\ \alphak
                & \in   & \TyVar^k\\
 \varrho        & \in   & \adhocreplacementl{\theidstatus}{4cm}{\RecType}{\RowType} = \finfun{\Lab}{\Type} \\
 \tau\rightarrow\tau'
                & \in   & \FunType = \Type\times\Type \\
                &       & \ConsType = \cup_{k\geq 0}\ConsType^{(k)}\\
        \tauk\t & \in   & \ConsType^{(k)} = \Type^k\times\TyNamesk  \\
\typefcn\ {\rm or}\ \typefcnk
                & \in   & \TypeFcn = \cup_{k\geq 0}\TyVar^k\times\Type\\
\tych\ {\rm or}\ \longtych
                & \in   & \TypeScheme = \cup_{k\geq 0}\TyVar^k\times\Type\\
\adhocdeletion{\thenostrsharing}{4cm}{\S\ {\rm or}\ (\m,\E)
                & \in   & \Str = \StrNames\times\Env  \\ }
(\theta,\adhocreplacementl{\thece}{2cm}{\CE}{\VE})    & \in   & \adhocreplacementl{\thece}{-8cm}{\TyStr = \TypeFcn\times\ConEnv}{\TyStr = \TypeFcn\times\ValEnv}\\
\SE             & \in   & \adhocreplacementl{\thenostrsharing}{3cm}{\StrEnv = \finfun{\StrId}{\Str}}{\StrEnv = \finfun{\StrId}{\Env}}\\
\TE             & \in   & \TyEnv = \finfun{\TyCon}{\TyStr}\\
\adhocdeletion{\thece}{4cm}{\CE             & \in   & \ConEnv = \finfun{\Con}{\TypeScheme}\\ }\VE             & \in   & \adhocreplacementl{\theidstatus}{3cm}{\VarEnv = \finfun{(\Var\cup\Con\cup\Exn)}{\TypeScheme}}{\ValEnv = \finfun{\VId}{\TypeScheme\times\IdStatus}}\\
\adhocdeletion{\theidstatus}{5mm}{\EE             & \in   & \ExnEnv = \finfun{\Exn}{\Type}\\ }\E\ {\rm or}\ \adhocreplacementl{\theidstatus}{3cm}{\longE{}}{\newlongE{}}
                & \in   & \adhocreplacementl{\theidstatus}{-9cm}{\Env = \StrEnv\times\TyEnv\times\VarEnv\times\ExnEnv}{\Env = \StrEnv\times\TyEnv\times\ValEnv}\\
\T              & \in   & \TyNameSets = \Fin(\TyNames)\\
\U              & \in   & \TyVarSet = \Fin(\TyVar)\\
\C\ {\rm or}\ \T,\U,\E   & \in   & \Context = \TyNameSets\times\TyVarSet\times\Env
\end{array}
\end{displaymath}
\caption{Compound Semantic Objects\index{21.2}}
\label{compound-objects}
%\vspace{3pt}
\end{figure}


Note that $\Lambda$\index{21.3} and $\forall$ bind type variables.  For any semantic object
$A$, $\TyNamesFcn A$ and $\TyVarsFcn A$ denote respectively the set of
type names and the set of type variables occurring free in $A$.
\deletion{\thenoimptypes}{Moreover, $\imptyvars A$ and $\apptyvars A$ denote respectively the set
of imperative type variables and the set of applicative
type variables occurring free in $A$.}\index{21.4}
\insertion{\theidstatus}{\par Also note that a value environment maps
value identifiers to a pair of a type scheme and an identifier status.
If $\VE(\vid) = (\sigma,\is)$, we say that $\vid$ {\sl has status $\is$
in $\VE$}. An occurrence of a value identifier which is elaborated
in $\VE$ is referred to as a {\sl value variable}, a {\sl value constructor}
or an {\sl exception constructor}, depending on whether its status in $\VE$
is $\isv$, $\isc$ or $\ise$, respectively. }

\subsection{Projection, Injection and Modification}
\label{stat-proj}\index{22.1}
{\bf Projection}: We often need to select components of tuples -- for example,
the \replacement{\theidstatus}{variable-environment}{value-environment} component of a context. In such cases we
rely on \replacement{\theidstatus}{variable}{metavariable} names to indicate which component
is selected. For instance ``$\of{\VE}{\E}$'' means ``the \replacement{\theidstatus}{variable-environment}{value-environment}
component
of $\E$''\deletion{\thenostrsharing}{ and ``$\of{\m}{\S}$'' means ``the structure name of $\S$''}.

Moreover, when a tuple contains a finite map we shall ``apply'' the
tuple to an argument, relying on the syntactic class of the argument to
determine the relevant function. \replacement{\theidstatus}{For instance $\C(\tycon)$ means
$(\of{\TE}{\C})\tycon$.

A particular case needs mention:  $\C(\con)$ is taken to stand for
$(\of{\VE}{\C})\con$; similarly, $\C(\exn)$ is taken to stand for
$(\of{\VE}{\C})\exn$.
  The type scheme of a value constructor is
held in $\VE$ as well as in $\TE$ (where it will be recorded within
a $\CE$); similarly, the type of an exception constructor is held in
$\VE$ as well as in $\EE$.
Thus the re-binding of a constructor of either kind is given proper
effect by accessing it in $\VE$, rather than in $\TE$ or in $\EE$.}{For 
instance $\C(\tycon)$ means
$(\of{\TE}{\C})\tycon$ and $\C(\vid)$ means $(\of{\VE}{(\of{E}{\C})})(\vid)$.}

Finally, environments may be applied to long identifiers.
\replacement{\theidstatus}{For instance if $\longcon = \strid_1.\cdots.\strid_k.\con$ then
$\E(\longcon)$ means
\[ (\of{\VE}
       {(\of{\SE}
            {\cdots(\of{\SE}
                       {(\of{\SE}{\E})\strid_1}
                   )\strid_2\cdots}
        )\strid_k}
    )\con.
\]
}{For instance if $\longvid = \strid_1.\cdots.\strid_k.\vid$ then
$\E(\longvid)$ means
\[ (\of{\VE}
       {(\of{\SE}
            {\cdots(\of{\SE}
                       {(\of{\SE}{\E})\strid_1}
                   )\strid_2\cdots}
        )\strid_k}
    )\vid.
\]
}

{\bf Injection}: Components may be injected into tuple classes; for example,\linebreak
``$\VE\ \In\ \Env$'' means the environment
\replacement{\theidstatus}{$(\emptymap,\emptymap,\VE,\emptymap)$.}{$(\emptymap,\emptymap,\VE)$.}

{\bf Modification}: The modification of one map $f$ by another map $g$,
written $f+g$, has already been mentioned.  It is commonly used for
environment modification, for example $\E+\E'$.  Often, empty components
will be left implicit in a modification; for example $\E+\VE$ means
\replacement{\theidstatus}{$\E+(\emptymap,\emptymap,\VE,\emptymap)$.}{$\E+(\emptymap,\emptymap,\VE)$.}  For set components, modification
means union, so that $\C+(\T,\VE)$ means
\[ (\ (\of{\T}{\C})\cup\T,\ \of{\U}{\C},\ (\of{\E}{\C})+\VE\ ) \]
Finally, we frequently need to modify a context $\C$ by an environment $\E$
(or a type environment $\TE$ say),
at the same time extending $\of{\T}{\C}$ to include the type names of $\E$
(or of $\TE$ say).
We therefore define $\C\oplus\TE$,\index{22.2} for example, to mean
$\C+(\TyNamesFcn\TE,\TE)$.
%\vspace*{12pt}

\subsection{Types and Type functions}
\label{tyfun-sec}
A type $\tau$ is an {\sl equality type},\index{22.3} or {\sl admits equality}, if it is
of one of the forms
\begin{itemize}
\item $\alpha$, where $\alpha$ admits equality;
\item $\{\lab_1\mapsto\tau_1,\ \cdots,\ \lab_n\mapsto\tau_n\}$,
      where each $\tau_i$ admits equality;
\item $\tauk\t$, where $t$ and all members of $\tauk$ admit equality;
\item $(\tau')\REF$\ \ADD{or $(tau')\ARRAY$}.\index{23.1}
\end{itemize}
\label{tyfcn-lab}
A type function $\theta=\Lambda\alphak.\tau$\index{23.2}
 has arity $k$; \deletion{\theidstatus}{it must be
{\sl closed} -- i.e.
$\TyVarFcn(\tau)\subseteq\alphak$ -- and} the bound variables must
be distinct. Two type functions are considered equal
if they only differ in their choice of bound variables (alpha-conversion).
In particular, the equality attribute has no significance in a 
bound variable of a type function; for example, $\Lambda\alpha.\alpha\to
\alpha$ and $\Lambda\beta.\beta\to\beta$ are equal type functions
even if $\alpha$ admits equality but $\beta$ does not.
%poly 
\deletion{\thenoimptypes}{Similarly, the imperative attribute has no significance 
in the bound variable of a type function.}
If $t$ has arity $k$, then we write $t$ to mean $\Lambda\alphak.\alphak\t$
(eta-conversion); thus $\TyNames\subseteq\TypeFcn$. $\theta=\Lambda\alphak.\tau$
is an {\sl equality} type function, or {\sl admits equality}, if when the
type variables $\alphak$ are chosen to admit equality then $\tau$ also admits
equality.

We write the application of a type function $\theta$ to a vector
$\tauk$ of types as $\tauk\theta$.
If $\theta=\Lambda\alphak.\tau$ we set $\tauk\theta=\tau\{\tauk/\alphak\}$
(beta-conversion). 

We write $\tau\{\thetak/\tk\}$ for the result of substituting type
functions $\thetak$ for type names $\tk$ in $\tau$.
We assume that all beta-conversions
are carried out after substitution, so that for example
\[(\tauk\t)\{\Lambda\alphak.\tau/\t\}=\tau\{\tauk/\alphak\}.\]
%poly 
\label{imp-ty-lab}
\deletion{\thenoimptypes}{A type is {\sl imperative} if all type variables occurring in it are
imperative.}
\subsection{Type Schemes}
\label{type-scheme-sec}
A type scheme $\tych=\forall\alphak.\tau$\index{23.3}
 {\sl generalises} a type $\tau'$,
written $\tych \succ\tau'$,
\replacement{\thenoimptypes}{if $\tau'=\tau\{\tauk/\alphak\}$ for some $\tauk$, where each member $\tau_i$
of $\tauk$ admits equality if $\alpha_i$ does,  
%poly 
and $\tau_i$ is imperative if $\alpha_i$ is imperative.}{if $\tau'=\tau\{\tauk/\alphak\}$ for some $\tauk$, where each member $\tau_i$
of $\tauk$ admits equality if $\alpha_i$ does.}
If $\tych'=\forall\beta^{(l)}.\tau'$ then $\tych$ {\sl generalises} $\tych'$,
written $\tych\succ\tych'$, if $\tych\succ\tau'$ and $\beta^{(l)}$ contains
no free type variable of $\tych$.
It can be shown that $\tych\succ\tych'$ iff, for all $\tau''$, whenever
$\tych'\succ\tau''$ then also $\tych\succ\tau''$.

Two type schemes $\tych$ and $\tych'$ are considered equal
if they can be obtained from each other by
renaming and reordering of bound type variables, and deleting type
variables from the prefix which do not occur in the body.
Here, in contrast to the case for type functions, the equality attribute
must be preserved in renaming; for example $\forall\alpha.\alpha\to\alpha$
and $\forall\beta.\beta\to\beta$ are only equal if either both $\alpha$
and $\beta$ admit equality, or neither does.
%poly 
\deletion{\thenoimptypes}{Similarly, the imperative attribute of a bound type variable of a
type scheme {\sl is} significant.}
It can be shown that $\tych=\tych'$ iff $\tych\succ\tych'$ and
$\tych'\succ\tych$.

We consider a type $\tau$ to be a type scheme, identifying it with
$\forall().\tau$.
\oldpagebreak

\subsection{Scope of Explicit Type Variables}
\label{scope-sec}

In\index{23.10} the Core language, a type or datatype binding can 
explicitly introduce type variables whose scope is that binding.
\insertion{\theexplicittyvars}{Moreover, in a value declaration
{\tt val $\tyvarseq$ $\valbind$}, the sequence $\tyvarseq$ binds
type variables: a type variable occurs free in 
{\tt val $\tyvarseq$ $\valbind$} iff it occurs free in $\valbind$
and is not in the sequence $\tyvarseq$.}
\deletion{\theexplicittyvars}{
In the modules, a description of a value, type, or datatype
may contain explicit type variables whose scope is that
description.} However, \insertion{\theexplicittyvars}{explicit binding of type
variables at {\tt val} is optional, so} we\index{23.11} still have to account for the
scope of an explicit type variable occurring in the ``\ml{:}~$\ty$'' 
of a typed expression or pattern 
or in the ``\ml{of} $\ty$'' of an exception binding. For the rest
of this section, we consider such \insertion{\theexplicittyvars}{free} occurrences of type variables only.

Every occurrence of a value declaration is said to
{\sl scope} a set of explicit type variables determined as follows.



%Every explicit type variable $\alpha$ is {\sl scoped at} a value binding
%which is determined as follows.

First, a free occurrence of $\alpha$ in a value declaration 
$\explicitvaldec$ is said
to be {\sl unguarded} if the occurrence is not part of a smaller value
declaration within $\valbind$.
In this case we say that $\alpha$ {\sl occurs unguarded} in the 
value declaration.

\replacement{\theexplicittyvars}{Then we say that $\alpha$ is {\sl scoped at} 
a particular occurrence
$O$ of $\valdec$ in a program if}{Then we say that $\alpha$ is {\sl implicitly scoped at} a particular value declaration
{\tt val $\tyvarseq$ $\valbind$} in a program if} 
(1) $\alpha$ occurs unguarded in this value declaration, and 
(2) $\alpha$ does not occur unguarded in any larger value declaration
containing the \replacement{\theexplicittyvars}{occurrence $O$.}{given one.}\label{scope-def-lab}

\deletion{\theexplicittyvars}{
Hence, associated with every occurrence of a value declaration there is 
a set $\U$ of the explicit type variables that are 
scoped at that
occurrence. One may think of each occurrence of $\VAL$ as being implicitly
decorated with such a  set, for instance:

\vspace*{3mm}
\halign{\indent$#$&$#$&$#$\cr
\mbox{$\VAL_{\{\}}$ \ml{x = }}&\mbox{\ml{(}}&
\mbox{\ml{let $\VAL_{\{\mbox{\ml{'a}}\}}$ Id1:'a->'a = fn z=>z in Id1 Id1 end,}}\cr
& &\mbox{\ml{let $\VAL_{\{\mbox{\ml{'a}}\}}$ Id2:'a->'a = fn z=>z in Id2 Id2 end)}}\cr
\noalign{\vspace*{3mm}}
\mbox{$\VAL_{\{\mbox{\ml{'a}}\}}$ \ml{x = }}&\mbox{\ml{(}}&
\mbox{\ml{let $\VAL_{\{\}}$ Id:'a->'a = fn z=>z in Id Id end,}}\cr
& &\mbox{\ml{fn z=> z:'a)}}\cr}
}
\insertion{\thenoimptypes}{Henceforth, we assume that for every
value declaration $\boxml{val}\,\tyvarseq\cdots$ occurring in the
program, every explicit type variable implicitly scoped at the {\tt val}
has been added to $\tyvarseq$ (subject to the syntactic constraint in Section~\ref{synres.sec}). Thus for example, in the two declarations
\begin{tabbing}
\indent\=\tt  val x =  let val id:'a->'a = fn z=>z in id id end\\
       \>\tt  val x = (let val id:'a->'a = fn z=>z in id id end; fn z=>z:'a)
\end{tabbing}
the type variable \boxml{'a} is scoped differently; they become respectively
\begin{tabbing}
\indent\=\tt val x =  let val 'a id:'a->'a = fn z=>z in id id end\\
       \>\tt val 'a x = (let val id:'a->'a = fn z=>z in id id end; fn z=>z:'a)
\end{tabbing}
}

Then, according to the 
inference rules in Section~\ref{stat-cor-inf-rules}
the first example can be elaborated, but the second cannot since \ml{'a}
is bound at the outer value declaration leaving no possibility of two 
different instantiations of the type of \ml{id} in the application
\ml{id id}.


\subsection{Non-expansive Expressions}
\label{expansive-sec}
In\index{23.4} order to treat polymorphic references and exceptions,
the set Exp of expressions is partitioned into two classes, the {\sl
expansive} and the {\sl non-expansive} expressions. 
An expression
     is {\sl non-expansive in context $\C$} if, after replacing infixed forms 
     by their equivalent prefixed forms, and derived forms by their equivalent
     forms, it can be generated  by the following grammar from the 
     non-terminal $\nexp$:
\medskip

\begin{displaymath}
  \begin{array}{rcl@{\hskip18mm}rcl}
    \nexp & ::= & \scon & \nexprow & ::= & \boxml{$\lab$ = $\nexp\langle$, $\nexprow\rangle$} \\
    & & \langle\OP\rangle\longvid & & & \ADD{\boxml{...}\ \boxml{=}\ \nexp} \\
    & & \ttlbrace\langle\nexprow\rangle\ttrbrace \\
    & &\boxml{($\nexp$)} & \conexp & ::= & \boxml{($\conexp\langle$:$\ty\rangle$)} \\
    & &\boxml{$\conexp\;\nexp$} & & & \hbox{$\langle\OP\rangle\longvid$} \\
    & &\nexp \boxml{:} \ty \\
    & &\boxml{fn $\match$}
  \end{array}%
\end{displaymath}%

%\halign{&\indent\hfil$#$\ &\ $#$\hfil\ &\ $#$\hfil\cr
%\nexp&::=&\scon & \hskip20mm\nexprow&::=&\boxml{$\lab$ = $\nexp\langle$, $\nexprow\rangle$}\cr
%&&\langle\OP\rangle\longvid\cr
%&&\ttlbrace\langle\nexprow\rangle\ttrbrace&\conexp&::=&\boxml{($\conexp\langle$:$\ty\rangle$)}\cr
%&&\boxml{($\nexp$)}&&&\hbox{$\langle\OP\rangle\longvid$}\cr
%&&\boxml{$\conexp\;\nexp$}\cr
%&&\nexp \boxml{:} \ty\cr
%%&&\boxml{$\nexp$ handle $\match$}\cr
%&&\boxml{fn $\match$}\cr\noalign{\vskip6pt}}
%\medskip

\hangindent=\parindent\hangafter=0\noindent
{\sl Restriction:}\/ Within a $\conexp$, we require $\longvid\neq\REF$ and
$\of{\is\,}{\,\C(\longvid)}\in\{\isc,\ise\}$.\medskip

\noindent
All other expressions are said to be {\sl expansive (in $C$)}.
The idea is that the dynamic evaluation of a
non-expansive expression will neither generate an exception nor extend
the domain of the memory, while the evaluation of an expansive
expression might.

\oldpagebreak
\subsection{Closure}
\label{closure-sec}
Let\index{24.2} $\tau$ be a type and $A$ a semantic object. Then $\cl{A}{(\tau)}$,
the {\sl closure} of $\tau$ with respect to $A$, is the type scheme
$\forall\alphak.\tau$, where $\alphak=\TyVarFcn(\tau)\setminus\TyVarFcn A$.
Commonly, $A$ will be a context $\C$.
We abbreviate the {\sl total} closure $\cl{\emptymap}{(\tau)}$ to
$\cl{}{(\tau)}$.
If the range of a \replacement{\theidstatus}{variable environment}{value 
environment} $\VE$ contains only types (rather than
arbitrary type schemes) we set
\[\cl{A}{\VE}=\{\vid\mapsto(\cl{A}{(\tau)},\is)\ ;\ \VE(\vid)=(\tau,\is)\}\]%

\label{clos-def-lab}
Closing\index{24.3} a \replacement{\theidstatus}{variable environment}{value environment} $\VE$ that stems from
the elaboration of a value binding $\valbind$ requires extra
care to ensure type security of references and exceptions and correct
scoping of explicit type variables.
Recall that $\valbind$ is not allowed to bind the
same variable twice. \replacement{\theidstatus}{Thus, for each $\var\in\Dom\VE$ 
there is a unique \mbox{\pat\ \ml{=} \exp}
in $\valbind$ which binds $\var$.}{Thus, for each $\vid\in\Dom\VE$ 
there is a unique \mbox{\pat\ \ml{=} \exp}
in $\valbind$ which binds $\vid$.}
If $\VE(\vid)=(\tau,\is)$, let 
$\cl{\C,\valbind}{\VE(\vid)}=(\longtych,\is)$, 
where
\[\alphak=\cases{\TyVarFcn\tau\setminus\TyVarFcn\C,
	& if \REPL{$\pat$ is exhaustive and }{is} \cr & $\exp$ is non-expansive in $\C$;\cr
                 (), & \REPL{otherwise }{if $\exp$ is expansive in $\C$}.}
\]
\ADD{Where a pattern is said to be {\sl exhaustive} if it matches all possible values of its
type (cf.\ Section~\ref{further-restrictions-sec}).
Since whether a nested match matches a value is undecidable in general, we classify any
pattern involving a nested match as non-exhaustive.}

\subsection{Type Structures and Type Environments}
\label{typeenv-wf-sec}
A type\index{24.4} structure 
\replacement{\thece}{$(\theta,\CE)$}{$(\theta,\VE)$}\ 
is {\sl well-formed} if either
\replacement{\thece}{$\CE=\emptymap$}{$\VE=\emptymap$}, or $\theta$ is a type name $t$.
(The latter case arises, with \replacement{\thece}{$\CE\neq\emptymap$}{$\VE\neq\emptymap$}, in $\DATATYPE$
declarations.)
\insertion{\thenostrsharing}{An object or assembly $A$ of semantic objects is {\sl well-formed} if every type structure
occurring in $A$ is well-formed.}
\deletion{\thenostrsharing}{All type structures occurring in elaborations are 
assumed to
be well-formed.}

A type structure \replacement{\thece}{$(\t,\CE)$}{$(\t,\VE)$}\ is said to
{\sl respect equality} if, whenever $\t$ admits equality, then
either $\t=\REF$ \ADD{or $\t=\ARRAY$} (see Appendix~\ref{init-stat-bas-app}) or,
for each \replacement{\theidstatus}{$\CE(\con)$}{$\VE(\vid)$} of the form 
\replacement{\thece}{$\forall\alphak.(\tau\rightarrow\alphak\t)$,}{$(\forall\alphak.(\tau\rightarrow\alphak\t), \is)$,}
the type function $\Lambda\alphak.\tau$ also admits equality.
(This ensures that the equality
predicate ~{\tt =}~ will be applicable to a constructed value 
\replacement{\theidstatus}{$(\con,v)$}{$(\vid,v)$} of
type $\tauk\t$ only when it is applicable to the value $v$ itself,
whose type is $\tau\{\tauk/\alphak\}$.)
A type environment $\TE$ {\sl respects equality} if all its type
structures do so.

Let $\TE$ be a type environment, and let $T$ be the set of type names
$\t$ such that \replacement{\thece}{$(\t,\CE)$ }{$(\t,\VE)$ } occurs in $\TE$ for some
\replacement{\thece}{$\CE\neq\emptymap$}{$\VE\neq\emptymap$}.  
Then $\TE$ is said to {\sl maximise equality}
if (a) $\TE$ respects equality, and also (b) if any larger subset of
$T$ were to admit equality (without any change in the equality
attribute of any type names not in $T$) then $\TE$ would cease to
respect equality.


\CUT{For any $\TE$ of the form}
\[
  \CUT{\TE=\{\tycon_i\mapsto(t_i,\VE_i)\ ;\ 1\leq i\leq k\},}
\]%
\CUT{where no $\VE_i$
is the empty map, and for any $\E$ we define
$\Abs(\TE,\E)$ to\index{25.1} be the environment obtained from 
$\E$ and $\TE$ as
follows.
First, let $\Abs(\TE)$ be the type environment}\linebreak
\CUT{$\{\tycon_i \mapsto (t_i,\emptymap)\ ;\ 1\leq i\leq k\}$
in which all value
environments $\VE_i$
have been replaced by the empty map. 
Let $t_1',\cdots,t_k'$ be new distinct type names none of which
admit equality. Then $\Abs(\TE,\E)$ is the result of simultaneously
substituting
$t_i'$ for $t_i$, $1\leq i\leq k$,  throughout $\Abs(\TE)+\E$.
(The effect of the latter substitution is to ensure that the use of 
equality on an $\ABSTYPE$ is restricted to the $\WITH$ part.)}
\label{abs-lab}
%\clearpage

\subsection{Inference Rules}
\label{stat-cor-inf-rules}
Each rule\index{26.1} of the semantics allows inferences among sentences of the form
\[A\ts{\it phrase}\ra A'\]
where
$A$ is usually \deletion{\theidstatus}{an environment or }a context, {\it phrase} is a phrase of
the Core, and $A'$ is a semantic object -- usually a type or an
environment.  It may be pronounced ``{\it phrase} elaborates to $A'$ in
(context\deletion{\theidstatus}{ or environment}) $A$''.  Some rules have extra hypotheses not of
this form; they are called {\sl side conditions}.  

In the presentation of the rules, phrases within single
angle brackets ~$\langle\ \rangle$~ are called {\sl
first options}, and those within double
angle brackets ~$\langle\langle\ \rangle\rangle$~ are called {\sl
second options}.  To reduce the number of rules, we have adopted the
following convention:
\begin{quote} In each instance of a rule, the
first options must be either all present or all absent;
similarly the second options must be either all present or all absent.
\end{quote}

Although not assumed in our definitions, it is intended that every
context $\C=\T,\U,\E$ has the property that $\TyNamesFcn\E\subseteq\T$.
Thus $\T$ may be thought of, loosely, as containing all type names
which ``have been generated''. It is necessary to include $\T$ as a
separate component in a context, since $\TyNamesFcn\E$ may not contain
all the type names which have been generated; one reason is that a
context $\T,\emptyset,\E$ is a projection of the basis
\replacement{\thenostrsharing}{$\B=(\M,\T),\F,\G,\E$}{$\B=\T,\F,\G,\E$} 
whose other components $\F$ and $\G$
could contain other such names -- recorded in $\T$ but not present in
$\E$.  Of course, remarks about what ``has been generated'' are not
precise in terms of the semantic rules. But the following precise result
may easily be demonstrated:
\begin{quote}
Let S be a sentence
~$\T,\U,\E\ts{\it phrase}\ra A$~ such that
$\TyNamesFcn\E\subseteq\T$,
and let S$'$ be a sentence
~$\T',\U',\E'\ts{\it phrase}'\ra A'$~
occurring in a proof of S; then also
$\TyNamesFcn\E'\subseteq\T'$.
\end{quote}



%                       Atomic Expressions
%
\rulesec{Atomic Expressions\index{26.2}}{\C\vdash\atexp\ra\tau}
%\begin{figure}[h]

\begin{equation}        % special constant
\label{sconexp-rule}
\frac{}
     {\C\ts\scon\ra\scontype(\scon)}\index{26.3}
\end{equation}

\replacement{\theidstatus}{\begin{equation}        % value variable
\label{varexp-rule}
\frac{\C(\longvar)\succ\tau}
     {\C\ts\longvar\ra\tau}
\end{equation}}{\begin{equation}        % value variable
\label{varexp-rule}
\frac{\C(\longvid) = (\sigma,\is)\qquad\sigma\succ\tau}
     {\C\ts\longvid\ra\tau}
\end{equation}}

\deletion{\theidstatus}{\begin{equation}        % value constructor
\label{conexp-rule}
\frac{\C(\longcon)\succ\tau}
     {\C\ts\longcon\ra\tau}
\end{equation}

\begin{equation}      % exception constant
%\label{exconexp-rule}
\frac{\C(\longexn)=\tau}
     {\C\ts\longexn\ra\tau}
\end{equation}}
\oldpagebreak

\begin{equation}        % record expression
%\label{recexp-rule}
\frac{\langle\C\ts\labexps\ra\varrho\rangle}
     {\C\ts\ttlbrace\ \recexp\ \ttrbrace\ra\emptymap\langle +\ \varrho\rangle{\rm\ in\ \Type}}\index{27.0}
\end{equation}
\vskip6pt

\replacement{\thesafelet}{\begin{equation}        % local declaration
\label{let-rule}
\frac{\C\ts\dec\ra\E\qquad\C\oplus\E\ts\exp\ra\tau}
     {\C\ts\letexp\ra\tau}\index{27.1}
\end{equation}}{\begin{equation}        % local declaration
\label{let-rule}
\frac{\C\ts\dec\ra\E\qquad\C\oplus\E\ts\exp\ra\tau\qquad\TyNamesFcn\tau\subseteq\of{\T}{\C}}
     {\C\ts\letexp\ra\tau}\index{27.1}
\end{equation}}

\begin{equation}        % paren expression
%\label{parexp-rule}
\frac{\C\ts\exp\ra\tau}
     {\C\ts\parexp\ra\tau}
\end{equation}
\comments
\begin{description}
\replacement{\theidstatus}{\item{(\ref{varexp-rule}),(\ref{conexp-rule})}
The instantiation of 
type schemes allows different occurrences of a single $\longvar$ 
or $\longcon$ to assume different types.}{\item{(\ref{varexp-rule})}
The instantiation of 
type schemes allows different occurrences of a single $\longvid$ 
to assume different types. Note that the identifier status is not
used in this rule.}
\item{(\ref{let-rule})} 
The use of $\oplus$, here and elsewhere, ensures that
type names generated by the first sub-phrase are different from 
type names generated by the second sub-phrase.\insertion{\thefixtypos}{The side condition
prevents type names generated by $\dec$ from escaping outside the local declaration.}
\end{description}

\rulesec{Expression Rows}{\C\ts\labexps\ra\varrho}
\begin{equation}        % expression rows
%\label{labexps-rule}
\frac{\C\ts\exp\ra\tau\qquad\langle\C\ts\labexps\ra\varrho\qquad\ADD{\lab\not\in\Dom\varrho}\rangle}
     {\C\ts\longlabexps\ra\{\lab\mapsto\tau\}\langle +\ \varrho\rangle}\index{27.2}
\end{equation}%

\BeginNewEqns%
\begin{equation} % expression row extension
\color{\addcolor}
\frac{\C\ts\exp\ra\varrho\ \In\ \Type}
     {\C\ts\boxml{...}\ \boxml{=}\ \exp\ra\varrho}
\end{equation}%
\EndNewEqns%
%                        Expressions
%
\rulesec{Expressions}{\C\ts\exp\ra\tau}
%\vspace{6pt}
%\fbox{$\C\ts\exp\ra\tau$}
\begin{equation}        % atomic
\label{atexp-rule}
\frac{\C\ts\atexp\ra\tau}
     {\C\ts\atexp\ra\tau}\index{27.3}
\end{equation}

\begin{equation}        % application
%\label{app-rule}
\frac{\C\ts\exp\ra\tau'\rightarrow\tau\qquad\C\ts\atexp\ra\tau'}
     {\C\ts\appexp\ra\tau}
\end{equation}

\begin{equation}        % typed
\label{typedexp-rule}
\frac{\C\ts\exp\ra\tau\qquad\C\ts\ty\ra\tau}
     {\C\ts\typedexp\ra\tau}
\end{equation}

\begin{equation}        % handle exception
%\label{handlexp-rule}
\frac{\C\ts\exp\ra\tau\qquad\C\ts\match\ra\EXCN\rightarrow\tau}
     {\C\ts\handlexp\ra\tau}
\end{equation}

\begin{equation}        % raise exception
\label{raiseexp-rule}
\frac{\C\ts\exp\ra\EXCN}
     {\C\ts\raisexp\ra\tau}
\end{equation}

\begin{equation}        % function
%\label{fnexp-rule}
\frac{\C\ts\match\ra\tau}
     {\C\ts\fnexp\ra\tau}
\end{equation}
\comments
\begin{description}
\item{(\ref{atexp-rule})}
The relational symbol $\ts$ is overloaded for all syntactic classes (here
atomic expressions and expressions).
\item{(\ref{typedexp-rule})}
Here $\tau$ is determined by $\C$ and $\ty$. Notice that type variables
in $\ty$ cannot be instantiated in obtaining $\tau$; thus the expression
\verb+1:'a+ will not elaborate successfully, nor will the expression
\verb+(fn x=>x):'a->'b+.
The effect of type variables in an explicitly typed expression is
to indicate exactly the degree of polymorphism present in the expression.\index{27.4}
\item{(\ref{raiseexp-rule})}
Note that $\tau$ does not occur in the premise; thus a $\RAISE$
expression has ``arbitrary'' type.
\end{description}
%                        Matches 
%
\rulesec{Matches}{\C\ts\match\ra\tau}
\begin{equation}        % match
%\label{match-rule}
\frac{\C\ts\mrule\ra\tau\qquad\langle\C\ts\match\ra\tau\rangle}
     {\C\ts\longmatch\ra\tau}\index{28.1}
\end{equation}
\rulesec{Match Rules}{\C\ts\mrule\ra\tau}
\begin{equation}        % mrule
\label{mrule-rule}
\frac{\C\ts\pat\ra(\VE,\tau)\qquad\C+\VE\ts\exp\ra\tau'\qquad\TyNamesFcn\VE\subseteq\of{\T}{\C}}
     {\C\ts\longmrule\ \ra\tau\rightarrow\tau'}
\end{equation}%
\comment  This rule allows new free type variables to enter
the context. These new type variables will be chosen, in effect, during
the elaboration of $\pat$ (i.e., in the inference of the first
hypothesis). In particular, their choice may have to be made to
agree with type variables present in any explicit type expression
occurring within $\exp$ (see rule~\ref{typedexp-rule}).

%
%                        Declarations
%
\rulesec{Declarations}{\C\ts\dec\ra\E}
%poly
\begin{equation}        % value declaration
\label{valdec-rule}
\frac{\begin{array}{c}
      U = \TyVarsFcn (\tyvarseq) \qquad \ADD{\langle \TyNamesFcn\VE\subseteq\of{\T}{\C} \rangle} \\
      \ADD{\langle\,\forall\vid\in\Dom\ \VE,\,\mbox{either $\vid\not\in\Dom\ C$ or $\of{\is}{\C} = \isv$}\,\rangle} \\
     \plusmap{\C}{\U}\ADD{\langle+\VE\rangle}\ts\valbind\ra\VE \\ 
      \VE'=\cl{\C,\valbind}{\VE}\qquad
      \U\cap\TyVarFcn\VE'=\emptyset
      \end{array}}
     {\C\ts\boxml{val \ADD{$\langle$rec$\rangle$} $\tyvarseq$ $\valbind$}\ra\VE'\ \In\ \Env}\index{28.2}
\end{equation}

\begin{equation}        % type declaration
%\label{typedec-rule}
\frac{\C\ts\typbind\ra\TE}
     {\C\ts\typedec\ra\TE\ \In\ \Env}
\end{equation}

\begin{equation}        % datatype declaration
\label{datatypedec-rule}
\frac{\begin{array}{c}
\C\oplus\TE\ts\datbind\ra\VE,\TE\qquad
      \forall(\t,\adhocreplacementl{\thece}{9cm}{\CE}{\VE'})\in\Ran\TE,\ \t\notin(\of{\T}{\C}) \\
     \mbox{$\TE$ maximises equality}
     \end{array}
     }
     {\C\ts\datatypedec\ra(\VE,\TE)\ \In\ \Env}
\end{equation}

\insertion{\thedatatyperepl}{\begin{equation}        % datatype replication
\label{datatyperepldec-rule}
\frac{\C(\longtycon) = (\typefcn,\VE)\qquad
      \TE=\{\tycon\mapsto(\typefcn,\VE)\}
     }
     {\C\ts\datatyperepldec\ra(\VE,\TE)\ \In\ \Env}
\end{equation}}

\begin{equation}        % abstype declaration
\label{abstypedec-rule}
\frac{\begin{array}{rl}
      \CUT{\C\oplus\TE\ts\datbind\ra\VE,\TE}\qquad &
      \CUT{\forall(\t,\VE')\in\Ran\TE,\ \t\notin(\of{\T}{\C})}\\
      \CUT{\C\oplus(\VE,\TE)\ts\dec\ra\E}\qquad & 
     \mbox{\CUT{$\TE$ maximises equality}}
      \end{array}
     }
     {\CUT{\C\ts\abstypedec\ra\Abs(\TE,\E)}}
\end{equation}
\vskip6pt

\replacement{\theidstatus}{\begin{equation}        % exception declaration
\label{exceptiondec-rule}
\frac{\C\ts\exnbind\ra\EE\quad\VE=\EE }
     {\C\ts\exceptiondec\ra(\VE,\EE)\ \In\ \Env }
\end{equation}}{\begin{equation}        % exception declaration
\label{exceptiondec-rule}
\frac{\C\ts\exnbind\ra\VE}
     {\C\ts\exceptiondec\ra\VE\ \In\ \Env }
\end{equation}}

\oldpagebreak
\begin{equation}        % local declaration
%\label{localdec-rule}
\frac{\C\ts\dec_1\ra\E_1\qquad\C\oplus\E_1\ts\dec_2\ra\E_2}
     {\C\ts\localdec\ra\E_2}\index{28.3}
\end{equation}
\vskip6pt

\replacement{\thenostrsharing}{\begin{equation}                % open declaration
%\label{open-dec-rule}
\frac{ \C(\longstrid_1)=(\m_1,\E_1)
            \quad\cdots\quad
       \C(\longstrid_n)=(\m_n,\E_n) }
     { \C\ts\openstrdec\ra \E_1 + \cdots + \E_n }
\end{equation}}{\begin{equation}                % open declaration
%\label{open-dec-rule}
\frac{ \C(\longstrid_1)= \E_1 
            \quad\cdots\quad
       \C(\longstrid_n)= \E_n  }
     { \C\ts\openstrdec\ra \E_1 + \cdots + \E_n }
\end{equation}}
\vskip-4pt

\begin{equation}        % empty declaration
%\label{emptydec-rule}
\frac{}
     {\C\ts\emptydec\ra\emptymap\ \In\ \Env}
\end{equation}
\vskip4pt

\begin{equation}        % sequential declaration
%\label{seqdec-rule}
\frac{\C\ts\dec_1\ra\E_1\qquad\C\oplus\E_1\ts\dec_2\ra\E_2}
     {\C\ts\seqdec\ra\plusmap{E_1}{E_2}}
\end{equation}
\comments
\begin{description}

\item{(\ref{valdec-rule})}
Here $\VE$ will contain types rather than general
type schemes.
The closure of $\VE$  
allows value identifiers  to
be used polymorphically, via rule~\ref{varexp-rule}.

The side-condition on $\U$
ensures that the type variables in $\tyvarseq$  are bound 
by the closure operation,
if they occur free in the range of $\VE$.

On the other hand,
if the phrase $\boxml{val}\,\tyvarseq\,\valbind$ occurs inside
some larger value binding $\boxml{val}\,\tyvarseq'\,\valbind'$
then no type variable $\alpha$ listed in $\tyvarseq'$ will become
bound by the $\cl{\C,\valbind}{\VE}$ operation; for $\alpha$ must 
be in $\of{\U}{\C}$ and hence excluded from closure by the definition of the closure operation
(Section~\ref{closure-sec}, page~\pageref{clos-def-lab})
since $\of{\U}{\C}\subseteq\TyVarFcn\C$.

\ADD{Modifying $\C$ by $\VE$ on the left captures the 
recursive nature of the binding.
From rule~\ref{valbind-rule} we see that any
type scheme occurring in $\VE$ will have to be a type.
Thus each use of a
recursive function in its own body must be assigned the same type.
The side condition on the value identifiers in $C$ ensures that $C + \VE$
does not overwrite identifier status in the recursive case.
For example, the program}
\begin{center}
\ADD{\ml{datatype t = f; val rec f = fn x => x;}}
\end{center}%
\ADD{is not legal.}

\item{(\ref{datatypedec-rule})\CUT{,(\ref{abstypedec-rule}})}
The side conditions
express that the elaboration of each datatype binding
generates new type names and that as many of these new names
as possible admit equality.  Adding $\TE$ to the context on the left
of the $\ts$ captures the recursive nature of the binding.
%The side condition is
%the formal way of expressing that the elaboration of each datatype binding
%generates new type names.  Adding $\TE$ to the context on the left
%of the $\ts$ captures the recursive nature of the binding. Recall that $\TE$
%is assumed well-formed (as defined in Section~\ref{typeenv-wf-sec}). If
%$\TyNamesFcn(\of{\E}{\C})\subseteq\of{\T}{\C}$ and the side condition is
%satisfied then $\C\oplus\TE$ is well-formed.

\item{(\ref{datatyperepldec-rule})}
Note that no new type name is generated (i.e., datatype replication is
not generative).

\item{\CUT{(\ref{abstypedec-rule})}}
\CUT{The $\Abs$ operation was defined in Section~\ref{typeenv-wf-sec}, page~\pageref{abs-lab}.}

\item{(\ref{exceptiondec-rule})}
No closure operation is used here, as this would make the type system unsound.
Example: {\tt exception E of 'a; val it = (raise E 5) handle E f => f(2)}~.
\end{description}%

%                        Bindings
%
\rulesec{Value Bindings}{\C\ts\valbind\ra\VE}
%\vspace{6pt}
\begin{equation}        % value binding
\label{valbind-rule}
\frac{\C\ts\pat\ra(\VE,\tau)\qquad\C\ts\exp\ra\tau\qquad
      \langle\C\ts\valbind\ra\VE'\rangle }
     {\C\ts\longvalbind\ra\VE\ \langle +\ \VE'\rangle}\index{29.1}
\end{equation}

\begin{equation}        % recursive value binding
\label{recvalbind-rule}
\frac{\CUT{\C+\VE\ts\valbind\ra\VE\qquad\TyNamesFcn\VE\subseteq\of{\T}{\C}}}
     {\CUT{\C\ts\recvalbind\ra\VE}}
\end{equation}
\comments
\begin{description}
\item{(\ref{valbind-rule})}
When the option is present we have $\Dom\VE\cap
\Dom\VE' = \emptyset$ by the syntactic restrictions.\index{29.2}
\oldpagebreak
\item{\CUT{(\ref{recvalbind-rule})}}
\CUT{Modifying $\C$ by $\VE$ on the left captures the 
recursive nature of the binding. From rule~\ref{valbind-rule} we see that any
type scheme occurring in $\VE$ will have to be a type. Thus each use of a
recursive function in its own body must be assigned the same type.
Also note that $\C+\VE$ may overwrite 
identifier status. For example, the program
    {\tt datatype t = f; val rec f = fn x => x;}~~  is legal.}
\end{description}

\rulesec{Type Bindings}{\C\ts\typbind\ra\TE}
%\fbox{$\C\ts\typbind\ra\TE$}
\begin{equation}        % type binding
%\label{typbind-rule}
\frac{\tyvarseq=\alphak\qquad\C\ts\ty\ra\tau\qquad
      \langle\C\ts\typbind\ra\TE\rangle}
     {\begin{array}{c}
      \C\ts\longtypbind\ra\\
      \qquad\qquad\qquad
      \{\tycon\mapsto(\typefcnk,\emptymap)\}\ \langle +\ \TE\rangle
      \end{array}
     }\index{29.3}
\end{equation}
\comment The syntactic restrictions ensure that the type function
$\typefcnk$ satisfies the well-formedness \replacement{\theidstatus}{constraints }{constraint }of 
Section~\ref{tyfun-sec} and they ensure $tycon\notin\Dom\TE$.

\rulesec{Datatype Bindings}{\C\ts\datbind\ra\VE,\TE}
%\fbox{$\C\ts\datbind\ra\VE,\TE$}
\begin{equation}        % datatype binding
\frac{\begin{array}{c}
        \tyvarseq=\alphak\qquad\C,\alphakt\ts\constrs\ra\VE\qquad\arity\t=k\\
        \langle\C\ts\datbind'\ra\VE',\TE'\qquad
        \forall(\t',\VE'')\in\Ran\TE', \t\neq\t'\rangle
      \end{array}
     }
     {\begin{array}{l}
        \C\ts\newlongdatbind\ra\\
        \qquad(\cl{}{\VE}\langle +\ \VE'\rangle,\
        \{\tycon\mapsto(\t,\cl{}{\VE})\}\ \langle +\ \TE'\rangle\FIX{)}
      \end{array}
     }\index{30.1}
\end{equation}%
\comment \replacement{\thece}{The syntactic restrictions ensure $\Dom\VE\cap\Dom\CE = \emptyset$
and $\tycon\notin\Dom\TE$.}{The syntactic restrictions ensure $\Dom\VE\cap\Dom\VE' = \emptyset$
and $\tycon\notin\Dom\TE'$.}

\replacement{\thece}{\rulesec{Constructor Bindings}{\C,\tau\ts\constrs\ra\CE}}{\rulesec{Constructor Bindings}{\C,\tau\ts\constrs\ra\VE}}
%\fbox{$\C,\tau\ts\constrs\ra\CE$}
\replacement{\theidstatus}{\begin{equation}        % data constructors
%\label{constrs-rule}
\frac{\langle\C\ts\ty\ra\tau'\rangle\qquad
      \langle\langle\C,\tau\ts\constrs\ra\CE\rangle\rangle }
     {\begin{array}{c}
      \C,\tau\ts\longerconstrs\ra\\
      \qquad\qquad\qquad\{\con\mapsto\tau\}\
     \langle +\ \{\con\mapsto\tau'\to\tau\}\ \rangle\
      \langle\langle +\ \CE\rangle\rangle
      \end{array}
     }\index{30.2}
\end{equation}}{\begin{equation}        % data constructors
%\label{constrs-rule}
\frac{\langle\C\ts\ty\ra\tau'\rangle\qquad
      \langle\langle\C,\tau\ts\constrs\ra\VE\rangle\rangle }
     {\begin{array}{c}
      \C,\tau\ts\longervidconstrs\ra\\
      \qquad\qquad\qquad\{\vid\mapsto(\tau,\isc)\}\
     \langle +\ \{\vid\mapsto(\tau'\to\tau,\isc)\}\ \rangle\
      \langle\langle +\ \VE\rangle\rangle
      \end{array}
     }\index{30.2}
\end{equation}}
\comment By the syntactic restrictions \replacement{\theidstatus}{$\con\notin\Dom\CE$.}{$\vid\notin\Dom\VE$.}

\replacement{\theidstatus}{
\rulesec{Exception Bindings}{\C\ts\exnbind\ra\EE}}{
\rulesec{Exception Bindings}{\C\ts\exnbind\ra\VE}}
%poly with polymorphic exceptions:
% \ replacement{\thenoimptypes}{\begin{equation}        % exception binding
%  \label{exnbind1-rule}
%  \frac{\langle\C\ts\ty\ra\tau\quad\mbox{$\tau$ is imperative}\rangle\qquad
%        \langle\langle\C\ts\exnbind\ra\EE\rangle\rangle }
%       {\begin{array}{c}
%        \C\ts\longexnbinda\ra\\
%        \qquad\qquad\qquad\{\exn\mapsto\EXCN\}\
%        \langle +\ \{\exn\mapsto\tau\rightarrow\EXCN\}\ \rangle\
%        \langle\langle +\ \EE\rangle\rangle
%        \end{array}
%       }\index{30.3}
%  \end{equation}}{\begin{equation}        % exception binding
%  \label{exnbind1-rule}
%  \frac{\langle\C\ts\ty\ra\tau\rangle\qquad
%        \langle\langle\C\ts\exnbind\ra\EE\rangle\rangle }
%       {\begin{array}{c}
%        \C\ts\longexnbinda\ra\\
%        \qquad\qquad\qquad\{\exn\mapsto\EXCN\}\
%        \langle +\ \{\exn\mapsto\tau\rightarrow\EXCN\}\ \rangle\
%        \langle\langle +\ \EE\rangle\rangle
%        \end{array}
%       }\index{30.3}
%  \end{equation}}
\replacement{\theidstatus}{\begin{equation}        % exception binding
\label{exnbind1-rule}
\frac{\langle\C\ts\ty\ra\tau\quad\mbox{$\tau$ is imperative}\rangle\qquad
      \langle\langle\C\ts\exnbind\ra\EE\rangle\rangle }
     {\begin{array}{c}
      \C\ts\longexnbinda\ra\\
      \qquad\qquad\qquad\{\exn\mapsto\EXCN\}\
      \langle +\ \{\exn\mapsto\tau\rightarrow\EXCN\}\ \rangle\
      \langle\langle +\ \EE\rangle\rangle
      \end{array}
     }\index{30.3}
\end{equation}}{\begin{equation}        % exception binding
\label{exnbind1-rule}
\frac{\langle\C\ts\ty\ra\tau\rangle\qquad
      \langle\langle\C\ts\exnbind\ra\VE\rangle\rangle }
     {\begin{array}{l}
      \C\ts\longvidexnbinda\ra\\
      \qquad\{\vid\mapsto(\EXCN,\ise)\}\
      \langle +\ \{\vid\mapsto(\tau\rightarrow\EXCN,\ise)\}\ \rangle\
      \langle\langle +\ \VE\rangle\rangle
      \end{array}
     }\index{30.3}
\end{equation}}
\vskip-4pt

%\vspace*{4mm}
\replacement{\theidstatus}{\begin{equation}        % exception binding
\label{exnbind2-rule}
\frac{\C(\longexn)=\tau\qquad
      \langle\C\ts\exnbind\ra\EE\rangle }
      {\C\ts\longexnbindb\ra\{\exn\mapsto\tau\}\ \langle +\ \EE\rangle}
\end{equation}}{\begin{equation}        % exception binding
\label{exnbind2-rule}
\frac{\C(\longvid)=(\tau,\ise)\qquad
      \langle\C\ts\exnbind\ra\VE\rangle }
      {\C\ts\longvidexnbindb\ra\{\vid\mapsto(\tau,\ise)\}\ \langle +\ \VE\rangle}
\end{equation}}
\comments
\begin{description}
\replacement{\thenoimptypes}{\item{(\ref{exnbind1-rule})} Notice that $\tau$ must not contain
any applicative type variables.}{\item{(\ref{exnbind1-rule})} Notice that $\tau$ may contain
type variables.}\index{30.35}
%with monotyped exceptions:
%\item{(\ref{exnbind1-rule})} Notice that $\tau$ must be a monotype
%(see also restriction~\ref{monotypes-res} in 
%Section~\ref{further-restrictions-sec}).
\oldpagebreak
\item{(\ref{exnbind1-rule}),(\ref{exnbind2-rule})}
\replacement{\theidstatus}{
There is a unique $\EE$, for each $\C$ and $\exnbind$,
such that $\C\ts\exnbind\ra\EE$.}{For each $\C$ and $\exnbind$,
there is at most one $\VE$ satisfying $\C\ts\exnbind\ra\VE$.}
\end{description}

%\caption{Rules for Bindings}
%\end{figure}

%                        Atomic Patterns
%
\rulesec{Atomic Patterns}{\C\ts\atpat\ra(\VE,\tau)}
%\vspace{6pt}
%\fbox{$\C\ts\atpat\ra(\VE,\tau)$}
\begin{equation}        % wildcard pattern
%\label{wildcard-rule}
\frac{}
     {\C\ts\wildpat\ra (\emptymap,\tau)}\index{30.4}
\end{equation}

\begin{equation}        % special constant in pattern
\frac{}
     {\C\ts\scon\ra (\emptymap,\scontype(\scon))}\index{30.5}
\end{equation}

\replacement{\theidstatus}{
\begin{equation}        % variable pattern
\label{varpat-rule}
\frac{}
     {\C\ts\var\ra (\{\var\mapsto\tau\},\tau) }
\end{equation}}{\begin{equation}        % variable pattern
\label{varpat-rule}
\frac{\hbox{$\vid\notin\Dom(\C)$ or $\of{\is}{C(\vid)} = \isv$}}
     {\C\ts\vid\ra (\{\vid\mapsto(\tau,\isv)\},\tau) }
\end{equation}}

\replacement{\theidstatus}{
\begin{equation}        % constant pattern
%\label{constpat-rule}
\frac{\C(\longcon)\succ\tauk\t }
     {\C\ts\longcon\ra (\emptymap,\tauk\t)}
\end{equation}}{\begin{equation}        % constant pattern
\label{constpat-rule}
\frac{\C(\longvid)=(\sigma,\is)\qquad\is\neq\isv\qquad\sigma\succ\tauk\t }
     {\C\ts\longvid\ra (\emptymap,\tauk\t)}
\end{equation}}

\deletion{\theidstatus}{
\begin{equation}       % exception constant
%\label{exconapat-rule}
\frac{\C(\longexn)=\EXCN}
     {\C\ts\longexn\ra (\emptymap,\EXCN)}
\end{equation}}



\begin{equation}        % record pattern
%\label{recpat-rule}
\frac{\langle\C\ts\labpats\ra(\VE,\varrho)\rangle}
     {\C\ts\lttbrace\ \recpat\ \rttbrace\ra(\ \emptymap\langle +\ \VE\rangle,\ \emptymap
      \langle +\ \varrho\rangle\ \In\ \Type\ ) }\index{31.1}
\end{equation}

\begin{equation}        % parenthesised pattern
%\label{parpat-rule}
\frac{\C\ts\pat\ra(\VE,\tau)}
     {\C\ts\parpat\ra(\VE,\tau)}
\end{equation}
\comments
\begin{description}
\replacement{\theidstatus}{\item{(\ref{varpat-rule})} 
Note that $\var$ can assume a type, not a general type scheme.}{\item{(\ref{varpat-rule}), (\ref{constpat-rule})} The context $\C$ determines which of these
two rules applies. In rule~\ref{varpat-rule}, note that  
$\vid$ can assume a type, not a general type scheme.}
\end{description}

\rulesec{Pattern Rows}{\C\ts\labpats\ra(\VE,\varrho)}
%\fbox{$\C\ts\labpats\ra(\VE,\varrho)$}
\begin{equation}        % wildcard record
%\label{wildrec-rule}
\frac{\ADD{\C\ts\pat\ra(\VE,\varrho\ \In\ \Type)}}
     {\C\ts\wildrec\ra(\REPL{\VE\ }{\emptymap},\varrho)}\index{31.2}
\end{equation}

\begin{equation}        % record component
\label{longlab-rule}
\frac{\begin{array}{c}\C\ts\pat\ra(\VE,\tau)\\
      \langle\C\ADD{+\VE}\ts\labpats\ra(\VE',\varrho) \qquad \Dom\VE\cap\Dom\VE' = \emptyset\rangle
        \qquad \ADD{\lab\not\in\Dom\varrho}
      \end{array}
}
     {\C\ts\longlabpats\ra
     (\VE\langle +\ \VE'\rangle,\
      \{\lab\mapsto\tau\}\langle +\ \varrho\rangle) }
\end{equation}%
\deletion{\theidstatus}{\comment 
\begin{description}
\item{(\ref{longlab-rule})} 
 By the syntactic restrictions, $\Dom\VE\cap\Dom\VE' = \emptyset$.
\end{description}}
\insertion{\theidstatus}{\comment 
\begin{description}
\item{\CUT{(\ref{longlab-rule})}} 
 \CUT{The syntactic restrictions ensure $\lab\notin\Dom\varrho$.}
\end{description}}
%                        Patterns
%
%\begin{figure}[h]
%\label{pat-rules}
\oldpagebreak
\rulesec{Patterns}{\C\ts\pat\ra(\VE,\tau)}
\begin{equation}        % atomic pattern
%\label{atpat-rule}
\frac{\C\ts\atpat\ra (\VE,\tau)}
     {\C\ts\atpat\ra (\VE,\tau)}\index{31.3}
\end{equation}

\replacement{\theidstatus}{
\begin{equation}        % construction pattern
%\label{conpat-rule}
\frac{\C(\longcon)\succ\tau'\to\tau\qquad\C\ts\atpat\ra(\VE,\tau')}
     {\C\ts\conpat\ra (\VE,\tau)}
\end{equation}}{\begin{equation}        % construction pattern
%\label{conpat-rule}
\frac{\C(\longvid) = (\sigma, \is)\qquad\is\neq\isv\qquad \sigma\succ\tau'\to\tau\qquad\C\ts\atpat\ra(\VE,\tau')}
     {\C\ts\vidpat\ra (\VE,\tau)}
\end{equation}}

\deletion{\theidstatus}{
\begin{equation}       %  exception construction pattern
%\label{exconpat-rule}
\frac{\C(\longexn)=\tau\rightarrow\EXCN\qquad
      \C\ts\atpat\ra(\VE,\tau)}
     {\C\ts\exconpat\ra(\VE,\EXCN)}
\end{equation}}

\begin{equation}        % typed pattern
%\label{typedpat-rule}
\frac{\C\ts\pat\ra(\VE,\tau)\qquad\C\ts\ty\ra\tau}
     {\C\ts\typedpat\ra (\VE,\tau)}
\end{equation}

\begin{equation}        % layered pattern
\label{layeredpat-rule}
\frac{\ADD{\C\ts\pat_1\ra(\VE_1,\tau) \qquad \plusmap{\C}{\VE_1}\ts\pat_2\ra(\VE_2,\tau) \qquad
      \Dom\VE_0\cap\Dom\VE_1 = \emptyset}}
     {\ADD{\C\ts\aspat\ra(\plusmap{\VE_1}{\VE_2},\tau)}}
\end{equation}

\SameEqn
\begin{equation}        % layered pattern
\frac{\begin{array}{c}
     \CUT{\hbox{$\vid\notin\Dom(\C)$ or $\of{\is}{C(\vid)} = \isv$}}\\[-0.5em]
      \CUT{\langle\C\ts\ty\ra\tau\rangle\qquad
      \C\ts\pat\ra(\VE,\tau)\qquad \vid\notin\Dom\VE}\\[-0.5em]
      \end{array}
     }
     {\CUT{\C\ts\layeredvidpat\ra(\plusmap{\{\vid\mapsto(\tau,\isv)\}}{\VE},\tau)}}
\end{equation}%
\NextEqn

\BeginNewEqns
\begin{equation}        % or pattern
\label{orpat-rule}
\frac{\ADD{\C\ts\pat_1\ra(\VE,\tau) \qquad \C\ts\pat_2\ra(\VE,\tau)}}
     {\ADD{\C\ts\orpat\ra(\VE,\tau)}}
\end{equation}

\begin{equation}        % nested match pattern
\label{nestedmatch-rule}
\frac{\begin{array}{cc}
        \ADD{\C\ts\pat_1\ra(\VE_1,\tau)} & \ADD{\plusmap{\C}{\VE_1}\ts\exp\ra\tau'} \\
        \ADD{\plusmap{\C}{\VE_1}\ts\pat_2\ra(\VE_2,\tau')} & \ADD{\Dom\VE_1\cap\Dom\VE_2=\emptyset} \\
      \end{array}}
     {\ADD{\C\ts\nestedpat\ra(\plusmap{\VE_1}{\VE_2},\tau)}}
\end{equation}

\EndNewEqns

%                        Type Expressions
\rulesec{Type Expressions}{\C\ts\ty\ra\tau}
\begin{equation}        % atype variable
%\label{tyvar-rule}
\frac{\tyvar=\alpha}
     {\C\ts\tyvar\ra\alpha}\index{32.1}
\end{equation}

\begin{equation}        % record type
%\label{rectype-rule}
\frac{\langle\C\ts\labtys\ra\varrho\rangle}
     {\C\ts\lttbrace\ \rectype\ \rttbrace\ra\emptymap\langle +\ \varrho\rangle\ \In\ \Type}
\end{equation}

\begin{equation}        % constructed type
\label{constype-rule}
\frac{\begin{array}{c}
      \tyseq=\ty_1\cdots\ty_k\qquad\C\ts\ty_i\ra\tau_i\ (1\leq i\leq k)\\
      \C(\longtycon)=(\theta,\adhocreplacementl{\thece}{8cm}{\CE}{\VE})
      \end{array}
     }
     {\C\ts\constype\ra\tauk\theta}
\end{equation}

\begin{equation}        % function type
%\label{funtype-rule}
\frac{\C\ts\ty\ra\tau\qquad\C\ts\ty\,'\ra\tau'}
     {\C\ts\funtype\ra\tau\to\tau'}
\end{equation}

\begin{equation}        % parenthesised type
%\label{partype-rule}
\frac{\C\ts\ty\ra\tau}
     {\C\ts\partype\ra\tau}
\end{equation}
\comments
\begin{tabbing}
(\ref{constype-rule}) \= Recall that for $\tauk\theta$ to be defined, $\theta$
must have arity $k$.
\end{tabbing}

\oldpagebreak
\rulesec{Type-expression Rows}{\C\ts\labtys\ra\varrho}
%\fbox{$\C\ts\labtys\ra\varrho$}
\begin{equation}        % record type components
%\label{longlabtys-rule}
\frac{\C\ts\ty\ra\tau\qquad\langle\C\ts\labtys\ra\varrho \qquad \ADD{\lab\not\in\Dom\varrho}\rangle}
     {\C\ts\longlabtys\ra\{\lab\mapsto\tau\}\langle +\ \varrho\rangle}\index{32.15}
\end{equation}

\BeginNewEqns%
\begin{equation} % expression row extension
\color{\addcolor}
\frac{\C\ts\ty\ra\varrho\ \In\ \Type}
     {\C\ts\boxml{...}\ \boxml{:}\ \ty\ra\varrho}
\end{equation}%
\EndNewEqns%

\CUT{\comment The syntactic constraints ensure $\lab\notin\Dom\varrho$.}
%\caption{Rules for Types}
%\end{figure}

\subsection{Further Restrictions}
\label{further-restrictions-sec}
There\index{32.2} are a few restrictions on programs which should be enforced by a
compiler, but are better expressed apart from the preceding
Inference Rules.  They are:
\begin{enumerate}
\item
\FIX{%
For each occurrence of a record expression containing ellipses,
i.e., of the form
%\begin{quote}
\ttlbrace$\lab_1$\ml{=}$\exp_1$\ml{,} $\cdots$\ml{,} $\lab_m$\ml{=}$\exp_m$\ml{,...=}$\exp_0$\ttrbrace\
%\end{quote}
the program context consisting of the smallest enclosing declaration must determine uniquely the domain
$\{{\it lab}_1,\cdots,{\it lab}_n\}$
of its row type, where $m\leq n$; thus, the context must
determine the labels $\{{\it lab}_{m+1},\cdots,{\it lab}_n\}$ of the fields of $\exp_0$.
Likewise for record patterns that contain ellipses.
In these situations, an explicit type
constraint may be needed.
}%

\CUT{
For each occurrence of a record pattern containing a record wildcard,
i.e., of the form
%\begin{quote}
\ttlbrace${\it lab}_1$\ml{=}$\pat_1$\ml{,}$\cdots$\ml{,}$\lab_m$\ml{=}$\pat_m$\ml{,...}\ttrbrace\
%\end{quote}
the program context must determine uniquely the domain
$\{\lab_1,\cdots,\lab_n\}$
of its row type, where $m\leq n$; thus, the context must
determine the labels $\{\lab_{m+1},\cdots,\lab_n\}$ of the fields
to be matched by the wildcard. For this purpose, an explicit type
constraint may be needed.
}

\item\label{irredundant-pat-restriction}
In a match of the form ${\it pat}_1$ \ml{=>} $\exp_1$ \ml{|}$\;\cdots\;$
\ml{|} ${\tt pat}_n$ \ml{=>} $\exp_n$ 
the pattern sequence $\pat_1,\ldots,\pat_n$ should be {\sl irredundant};
that is, each $\pat_j$ must match some value
(of the right type) which is not matched by $\pat_i$ for any $i<j$.
In the context {\fnexp}, the $\match$ must also be {\sl exhaustive}; that is,
every value (of the right type) must be matched by some $\pat_i$.
\ADD{For the purposes of checking exhaustiveness, any contained nested match
``$\nestedpat$'' may be assumed to fail, unless $\pat_2$ is exhaustive itself.
Furthermore, note that $\exp$ may contain side effects that could alter the
contents of any ref cells being matched against.
}
The compiler must give warning on violation of these restrictions, 
but should still compile the match.
The restrictions are inherited by derived forms; in particular,
this means that in the function-value binding\index{33.1}
$\vid\ \atpat_1\ \cdots\ \atpat_n\langle : \ty\rangle$\ \ml{=}\ $\exp$
(consisting of one clause only), each separate $\atpat_i$ should be
exhaustive by itself.
%must be {\sl irredundant} and {\sl exhaustive}.  That is, each ${\it pat}_j$
%must match {\sl some} value (of the right type) which is not matched by
%${\it pat}_i$ for any $i <j$, and {\sl every} value (of the right type) must be
%matched by some ${\it pat}_i$. The compiler must give a warning on violation
%of this restriction, but should still compile the match.

\item
\ADD{%
	A disjunctive pattern of the form ``$\orpat$ should be {\sl irredundant};
	that is, $\pat_2$ should match some value not matched by $\pat_1$.
	As in \ref{irredundant-pat-restriction} above, a pattern that contains a guard
	may be assumed to possibly fail.
}

\item
        For each value binding   $\pat\ \mbox{\ml{=}}\ \exp$
        the compiler must issue a report (but
        still compile) if   $\pat$  is not exhaustive. 
        This will detect a mistaken
        declaration like  $\VAL\ \ml{nil}\ \mbox{\ml{=}}\ \exp$
        in which the user expects to declare
        a new variable \ml{nil} 
        (whereas the language dictates that \ml{nil} is here a 
        constant pattern, so no variable gets declared). 
        However, this warning
        should not be given when the binding is a component 
        of a top-level declaration
        $\valdec$; e.g. 
        $\VAL\ \mbox{\ml{x::l = }}\exp_1\ \mbox{\ml{\AND\ y = }}\exp_2$ 
        is not faulted by the
        compiler at top level, but may of course generate a \ml{Bind} 
        exception (see Section~\ref{bas-exc}).

\item	\ADD{Every pattern of the form ``$\aspat$'' must be consistent; i.e., there must exist at least one
	value that is matched by both $\pat_1$ and $\pat_2$.}
\end{enumerate}

\clearpage{}
\thispagestyle{empty}
%!TEX root = root.tex
%

\section{Static Semantics for Modules}
\label{statmod-sec}
\subsection{Semantic Objects}
\label{statmod-sem-obj-sec}
The\index{34.1} simple 
objects for Modules static semantics are exactly as for the Core.
The compound objects are those for the Core,
augmented by those in Figure~\ref{module-objects}.


\begin{figure}[h]
%\vspace{2pt}
\adhocreplacementl{\thenostrsharing}{0mm}{
\begin{displaymath}
\begin{array}{rcl}
\M              & \in   & \StrNameSets = \Fin(\StrNames)\\
\N\ {\rm or}\ (\M,\T)
                & \in   & \NameSets = \StrNameSets\times\TyNameSets\\
\sig\ {\rm or}\ \longsig{}
                & \in   & \Sig =  \NameSets\times\Str \\
\funsig\ {\rm or}\ \longfunsig{}
                & \in   & \FunSig = \NameSets\times
                                         (\Str\times\Sig)\\
\G              & \in   & \SigEnv        =       \finfun{\SigId}{\Sig} \\
\F              & \in   & \FunEnv        =       \finfun{\FunId}{\FunSig} \\
\B\ {\rm or}\ \N,\F,\G,\E
                & \in   & \Basis = \NameSets\times
                                              \FunEnv\times\SigEnv\times\Env\\
\end{array}
\end{displaymath}}{\begin{displaymath}
\begin{array}{rcl}
\sig\ {\rm or}\ \newlongsig{}
                & \in   & \Sig =  \TyNameSets\times\Env \\
\funsig\ {\rm or}\ \newlongfunsig{}
                & \in   & \FunSig = \TyNameSets\times
                                         (\Env\times\Sig)\\
\G              & \in   & \SigEnv        =       \finfun{\SigId}{\Sig} \\
\F              & \in   & \FunEnv        =       \finfun{\FunId}{\FunSig} \\
\B\ {\rm or}\ \T,\F,\G,\E
                & \in   & \Basis = \TyNameSets\times
                                              \FunEnv\times\SigEnv\times\Env\\
\end{array}
\end{displaymath}}
\caption{Further Compound Semantic Objects}
\label{module-objects}
%\vspace{3pt}
\end{figure}
%
\replacement{\thenostrsharing}{
The prefix $(\N)$, in signatures and functor signatures, binds both type names
and structure names. We shall always consider a set $\N$ of names as
partitioned into a pair $(\M,\T)$ of sets of the two kinds of name.}{The 
prefix $(\T)$, in signatures and functor signatures, binds  type names.}
\deletion{\thenostrsharing}{It is sometimes convenient to work with an arbitrary semantic object $A$, or
assembly $A$ of such objects.
As with the function $\TyNamesFcn$,
$\StrNamesFcn(A)$ and $\NamesFcn(A)$ denote respectively the set of structure names
and the set of names occurring free in $A$.}
Certain operations require a change of bound names in semantic objects;
see for example \replacement{\thenostrsharing}{Section~\ref{realisation-sec}}{Section~\ref{tyrea.sec}}. When bound type names are
changed, we demand that all of their attributes (i.e. \deletion{\thenoimptypes}{imperative, }equality
and arity) are preserved.\index{34.2}

\deletion{\thenotypexplication}{For any structure $\S=\longS{}$ we call $m$ the {\sl structure name} or
{\sl name} of $\S$; also, the {\sl proper substructures} of $\S$ are
the members of $\Ran\SE$ and their proper substructures.  The 
{\sl substructures} of
$\S$ are $\S$ itself and its proper substructures.  The structures
{\sl occurring in}
an object or assembly $A$ are the structures and
substructures from which it is built.}

The operations of projection, injection and modification are as for the
Core. Moreover, we define $\of{\C}{\B}$ to be the context
$(\of{\T}{\B},\emptyset,\of{\E}{\B})$, i.e.~with an empty set of
explicit type variables.
Also,
we frequently need to modify a basis $\B$ by an environment $\E$
(or a structure environment $\SE$ say),
at the same time extending \replacement{\thenostrsharing}{$\of{\N}{\B}$}{$\of{\T}{\B}$} to include the type names 
\deletion{\thenostrsharing}{and
structure names }of $\E$ (or of $\SE$ say).
We therefore define $\B\oplus\SE$, for example, to mean
\replacement{\thenostrsharing}{$\B+(\NamesFcn\SE,\SE)$}{$\B+(\TyNamesFcn\SE,\SE)$}.
\index{34.3}
 
\insertion{\thenostrsharing}{
There is no separate kind of semantic object to represent structures: 
structure expressions elaborate to environments, just as structure-level
declarations do. Thus, notions which are commonly associated with structures
(for example the notion of matching a structure against a signature) are defined
in terms of environments.}

\deletion{\thenostrsharing}{
\subsection{Consistency}
\label{consistency-sec}
A\index{35.1} set of type structures is said to be {\sl consistent} if, for all
$(\theta_1,\CE_1)$ and $(\theta_2,\CE_2)$ in the set, if $\theta_1 = \theta_2$
then
\[\CE_1=\emptymap\ {\rm or}\ 
\CE_2=\emptymap\ {\rm or}\ \Dom\CE_1=\Dom\CE_2\]
A semantic object $A$ or assembly $A$ of objects is said to be
{\sl consistent} if (after changing bound names to make all nameset prefixes
in $A$ disjoint) 
for all $\S_1$ and
$\S_2$ occurring in $A$ and for every $\longstrid$ 
and every $\longtycon$
\begin{enumerate}
\item If $\of{\m}{\S_1}=\of{\m}{\S_2}$, and both
      $\S_1(\longstrid)$ and $\S_2(\longstrid)$ exist, then
      \[ \of{\m}{\S_1(\longstrid)}\ =\ \of{\m}{\S_2(\longstrid)}\]

\item If $\of{\m}{\S_1}=\of{\m}{\S_2}$, and both
      $\S_1(\longtycon)$ and $\S_2(\longtycon)$ exist, then
      \[ \of{\theta}{\S_1(\longtycon)}\ =\ \of{\theta}{\S_2(\longtycon)}\]

\item The set of all type structures in $A$ is consistent
\end{enumerate}

As an example, a functor signature 
$\longfunsig{}$ is
consistent if, assuming first that 
$\N\cap\N'=\emptyset$,
the assembly $A=\{\S,\S'\}$ is consistent.

We may loosely say that two 
structures $\S_1$ and $\S_2$
are consistent if
$\{\S_1,\S_2\}$ is consistent, but must remember that this is stronger than
the assertion that $\S_1$ is consistent and $\S_2$ is consistent.

Note that if $A$ is a consistent assembly and $A'\subset A$ then $A'$ is
also a consistent assembly.
}

\deletion{\thenostrsharing}{
\subsection{Well-formedness}
A signature\index{35.2} $\longsig{}$ is {\sl well-formed} 
if $\N\subseteq\NamesFcn\S$,
and also, whenever $(\m,\E)$ is a
substructure of $\S$ and $\m\notin\N$, then $\N\cap(\NamesFcn\E)=\emptyset$.
A functor signature $\longfunsig{}$ is {\sl well-formed} if
$\longsig{}$ and  $(\N')\S'$ are well-formed, and also, whenever
$(\m',\E')$ is a substructure of $\S'$ and $\m'\notin\N\cup\N'$,
then $(\N\cup\N')\cap(\NamesFcn\E')=\emptyset$.

An object or assembly $A$ is {\sl well-formed} if every type environment,
signature and functor signature occurring in $A$ is well-formed.}

\deletion{\thenostrsharing}{\subsection{Cycle-freedom}
An\index{35.3} object or assembly $A$ is {\sl cycle-free} if it contains no
cycle of structure names; that is, there is no sequence
\[\m_0,\cdots,\m_{k-1},\m_k=m_0\ \ (k>0)\]
of structure names such that, for each $i\ (0\leq i<k)$ some structure
with name $m_i$ occurring in $A$ has a proper substructure with name
$m_{i+1}$.
}

\deletion{\thenostrsharing}{
\subsection{Admissibility}
\label{admis-sec}
An\index{36.1} object or assembly $A$ is {\sl admissible} if it is
consistent, well-formed and cycle-free. 
Henceforth it is assumed
that
all objects mentioned are admissible.  
We also require that
\begin{enumerate}
\item In every sentence $A\ts\phrase\ra A'$  inferred by the rules
given in Section~\ref{statmod-rules-sec}, the assembly $\{A,A'\}$ is
admissible.  
\item In the special case of a sentence $\B\ts\sigexp\ra\S$,
we further require that the assembly consisting of all semantic
objects occurring in the entire inference of this sentence be
admissible. This  is important for the definition of principal
signatures in Section~\ref{prinsig-sec}.
\end{enumerate}
In our semantic definition we have not undertaken to
indicate how admissibility should be checked in an implementation.
}

\subsection{Type Realisation}
\label{tyrea.sec}
\replacement{\thenostrsharing}{
A {\sl type realisation}\index{36.2} is a map
$\tyrea:\TyNames\to\TypeFcn$
such that
$\t$ and $\tyrea(\t)$ have the same arity, and
if $t$ admits equality then so does $\tyrea(\t)$.

The {\sl support} $\Supp\tyrea$ of a type realisation $\tyrea$ is the set of
type names $\t$ for which $\tyrea(\t)\ne\t$.}{A 
{\sl (type) realisation}\index{36.2} is a map
$\rea:\TyNames\to\TypeFcn$
such that
$\t$ and $\rea(\t)$ have the same arity, and
if $t$ admits equality then so does $\rea(\t)$.

The {\sl support} $\Supp\rea$ of a type realisation $\rea$ is the set of
type names $\t$ for which $\rea(\t)\ne\t$.}
%
\deletion{\thenostrsharing}{\subsection{Realisation}
\label{realisation-sec}
A {\sl realisation}\index{36.3} is a function $\rea$ of names,
partitioned into a type realisation $\tyrea:\TyNames\to\TypeFcn$
and a function $\strrea : \StrNames\to\StrNames$.
The {\sl support} $\Supp\rea$
of a realisation $\rea$ is the set of
names $\n$ for which $\rea(\n)\ne\n$.}\replacement{\thenostrsharing}{The {\sl yield}
$\Yield\rea$ of a realisation $\rea$ is the set of
names which occur in some $\rea(\n)$ for which $\n\in\Supp\rea$.}{The
{\sl yield} $\Yield\rea$ of a realisation $\rea$ is the set of
type names which occur in some $\rea(\t)$ for which $\t\in\Supp\rea$.}

Realisations $\rea$ are extended to apply to all semantic objects; their
effect is to
replace each name \replacement{\thenostrsharing}{$n$ by $\rea(\n)$}{$\t$ by $\rea(\t)$}.  In applying $\rea$ to an object with
bound names, such as a signature \replacement{\thenostrsharing}{$\longsig{}$}{$\newlongsig{}$}, first bound names must be
changed so that, for each binding prefix \replacement{\thenostrsharing}{$(\N)$}{$(\T)$},
\replacement{\thenostrsharing}{
\[\N\cap(\Supp\rea\cup\Yield\rea)=\emptyset\ .\]}{\[\T\cap(\Supp\rea\cup\Yield\rea)=\emptyset\ .\]}
%
\deletion{\thenotypexplication}{
\subsection{Type Explication}
\label{type-explication-sec}
A\index{36.35} signature $(\N)\S$ is {\sl type-explicit\/} if,
whenever $\t\in\N$ and occurs free in $\S$, then some substructure of
$\S$ contains a type environment $\TE$ such that
$\TE(\tycon)=(\t,\CE)$ for some $\tycon$ and some $\CE$.} 
%
\subsection{Signature Instantiation}
\replacement{\thenostrsharing}{
A\index{36.4} structure $\S_2$ {\sl is an instance of} a signature
$\sig_1=\longsig{1}$,
written $\siginst{\sig_1}{}{\S_2}$, if there exists a realisation
$\rea$
such that $\rea(\S_1)=\S_2$ and $\Supp\rea\subseteq\N_1$.}{An\index{36.4} environment $\E_2$ {\sl is an instance of} a signature
$\sig_1=\newlongsig{1}$,
written $\siginst{\sig_1}{}{\E_2}$, if there exists a realisation
$\rea$
such that $\rea(\E_1)=\E_2$ and $\Supp\rea\subseteq\T_1$.}
\deletion{\thenotypexplication}{(Note that if $\sig_1$ is type-explicit then there is at most one
such $\rea$.)}\ 
\deletion{\thenostrsharing}{A signature
$\sig_2=\longsig{2}$ {\sl is an instance of}
$\sig_1 =\longsig{1}$,
written $\siginst{\sig_1}{}{\sig_2}$, if
$\siginst{\sig_1}{}{\S_2}$ and $\N_2\cap(\NamesFcn\sig_1)=\emptyset$.
It can be shown that $\siginst{\sig_1}{}{\sig_2}$ iff, for all $\S$,
whenever $\siginst{\sig_2}{}{\S}$ then $\siginst{\sig_1}{}{\S}$.}

\subsection{Functor Signature Instantiation}
\replacement{\thenostrsharing}{
A\index{36.5} pair $(\S,(\N')\S')$ is called a {\sl functor instance}.
Given $\funsig=\longfunsig{1}$,
a functor instance $(\S_2,(\N_2')\S_2')$ is an {\sl instance} of
$\funsig$,
written $\funsiginst{\funsig}{}{(\S_2,(\N_2')\S_2')}$,
if there exists a realisation $\rea$
such that
$\rea(\S_1,(\N_1')\S_1')=(\S_2,(\N_2')\S_2')$ and
$\Supp\rea\subseteq\N_1$.}{
A\index{36.5} pair $(\E,(\T')\E')$ is called a {\sl functor instance}.
Given $\funsig=\newlongfunsig{1}$,
a functor instance $(\E_2,(\T_2')\E_2')$ is an {\sl instance} of
$\funsig$,
written $\funsiginst{\funsig}{}{(\E_2,(\T_2')\E_2')}$,
if there exists a realisation $\rea$
such that
$\rea(\E_1,(\T_1')\E_1')=(\E_2,(\T_2')\E_2')$ and
$\Supp\rea\subseteq\T_1$.}
%
\subsection{Enrichment}
\label{enrichment-sec}
In\index{37.1} matching \replacement{\thenostrsharing}{a structure}{an environment} to a signature, the \replacement{\thenostrsharing}{structure}{environment} will be allowed both to
have more components, and to be more polymorphic, than (an instance of) the
signature.  Precisely, we  define enrichment of \deletion{\thenostrsharing}{structures, }environments and
type structures \replacement{\thenostrsharing}{by mutual recursion}{recursively} as follows.

\deletion{\thenostrsharing}{A structure $\S_1=(\m_1,\E_1)$
{\sl enriches} another structure
$\S_2=(\m_2,\E_2)$, written $\S_1\succ\S_2$, if
\begin{enumerate}
\item $\m_1=\m_2$
\item $\E_1\succ\E_2$
\end{enumerate}}
An environment \replacement{\theidstatus}{$\E_1=\longE{1}$}{$\E_1=\newlongE{1}$}
{\sl enriches} another environment \replacement{\theidstatus}{$\E_2=$ $\longE{2}$}{$\E_2=
( \SE_2,$\linebreak$\TE_2,\VE_2)$},
written $\E_1\succ\E_2$,
if
\begin{enumerate}
\item $\Dom\SE_1\supseteq\Dom\SE_2$, and $\SE_1(\strid)\succ\SE_2(\strid)$
                                               for all $\strid\in\Dom\SE_2$
\item $\Dom\TE_1\supseteq\Dom\TE_2$, and $\TE_1(\tycon)\succ\TE_2(\tycon)$
                                               for all $\tycon\in\Dom\TE_2$
\item \replacement{\theidstatus}{$\Dom\VE_1\supseteq\Dom\VE_2$, and $\VE_1(\id)\succ\VE_2(\id)$
                                               for all $\id\in\Dom\VE_2$}{$\Dom\VE_1\supseteq\Dom\VE_2$, and $\VE_1(\vid)\succ\VE_2(\vid)$
                                               for all $\vid\in\Dom\VE_2$,
where $(\sigma_1,\is_1)\succ(\sigma_2,\is_2)$ means $\sigma_1\succ\sigma_2$ and
$$\is_1 = \is_2\quad\hbox{or}\quad \is_2 = \isv$$}
\deletion{\theidstatus}{
\item $\Dom\EE_1\supseteq\Dom\EE_2$, and $\EE_1(\exn)=\EE_2(\exn)$
                                               for all $\exn\in\Dom\EE_2$}
\end{enumerate}
Finally, a type structure $(\theta_1,\adhocreplacementl{\thece}{6cm}{\CE}{\VE}_1)$
{\sl enriches} another type structure $(\theta_2,\adhocreplacementl{\thece}{-4cm}{\CE}{\VE}_2)$,
written $(\theta_1,\adhocreplacementl{\thece}{10mm}{\CE}{\VE}_1)\succ(\theta_2,\adhocreplacementl{\thece}{-12cm}{\CE}{\VE}_2)$,
if
\begin{enumerate}
\item $\theta_1=\theta_2$
\item Either $\adhocreplacementl{\thece}{3cm}{\CE}{\VE}_1=\adhocreplacementl{\thece}{-8cm}{\CE}{\VE}_2$ or $\adhocreplacementl{\thece}{-11cm}{\CE}{\VE}_2=\emptymap$
\end{enumerate}

\oldpagebreak
\subsection{Signature Matching}
\label{sigmatch-sec}
\replacement{\thenostrsharing}{
A\index{37.2} structure $\S$ {\sl matches} a signature $\sig_1$ if there exists
a structure $\S^-$ such that $\sig_1\geq\S^-\prec\S$. Thus matching
is a combination of instantiation and enrichment. There is at most
one such $\S^-$, given $\sig_1$ and $\S$.}{An\index{37.2} environment $\E$ {\sl matches} a signature $\sig_1$ if there exists
an environment $\E^-$ such that $\sig_1\geq\E^-\prec\E$. Thus matching
is a combination of instantiation and enrichment. There is at most
one such $\E^-$, given $\sig_1$ and $\E$.} \deletion{\thenotypexplication}{Moreover, writing $\sig_1=
\longsig{1}$, if $\sig_1\geq\S^-$ then there exists a realisation $\rea$
with $\Supp\rea\subseteq\N_1$ and $\rea(\S_1)=\S^-$.
We shall then say that $\S$ matches $\sig_1$ {\em via} $\rea$.
(Note that if $\sig_1$ is type-explicit 
then $\rea$ is uniquely determined by $\sig_1$ and $\S$.)}

\deletion{\thenostrsharing}{A\index{37.2.5} signature $\sig_2$ {\em matches} a signature $\sig_1$
if for all structures $\S$, if $\S$ matches $\sig_2$ then $\S$
matches $\sig_1$. It can be shown that $\sig_2=\longsig{2}$ matches
$\sig_1=\longsig{1}$ if and only if there exists a realisation
$\rea$ with $\Supp\rea\subseteq\N_1$ and $\rea(\S_1)\prec\S_2$
and $\N_2\cap\NamesFcn\sig_1=\emptyset$.}

\deletion{\thenostrsharing}{\subsection{Principal Signatures}
\label{prinsig-sec}
The definitions in this section concern the elaboration of signature
expressions; more precisely they concern inferences of sentences of the
form $\B\ts\sigexp\ra\S$, where $\S$ is a structure and $\B$ is a basis.
Recall, from Section~\ref{admis-sec}, that the assembly of all semantic
objects in such an inference must be admissible.

For any basis $\B$ and any structure $\S$, 
we say that $\B$ {\sl covers} $\S$
if for every substructure $(m,E)$ of $\S$ such that
$m\in\of{\N}{\B}$:
\begin{enumerate}
\item
For every structure identifier $\strid\in\Dom\E$,
$\B$ contains a substructure $(m,\E')$ with $m$
free and $\strid\in\Dom\E'$
\item
For every type constructor $\tycon\in\Dom\E$,
$\B$ contains a substructure $(m,\E')$ with $m$ free
and $\tycon\in\Dom\E'$
\end{enumerate}
(This condition is not a consequence of consistency of $\{\B,\S\}$; 
informally, it states that if $\S$ shares a substructure with $\B$,
then $\S$ mentions no more components of the substructure than
$\B$ does.)



We\index{38.1} say that a signature
$\longsig{}$ is {\sl principal for $\sigexp$ in $\B$} if, choosing $\N$
so that $(\of{\N}{\B})\cap\N=\emptyset$,
\begin{enumerate}
\item $\B$ covers $\S$ 
\item $\B\vdash\sigexp\ra\S$
\item Whenever $\B\vdash\sigexp\ra\S'$, then $\sigord{\longsig{}}{}{\S'}$
\end{enumerate}
We claim that if $\sigexp$ elaborates in $\B$ to some structure covered
by $\B$, then it possesses a principal signature in $\B$.

Analogous to the definition given for type environments in
Section~\ref{typeenv-wf-sec}, we say that a semantic object $A$
{\sl respects equality} if every type environment occurring in 
$A$ respects equality. 
%
%
%Further, let $T$ be the set of type names
%$\t$ such that $(\t,\CE)$ occurs in $A$ for some
%$\CE\neq\emptymap$.  Then $A$ is said to {\sl maximise equality}
%if (a) $A$ respects equality, and also (b) if any larger subset of
%$T$ were to admit equality (without any change in the equality
%attribute of any type names not in $T$) then $A$ would cease to
%respect equality.
%
\oldpagebreak
Now\index{38.5} let us assume that $\sigexp$ possesses a principal signature
$\sig_0=\longsig{0}$ in $B$. We wish to
define, in terms of $\sig_0$, another signature $\sig$ which provides more
information about the equality attributes of structures which will
match $\sig_0$. To this end, let $\T_0$ be the set of type names $\t\in\N_0$
which do not admit equality, and such that $(\t,\CE)$ occurs in $\S_0$
for some $\CE\neq\emptymap$.  Then we say $\sig$ is 
{\sl equality-principal for $\sigexp$ in $\B$} if
\begin{enumerate}
\item
$\sig$ respects equality
\item
$\sig$ is obtained from $\sig_0$ just by making as many
members of $\T_0$ admit equality as possible, subject to 1.~above
\end{enumerate}
It is easy to show that, if any such $\sig$ exists, it is determined
uniquely by $\sig_0$; moreover, $\sig$ exists if $\sig_0$ itself
respects equality.
\bigskip}


%
%\clearpage

%                   Inference Rules
%
\subsection{Inference Rules}
\label{statmod-rules-sec}
As\index{39.1} for the Core, the rules of the Modules static semantics allow
sentences of the form
\[ A\ts\phrase\ra A'\]
to be inferred, where in this case $A$ is either a basis, a context or
an environment and $A'$ is a semantic object.  The convention for options
is as in the Core semantics. 

Although not assumed in our definitions, it is intended that every basis
\replacement{\thenostrsharing}{$\B=\N,\F,\G,\E$}{$\B=\T,\F,\G,\E$} in which a $\topdec$ is elaborated has the property
that 
\replacement{\thenostrsharing}{$\NamesFcn\F\ \cup\NamesFcn\G\cup\NamesFcn\E\subseteq\N$}{$\TyNamesFcn\F
\ \cup\TyNamesFcn\G\cup\TyNamesFcn\E\subseteq\T$}. 
\replacement{\thenostrsharing}{
This is not
the case for bases in which signature expressions and specifications are
elaborated, but the following Theorem can be proved:}{
The following Theorem can be proved:}
\begin{quote}
Let S be an inferred sentence $\B\ts\topdec\ra\B'$ in which $\B$ satisfies
the above condition. Then $\B'$ also satisfies the condition.

\replacement{\thenostrsharing}{
Moreover, if S$'$ is a sentence of the form
$\B''\ts\phrase\ra A$ occurring in a proof of S, where $\phrase$ is
either a structure expression or a structure-level declaration, then $\B''$
also satisfies the condition.}{
Moreover, if S$'$ is a sentence of the form
$\B''\ts\phrase\ra A$ occurring in a proof of S, where $\phrase$ is
any Modules phrase, then $\B''$ also satisfies the condition.}

\replacement{\thenostrsharing}{Finally, if $\T,\U,\E\ts\phrase\ra A$ occurs
in a proof of S, where $\phrase$ is a phrase of the Core, then
$\TyNamesFcn\E\subseteq\T$.}{Finally, if $\T,\U,\E\ts\phrase\ra A$ occurs
in a proof of S, where $\phrase$ is a phrase of Modules or of the Core, then
$\TyNamesFcn\E\subseteq\T$.}
\end{quote}



%               SEMANTICS
%
%                       Structure Expressions
%
\replacement{\thenostrsharing}{
\rulesec{Structure Expressions}{\B\ts\strexp\ra \S}}
{\rulesec{Structure Expressions}{\B\ts\strexp\ra \E}}
\replacement{\thenostrsharing}{\begin{equation}        % generative strexp
\label{generative-strexp-rule}
\frac{\B\ts\strdec\ra\E\qquad\m\notin(\of{\N}{\B})\cup\NamesFcn\E}
     {\B\ts\encstrexp\ra(\m,\E)}\index{39.2}
\end{equation}}{\begin{equation}        % generative strexp
\label{generative-strexp-rule}
\frac{\B\ts\strdec\ra\E}
     {\B\ts\encstrexp\ra \E }\index{39.2}
\end{equation}}
\replacement{\thenostrsharing}{
\begin{equation}        % longstrid
%\label{longstrid-strexp-rule}
\frac{\B(\longstrid)=\S}
     {\B\ts\longstrid\ra\S}
\end{equation}}{\begin{equation}        % longstrid
%\label{longstrid-strexp-rule}
\frac{\B(\longstrid)=\E}
     {\B\ts\longstrid\ra\E}
\end{equation}}

\insertion{\thenostrsharing}{
\begin{equation}
\label{transparent-constraint-rule}
\frac{B\ts\strexp\ra\E\quad\B\ts\sigexp\ra\Sigma\quad\Sigma\geq\E'\prec\E}
     {\B\ts\transpconstraint\ra\E'}
\end{equation}
}

\insertion{\thenostrsharing}{
\begin{equation}
\label{opaque-constraint-rule}
\frac{\begin{array}{c}
   B\ts\strexp\ra\E\quad\B\ts\sigexp\ra(\T')\E'\\
   (\T')\E'\geq\E''\prec\E\quad \T' \cap(\of{\T}{\B}) = \emptyset
      \end{array}}
     {\B\ts\opaqueconstraint\ra\E'}
\end{equation}
}

\vspace{6pt}
\replacement{\thenostrsharing}{
\begin{equation}                % functor application
\label{functor-application-rule}
\frac{ \begin{array}{c}
        \B\ts\strexp\ra\S\\
        \funsiginst{\B(\funid)}{}{(\S'',(\N')\S')}\ ,
                                                    \ \S\succ\S''\\
        (\of{\N}{\B})\cap\N'=\emptyset
       \end{array}
     }
     {\B\ts\funappstr\ra\S'}
\end{equation}}{\begin{equation}                % functor application
\label{functor-application-rule}
\frac{ \begin{array}{c}
        \B\ts\strexp\ra\E\\
        \funsiginst{\B(\funid)}{}{(\E'',(\T')\E')}\ ,
                                                    \ \E\succ\E''\\
        (\TyNamesFcn \E\; \cup\; \of{\T}{\B})\cap\T'=\emptyset
       \end{array}
     }
     {\B\ts\funappstr\ra\E'}
\end{equation}}

\vspace{6pt}
\replacement{\thenostrsharing}{
\begin{equation}        % let strexp
\label{letstrexp-rule}
\frac{\B\ts\strdec\ra\E\qquad\B\oplus\E\ts\strexp\ra\S}
     {\B\ts\letstrexp\ra\S}
\end{equation}}{\begin{equation}        % let strexp
\label{letstrexp-rule}
\frac{\B\ts\strdec\ra\E_1\qquad\B\oplus\E_1\ts\strexp\ra\E_2}
     {\B\ts\letstrexp\ra\E_2}
\end{equation}}
\comments
\begin{description}
%
\oldpagebreak
\item{(\ref{functor-application-rule})}
   The side condition 
\replacement{\thenostrsharing}{$ (\of{\N}{\B})\cap\N'=\emptyset$}{$(\TyNamesFcn\E \cup \of{\T}{\B})\cap\T'=\emptyset$}
  can always
be satisfied by renaming bound names in \replacement{\thenostrsharing}{$(\N')S'$}{$(\T')E'$}; it ensures that the
generated \replacement{\thenostrsharing}{structures}{datatypes} receive new names.\index{40.1}

\replacement{\thenostrsharing}{Let $\B(\funid)=(N)(\S_f,(N')\S_f')$.}{Let $\B(\funid)=(\T)(\E_f,(T')\E_f')$.} 
\replacement{\thenotypexplication}{Assuming that $(\N)\S_f$ is
type-explicit, the realisation $\rea$ for which
$\rea(\S_f,(N')\S_f')=(\S'',(\N')\S')$ is uniquely determined by $\S$,
since $\S\succ\S''$ can only hold if the type names and structure
names in $\S$ and $\S''$ agree.  Recall that enrichment $\succ$ allows
more components and more polymorphism, while instantiation $\geq$ does
not.\par}{Let $\rea$ be a realisation such that\linebreak
$\rea(\E_f,(T')\E_f')=(\E'',(\T')\E')$.}
\replacement{\thenostrsharing}{
Sharing between argument and result specified in the declaration of
the functor $\funid$ is represented by the occurrence of the same name
in both $\S_f$ and $\S_f'$, and this repeated occurrence is preserved
by $\rea$, yielding sharing between the argument structure $\S$ and
the result structure $\S'$ of this functor application.}{
Sharing between argument and result specified in the declaration of
the functor $\funid$ is represented by the occurrence of the same name
in both $\E_f$ and $\E_f'$, and this repeated occurrence is preserved
by $\rea$, yielding sharing between the argument structure $\E$ and
the result structure $\E'$ of this functor application.}
%
\item{(\ref{letstrexp-rule})}
   The use of $\oplus$, here and elsewhere, ensures that \deletion{\thenostrsharing}{structure
and }type names generated by
the first sub-phrase
are distinct from names generated by the second
sub-phrase.
\end{description}

%                              declarations
\rulesec{Structure-level Declarations}{\B\ts\strdec\ra\E}               
\replacement{\thenostrsharing}{
\begin{equation}                % core declaration
\label{dec-rule}
\frac{ \of{\C}{\B}\ts\dec\ra\E
       \quad\E\ {\rm principal\ for\ \dec\ in\ } (\of{\C}{\B})
}
     { \B\ts\dec\ra\E }\index{40.2}
\end{equation}}{\begin{equation}                % core declaration
\label{dec-rule}
\frac{ \of{\C}{\B}\ts\dec\ra\E
}
     { \B\ts\dec\ra\E }\index{40.2}
\end{equation}}

\vspace{6pt}
\begin{equation}                % structure declaration
%\label{structure-decl-rule}
\frac{ \B\ts\strbind\ra\SE }
     { \B\ts\singstrdec\ra\SE\ \In\ \Env }
\end{equation}

\vspace{6pt}
\begin{equation}                % local structure-level declaration
%\label{local structure-level declaration}
\frac{ \B\ts\strdec_1\ra\E_1\qquad
       \B\oplus\E_1\ts\strdec_2\ra\E_2 }
     { \B\ts\localstrdec\ra\E_2 }
\end{equation}

\vspace{6pt}
\begin{equation}                % empty declaration
%\label{empty-strdec-rule}
\frac{}
     {\B\ts\emptystrdec\ra \emptymap{\rm\ in}\ \Env}
\end{equation}

\vspace{6pt}
\begin{equation}                % sequential declaration
%\label{sequential-strdec-rule}
\frac{ \B\ts\strdec_1\ra\E_1\qquad
       \B\oplus\E_1\ts\strdec_2\ra\E_2 }
     { \B\ts\seqstrdec\ra\plusmap{\E_1}{\E_2} }
\end{equation}
\deletion{\thenostrsharing}{
\comments
\begin{description}
\item{(\ref{dec-rule})}
The side condition ensures that all type schemes in $\E$ are as
general as possible.
% and that no imperative type variables occur
%free in $\E$.
%from version 1:
%   The side condition ensures that all type schemes in $\E$ are as
%general as possible and that all new type names in $\E$ admit
%equality, if possible.
\end{description}}
\oldpagebreak
\rulesec{Structure Bindings}{\B\ts\strbind\ra\SE}
\begin{equation}                % structure binding
\label{structure-binding-rule}
\frac{ 
       \B\ts\strexp\ra\E\quad
       \langle\plusmap{\B}{\TyNamesFcn\E}\ts
                                      \strbind\ra\SE\rangle
     }
     { \B\ts\barestrbindera\ra\{\strid\mapsto\E\}
       \ \langle +\ \SE\rangle}\index{41.1}
\end{equation}%
%
%                   Signature Rules
%
\rulesec{Signature Expressions}{\B\ts\sigexp\ra\E}
\begin{equation}                % encapsulation sigexp
\label{encapsulating-sigexp-rule}
\frac{\B\ts\spec\ra\E }
     {\B\ts\encsigexp\ra  \E }\index{41.2}
\end{equation}

\begin{equation}                % signature identifier
\label{signature-identifier-rule}
\frac{  \B(\sigid) = (\T)\E \quad \T\cap (\of{\T}{\B}) = \emptyset}
     { \B\ts\sigid\ra\E }
\end{equation}

\begin{equation}
\label{wheretype-rule}
\frac{
  \begin{array}{c}
     \B\ts\sigexp\ra \E\quad \tyvarseq = \alphak\quad \of{\C}{\B}\ts\ty\ra \tau\\
     \E(\longtycon) = (\t, \VE)\quad t\notin\of{\T}{\B} \quad \FIX{t\in\TyNamesk} \\
     \rea = \{\t\mapsto \Lambda\alphak.\tau\}\quad
     \hbox{$\Lambda\alphak.\tau$ admits equality, if $\t$ does\quad $\rea(\E)$ well-formed}
  \end{array}
 }
 {\B\ts\wheretypesigexp\ra\rea(\E)}
\end{equation}%
\comments
\begin{description}
\deletion{\thenostrsharing}{
\item{(\ref{encapsulating-sigexp-rule})}
   In contrast to rule~\ref{generative-strexp-rule}, $m$ is not here 
required to be new. 
The name $m$ may be chosen to achieve the sharing required
in rule~\ref{strshareq-rule}, or to achieve the enrichment side conditions
of rule~\ref{structure-binding-rule} or \ref{funbind-rule}. 
The choice of $m$ must result in an admissible object.}
\item{(\ref{signature-identifier-rule})}
   \replacement{\thenostrsharing}{The instance $\S$ of $\B(\sigid)$ is not determined by this rule,
but -- as in rule~\ref{encapsulating-sigexp-rule} -- the instance
may  be chosen to achieve sharing properties or enrichment
conditions.}{The bound names of $\B(\sigid)$ can always be renamed to satisfy $\T\cap(\of{\T}{\B}) = \emptyset$,
if necessary.}
\end{description}

\rulesec{}{\B\ts\sigexp\ra\sig}
\replacement{\thenostrsharing}{\begin{equation}                % any sigexp
\label{topmost-sigexp-rule}
\frac{\begin{array}{c}
\B\ts\sigexp\ra\S\quad\mbox{$(\N)\S$ equality-principal for $\sigexp$ in $\B$}\\
\mbox{$(\N)\S$ type-explicit}
      \end{array}}
     {\B\ts\sigexp\ra (\N)\S}\index{41.25}
\end{equation}}{\begin{equation}                % any sigexp
\label{topmost-sigexp-rule}
\frac{
\B\ts\sigexp\ra\E\quad\T= \TyNamesFcn\E\setminus(\of{\T}{\B})
}
     {\B\ts\sigexp\ra (\T)\E}\index{41.25}
\end{equation}}%
\noindent\comment
A signature expression $\sigexp$ which is an immediate constituent of
\deletion{\thenostrsharing}{a structure binding, } a signature binding\replacement{\thenostrsharing}{, a 
functor binding or a
functor signature}{, a signature constraint, or a
functor binding }is elaborated to \replacement{\thenostrsharing}{an equality-principal and type-explicit }{a }signature,
 see rules~\replacement{\thenostrsharing}{\ref{structure-binding-rule}, }{
\ref{transparent-constraint-rule}, \ref{opaque-constraint-rule}, }\ref{sigbind-rule}\deletion{\thenostrsharing}{, 
\ref{funsigexp-rule}} and \ref{funbind-rule}.  \deletion{\thenostrsharing}{By contrast, signature 
expressions occurring in structure descriptions are elaborated to
structures using the liberal rules
\ref{encapsulating-sigexp-rule} and \ref{signature-identifier-rule}, 
see rule~\ref{strdesc-rule}, so that names can be chosen to achieve
sharing, when necessary.}
\oldpagebreak

\rulesec{Signature Declarations}{\B\ts\sigdec\ra\G}
\begin{equation}        % single signature declaration
\label{single-sigdec-rule}
\frac{ \B\ts\sigbind\ra\G }
     { \B\ts\singsigdec\ra\G }\index{41.3}
\end{equation}
\deletion{\thenostrsharing}{
\begin{equation}        % empty signature declaration
%\label{empty-sigdec-rule}
\frac{}
     { \B\ts\emptysigdec\ra\emptymap }
\end{equation}

\begin{equation}        % sequential signature declaration
\label{sequence-sigdec-rule}
\frac{ \B\ts\sigdec_1\ra\G_1 \qquad \plusmap{\B}{\G_1}\ts\sigdec_2\ra\G_2 }
     { \B\ts\seqsigdec\ra\plusmap{\G_1}{\G_2} }
\end{equation}}
\deletion{\thenostrsharing}{\comments
\begin{description}
%
\item{(\ref{single-sigdec-rule})}
The first closure restriction of Section~\ref{closure-restr-sec}
can be  enforced by replacing the $\B$ in the premise by $\B_0+\of{\G}{\B}$.
\item{(\ref{sequence-sigdec-rule})}
   A signature declaration does not create any new structures
or types; hence the use of $+$ instead of $\oplus$.
\end{description}
}

\rulesec{Signature Bindings}{\B\ts\sigbind\ra\G}
\begin{equation}        % signature binding
\label{sigbind-rule}
\frac{ \B\ts\sigexp\ra\sig
        \qquad\langle\B\ts\sigbind\ra\G\rangle }
     { \B\ts\sigbinder\ra\{\sigid\mapsto\sig\}
       \ \langle +\ \G\rangle }\index{42.1}
\end{equation}%
%
                     % Specifications
\rulesec{Specifications}{\B\ts\spec\ra\E}
\begin{equation}        % value specification
\label{valspec-rule}
\frac{ \of{\C}{\B}\ts\valdesc\ra\VE }
     { \B\ts\valspec\ra\cl{}{\VE}\ \In\ \Env }\index{42.2}
\end{equation}

\replacement{\thenostrsharing}{
\begin{equation}        % type specification
\label{typespec-rule}
\frac{ \of{\C}{\B}\ts\typdesc\ra\TE }
     { \B\ts\typespec\ra\TE\ \In\ \Env }
\end{equation}}{\begin{equation}        % type specification
\label{typespec-rule}
\frac{ 
         \of{\C}{\B}\ts\typdesc\ra\TE \quad
         \forall(\t,\VE)\in\Ran\TE,\hbox{\ $t$ does not admit equality}
     }
     { \B\ts\typespec\ra\TE\ \In\ \Env }
\end{equation}}

\replacement{\thenostrsharing}{
\begin{equation}        % eqtype specification
\label{eqtypspec-rule}
\frac{ \of{\C}{\B}\ts\typdesc\ra\TE \qquad
       \forall(\theta,\CE)\in \Ran\TE,\ \theta {\rm\ admits\ equality} }
     { \B\ts\eqtypespec\ra\TE\ \In\ \Env }
\end{equation}}{\begin{equation}        % eqtype specification
\label{eqtypspec-rule}
\frac{ \of{\C}{\B}\ts\typdesc\ra\TE \qquad
       \forall(\t,\VE)\in \Ran\TE,\ \t {\rm\ admits\ equality} }
     { \B\ts\eqtypespec\ra\TE\ \In\ \Env }
\end{equation}}

\replacement{\thenostrsharing}{
\begin{equation}        % data specification
\label{datatypespec-rule}
\frac{ \plusmap{\of{\C}{\B}}{\TE}\ts\datdesc\ra\VE,\TE }
     { \B\ts\datatypespec\ra(\VE,\TE)\ \In\ \Env }
\end{equation}}{
\begin{equation}        % data specification
\label{datatypespec-rule}
\frac{ \begin{array}{c}
       \of{\C}{\B}\oplus\TE\ts\datdesc\ra\VE,\TE 
       \quad \forall(\t,\VE')\in\Ran\TE, \t\notin\of{\T}{\B}\\
       \hbox{$\TE$ maximises equality}
       \end{array}}
     { \B\ts\datatypespec\ra(\VE,\TE)\ \In\ \Env }
\end{equation}}

\insertion{\thedatatyperepl}{\begin{equation}
\label{datatypereplspec-rule}
\frac{ \B(\longtycon) = (\typefcn,\VE)\qquad
       \TE = \{\tycon\mapsto(\typefcn,\VE)\}
     }
     {\B\ts\datatypereplspec\ra (\VE,\TE)\ \In\ \Env}
\end{equation}}

\replacement{\theidstatus}{
\begin{equation}        % exception specification
\label{exceptionspec-rule}
\frac{ \of{\C}{\B}\ts\exndesc\ra\EE\quad\VE=\EE }
     { \B\ts\exceptionspec\ra(\VE,\EE)\ \In\ \Env }
\end{equation}}{\begin{equation}        % exception specification
\label{exceptionspec-rule}
\frac{ \of{\C}{\B}\ts\exndesc\ra\VE }
     { \B\ts\exceptionspec\ra\VE\ \In\ \Env }
\end{equation}}

\begin{equation}        % structure specification
%\label{structurespec-rule}
\frac{ \B\ts\strdesc\ra\SE }
     { \B\ts\structurespec\ra\SE\ \In\ \Env }
\end{equation}
\oldpagebreak

\deletion{\thetypabbr}{
\begin{equation}        % sharing specification
%\label{sharingspec-rule}
\frac{ \B\ts\shareq\ra\emptymap }
     { \B\ts\sharingspec\ra\emptymap\ \In\ \Env }\index{42.3}
\end{equation}}

\deletion{\thenolocalspec}{\begin{equation}        % local specification
%\label{localspec-rule}
\frac{ \B\ts\spec_1\ra\E_1 \qquad \plusmap{\B}{\E_1}\ts\spec_2\ra\E_2 }
     { \B\ts\localspec\ra\E_2 }
\end{equation}}

\deletion{\thenoopenspec}{\begin{equation}        % open specification
%\label{openspec-rule}
\frac{ \B(\longstrid_1)=(\m_1,\E_1)\quad\cdots\quad
       \B(\longstrid_n)=(\m_n,\E_n) }
     { \B\ts\openspec\ra\E_1 + \cdots +\E_n }
\end{equation}}

\replacement{\thesingleincludespec}{\begin{equation}        % include signature specification
\label{inclspec-rule}
\frac{ \sigord{\B(\sigid_1)}{}{(\m_1,\E_1)} \quad\cdots\quad
       \sigord{\B(\sigid_n)}{}{(\m_n,\E_n)} }
     { \B\ts\inclspec\ra\E_1 + \cdots +\E_n }
\end{equation}}{\begin{equation}        % include signature specification
\label{inclspec-rule}
\frac{  \B\ts\sigexp\ra\E}
     { \B\ts\singleinclspec\ra\E }
\end{equation}}

\begin{equation}        % empty specification
%\label{emptyspec-rule}
\frac{}
     { \B\ts\emptyspec\ra\emptymap{\rm\ in}\ \Env }
\end{equation}

\replacement{\thenostrsharing}{
\begin{equation}        % sequential specification
%\label{seqspec-rule}
\frac{ \B\ts\spec_1\ra\E_1 \qquad \plusmap{\B}{\E_1}\ts\spec_2\ra\E_2 }
     { \B\ts\seqspec\ra\plusmap{\E_1}{\E_2} }
\end{equation}}{\begin{equation}        % sequential specification
\label{seqspec-rule}
\frac{ \B\ts\spec_1\ra\E_1 \qquad \B\oplus\E_1\ts\spec_2\ra\E_2\qquad\Dom(\E_1)\cap\Dom(\E_2) = \emptyset }
     { \B\ts\seqspec\ra\plusmap{\E_1}{\E_2} }
\end{equation}}

\begin{equation}
\label{sharspec-rule}
\frac{\begin{array}{c}
        \B\ts \spec \ra \E\quad
          \hbox{$\E(\longtycon_i) = (\t_i,\VE_i)$ \FIX{and $\t_i\in\TyNamesk$}, $i = 1..n$}\\
        t\in\{\t_1,\ldots,\t_n\}\quad\hbox{$\t$ admits equality, if some $\t_i$ does}\\
        \{\t_1,\ldots,\t_n\}\cap\of{\T}{\B} = \emptyset\quad
           \rea = \{\t_1\mapsto\t,\ldots,\t_n\mapsto \t\}
      \end{array}
     }
     {\B\ts \newsharingspec\ra \rea(\E)}
\end{equation}%

\pagebreak
\noindent\comments
\begin{description}
\item{(\ref{valspec-rule})}
   $\VE$ is determined by $\B$ and $\valdesc$.
\item{(\ref{typespec-rule})--(\ref{datatypespec-rule})}
   \replacement{\thenostrsharing}{The type functions in $\TE$ may be chosen to achieve the sharing hypothesis
of rule~\ref{typshareq-rule} or the enrichment conditions of 
rules~\ref{structure-binding-rule} and~\ref{funbind-rule}. In particular, the type
names in $\TE$ in rule~\ref{datatypespec-rule} need not be new.
Also, in rule~\ref{typespec-rule} the type functions in $\TE$ may admit
equality.}{The type
names in $\TE$ are new.}
%
\item{(\ref{exceptionspec-rule})}
   \replacement{\theidstatus}{$\EE$}{$\VE$} is determined by $\B$ and $\exndesc$ and contains monotypes only.
\deletion{\thenostrsharing}{\item{(\ref{inclspec-rule})}
   The names $\m_i$ in the instances may be chosen to achieve sharing or
enrichment conditions.\index{43.0}}
\insertion{\thenostrsharing}{\item{(\ref{seqspec-rule})}
   Note that no sequential specification is allowed to specify the
   same identifier twice.}
\end{description} 

                  % Descriptions
\rulesec{Value Descriptions}{\C\ts\valdesc\ra\VE}
\replacement{\theidstatus}{
\begin{equation}         % value description
%\label{valdesc-rule}
\frac{ \C\ts\ty\ra\tau\qquad
       \langle\C\ts\valdesc\ra\VE\rangle }
     { \C\ts\valdescription\ra\{\var\mapsto\tau\}
       \ \langle +\ \VE\rangle }\index{43.1}
\end{equation}}{\begin{equation}         % value description
%\label{valdesc-rule}
\frac{ \C\ts\ty\ra\tau\qquad
       \langle\C\ts\valdesc\ra\VE\rangle }
     { \C\ts\valviddescription\ra\{\vid\mapsto(\tau,\isv)\}
       \ \langle +\ \VE\rangle }\index{43.1}
\end{equation}}

\rulesec{Type Descriptions}{\C\ts\typdesc\ra\TE}
\begin{equation}         % type description
\label{typdesc-rule}
\frac{ \begin{array}{c}
         \tyvarseq = \alphak\quad\t\notin\of{\T}{\C}\quad\arity\t=k \\
       \langle \C\ts\typdesc\ra\TE\qquad\t\notin\TyNamesFcn\TE\rangle
      \end{array}}
     { \C\ts\typdescription\ra\{\tycon\mapsto(\t,\emptymap)\}
       \ \langle +\ \TE\rangle }\index{43.2}
\end{equation}%
\comment Note that \deletion{\thenostrsharing}{any $\theta$ of arity $k$ may be chosen but that}
the \replacement{\thece}{constructor}{value} environment in the resulting type structure must be
empty. \replacement{\thefixtypos}{For example, \mbox{\ml{datatype s=c type t sharing s=t}}\  }{For example, \mbox{\ml{datatype s=C type t}} \mbox{\ml{sharing type t=s}}\  }
is a legal specification, but the type structure bound to \ml{t}
does not bind any value constructors.
\oldpagebreak

\rulesec{Datatype Descriptions}{\C\ts\datdesc\ra\VE,\TE}
\replacement{\thenostrsharing}{\begin{equation}         % datatype description
\label{datdesc-rule}
\frac{ \tyvarseq = \alphak\qquad\C,\alphakt\ts\condesc\ra\CE
       \qquad\langle\C\ts\datdesc\ra\VE,\TE\rangle }
     { \begin{array}{cl}
       \C\ts\datdescription\ra\\
       \qquad\qquad\cl{}{\CE}\langle +\ \VE\rangle,\
       \{\tycon\mapsto(t,\cl{}{\CE})\}\ \langle +\ \TE\rangle
       \end{array}
     }\index{43.3}
\end{equation}}{\begin{equation}         % datatype description
\label{datdesc-rule}
\frac{ \begin{array}{c}
        \tyvarseq = \alphak\qquad\C,\alphakt\ts\condesc\ra\VE
           \quad\arity \t  =  k\\
        \langle\C\ts\datdesc'\ra\VE',\TE'\qquad \forall(\t',\VE'')\in\Ran\TE',\,\t\neq\t'\rangle 
       \end{array}}
     { \begin{array}{l}
       \C\ts\newdatdescription\ra\\
       \qquad\cl{}{\VE} \langle +\ \VE'\rangle,\
       \{\tycon\mapsto(t,\cl{}{\VE})\}\ \langle +\ \TE'\rangle
       \end{array}
     }\index{43.3}
\end{equation}}

\replacement{\thece}{\rulesec{Constructor Descriptions}{\C,\tau\ts\condesc\ra\CE}}{\rulesec{Constructor Descriptions}{\C,\tau\ts\condesc\ra\VE}}
\replacement{\theidstatus}{\begin{equation}         % constructor description
%\label{condesc-rule}
\frac{\langle\C\ts\ty\ra\tau'\rangle\qquad
      \langle\langle\C,\tau\ts\condesc\ra\CE\rangle\rangle }
     {\begin{array}{l}
      \C,\tau\ts\longcondescription\ra\\
      \qquad\{\con\mapsto\tau\}\
     \langle +\ \{\con\mapsto\tau'\to\tau\}\ \rangle\
      \langle\langle +\ \CE\rangle\rangle
      \end{array}
     }\index{43.35}
\end{equation}}{\begin{equation}         % constructor description
%\label{condesc-rule}
\frac{\langle\C\ts\ty\ra\tau'\rangle\qquad
      \langle\langle\C,\tau\ts\condesc\ra\VE\rangle\rangle }
     {\begin{array}{l}
      \C,\tau\ts\longconviddescription\ra\\
      \qquad\{\vid\mapsto(\tau,\isc)\}\
     \langle +\ \{\vid\mapsto(\tau'\to\tau,\isc)\}\ \rangle\
      \langle\langle +\ \VE\rangle\rangle
      \end{array}
     }\index{43.35}
\end{equation}}

\replacement{\theidstatus}{
\rulesec{Exception Descriptions}{\C\ts\exndesc\ra\EE}}{\rulesec{Exception 
Descriptions}{\C\ts\exndesc\ra\VE}}
\replacement{\theidstatus}{
\begin{equation}         % exception description
\label{exndesc-rule}
\frac{ \langle\C\ts\ty\ra\tau\qquad\TyVarsFcn(\tau)=\emptyset\rangle\qquad
       \langle\langle\C\ts\exndesc\ra\EE\rangle\rangle }
     { \begin{array}{l}
        \C\ts\exndescriptiona\ra\\
        \quad\quad\{\exn\mapsto\EXCN\}\ \langle +\ \{\exn\mapsto\tau\rightarrow\EXCN\}\rangle\ \langle\langle +\ \EE\rangle\rangle 
       \end{array}
     }\index{43.4}
\end{equation}}{\begin{equation}         % exception description
\label{exndesc-rule}
\frac{ \langle\C\ts\ty\ra\tau\qquad\TyVarsFcn(\tau)=\emptyset\rangle\qquad
       \langle\langle\C\ts\exndesc\ra\VE\rangle\rangle }
     { \begin{array}{l}
        \C\ts\exnviddescriptiona\ra\\
        \quad\quad\{\vid\mapsto(\EXCN,\ise)\}\ \langle +\ \{\vid\mapsto(\tau\rightarrow\EXCN,\ise)\}\rangle\ \langle\langle +\ \VE\rangle\rangle 
       \end{array}
     }\index{43.4}
\end{equation}}


\rulesec{Structure Descriptions}{\B\ts\strdesc\ra\SE}
\replacement{\thenostrsharing}{
\begin{equation}
\label{strdesc-rule}
\frac{ \B\ts\sigexp\ra\S\qquad\langle\B\ts\strdesc\ra\SE\rangle }
     { \B\ts\strdescription\ra\{\strid\mapsto\S\}\ \langle +\ \SE\rangle }\index{43.5}
\end{equation}}{\begin{equation}
\label{strdesc-rule}
\frac{ \B\ts\sigexp\ra\E\qquad\langle\B + \TyNamesFcn\E\ts\strdesc\ra\SE\rangle }
     { \B\ts\strdescription\ra\{\strid\mapsto\E\}\ \langle +\ \SE\rangle }\index{43.5}
\end{equation}}


\deletion{\thenostrsharing}{
\rulesec{Sharing Equations}{\B\ts\shareq\ra\emptymap}
\begin{equation}          % structure sharing equation
\label{strshareq-rule}
\frac{ \of{\m}{\B(\longstrid_1)}=\cdots =\of{\m}{\B(\longstrid_n)} }
     { \B\ts\strshareq\ra\E,\emptymap }\index{44.1}
\end{equation}
\vspace{6pt}
\begin{equation}          % type sharing equation
\label{typshareq-rule}
\frac{ \of{\typefcn}{\B(\longtycon_1)}=\cdots=\of{\typefcn}{\B(\longtycon_n)} }
     { \B\ts\typshareq\ra\emptymap }
\end{equation}

\vspace{6pt}
\begin{equation}          % multiple sharing equation
%\label{multshareq-rule}
\frac{ \B\ts\shareq_1\ra\emptymap\qquad\B\ts\shareq_2\ra\emptymap }
     { \B\ts\multshareq\ra\emptymap }
\end{equation}%

%
\comments
\begin{description}
\item{(\ref{strshareq-rule})}
   By the definition of consistency the premise is weaker than\linebreak
$\B(\longstrid_1) = \cdots = \B(\longstrid_n)$.
Two different structures with the same name may be thought of
as representing different views. The requirement that $\B$ is 
consistent forces different views to be consistent.
\end{description}
%
\oldpagebreak
\begin{description}
\item{(\ref{typshareq-rule})}
   By\index{44.1.5} 
the definition of consistency the premise is weaker than\linebreak
$\B(\longtycon_1) = \cdots = \B(\longtycon_n)$.
A type structure with empty constructor environment may have the
same type name as one with a non-empty constructor environment;
the former could arise from a type description, and the latter
from a datatype description. 
However, the requirement that $\B$ is
consistent will prevent two type structures with constructor
environments which have different 
non-empty domains from sharing the same type name.
end{description}
}
%
%
%                       Type abbreviation rules
%
%
%
\deletion{\thenofuncspec}{
%
%                       Functor Specification rules
%
\rulesec{Functor Specifications}{\B\ts\funspec\ra\F}
\begin{equation}        % single functor specification
\label{singfunspec-rule}
\frac{ \B\ts\fundesc\ra\F }
     { \B\ts\singfunspec\ra\F }\index{44.2}
\end{equation}

\vspace{6pt}
\begin{equation}        % empty functor specification
%\label{emptyfunspec-rule}
\frac{}
     { \B\ts\emptyfunspec\ra\emptymap }
\end{equation}

\vspace{6pt}
\begin{equation}        % sequential functor specification
%\label{seqfunspec-rule}
\frac{ \B\ts\funspec_1\ra\F_1\qquad
       \B+\F_1\ts\funspec_2\ra\F_2 }
     { \B\ts\seqfunspec\ra\plusmap{\F_1}{\F_2} }
\end{equation}
\comments
\begin{description}
\item{(\ref{singfunspec-rule})}
The second closure restriction of Section~\ref{closure-restr-sec}
can be enforced by replacing the $\B$ in the premise by $\B_0+\of{\G}{\B}$.
\end{description}
\rulesec{Functor Descriptions}{\B\ts\fundesc\ra\F}
\begin{equation}        % functor description
%\label{fundesc-rule}
\frac{ \B\ts\funsigexp\ra\funsig\qquad
       \langle\B\ts\fundesc\ra\F\rangle}
     { \B\ts\longfundesc\ra\{\funid\mapsto\funsig\}
       \langle +\ \F\rangle}\index{44.3}
\end{equation}

\rulesec{Functor Signature Expressions}{\B\ts\funsigexp\ra\funsig}
\begin{equation}        % functor signature
\label{funsigexp-rule}
%version 1:
%\frac{
%      \begin{array}{c}
%      \B\ts\sigexp\ra\S\qquad\longsig{}{\rm\ principal\ in\ }\B\\
%      \B\oplus\{\strid\mapsto\S\} \ts\sigexp'\ra\S'\\
%      \N' = \NamesFcn\S'\setminus((\of{\N}{\B})\cup\N) 
%      \end{array}
%     }
%     {\B\ts\longfunsigexpa\ra(\N)(\S,(\N')\S')}\index{44.4}
%version2: \frac{\begin{array}{rl}
%      \B\ts\sigexp\ra\S&\mbox{$(N)S$ principal in $\B$}\\
%      \B\oplus\{\strid\mapsto\S\}\ts\sigexp'\ra\S'&
%      \mbox{$(N')S'$ principal in $\B\oplus\{\strid\mapsto\S\}$}
%      \end{array}}
%     {\B\ts\longfunsigexpa\ra(N)(S,(N')S')}\index{44.4}
%\end{equation}
\frac{\B\ts\sigexp\ra(\N)\S\qquad
      \B\oplus\{\strid\mapsto\S\}\ts\sigexp'\ra(\N')\S'}
     {\B\ts\longfunsigexpa\ra(N)(S,(N')S')}\index{44.4}
\end{equation}%
\comment
The signatures $(\N)\S$ and $(\N')\S'$ are equality-principal 
and type-explicit, see rule~\ref{topmost-sigexp-rule}.
} %deletion
%                       Functor and Program rules

\rulesec{Functor Declarations}{\B\ts\fundec\ra\F}
\begin{equation}        % single functor declaration
\label{singfundec-rule}
\frac{ \B\ts\funbind\ra\F }
     { \B\ts\singfundec\ra\F }\index{45.1}
\end{equation}

\deletion{\thenostrsharing}{
\vspace{6pt}
\begin{equation}        % empty functor declaration
%\label{emptyfundec-rule}
\frac{}
     { \B\ts\emptyfundec\ra\emptymap }
\end{equation}

\vspace{6pt}
\oldpagebreak

\begin{equation}        % sequential functor declaration
%\label{seqfundec-rule}
\frac{ \B\ts\fundec_1\ra\F_1\qquad
       \B+\F_1\ts\fundec_2\ra\F_2 }
     { \B\ts\seqfundec\ra\plusmap{\F_1}{\F_2} }\index{45.1.5}
\end{equation}}
\deletion{\thenoclosurerestriction}{\comments
\begin{description}
\item{(\ref{singfundec-rule})}
The third closure restriction of Section~\ref{closure-restr-sec}
can be enforced by replacing the $\B$ in the premise 
by $\B_0+(\of{\G}{\B})+(\of{\F}{\B})$.
\end{description}}
\rulesec{Functor Bindings}{\B\ts\funbind\ra\F}
\replacement{\thenostrsharing}{\begin{equation}        % functor binding
\label{funbind-rule}
\frac{
      \begin{array}{c}
      \B\ts\sigexp\ra(\N)\S\qquad
      \B\oplus\{\strid\mapsto\S\} \ts\strexp\ra\S' \\
       \langle
      \B\oplus\{\strid\mapsto\S\} \ts\sigexp'\ra\sig',\ \sig'\geq\S''\prec\S'
       \rangle\\
      \N' = \NamesFcn\S'\setminus((\of{\N}{\B})\cup\N) \\
       \langle\langle\B\ts\funbind\ra\F\rangle\rangle
      \end{array}
     }
     {
      \begin{array}{c}
       \B\ts\funstrbinder\ \optfunbind\ra\\
       \qquad\qquad \qquad
              \{\funid\mapsto(\N)(\S,(\N')\S'\langle'\rangle)\}
              \ \langle\langle +\ \F\rangle\rangle
      \end{array}
     }\index{45.2}
\end{equation}}{\begin{equation}        % functor binding
\label{funbind-rule}
\frac{
      \begin{array}{c}
      \B\ts\sigexp\ra(\T)\E\qquad
      \B\oplus\{\strid\mapsto\E\} \ts\strexp\ra\E' 
      \\
      \T\cap(\of{\T}{\B}) = \emptyset\quad \T' = \TyNamesFcn\E'\setminus((\of{\T}{\B})\cup\T) \\
       \langle\B\ts\funbind\ra\F\rangle
      \end{array}
     }
     {
      \begin{array}{c}
       \B\ts\barefunstrbinder\ \langle\boxml{and \funbind}\rangle\ra\\
       \qquad\qquad \qquad
              \{\funid\mapsto(\T)(\E,(\T')\E')\}
              \ \langle +\ \F\rangle
      \end{array}
     }\index{45.2}
\end{equation}
}\comment \deletion{\thenostrsharing}{The  requirement that $(\N)\S$ be equality-principal,
implicit in the first premise, forces $(\N)\S$ to be
as general as possible given the sharing constraints in $\sigexp$.
The requirement that $(\N)\S$ be type-explicit ensures that there is
at most one realisation via which an actual argument can match
$(\N)\S$.}Since $\oplus$ is used, any \deletion{\thenostrsharing}{structure name $\m$ and}type name $\t$ in
\replacement{\thenostrsharing}{$\S$ }{$\E$ }acts like a constant in the functor body; in particular,
it ensures that further names generated during elaboration of the
body are distinct from \deletion{\thenostrsharing}{$\m$ and }$\t$. 
\replacement{\thenostrsharing}{The set $\N'$ is
chosen such that every  name free
in $(\N)\S$ or $(\N)(\S,(\N')\S')$ is free in $\B$.}{The set $\T'$ is
chosen such that every  name free
in $(\T)\E$ or $(\T)(\E,(\T')\E')$ is free in $\B$.}
\rulesec{Top-level Declarations}{\B\ts\topdec\ra\B'}
%\rulesec{Programs}{\B\ts\program\ra\B'}
\replacement{\thenostrsharing}{\begin{equation}	% structure-level declaration
\label{strdectopdec-rule}
\frac{\B\ts\strdec\ra\E \quad\imptyvars\E=\emptyset}
     {\B\ts\strdec\ra
      (\NamesFcn\E,\E)\ \In\ \Basis
     }\index{45.3}
\end{equation}}{\begin{equation}        % structure-level declaration
\label{strdectopdec-rule}
\frac{\begin{array}{c}
        \B\ts\strdec\ra\E \quad \langle \B\oplus \E\ts \topdec\ra \B'\rangle\\
        B'' = (\TyNamesFcn\E,\E)\In\ \Basis\; \langle + \B'\rangle\quad\TyVarFcn\B''=\emptyset
      \end{array}}
     {\B\ts\strdecintopdec\ra\B''}
\end{equation}}

\vspace{6pt}
\replacement{\thenostrsharing}{\begin{equation}	% signature declaration
%\label{sigdectopdec-rule}
\frac{\B\ts\sigdec\ra\G \quad\imptyvars\G=\emptyset}
     {\B\ts\sigdec\ra
      (\NamesFcn\G,\G)\ \In\ \Basis
     }\index{46.0}
\end{equation}}{\begin{equation}        % signature declaration
\frac{\begin{array}{c}
        \B\ts\sigdec\ra \G\quad \langle\B\oplus\G\ts\topdec\ra\B'\rangle\\
        \B'' = (\TyNamesFcn\G,G)\ \In\ \Basis\;\langle + \B'\rangle
      \end{array}}
     {\B\ts\sigdecintopdec\ra \B''
     }\index{46.0}
\end{equation}}

\vspace{6pt}
\replacement{\thenostrsharing}{\begin{equation}	% functor declaration
\label{fundectopdec-rule}
\frac{\B\ts\fundec\ra\F \quad\imptyvars\F=\emptyset}
     {\B\ts\fundec\ra
      (\NamesFcn\F,\F)\ \In\ \Basis
     }
\end{equation}}{\begin{equation}        % functor declaration
\label{fundectopdec-rule}
\frac{\begin{array}{c}
          \B\ts\fundec\ra\F\quad\langle\B\oplus\F\ts\topdec\ra\B'\rangle\\
          B'' = (\TyNamesFcn\F,\F)\ \In\ \Basis\; \langle+\B'\rangle\quad \TyVarsFcn\B''=\emptyset
      \end{array}}
     {\B\ts\fundecintopdec\ra\B''}
\end{equation}}
  
\comments
\replacement{\thenostrsharing}{
\begin{description}
\item{(\ref{strdectopdec-rule})--(\ref{fundectopdec-rule})} The side
conditions ensure that no free imperative 
type variables enter the 
basis.\index{46.01}
\end{description}}{
\begin{description}
\item{(\ref{strdectopdec-rule})--(\ref{fundectopdec-rule})} 
No free type variables enter the  basis: if $\B\ts\topdec\ra\B'$
then $\TyVarsFcn(\B') = \emptyset$.\index{46.01}
\end{description}}
\oldpagebreak

\deletion{\thenofuncspec}{\subsection{Functor Signature Matching}
\label{fun-sig-match-sec}
As\index{46} pointed out in Section~\ref{mod-gram-sec} on the 
grammar for Modules, there is no phrase class whose elaboration 
requires matching one functor signature to another functor signature.
But a precise definition of this matching is needed, since a 
functor $g$ may only be separately compiled in the presence of 
specification of any functor $f$ to which $g$ refers, and then a 
real functor $f$ must match this specification.
In the case, then, that $f$ has been specified by a functor signature
\[\funsig_1\ =\ \longfunsig{1}\]
and that later $f$ is declared with functor signature
\[\funsig_2\ =\ \longfunsig{2}\]
the following matching rule will be employed:

A functor signature
$\funsig_2\ =\ \longfunsig{2}$ {\sl matches} another functor signature,
$\funsig_1\ =\ \longfunsig{1}$, if there exists a realisation $\rea$ 
such that
\begin{enumerate}
\item $\longsig{1}$ matches $\longsig{2}$ via $\rea$, and
\item $\rea((\N_2')\S_2')$ matches $(\N_1')\S_1'$.
\end{enumerate}
The first condition ensures that the real functor signature $\funsig_2$
for $f$ requires the argument $\strexp$ of any application $\f(\strexp)$
to have no more sharing, and no more richness, than was predicted by
the specified signature $\funsig_1$.
The second condition ensures that the real functor signature $\funsig_2$,
instantiated to $(\rea\S_2,\rea((\N_2')\S_2'))$, provides in the result of
the application $\f(\strexp)$
no less sharing, and no less richness, than was predicted by
the specified signature $\funsig_1$.

%We claim that any phrase -- e.g. the declaration of the functor $g$ above --
%which elaborates successfully in a basis $\B$ with $\B(f)=\funsig_1$ will
%also elaborate successfully in the basis $\B+\{f\mapsto\funsig_2\}$.  This
%claim justifies our definition of functor matching.
% -- this claim is false because of open.

} % deletion

\clearpage{}
\thispagestyle{empty}
%!TEX root = root.tex
%

\section{Dynamic Semantics for the Core}
\subsection{Reduced Syntax}
\replacement{\theconstructors}{
Since\index{47.1} types are fully dealt with in the static semantics,
the dynamic semantics ignores them.  The Core syntax is therefore
reduced by the following transformations, for the purpose of the dynamic
semantics:}{Since\index{47.1} types are mostly dealt with in the static semantics,
the Core syntax is 
reduced by the following transformations, 
for the purpose of the dynamic
semantics:}
\begin{itemize}
\item All explicit type ascriptions ``\ml{:} $\ty$\,'' are omitted, and
      qualifications ``$\OF\  \ty\,$'' are omitted from 
      \insertion{\theconstructors}{constructor and}
      exception bindings.
\item The Core phrase classes \deletion{\thedatatyperepl}{TypBind, DatBind, $\ConBind$, }Ty and
      TyRow are omitted.
\end{itemize}


\subsection{Simple Objects}
All\index{47.2} objects in the dynamic semantics are built from
identifier classes together with the simple object classes shown (with the
variables which range over them) in Figure~\ref{simp-dyn-obj}.

\begin{figure}[h]
\vspace{2pt}
\begin{displaymath}
\begin{array}{rclr}
\A               & \in   & \Addr	& \mbox{addresses}\\
\e               & \in   & \Exc 	& \mbox{exception names}\\
b      		& \in	& \BasVal	& \mbox{basic values}\\
\sv             & \in   & \SVal         & \mbox{special values}\\
                &       & \{\FAIL\}     & \mbox{failure}\\   
\end{array}
\end{displaymath}
\caption{Simple Semantic Objects}
\label{simp-dyn-obj}
\vspace{3pt}
\end{figure}

$\Addr$ and $\Exc$ are infinite sets. BasVal is described below.
{\SVal} is the class of values denoted by the special constants
\SCon. Each integer\insertion{\thelibrary}{, word} or real constant denotes a value according to normal 
mathematical conventions; each string \insertion{\thescon}{or character}
constant denotes a sequence of \deletion{\thelibrary}{ASCII }characters as explained in Section~\ref{cr:speccon}. The value denoted
by {\scon} is written $\sconval(\scon)$.
$\FAIL$ is the result of a failing attempt to match a value and a
pattern. Thus $\FAIL$ is neither a value nor an exception, but simply
a semantic object used in the rules to express operationally
how matching proceeds.

Exception constructors evaluate to exception names.
This is to accommodate the generative
nature of exception bindings;\index{48.0} each evaluation of a declaration of a
exception constructor binds it to a new unique name.

\subsection{Compound Objects}

The\index{48.2} compound objects for the dynamic semantics are
shown in Figure~\ref{comp-dyn-obj}.
Many conventions and notations are adopted as in the static semantics; in
particular projection, injection and modification all retain
their meaning.
We generally omit the injection functions taking $\VId$,
$\VId\times\Val$ etc into $\Val$.
For records $\r\in\Record$ however,
we write this injection explicitly as ``$\In\ \Val$''; this accords with
the fact that there is a separate phrase class ExpRow, whose members
evaluate to records. 

We take $\cup$ to mean disjoint union over
semantic object classes. We also understand all the defined object
classes to be disjoint. A particular case deserves mention; $\ExVal$
and $\Pack$ (exception values and packets) are isomorphic classes,
but the latter class corresponds to exceptions which have been
raised, and therefore has different semantic significance from the
former, which is just a subclass of values.


\begin{figure}[t]
\vspace{2pt}
\begin{displaymath}
\begin{array}{rcl}
        \V	&\in	&\adhocreplacementl{\theidstatus}{-9cm}{\Val =\{\mbox{\tt :=}\}\cup\SVal\cup\BasVal\cup\Con}{\Val =\{\mbox{\tt :=}\}\cup\SVal\cup\BasVal\cup\VId}\\
                &       &\adhocreplacementl{\theidstatus}{-9cm}{\qquad\cup(\Con\times\Val)\cup\ExVal}{\qquad\cup(\VId\times\Val)\cup\ExVal}\\
                &       &\qquad\cup\Record\cup\Addr\cup\adhocreplacementl{\theidstatus}{-5cm}{\Closure}{\FcnClosure}\\
        \r      & \in   & \Record =  \finfun{\Lab}{\Val}\\
{\exval}      & \in   & \ExVal = \Exc \cup (\Exc\times\Val)\\
{[\exval]}\ {\rm or}\ \p
                & \in   & \Pack = \ExVal\\
(\match,\E,\VE) & \in   & \adhocreplacementl{\theidstatus}{-9cm}{\Closure = \Match\times\Env\times\VarEnv}{\FcnClosure = \Match\times\Env\times\ValEnv}\\
        \mem    & \in   & \Mem = \finfun{\Addr}{\Val}\\
        \excs   & \in   & \ExcSet = \Fin(\Exc)\\
(\mem,\excs)\ {\rm or}\ \s
                & \in   & \State = \Mem\times\ExcSet\\
(\SE,\adhocinsertion{\thedatatyperepl}{25mm}{\TE,}\VE\adhocdeletion{\theidstatus}{2cm}{,\EE})\ {\rm or}\ \E
                & \in   & \adhocreplacementl{\theidstatus}{-9cm}{\Env = \StrEnv\times\VarEnv\times\ExnEnv}{\Env = \StrEnv\times\TyEnv\times\ValEnv}\\
        \SE     & \in   & \StrEnv = \finfun{\StrId}{\Env}\\
\adhocinsertion{\thedatatyperepl}{1cm}{        \TE     & \in   & \TyEnv = \finfun{\TyCon}{\ValEnv}\\}
        \VE	& \in	& \adhocreplacementl{\theidstatus}{-9cm}{\VarEnv = \finfun{\Var}{\Val}}{\ValEnv = \finfun{\VId}{\Val\times\IdStatus}}\\
\adhocdeletion{\theidstatus}{2mm}{        \EE	& \in	& \ExnEnv = \finfun{\Exn}{\Exc}\\ }
\end{array}
\end{displaymath}
\caption{Compound Semantic Objects\index{48.1}}
\label{comp-dyn-obj}
\vspace{3pt}
\end{figure}
%
%
Although the same names, e.g. $\E$ for an environment, are used
as in the static semantics, the objects denoted are different.  This need cause
no confusion since the static and dynamic semantics are presented %completely
separately. 

\subsection{Basic Values}
The\index{49.1} basic values in $\BasVal$ are \deletion{\thelibrary}{the }values bound to predefined 
\replacement{\theidstatus}{identifiers}{value variables}.
\replacement{\thelibrary}{
These values are denoted by the identifiers to which they are bound in the
initial dynamic basis (see Appendix~\ref{init-dyn-bas-app}), 
and are as follows:
\begin{verbatim}
           abs  floor  real  sqrt  sin  cos  arctan  exp  ln
              size  chr  ord  explode  implode  div  mod
                  ~  /  *  +  -  =  <>  <  >  <=  >=
        std_in  std_out  open_in  open_out  close_in  close_out
                input  output  lookahead  end_of_stream
\end{verbatim}
The meaning of basic values (almost all of which are functions) is
represented by the function
\[ \APPLY\ :\ \BasVal\times\Val\to\Val\cup\Pack \]
 which is detailed in Appendix~\ref{init-dyn-bas-app}.
}{In this document, we take $\BasVal$ to be the singleton set $\{\boxml{=}\}$;
however, libraries may define a larger set of basic values. 
The meaning of basic values is
represented by a function
\[ \APPLY\ :\ \BasVal\times\Val\to\Val\cup\Pack \]
which satisfies that $\APPLY(\boxml{=}, \{1\mapsto v_1, 2\mapsto v_2\})$ 
is {\tt true} or {\tt false} according as the values $\V_1$ and $\V_2$ are, or are
not, identical values. 
}
\subsection{Basic Exceptions}
\label{bas-exc}
\replacement{\theidstatus}{
A\index{49.2} subset $\BasExc\subset\Exc$ of the exception names are bound to predefined
exception constructors.
These names are denoted by the identifiers to which they are bound in the
initial dynamic basis (see Appendix~\ref{init-dyn-bas-app}), 
and are as follows:}{A\index{49.2} 
subset $\BasExc\subset\Exc$ of the exception names are bound to predefined
exception constructors in the initial dynamic basis 
(see Appendix~\ref{init-dyn-bas-app}).
These names are denoted by the identifiers 
to which they are bound in the
initial basis, 
and are as follows:}\replacement{\thelibrary}{\begin{verbatim}
        Abs  Ord  Chr   Div  Mod  Quot  Prod  
        Neg  Sum  Diff  Floor  Sqrt  Exp  Ln
        Io   Match  Bind  Interrupt
\end{verbatim}}{$$\boxml{ Match\ \  Bind}$$}\replacement{\thelibrary}{
The exceptions on the first two  lines are raised by 
corresponding basic functions, where \verb+~+ {\tt /} {\tt *}
{\tt +} {\tt -} correspond respectively to {\tt Neg} {\tt Quot}
{\tt Prod} {\tt Sum} {\tt Diff}. The details are given
in Appendix~\ref{init-dyn-bas-app}. The exception $(\mbox{{\tt Io}},s)$,
where $s$ is a string, is raised
by certain of the basic input/output functions,
as detailed in Appendix~\ref{init-dyn-bas-app}.  
}{The} exceptions ~\ml{Match}~ and
~\ml{Bind}~
are raised upon failure of pattern-matching in evaluating a 
function {\fnexp} or a
$\valbind$, as detailed in the rules to follow.   
\deletion{\thelibrary}{
Finally, ~\ml{Interrupt}~
is raised by external intervention.}
Recall from Section~\ref{further-restrictions-sec} 
that in the context {\fnexp}, the $\match$ 
must be irredundant and exhaustive and that the compiler should flag
the {\match} if it violates these restrictions. The exception~\ml{Match}
can only be raised for a match which is not exhaustive, and has therefore 
been flagged by the compiler.

%In a match of the form $\pat_1\ \ml{=>}\ \exp_1\ \ml{|}\ \ldots\ \ml{|}\ 
%            \pat_n\ \ml{=>}\ \exp_n$
%the pattern sequence $\pat_1,\ldots,\pat_n$ should be {\sl irredundant};
%that is, each $\pat_j$ must match some value
%(of the right type) which is not matched by $\pat_i$ for any $i<j$.
%In the context {\fnexp}, the $\match$ must also be {\sl exhaustive}; that is,
%every value (of the right type) must be matched by some $\pat_i$.
%The compiler must give warning on violation of these restrictions, 
%but should still compile the match. 
%The \ml{match} exception
%can only be raised for a match which is not exhaustive, and has therefore
%been flagged by the compiler.
%The restrictions are inherited by derived forms; in particular,
%this means that in the function binding 
% $\var\ \atpat_1\ \cdots\ \atpat_n\langle : \ty\rangle\ \ml{=}\ \exp$
%(consisting of one clause only), each separate $\atpat_i$ should be
%exhaustive by itself.

\replacement{\theidstatus}{\subsection{Closures}}{\subsection{Function Closures}}
The\index{50.1} informal understanding of a {\sl \insertion{\theidstatus}{function} closure} $(\match,\E,\VE)$ is as follows:
when the \insertion{\theidstatus}{function} closure is applied to a value $\V$,
$\match$ will be evaluated against $\V$, in the environment $\E$ modified in
a special sense by
$\VE$.  The domain $\Dom\VE$ of this third component contains those \deletion{\thenoimptypes}{function }identifiers to be treated recursively in the evaluation.  To achieve this
effect, the evaluation of $\match$ will take place not in $\plusmap{\E}{\VE}$
but in $\plusmap{\E}{\Rec\VE}$, 
\replacement{\theidstatus}{where\[ \Rec\ :\ \VarEnv\to\VarEnv \]is defined as follows:}{where\[ \Rec\ :\ \ValEnv\to\ValEnv \]is defined as follows:}
\medskip

\begin{itemize}
\item $\Dom(\Rec\VE)=\Dom\VE $
\item If \replacement{\theidstatus}{$\VE(\var)\notin\Closure$}{$\VE(\vid)\notin\FcnClosure\times\{\isv\}$}, 
then $(\Rec\VE)(\vid)=\VE(\vid)$
\item \replacement{\theidstatus}{If $\VE(\var)=(\match',\E',\VE')$
      then $(\Rec\VE)(\var)=(\match',\E',\VE)$}{If $\VE(\vid)=((\match',\E',\VE'), \isv)$
      then $(\Rec\VE)(\vid)=((\match',\E',\VE), \isv)$}
\end{itemize}
The effect is that, before application of $(\match,\E,\VE)$ to $\V$, the
\replacement{\theidstatus}{closure values}{function closures} 
in $\Ran\VE$ are ``unrolled'' once, to prepare for their possible
recursive application during the evaluation of $\match$ upon $\V$.

This device is adopted to ensure that all semantic objects are finite (by
controlling the unrolling of recursion).  The operator $\Rec$ is invoked in
just two places in the semantic rules: in the rule for recursive value 
\REPL{%
declarations of the form ``$\VAL\ \REC\ \valbind$'' 
}{%
bindings of the form ``$\REC\ \valbind$''}, and in the rule for evaluating
an application expression ``$\exp\ atexp$'' in the case that $\exp$
evaluates to a \insertion{\theidstatus}{function} closure.

\subsection{Inference Rules}
\label{dyncor-inf-rules-sec}
The\index{50.2} semantic rules allow sentences  of the form
\[ \s,A\ts\phrase\ra A',\s' \]
to be inferred, where $A$ is usually an environment, $A'$ is some semantic
object and $\s$,$\s'$ are the states before and after the evaluation
represented by the sentence.  Some hypotheses in rules are not of this form;
they are called {\sl side-conditions}.  The convention for options is
the same as for the Core static semantics.  

In most rules the states $\s$ and $\s'$ are omitted from sentences; they
are only included for those rules which are directly concerned with the state
-- either referring to its contents or changing it.  When omitted, the
convention for restoring them is as follows.  If the rule is presented in the
form
\[ \frac{ \begin{array}{r}
          A_1\ts\phrase_1\ra A_1'\qquad
          A_2\ts\phrase_2\ra A_2'\quad\cdots\\
          \cdots\quad A_n\ts\phrase_n\ra A_n'
          \end{array}
        }
        { A\ts\phrase\ra A'} \]
\oldpagebreak
then the full form is intended to be
\[ \frac{ \begin{array}{r}
          \s_0,A_1\ts\phrase_1\ra A_1',\s_1\qquad
          \s_1,A_2\ts\phrase_2\ra A_2',\s_2\quad\cdots\\
          \cdots\quad\s_{n-1},A_n\ts\phrase_n\ra A_n',\s_n
          \end{array}
        }
        { \s_0,A\ts\phrase\ra A',\s_n} \]
(Any side-conditions are left unaltered).
Thus the left-to-right order of the hypotheses indicates the order of
evaluation.  Note that in the case $\n=0$, when there are no hypotheses
(except possibly side-conditions), we have $\s_n=\s_0$; this implies that the
rule causes no side effect.
The convention is called the {\sl state convention},\index{51.1} and
must be applied to each version of a rule obtained by inclusion or
omission of its options.

A second convention, the {\sl exception convention}, is adopted to deal
with the propagation of exception packets $\p$.
For each rule whose full form (ignoring side-conditions) is
\[ \frac{ \s_1,A_1\ts\phrase_1\ra A_1',\s_1'\qquad\cdots\qquad
          \s_n,A_n\ts\phrase_n\ra A_n',\s_n' }
        { \s,A\ts\phrase\ra A',\s'} \]
and for each $k$, $1\leq k\leq n$, for which the result $A_k'$ is not a
packet $\p$, an extra rule is added of the form
\[ \frac{ \s_1,A_1\ts\phrase_1\ra A_1',\s_1'\qquad\cdots\qquad
          \s_k,A_k\ts\phrase_k\ra \p',\s' }
        { \s,A\ts\phrase\ra \p',\s'} \]
where $\p'$ does not occur in the original rule.\footnote{There is one
exception to the exception convention; no extra rule is added for
rule~\ref{handlexp-dyn-rule1} which deals with handlers, 
since a handler is the only
means by which propagation of an exception can be arrested.}
This indicates that evaluation of phrases in the hypothesis terminates with the
first whose result is a packet (other than one already treated in the rule),
and this packet is the result of the phrase in the conclusion.

A third convention is that we allow compound variables (variables built
from the variables in Figure~\ref{comp-dyn-obj} and the symbol ``/'')
to range over unions of semantic objects. For instance 
the compound variable $\V/\p$ ranges
over $\Val\cup\Pack$. 
We also allow $x/\FAIL$ to range over $X\cup\{\FAIL\}$ where $x$ 
ranges over $X$;
furthermore, we extend environment modification to allow for failure
as follows:
\[\VE+\FAIL=\FAIL.\]
%
%                       Atomic Expressions
%
\rulesec{Atomic Expressions}{\E\ts\atexp\ra\V/\p}
\begin{equation}	% special constant
%\label{sconexp-dyn-rule}
\frac{}
     {\E\ts\scon\ra\sconval(\scon)}\index{51.15}
\end{equation}

\replacement{\theidstatus}{\begin{equation}	% value variable
\label{varexp-dyn-rule}
\frac{\E(\longvar)=\V}
     {\E\ts\longvar\ra\V}\index{51.2}
\end{equation}
\oldpagebreak
\begin{equation}	% value constructor
\label{conexp-dyn-rule}
\frac{\longcon=\strid_1.\cdots.\strid_k.\con}
     {\E\ts\longcon\ra\con}\index{52.1}
\end{equation}

\begin{equation}       %  exception constant
\label{exconexp-dyn-rule}
\frac{\E(\longexn)=\e}
     {\E\ts\longexn\ra\e}
\end{equation}}{\begin{equation}	% value variable
\label{varexp-dyn-rule}
\frac{\E(\longvid)=(\V,\is)}
     {\E\ts\longvid\ra\V}\index{51.2}
\end{equation}}


\begin{equation}	% record expression
%\label{recexp-dyn-rule}
\frac{\langle\E\ts\labexps\ra\r\rangle}
     {\E\ts\lttbrace\ \recexp\ \rttbrace\ra\emptymap
                                  \langle +\ \r\rangle\ \In\ \Val}
\end{equation}

\begin{equation}        % local declaration
%\label{let-dyn-rule}
\frac{\E\ts\dec\ra\E'\qquad\E+\E'\ts\exp\ra\V}
     {\E\ts\letexp\ra\V}
\end{equation}

\begin{equation}	% paren expression
%\label{parexp-dyn-rule}
\frac{\E\ts\exp\ra\V}
     {\E\ts\parexp\ra\V}
\end{equation}
\comments
\begin{description}
\deletion{\theidstatus}{\item{(\ref{conexp-dyn-rule})}
   Value constructors denote themselves.}

\insertion{\theidstatus}{\item{(\ref{varexp-dyn-rule})}
As in the static semantics,
value identifiers are looked up in the environment and the
identifier status is not used.}

\deletion{\theidstatus}{
\item{(\ref{exconexp-dyn-rule})}
   Exception constructors are looked up in the exception environment
   component of $\E$.}
\end{description}

\rulesec{Expression Rows}{\E\ts\labexps\ra\r/\p}
\begin{equation}	% labelled expressions
\label{labexps-dyn-rule}
\frac{\E\ts\exp\ra\V\qquad\langle\E\ts\labexps\ra\r\rangle}
     {\E\ts\longlabexps\ra\{\lab\mapsto\V\}\langle +\ \r\rangle}\index{52.2}
\end{equation}%

\BeginNewEqns%
\begin{equation} % expression row extension
\color{\addcolor}
\frac{\E\ts\exp\ra\r\ \In\ \Val}
     {\E\ts\boxml{...}\ \boxml{=}\ \exp\ra\r}
\end{equation}%
\EndNewEqns%

\comments
\begin{description}
\item{(\ref{labexps-dyn-rule})}
  We may think of components as being evaluated from left to right,
  because of the state and exception conventions.
\end{description}%
%
%                        Expressions
%
\rulesec{Expressions}{\E\ts\exp\ra\V/\p}
\begin{equation}	% atomic
%\label{atexp-dyn-rule}
\frac{\E\ts\atexp\ra\V}
     {\E\ts\atexp\ra\V}\index{52.3}
\end{equation}

\replacement{\theidstatus}{\begin{equation}	% constructor application
\label{conapp-dyn-rule}
\frac{\E\ts\exp\ra\con\qquad\con\neq\REF\qquad\E\ts\atexp\ra\V}
     {\E\ts\appexp\ra(\con,\V)}
\end{equation}}{\begin{equation}	% constructor application
\label{conapp-dyn-rule}
\frac{\E\ts\exp\ra\vid\qquad\vid\neq\REF\qquad\E\ts\atexp\ra\V}
     {\E\ts\appexp\ra(\vid,\V)}
\end{equation}}

\begin{equation}        % exception constructor application
\frac{\E\ts\exp\ra\e\qquad\E\ts\atexp\ra\V}
     {\E\ts\appexp\ra(\e,\V)}
\end{equation}

\begin{equation}	% reference application
\label{refapp-dyn-rule}
\frac{\s,\E\ts\exp\ra~\ml{ref}~,\s'\qquad
      \s',\E\ts\atexp\ra\V,\s''\qquad
      \A\notin\Dom(\of{\mem}{\s''})}
     {\s,\E\ts\appexp\ra\A,\ \s''+\{\A\mapsto\V\} }
\end{equation}


\begin{equation}	% assignment application
\label{assapp-dyn-rule}
\frac{\s,\E\ts\exp\ra~\mbox{\tt :=}~,\s'\qquad
      \s',\E\ts\atexp\ra\{{\tt 1}\mapsto\A,\ {\tt 2}\mapsto\V\},\s''}
     {\s,\E\ts\appexp\ra\emptymap\ \In\ \Val,\ \s''+\{\A\mapsto\V\} }
\end{equation}
\oldpagebreak
\replacement{\thefixtypos}{
\begin{equation}	% basic function application
%\label{basapp-dyn-rule}
\frac{\E\ts\exp\ra b
      \qquad\E\ts\atexp\ra\V\qquad\APPLY(b,\V)=\V'}
     {\E\ts\appexp\ra\V'}\index{53.1}
\end{equation}}{
\begin{equation}	% basic function application
%\label{basapp-dyn-rule}
\frac{\E\ts\exp\ra b
      \qquad\E\ts\atexp\ra\V\qquad\APPLY(b,\V)=\V'/\p}
     {\E\ts\appexp\ra\V'/\p}\index{53.1}
\end{equation}}

\begin{equation}	% closure application
\label{closapp-dyn-rule}
\frac{\begin{array}{c}
      \E\ts\exp\ra(\match,\E',\VE)\qquad\E\ts\atexp\ra\V\\
      \E'+\Rec\VE,\ \V\ts\match\ra\V'
      \end{array}
     }
     {\E\ts\appexp\ra\V'}
\end{equation}

\begin{equation}        % failing closure application
\label{closapp-dyn-rule1}
\frac{\begin{array}{c}
      \E\ts\exp\ra(\match,\E',\VE)\qquad\E\ts\atexp\ra\V\\
      \E'+\Rec\VE,\ \V\ts\match\ra\FAIL
      \end{array}
     }
     {\E\ts\appexp\ra[{\tt Match}]}
\end{equation}

\begin{equation}        % handle exception 1
\label{handlexp-dyn-rule1}
\frac{\E\ts\exp\ra\V}
     {\E\ts\handlexp\ra\V}
\end{equation}

\begin{equation}        % handle exception 2
\label{handlexp-dyn-rule2}
\frac{\E\ts\exp\ra[\exval]\qquad\E,\exval\ts\match\ra\V}
     {\E\ts\handlexp\ra\V}
\end{equation}

\begin{equation}        % handle exception 3
\label{handlexp-dyn-rule3}
\frac{\E\ts\exp\ra[\exval]\qquad\E,\exval\ts\match\ra\FAIL}
     {\E\ts\handlexp\ra[\exval]}
\end{equation}

\begin{equation}        % raise exception
%\label{raisexp-dyn-rule}
\frac{\E\ts\exp\ra\exval}
     {\E\ts\raisexp\ra[\exval]}
\end{equation}

\begin{equation}        % function
\label{fnexp-dyn-rule}
\frac{}
     {\E\ts\fnexp\ra(\match,\E,\emptymap)}
\end{equation}%

\pagebreak
\noindent\comments
\begin{description}
\item{(\ref{refapp-dyn-rule})}
  The side condition ensures that a new address is chosen. There are
no rules concerning disposal of inaccessible addresses\deletion{\thefixtypos}{ (``garbage
collection'')}.
%
\item{(\ref{conapp-dyn-rule})--(\ref{closapp-dyn-rule1})}
  Note that none of the rules for function application has a
premise in which the operator evaluates to a constructed
value, a record or an address. This is because we are interested
in the evaluation of well-typed programs only, and in such programs $\exp$
will always have a functional type.
% so $\V$ will be either a closure,
%a constructor, a basic value or \ml{:=}.
%
\item{(\ref{handlexp-dyn-rule1})}
  This is the only rule to which the exception convention does not apply.
If the operator evaluates to a packet then rule~\ref{handlexp-dyn-rule2}
or rule~\ref{handlexp-dyn-rule3} must be used.
%
\item{(\ref{handlexp-dyn-rule3})}
 Packets that are not handled by the $\match$ propagate.
%
\item{(\ref{fnexp-dyn-rule})}
  The third component of the \insertion{\theidstatus}{function} closure is empty because the match does not
introduce new recursively defined values.
\end{description}
%
%                        Matches
%
\rulesec{Matches}{\E,\V\ts\match\ra\V'/\p/\FAIL}
\begin{equation}	% match 1
%\label{match-dyn-rule1}
\frac{\E,\V\ts\mrule\ra\V'}
     {\E,\V\ts\longmatch\ra\V'}\index{54.1}
\end{equation}

\begin{equation}	% match 2
%\label{match-dyn-rule2}
\frac{\E,\V\ts\mrule\ra\FAIL}
     {\E,\V\ts\mrule\ra\FAIL}
\end{equation}

\replacement{\thefixtypos}{
\begin{equation}	% match 3
%\label{match-dyn-rule}
\frac{\E,\V\ts\mrule\ra\FAIL\qquad\E,\V\ts\match\ra\V'/\FAIL}
     {\E\ts\longmatcha\ra\V'/\FAIL}
\end{equation}}{\begin{equation}	% match 3
%\label{match-dyn-rule}
\frac{\E,\V\ts\mrule\ra\FAIL\qquad\E,\V\ts\match\ra\V'/\FAIL}
     {\E,\V\ts\longmatcha\ra\V'/\FAIL}
\end{equation}}\comment A value $\V$ occurs on the left of the turnstile, in evaluating
a $\match$. We may think of a $\match$ as being evaluated {\sl against}
a value; similarly, we may think of a pattern as being evaluated {\sl
against} a value.
Alternative match rules are tried from left to right.

\rulesec{Match Rules}{\E,\V\ts\mrule\ra\V'/\p/\fail}
\begin{equation}	% mrule 1
%\label{mrule-dyn-rule1}
\frac{\E,\V\ts\pat\ra\VE\qquad\E+\VE\ts\exp\ra\V'}
     {\E,\V\ts\longmrule\ \ra\V'}\index{54.2}
\end{equation}

\begin{equation}	% mrule 2
%\label{mrule-dyn-rule2}
\frac{\E,\V\ts\pat\ra\FAIL}
     {\E,\V\ts\longmrule\ \ra\FAIL}
\end{equation}

%
%                        Declarations
%
\rulesec{Declarations}{\E\ts\dec\ra\E'/\p}
\begin{equation}	% value declaration
%\label{valdec-dyn-rule}
\frac{\E\ts\valbind\ra\VE}
     {\E\ts\explicitvaldec\ra \ADD{\langle\Rec\rangle} \VE\ \In\ \Env}\index{54.3}
\end{equation}

\insertion{\thedatatyperepl}{\begin{equation}
\frac{\ts\typbind\ra\TE}
     {\E\ts\typedec\ra\TE\ \In\ \Env}
\end{equation}}

\insertion{\thedatatyperepl}{
\begin{equation}
\frac{\ts\datbind\ra\VE,\TE}
     {\E\ts\datatypedec\ra(\VE,\TE)\ \In\ \Env}
\end{equation}}

%\insertion{\thedatatyperepl}{\begin{equation}
%\frac{\E(\longtycon) = \VE}
%     {\begin{array}{r}
%          \E\ts\datatyperepldec\ra\qquad\qquad\\
%          (\VE,\{\tycon\mapsto\VE\})\ \In\ \Env
%      \end{array}}
%\end{equation}}
\insertion{\thedatatyperepl}{\begin{equation}
\frac{\E(\longtycon) = \VE}
     {
          \E\ts\datatyperepldec\ra
          (\VE,\{\tycon\mapsto\VE\})\ \In\ \Env
      }
\end{equation}}

\begin{equation}
\frac{\CUT{\ts\datbind\ra\VE,\TE\qquad \E+\VE\ts\dec\ra \E'}}
     {\CUT{\E\ts\abstypedec \ra \E'}}
\end{equation}%

% \ replacement{\theconstructors}{
%  \begin{equation}	% exception declaration
%  %\label{exceptiondec-dyn-rule}
%  \frac{\E\ts\exnbind\ra\EE }
%       {\E\ts\exceptiondec\ra\EE\ \In\ \Env }
%  \end{equation}}{\begin{equation}	% exception declaration
%  %\label{exceptiondec-dyn-rule}
%  \frac{\E\ts\exnbind\ra\EE\qquad\VE=\EE }
%       {\E\ts\exceptiondec\ra(\VE,\EE)\ \In\ \Env }
%  \end{equation}}
\replacement{\theidstatus}{
\begin{equation}	% exception declaration
%\label{exceptiondec-dyn-rule}
\frac{\E\ts\exnbind\ra\EE }
     {\E\ts\exceptiondec\ra\EE\ \In\ \Env }
\end{equation}}{\begin{equation}	% exception declaration
%\label{exceptiondec-dyn-rule}
\frac{\E\ts\exnbind\ra\VE }
     {\E\ts\exceptiondec\ra\VE\ \In\ \Env }
\end{equation}}

\begin{equation}	% local declaration
%\label{localdec-dyn-rule}
\frac{\E\ts\dec_1\ra\E_1\qquad\E+\E_1\ts\dec_2\ra\E_2}
     {\E\ts\localdec\ra\E_2}
\end{equation}

\replacement{\thefixtypos}{
\begin{equation}                % open declaration
%\label{open-strdec-dyn-rule}
\frac{ \E(\longstrid_1)=\E_1
            \quad\cdots\quad
       \E(\longstrid_k)=\E_k }
     { \E\ts\openstrdec\ra \E_1 + \cdots + \E_k }
\end{equation}}{
\begin{equation}                % open declaration
%\label{open-strdec-dyn-rule}
\frac{ \E(\longstrid_1)=\E_1
            \quad\cdots\quad
       \E(\longstrid_n)=\E_n }
     { \E\ts\openstrdec\ra \E_1 + \cdots + \E_n }
\end{equation}}

\vspace{6pt}
\begin{equation}	% empty declaration
%\label{emptydec-dyn-rule}
\frac{}
     {\E\ts\emptydec\ra\emptymap\ \In\ \Env}
\end{equation}

\begin{equation}	% sequential declaration
%\label{seqdec-dyn-rule}
\frac{\E\ts\dec_1\ra\E_1\qquad\E+\E_1\ts\dec_2\ra\E_2}
     {\E\ts\seqdec\ra\plusmap{E_1}{E_2}}
\end{equation}
%
%                        Bindings
%
\rulesec{Value Bindings}{\E\ts\valbind\ra\VE/\p}
\begin{equation}	% value binding 1
%\label{valbind-dyn-rule1}
\frac{\E\ts\exp\ra\V\qquad\E,\V\ts\pat\ra\VE\qquad
      \langle\E\ts\valbind\ra\VE'\rangle }
     {\E\ts\longvalbind\ra\VE\ \langle +\ \VE'\rangle}\index{55.1}
\end{equation}

\begin{equation}	% value binding 2
%\label{valbind-dyn-rule2}
\frac{\E\ts\exp\ra\V\qquad\E,\V\ts\pat\ra\FAIL}
     {\E\ts\longvalbind\ra[\ml{Bind}]}
\end{equation}

\begin{equation}	% recursive value binding
%\label{recvalbind-dyn-rule}
\frac{\CUT{\E\ts\valbind\ra\VE}}
     {\CUT{\E\ts\recvalbind\ra\Rec\VE}}
\end{equation}

\insertion{\thedatatyperepl}{
\rulesec{Type Bindings}{\ts\typbind\ra\TE}
\begin{equation}
\frac{\langle\ts\typbind\ra\TE\rangle}
     {\ts\longtypbind\ra\{\tycon\mapsto\emptymap\}\langle+\TE\rangle}
\end{equation}}

\insertion{\thedatatyperepl}{
\rulesec{Datatype Bindings}{\ts\datbind\ra\VE,\TE}
\begin{equation}
\frac{\ts\constrs\ra\VE\qquad\langle\ts\datbind'\ra\VE',\TE'\rangle}
     {\ts\tyvarseq\;\tycon\boxml{=}\constrs\;\langle\boxml{and}\,\datbind'\rangle\ra\VE\langle+\VE'\rangle,\{\tycon\mapsto\VE\}\langle+\TE'\rangle}
\end{equation}

\rulesec{Constructor Bindings}{\ts\constrs\ra\VE}
\begin{equation}
\frac{\langle\ts\constrs\ra\VE\rangle}
     {\ts\vid\langle\boxml{|}\,\constrs\rangle\ra
               \{\vid\mapsto(\vid,\isc)\}\,\langle+\VE\rangle}
\end{equation}
}


% \ replacement{\thefixtypos}{
% \rulesec{Exception Bindings}{\E\ts\exnbind\ra\EE/\p}}{\rulesec{Exception 
% Bindings}{\E\ts\exnbind\ra\EE}}
\replacement{\theidstatus}{
\rulesec{Exception Bindings}{\E\ts\exnbind\ra\EE/\p}}{\rulesec{Exception 
Bindings}{\E\ts\exnbind\ra\VE}}
\replacement{\theidstatus}{
\begin{equation}	% exception binding 1
\label{exnbind-dyn-rule1}
\frac{\e\notin\of{\excs}{\s}\qquad\s'=\s+\{\e\}\qquad
      \langle\s',\E\ts\exnbind\ra\EE,\s''\rangle }
     {\s,\E\ts\longexnbindaa\ra\{\exn\mapsto\e\}\langle +\ \EE\rangle,\
                               \s'\langle'\rangle}\index{55.2}
\end{equation}}{\begin{equation}	% exception binding 1
\label{exnbind-dyn-rule1}
\frac{\e\notin\of{\excs}{\s}\qquad\s'=\s+\{\e\}\qquad
      \langle\s',\E\ts\exnbind\ra\VE,\s''\rangle }
     {\s,\E\ts\longvidexnbindaa\ra\{\vid\mapsto(\e,\ise)\}\langle +\ \VE\rangle,\
                               \s'\langle'\rangle}\index{55.2}
\end{equation}}

\replacement{\theidstatus}{
\begin{equation}	% exception binding 2
%\label{exnbind-dyn-rule2}
\frac{\E(\longexn)=\e\qquad
      \langle\E\ts\exnbind\ra\EE\rangle }
     {\E\ts\longexnbindb\ra\{\exn\mapsto\e\}\langle +\ \EE\rangle}
\end{equation}}{\begin{equation}	% exception binding 2
%\label{exnbind-dyn-rule2}
\frac{\E(\longvid)=(\e,\ise)\qquad
      \langle\E\ts\exnbind\ra\VE\rangle }
     {\E\ts\longvidexnbindb\ra\{\vid\mapsto(\e,\ise)\}\langle +\ \VE\rangle}
\end{equation}}
\comments
\begin{description}
\item{(\ref{exnbind-dyn-rule1})}
  The two side conditions ensure that a new exception name is generated and 
recorded as ``used'' in subsequent states.
\end{description}
%
%                        Atomic Patterns
%
\rulesec{Atomic Patterns}{\E,\V\ts\atpat\ra\VE/\fail}
\begin{equation}	% wildcard pattern
%\label{wildcard-dyn-rule}
\frac{}
     {\E,\V\ts\wildpat\ra \emptymap}\index{55.3}
\end{equation}

\begin{equation}	% special constant in pattern (1)
%\label{sconpat-dyn-rule1}
\frac{\V=\sconval(\scon)}
     {\E,\V\ts\scon\ra \emptymap}\index{55.35}
\end{equation}

\begin{equation}	% special constant in pattern (2)
\label{sconpat-dyn-rule2}
\frac{\V\neq\sconval(\scon)}
     {\E,\V\ts\scon\ra \FAIL}\index{55.36}
\end{equation}

\replacement{\theidstatus}{\begin{equation}	% variable pattern
%\label{varpat-dyn-rule}
\frac{}
     {\E,\V\ts\var\ra \{\var\mapsto\V\} }
\end{equation}}{\begin{equation}	% variable pattern
%\label{varpat-dyn-rule}
\frac{\hbox{$\vid\notin\Dom(\E)$ or $\of{\is}{\E(\vid)} = \isv$}}
     {\E,\V\ts\vid\ra \{\vid\mapsto(\V,\isv)\} }
\end{equation}}

\deletion{\theidstatus}{\begin{equation}	% constant pattern
%\label{conapat-dyn-rule1}
\frac{\longcon=\strid_1.\cdots.\strid_k.\con\qquad\V=\con }
     {\E,\V\ts\longcon\ra \emptymap}
\end{equation}

\begin{equation}
\label{conapat-dyn-rule2}
\frac{\longcon=\strid_1.\cdots.\strid_k.\con\qquad\V\neq\con}
     {\E,\V\ts\longcon\ra\FAIL}
\end{equation}}
%\begin{equation}	% constant pattern
%\label{conpat-dyn-rule}
%\frac{\longcon=\strid_1.\cdots.\strid_k.\con }
%     {\con\ts\longcon\ra \emptymap}
%\end{equation}

\replacement{\theidstatus}{\begin{equation}        % exception constant
%\label{exconapat-dyn-rule1}
\frac{\E(\longexn)=\V}
     {\E,\V\ts\longexn\ra\emptymap}
\end{equation}}{\begin{equation}        % exception constant
%\label{exconapat-dyn-rule1}
\frac{\E(\longvid)=(\V,\is)\qquad\is\neq\isv}
     {\E,\V\ts\longvid\ra\emptymap}
\end{equation}}
\oldpagebreak
\replacement{\theidstatus}{\begin{equation}	
\label{exconapat-dyn-rule2}
\frac{\E(\longexn)\neq\V}
     {\E,\V\ts\longexn\ra\FAIL}\index{56.0}
\end{equation}}{\begin{equation}	
\label{exconapat-dyn-rule2}
\frac{\E(\longvid)=(\V',\is)\qquad\is\neq\isv\qquad\V\neq\V'}
     {\E,\V\ts\longvid\ra\FAIL}\index{56.0}
\end{equation}}


\begin{equation}	% record pattern
%\label{recpat-dyn-rule}
\frac{\V=\emptymap\langle+\r\rangle\ \In\ \Val\qquad
      \langle\E,\r\ts\labpats\ra\VE/\fail\rangle}
     {\E,\V\ts\lttbrace\ \langle\labpats\rangle\ \rttbrace\ra\emptymap\langle+\VE/\fail\rangle}
\end{equation}
%\begin{equation}	% record pattern
%\label{recpat-dyn-rule}
%\frac{\langle\r\ts\labpats\ra\VE\rangle}
%     {\emptymap\langle +\ \r\rangle\ \In\ \Val
%      \ts\{\ \recpat\ \}\ra\emptymap\langle +\ \VE\rangle}
%\end{equation}

\begin{equation}	% parenthesised pattern
%\label{parpat-dyn-rule}
\frac{\E,\V\ts\pat\ra\VE/\fail}
     {\E,\V\ts\parpat\ra\VE/\fail}\index{56.1}
\end{equation}

%\begin{equation}	% failure of atomic pattern
%\label{failatpat-dyn-rule}
%\frac{\forall\VE\ (\V\ts\atpat\not\Rightarrow\VE)}
%     {\V\ts\atpat\ra\FAIL}
%\end{equation}
\comments
\begin{description}
\item{(\ref{sconpat-dyn-rule2}),
\deletion{\theidstatus}{(\ref{conapat-dyn-rule2}),}
(\ref{exconapat-dyn-rule2})}
  Any evaluation resulting in $\FAIL$ must do so because 
rule~\ref{sconpat-dyn-rule2},
\deletion{\theidstatus}{rule~\ref{conapat-dyn-rule2},}
rule~\ref{exconapat-dyn-rule2},
rule~\ref{conpat-dyn-rule2},
or rule~\ref{exconpat-dyn-rule2} has been
applied.
\end{description}

\rulesec{Pattern Rows}{\E,\r\ts\labpats\ra\VE/\fail}
\begin{equation}	% wildcard record
%\label{wildrec-dyn-rule}
\frac{\ADD{\E,r\ \In\ \Val\ts\pat\ra\VE/\fail}}
     {\E,\r\ts\wildrec\ra\REPL{\VE/\fail }{\emptymap}}\index{56.2}
\end{equation}
 

\begin{equation}	% record component with inherited FAIL
\label{longlab-dyn-rule1}
\frac{\E,\r(\lab)\ts\pat\ra\FAIL}
     {\E,\r\ts\longlabpats\ra\FAIL}
\end{equation}

\begin{equation}	% record component
\label{longlab-dyn-rule2}
\frac{\E,\r(\lab)\ts\pat\ra\VE\qquad
      \langle\E\ADD{+\VE},\r\ADD{\setminus\{\lab\}}\ts\labpats\ra\VE'/\fail\rangle }
     {\E,\r\ts\longlabpats\ra
      \VE\langle +\ \VE'/\fail\rangle}
\end{equation}
\comments
\begin{description}
\item{(\ref{longlab-dyn-rule1}),(\ref{longlab-dyn-rule2})}
For well-typed programs $\lab$ will be in the domain of $\r$.
\end{description}
%\begin{equation}	% record component
%\label{longlab-dyn-rule}
%\frac{\V\ts\pat\ra\VE\qquad
%      \langle\r\ts\labpats\ra\VE'\qquad\VE\sim\VE'\rangle }
%     {\{\lab\mapsto\V\}\langle +\r\rangle\ts\longlabpats\ra
%      \VE\langle +\ \VE'\rangle}
%\end{equation}

%\begin{equation}	% failure of labelled patterns
%\label{faillabpats-dyn-rule}
%\frac{\forall\VE\ (\r\ts\labpats\not\ra\VE)}
%     {\r\ts\labpats\ra\FAIL}
%\end{equation}
%
%                        Patterns
%

\rulesec{Patterns}{\E,\V\ts\pat\ra\VE/\fail}
\begin{equation}	% atomic pattern
%\label{atpat-dyn-rule}
\frac{\E,\V\ts\atpat\ra \VE/\fail}
     {\E,\V\ts\atpat\ra \VE/\fail}\index{56.3}
\end{equation}

%\begin{equation}	% atomic pattern
%%\label{atpat-dyn-rule}
%\frac{\V\ts\atpat\ra \VE}
%     {\V\ts\atpat\ra \VE}
%\end{equation}

\replacement{\theidstatus}{\begin{equation}	% construction pattern
%\label{conpat-dyn-rule1}
\frac{\begin{array}{c}
       \longcon=\strid_1.\cdots.\strid_k.\con\neq\REF\qquad
      \V=(\con,\V')\\
      \E,\V'\ts\atpat\ra\VE/\fail
      \end{array}}
     {\E,\V\ts\conpat\ra \VE/\fail}
\end{equation}}{\begin{equation}	% construction pattern
%\label{conpat-dyn-rule1}
\frac{\begin{array}{c}
       \E(\longvid) = (\vid,\isc)\qquad\vid\neq\REF\qquad
      \V=(\vid,\V')\\
      \E,\V'\ts\atpat\ra\VE/\fail
      \end{array}}
     {\E,\V\ts\vidpat\ra \VE/\fail}
\end{equation}}

\replacement{\theidstatus}{\begin{equation}	% construction pattern
\label{conpat-dyn-rule2}
\frac{\longcon=\strid_1.\cdots.\strid_k.\con\neq\REF\qquad
      \V\notin\{\con\}\times\Val}
     {\E,\V\ts\conpat\ra \FAIL}
\end{equation}}{\begin{equation}	% construction pattern
\label{conpat-dyn-rule2}
\frac{\E(\longvid) = (\vid,\isc)\qquad\vid\neq\REF\qquad
      \V\notin\{\vid\}\times\Val}
     {\E,\V\ts\vidpat\ra \FAIL}
\end{equation}}

%\begin{equation}	% construction pattern
%\label{conpat-dyn-rule}
%\frac{\longcon=\strid_1.\cdots.\strid_k.\con\neq\REF\qquad\V\ts\atpat\ra\VE}
%     {(\con,\V)\ts\conpat\ra \VE}
%\end{equation}

\replacement{\theidstatus}{\begin{equation}        % exception construction
%\label{exconpat-dyn-rule1}
\frac{\begin{array}{c}
      \E(\longexn)=\e\qquad\V=(\e,\V')\\
      \E,\V'\ts\atpat\ra\VE/\FAIL
      \end{array}
     }
     {\E,\V\ts\exconpat\ra\VE/\FAIL}
\end{equation}}{\begin{equation}        % exception construction
%\label{exconpat-dyn-rule1}
\frac{\begin{array}{c}
      \E(\longvid)=(\e,\ise)\qquad\V=(\e,\V')\\
      \E,\V'\ts\atpat\ra\VE/\FAIL
      \end{array}
     }
     {\E,\V\ts\vidpat\ra\VE/\FAIL}
\end{equation}}

\replacement{\theidstatus}{\begin{equation} 
\label{exconpat-dyn-rule2}
\frac{\E(\longexn)=\e\qquad\V\notin\{\e\}\times\Val}
     {\E,\V\ts\exconpat\ra\FAIL}
\end{equation}}{\begin{equation} 
\label{exconpat-dyn-rule2}
\frac{\E(\longvid)=(\e,\ise)\qquad\V\notin\{\e\}\times\Val}
     {\E,\V\ts\vidpat\ra\FAIL}
\end{equation}}
\oldpagebreak
\begin{equation}	% reference pattern
%\label{refpat-dyn-rule}
\frac{\s(\A)=\V\qquad\s,\E,\V\ts\atpat\ra\VE/\fail,\s}
     {\s,\E,\A\ts\REF\ \atpat\ra \VE/\fail,\s}\index{57.0}
\end{equation}

%\begin{equation}	% reference pattern
%\label{refpat-dyn-rule}
%\frac{\s(\A)=\V\qquad\s,\V\ts\atpat\ra\VE,\s}
%     {\s,\A\ts\REF\ \atpat\ra \VE,\s}
%\end{equation}

\begin{equation}	% layered pattern
\frac{\ADD{\E,\V\ts\pat_1\ra\VE_1 \qquad \plusmap{\E}{\VE_1},\V\ts\pat_2\ra\VE_2/\fail}}
     {\ADD{\E,\V\ts\aspat\ra(\VE_1+\VE_2)/\fail}}
\end{equation}

\SameEqn
\begin{equation}	% layered pattern
\frac{\CUT{\E,\V\ts\pat\ra\VE/\fail}}
     {\CUT{\E,\V\ts\layeredvidpata\ra\{\vid\mapsto(\V,\isv)\}+\VE/\fail}}
\end{equation}
\NextEqn

\BeginNewEqns
\begin{equation}	% layered pattern
\frac{\ADD{\E,\V\ts\pat_1\ra\fail}}
     {\ADD{\E,\V\ts\aspat\ra\fail}}
\end{equation}

\begin{equation}	% or pattern
\frac{\ADD{\E,\V\ts\pat_1\ra\VE}}
     {\ADD{\E,\V\ts\orpat\ra\VE}}
\end{equation}

\begin{equation}	% or pattern
\frac{\ADD{\E,\V\ts\pat_1\ra\fail \qquad \E,\V\ts\pat_2\ra\VE/\fail}}
     {\ADD{\E,\V\ts\orpat\ra\VE/\fail}}
\end{equation}

\begin{equation}	% nested match pattern
\frac{\ADD{\E,\V\ts\pat_1\ra\fail}}
     {\ADD{\E,\V\ts\nestedpat\ra\fail}}
\end{equation}

\begin{equation}	% nested match pattern
\frac{\ADD{\E,\V\ts\pat_1\ra\VE_1 \qquad \plusmap{\E}{\VE_1},\V\ts\exp\ra\V'
      \qquad \plusmap{\E}{\VE_1},\V'\ts\pat_2\ra\VE/\fail}}
     {\ADD{\E,\V\ts\nestedpat\ra(\plusmap{\VE_1}{\VE_2})/\fail}}
\end{equation}

\EndNewEqns

%
%\begin{equation}	% layered pattern
%\label{layeredpat-dyn-rule}
%\frac{\V\ts\pat\ra\VE\qquad\{\var\mapsto\V\}\sim\VE}
%     {\V\ts\layeredpat\ra\VE}
%\end{equation}
%
%\begin{equation}	% failure of pattern
%\label{failpat-dyn-rule}
%\frac{\forall\VE\ (\V\ts\pat\not\ra\VE)}
%     {\V\ts\pat\ra\FAIL}
%\end{equation}
\comments
\begin{description}
\item{(\ref{conpat-dyn-rule2}),(\ref{exconpat-dyn-rule2})}
  Any evaluation resulting in $\FAIL$ must do so because 
rule~\ref{sconpat-dyn-rule2},
\deletion{\theidstatus}{rule~\ref{conapat-dyn-rule2},}
rule~\ref{exconapat-dyn-rule2},
rule~\ref{conpat-dyn-rule2},
or rule~\ref{exconpat-dyn-rule2} has been
applied.\index{57.1}
\end{description}

\clearpage{}
\thispagestyle{empty}
%!TEX root = root.tex
%
\section{Dynamic Semantics for Modules}
\label{dynmod-sec}
\subsection{Reduced Syntax}
Since\index{58.1} signature expressions
are mostly dealt with in the static semantics,
the dynamic semantics need only take limited account of them.  
\replacement{\theconstructors}{Unlike types,
it cannot ignore them completely; }{However,
they cannot be ignored completely; }the reason is that an explicit signature
ascription plays the r\^ole of restricting the ``view'' of a structure -- that is,
restricting the domains of its component environments\insertion{\theidstatus}{ and 
imposing identifier status on value identifiers}.  
\replacement{\thenostrsharing}{However, the types
and the sharing properties of structures and signatures are irrelevant to
dynamic evaluation; the syntax is therefore
reduced by the following transformations (in addition to those for the Core),
for the purpose of the dynamic semantics of Modules:}{The syntax is therefore
reduced by the following transformations (in addition to those for the Core),
for the purpose of the dynamic semantics of Modules:}
\begin{itemize}
\item Qualifications ``$\OF\ \ty\,$'' are omitted from \insertion{\theconstructors}{constructor and} exception descriptions.
\deletion{\thedatatyperepl}{\item Any specification of the form ``$\typespec$'', ``$\eqtypespec$'',
``$\DATATYPE$\ $\datdesc$\,'' or
``$\sharingspec$'' is replaced by the empty specification.}
\item \deletion{\thedatatyperepl}{The Modules phrase classes TypDesc, DatDesc, ConDesc and SharEq
      are omitted.}
\insertion{\thenostrsharing}{Any qualification \boxml{sharing type $\cdots$} on
a specification or \boxml{where type $\cdots$} on a signature expression is omitted.}
\end{itemize}

\subsection{Compound Objects}
\label{dynmod-comp-obj-sec}
The\index{58.2} compound objects for the Modules dynamic semantics, extra to those for the
Core dynamic semantics, are shown in Figure~\ref{comp-dynmod-obj}.
\begin{figure}[h]
\vspace{2pt}
\begin{displaymath}
\begin{array}{rcl}
\adhocreplacementl{\thenostrsharing}{1cm}{(\strid:\I,\strexp\langle:\I'\rangle,\B)}{(\strid:\I,\strexp,\B)}
                & \in   & \FunctorClosure\\
                &       & \qquad  = (\StrId\times\Int)\times
                          \adhocreplacementl{\thenostrsharing}{-5cm}{(\StrExp\langle\times\Int\rangle)\times\Basis}{\StrExp\times\Basis}\\
\adhocreplacementl{\theidstatus}{5mm}{(\IE,\vars,\exns)\ {\rm or}\ \I}{\I\ {\rm or}\ (\SI,\TI,\VI)}
                & \in   & \adhocreplacementl{\theidstatus}{-8cm}{\Int = \IntEnv\times\Fin(\Var)\times\Fin(\Exn)}{\Int = \StrInt\times\TyInt\times\ValInt}\\
        \adhocreplacementl{\theidstatus}{3cm}{\IE}{\SI}     & \in   & \adhocreplacementl{\theidstatus}{-9cm}{\IntEnv}{\StrInt} = \finfun{\StrId}{\Int}\\\adhocinsertion{\theidstatus}{2cm}{
        \TI     & \in   & \TyInt =  \finfun{\TyCon}{\ValInt}\\
        \VI     & \in   & \ValInt = \finfun{\VId}{\IdStatus}\\ }
        \G      & \in   & \SigEnv = \finfun{\SigId}{\Int}\\
        \F      & \in   & \FunEnv = \finfun{\FunId}{\FunctorClosure}\\
(\F,\G,\E)\ {\rm or}\ \B
                & \in   & \Basis = \FunEnv\times\SigEnv\times\Env\\
(\G,\adhocreplacementl{\thedatatyperepl}{2cm}{\IE}{\I})\ {\rm or}\ \IB
                & \in   & \IntBasis = \SigEnv\times\adhocreplacementl{\thedatatyperepl}{-4cm}{\IntEnv}{\Int}
\end{array}
\end{displaymath}
\caption{Compound Semantic Objects}
\label{comp-dynmod-obj}
\vspace{3pt}
\end{figure}
%
%
\insertion{\thedatatyperepl}{
An {\sl interface} $\I\in\Int$ represents a ``view'' of a structure.
Specifications and signature expressions will evaluate to interfaces;
moreover, during the evaluation of a specification or signature expression, 
structures (to which a specification or signature expression may
refer via datatype replicating specifications) are represented 
only by their interfaces.  To extract a value interface from
a dynamic value environment we define the operation $\Inter: \ValEnv \to\ValInt$
as follows:
\[\Inter(\VE) = \{\vid\mapsto\is\;;\;\VE(\vid) = (\V,\is)\}\]
In other words, $\Inter(\VE)$ is the value interface obtained from $\VE$ by
removing all values from $\VE$. We then extend $\Inter$ to a function
$\Inter:\Env\to\Int$ as follows:
\[ \Inter(\SE,\TE,\VE)\ =\ (\SI,\TI,\VI)\]
where $\VI$ = $\Inter(\VE)$ and 
\begin{eqnarray*}
\SI & = & \{\strid\mapsto\Inter\E\;;\;\SE(\strid) = \E\}\\
\TI & = & \{\tycon\mapsto\Inter\VE'\;;\;\TE(\tycon) = \VE'\}
\end{eqnarray*}
An {\sl interface basis} $\IB=(\G,\I)$ is a value-free part of a basis, sufficient to
evaluate signature expressions and specifications.
The function $\Inter$ is extended to create an interface basis
from a basis $\B$ as follows:
\[ \Inter(\F,\G,\E)\ =\ (\G, \Inter\E) \]
}

\deletion{\thedatatyperepl}{
An {\sl interface} $\I\in\Int$ represents a ``view'' of a structure.
Specifications and signature expressions will evaluate to interfaces; 
moreover, during the evaluation of a specification or signature expression, 
structures (to which a specification or signature expression may
refer via ``$\OPEN$'') are represented only by their interfaces.  To extract an
interface from a dynamic environment we define the operation
\[ \Inter\ :\ \Env\to\Int \]
as follows:
\[ \Inter(\SE,\VE,\EE)\ =\ (\IE,\Dom\VE,\Dom\EE)\]
where
\[ \IE\ =\ \{\strid\mapsto\Inter\E\ ;\ \SE(\strid)=\E\}\ .\]
An {\sl interface basis}\index{59.1} $\IB=(\G,\IE)$ is that part of a basis needed to
evaluate signature expressions and specifications.
The function $\Inter$ is extended to create an interface basis
from a basis $\B$ as follows:
\[ \Inter(\F,\G,\E)\ =\ (\G, \of{\IE}{(\Inter\E)}) \]}

\insertion{\thedatatyperepl}{
A further operation
\[ \downarrow\ :\ \Env\times\Int\to\Env\]
is required, to cut down an environment $\E$ to a given interface $\I$,
representing the effect of an explicit signature ascription. We first
define $\downarrow: \ValEnv\times\ValInt\to\ValEnv$ by
\[\VE\downarrow\VI = \{\vid\mapsto(\V,\is)\;;\;\VE(\vid) = (\V,\is')\ {\rm and}\ \VI(\vid) = \is\}\]
(Note that $\VE$ and $\VI$ need not have the same
domain and that the identifier status is taken from $\VI$.) 
We then define $\downarrow: \StrEnv \times \StrInt \to \StrEnv$,
$\downarrow: \TyEnv \times\TyInt\to\TyEnv$ and
$\downarrow: \Env\times\Int\to\Env$ simultaneously as follows:
\label{downarrowdef}
\begin{center}
 $\SE\downarrow\SI  =  \{\strid\mapsto\E\downarrow\I\ ;\
          	\SE(\strid)=\E\ {\rm and}\ \SI(\strid)=\I\}$\\[6pt]
 $ \TE\downarrow\TI =  \{\tycon\mapsto \VE'\downarrow\VI'\ ;\ 
               \TE(\tycon) = \VE'\ {\rm and}\ \TI(\tycon) = \VI'\}$ \\[6pt]
 $
 (\SE,\TE,\VE)\downarrow(\SI,\REPL{\TI\ }{\TE},\VI)  = 
               (\SE\downarrow\SI, \TE\downarrow\TI, \VE\downarrow\VI) $
\end{center}}

\deletion{\thedatatyperepl}{
A further operation
\[ \downarrow\ :\ \Env\times\Int\to\Env\]
is required, to cut down an environment $\E$ to a given interface $\I$,
representing the effect of an explicit signature ascription.  It is defined
as follows:
\[ (\SE,\VE,\EE)\downarrow(\IE,\vars,\exns)\ =\ (\SE',\VE',\EE') \]
where
\[ \SE'\ =\ \{\strid\mapsto\E\downarrow\I\ ;\
          \SE(\strid)=\E\ {\rm and}\ \IE(\strid)=\I\} \]
and (taking $\downarrow$ now to mean restriction of a function domain)
\[\VE'=\VE\downarrow\vars,\ \EE'=\EE\downarrow\exns.\]
}

\noindent
It is important to note that an interface \replacement{\thedatatyperepl}{is also a projection of}{can
also be obtained from} the
{\sl static} value $\Sigma$ of a signature expression; 
it is obtained by \replacement{\thedatatyperepl}{omitting structure names $\m$ and type environments
$\TE$,}{first replacing every type
structure $(\typefcn, \VE)$
in the range of every type environment $\TE$ by $\VE$}
and \insertion{\thece}{then} replacing each \replacement{\theidstatus}{variable environment $\VE$ and each 
exception environment $\EE$ by its domain.}{pair $(\sigma,\is)$ in the range
of every value environment $\VE$ by $\is$.}
Thus in an implementation interfaces would naturally be obtained from the
static elaboration; we choose to give separate rules here for obtaining them
in the dynamic semantics since we wish to maintain our separation of the
static and dynamic semantics, for reasons of presentation.

\subsection{Inference Rules}
The\index{59.2} semantic rules allow sentences  of the form
\[ \s,A\ts\phrase\ra A',\s' \]
to be inferred, where $A$ is either a basis\replacement{\theidstatus}{ or an interface basis}{, a signature environment} or empty,
$A'$ is some semantic
object and $\s$,$\s'$ are the states before and after the evaluation
represented by the sentence.  Some hypotheses in rules are not of this form;
they are called {\sl side-conditions}.  The convention for options is
the same as for the Core static semantics.  

The state and exception conventions are adopted as in the Core dynamic
semantics.  However, it may be shown that the only Modules phrases whose 
evaluation
may cause a side-effect or generate an exception packet are of the form
$\strexp$, $\strdec$, $\strbind$ or $\topdec$.
%Also, as will be seen in Section~\ref{prog-sec}, a phrase of the
%form $\program$ can have side-effects, but not generate an
%exception packet.

%               SEMANTICS
%
%                       Structure Expressions
%
\rulesec{Structure Expressions}{\B\ts\strexp\ra \E/\p}
\begin{equation}        % generative strexp
%\label{generative-strexp-dyn-rule}
\frac{\B\ts\strdec\ra\E}
     {\B\ts\encstrexp\ra\E}\index{60.1}
\end{equation}

\begin{equation}        % longstrid
%\label{longstrid-strexp-dyn-rule}
\frac{\B(\longstrid)=\E}
     {\B\ts\longstrid\ra\E}
\end{equation}

\insertion{\theconstructors}{\begin{equation}        % transparent signature constraint
\frac{\B\ts\strexp\ra\E\qquad\Inter\B\ts\sigexp\ra\I}
     {\B\ts\transpconstraint\ra\E\downarrow\I}
\end{equation}

\begin{equation}        % opaque signature constraint
\frac{\B\ts\strexp\ra\E\qquad\Inter\B\ts\sigexp\ra\I}
     {\B\ts\opaqueconstraint\ra\E\downarrow\I}
\end{equation}}

%\vspace{6pt}
\replacement{\thenostrsharing}{\begin{equation}                % functor application
\label{functor-application-dyn-rule}
\frac{ \begin{array}{c}
        \B(\funid)=(\strid:\I,\strexp'\langle:\I'\rangle,\B')\\
        \B\ts\strexp\ra\E\qquad
       \B'+\{\strid\mapsto\E\downarrow\I\}\ts\strexp'\ra\E'\\
       \end{array}
     }
     {\B\ts\funappstr\ra\E'\langle\downarrow\I'\rangle}
\end{equation}}{\begin{equation}                % functor application
\label{functor-application-dyn-rule}
\frac{ \begin{array}{c}
        \B(\funid)=(\strid:\I,\strexp',\B')\\
        \B\ts\strexp\ra\E\qquad
       \B'+\{\strid\mapsto\E\downarrow\I\}\ts\strexp'\ra\E'\\
       \end{array}
     }
     {\B\ts\funappstr\ra\E'}
\end{equation}}

%\vspace{6pt}
\begin{equation}        % let strexp
%\label{letstrexp-dyn-rule}
\frac{\B\ts\strdec\ra\E\qquad\B+\E\ts\strexp\ra\E'}
     {\B\ts\letstrexp\ra\E'}
\end{equation}
\comments
\begin{description}
\item{(\ref{functor-application-dyn-rule})}
Before the evaluation of the functor body $\strexp'$, the
actual argument $\E$ is cut down by the formal parameter
interface $\I$, so that any opening of $\strid$ resulting
from the evaluation of $\strexp'$ will produce no more components
than anticipated during the static elaboration.
\end{description}

\rulesec{Structure-level Declarations}{\B\ts\strdec\ra\E/\p}
                % declarations
\begin{equation}                % core declaration
%\label{dec-dyn-rule}
\frac{ \of{\E}{\B}\ts\dec\ra\E' }
     { \B\ts\dec\ra\E' }\index{60.2}
\end{equation}

\vspace{6pt}
\begin{equation}                % structure declaration
%\label{structure-decl-dyn-rule}
\frac{ \B\ts\strbind\ra\SE }
     { \B\ts\singstrdec\ra\SE\ \In\ \Env }
\end{equation}

\vspace{6pt}
\begin{equation}                % local structure-level declaration
%\label{local structure-level declaration-dyn-rule}
\frac{ \B\ts\strdec_1\ra\E_1\qquad
       \B+\E_1\ts\strdec_2\ra\E_2 }
     { \B\ts\localstrdec\ra\E_2 }
\end{equation}

\vspace{6pt}
\begin{equation}                % empty declaration
%\label{empty-strdec-dyn-rule}
\frac{}
     {\B\ts\emptystrdec\ra \emptymap{\rm\ in}\ \Env}
\end{equation}

\vspace{6pt}
\begin{equation}                % sequential declaration
%\label{sequential-strdec-dyn-rule}
\frac{ \B\ts\strdec_1\ra\E_1\qquad
       \B+\E_1\ts\strdec_2\ra\E_2 }
     { \B\ts\seqstrdec\ra\plusmap{\E_1}{\E_2} }
\end{equation}

\rulesec{Structure Bindings}{\B\ts\strbind\ra\SE/\p}
\replacement{\thenostrsharing}{
\begin{equation}                % structure binding
\frac{ \begin{array}{cl}
       \B\ts\strexp\ra\E\qquad\langle\Inter\B\ts\sigexp\ra\I\rangle\\
       \langle\langle\B\ts\strbind\ra\SE\rangle\rangle
       \end{array}
     }
     {\begin{array}{c}
      \B\ts\strbinder\ra\\
      \qquad\qquad\qquad\{\strid\mapsto\E\langle\downarrow\I\rangle\}
      \ \langle\langle +\ \SE\rangle\rangle
      \end{array}
     }\index{61.1}
\end{equation}
\comment As in the static semantics, when present, $\sigexp$ constrains the
``view'' of the structure. The restriction must be done in the
dynamic semantics to ensure that any dynamic opening of the structure
produces no more components than anticipated during the static
elaboration.}{\begin{equation}                % structure binding
\frac{ 
       \B\ts\strexp\ra\E\qquad
       \langle\B\ts\strbind\ra\SE\rangle
     }
     {
      \B\ts\barestrbindera\ra\{\strid\mapsto\E\}
      \ \langle +\ \SE\rangle
     }\index{61.1}
\end{equation}
}
%
%                   Signature Rules
%

\rulesec{Signature Expressions}{\IB\ts\sigexp\ra\I}
\begin{equation}                % encapsulation sigexp
%\label{encapsulating-sigexp-dyn-rule}
\frac{\IB\ts\spec\ra\I }
     {\IB\ts\encsigexp\ra\I}\index{61.2}
\end{equation}

\begin{equation}                % signature identifier
%\label{signature-identifier-dyn-rule}
\frac{ \IB(\sigid)=\I}
     { \IB\ts\sigid\ra\I }
\end{equation}

\rulesec{Signature Declarations}{\IB\ts\sigdec\ra\G}
\begin{equation}        % single signature declaration
%\label{single-sigdec-dyn-rule}
\frac{ \IB\ts\sigbind\ra\G }
     { \IB\ts\singsigdec\ra\G }\index{61.3}
\end{equation}

\deletion{\thenostrsharing}{
\begin{equation}        % empty signature declaration
%\label{empty-sigdec-dyn-rule}
\frac{}
     { \IB\ts\emptysigdec\ra\emptymap }
\end{equation}

\begin{equation}        % sequential signature declaration
%\label{sequence-sigdec-dyn-rule}
\frac{ \IB\ts\sigdec_1\ra\G_1 \qquad \plusmap{\IB}{\G_1}\ts\sigdec_2\ra\G_2 }
     { \IB\ts\seqsigdec\ra\plusmap{\G_1}{\G_2} }
\end{equation}
}
\rulesec{Signature Bindings}{\IB\ts\sigbind\ra\G}
\begin{equation}        % signature binding
%\label{sigbind-dyn-rule}
\frac{ \IB\ts\sigexp\ra\I
        \qquad\langle\IB\ts\sigbind\ra\G\rangle }
     { \IB\ts\sigbinder\ra\{\sigid\mapsto\I\}
       \ \langle +\ \G\rangle }\index{61.4}
\end{equation}
%
                     % Specifications
%
\rulesec{Specifications}{\IB\ts\spec\ra\I}

\replacement{\theidstatus}{\begin{equation}        % value specification
%\label{valspec-dyn-rule}
\frac{ \ts\valdesc\ra\vars }
     { \IB\ts\valspec\ra\vars\ \In\ \Int }\index{61.5}
\end{equation}}{\begin{equation}        % value specification
%\label{valspec-dyn-rule}
\frac{ \ts\valdesc\ra\VI }
     { \IB\ts\valspec\ra\VI\ \In\ \Int }\index{61.5}
\end{equation}}

\insertion{\thedatatyperepl}{
\begin{equation}
\frac{\ts\typdesc\ra\TI}
     {\IB\ts\typespec\ra\TI\ \In\ \Int}
\end{equation}

\begin{equation}
\frac{\ts\typdesc\ra\TI}
     {\IB\ts\eqtypespec\ra\TI\ \In\ \Int}
\end{equation}}

\insertion{\thedatatyperepl}{
\begin{equation}
\frac{\ts\datdesc\ra\VI,\TI}
     {\IB\ts\datatypespec\ra(\VI,\TI)\ \In\ \Int}
\end{equation}
}

\insertion{\thedatatyperepl}{
\begin{equation}
\frac{\IB(\longtycon) = \VI\qquad \TI = \{\tycon\mapsto\VI\}}
     {\IB\ts\datatypereplspec\ra(\VI,\TI)\ \In\ \Int}
\end{equation}
}

\replacement{\theidstatus}{
\begin{equation}        % exception specification
%\label{exceptionspec-dyn-rule}
\frac{ \ts\exndesc\ra\exns}
     { \IB\ts\exceptionspec\ra\exns\ \In\ \Int }
\end{equation}}{\begin{equation}        % exception specification
%\label{exceptionspec-dyn-rule}
\frac{ \ts\exndesc\ra\VI}
     { \IB\ts\exceptionspec\ra \VI\ \In\ \Int }
\end{equation}}
\oldpagebreak
\replacement{\theidstatus}{\begin{equation}        % structure specification
\label{structurespec-dyn-rule}
\frac{ \IB\ts\strdesc\ra\IE }
     { \IB\ts\structurespec\ra\IE\ \In\ \Int }\index{62.1}
\end{equation}}{\begin{equation}        % structure specification
\label{structurespec-dyn-rule}
\frac{ \IB\ts\strdesc\ra\SI }
     { \IB\ts\structurespec\ra\SI\ \In\ \Int }\index{62.1}
\end{equation}}

\deletion{\thenolocalspec}{
\begin{equation}        % local specification
\label{localspec-dyn-rule}
\frac{ \IB\ts\spec_1\ra\I_1 \qquad
       \plusmap{\IB}{\of{\IE}{\I_1}}\ts\spec_2\ra\I_2 }
     { \IB\ts\localspec\ra\I_2 }
\end{equation}}

\deletion{\thenoopenspec}{
\begin{equation}        % open specification
%\label{openspec-dyn-rule}
\frac{ \IB(\longstrid_1)=\I_1\quad\cdots\quad
       \IB(\longstrid_n)=\I_n }
     { \IB\ts\openspec\ra\I_1 + \cdots +\I_n }
\end{equation}}

\replacement{\thesingleincludespec}{
\begin{equation}        % include signature specification
%\label{inclspec-dyn-rule}
\frac{ \IB(\sigid_1)=\I_1 \quad\cdots\quad
       \IB(\sigid_n)=\I_n }
     { \IB\ts\inclspec\ra\I_1 + \cdots +\I_n }
\end{equation}}{\begin{equation}        % include signature specification
%\label{inclspec-dyn-rule}
\frac{ \IB\ts\sigexp\ra\I}
     { \IB\ts\singleinclspec\ra\I}
\end{equation}}

\begin{equation}        % empty specification
%\label{emptyspec-dyn-rule}
\frac{}
     { \IB\ts\emptyspec\ra\emptymap{\rm\ in}\ \Int }
\end{equation}

\replacement{\theidstatus}{\begin{equation}        % sequential specification
\label{seqspec-dyn-rule}
\frac{ \IB\ts\spec_1\ra\I_1
       \qquad \plusmap{\IB}{\of{\IE}{\I_1}}\ts\spec_2\ra\I_2 }
     { \IB\ts\seqspec\ra\plusmap{\I_1}{\I_2} }
\end{equation}}{\begin{equation}        % sequential specification
\label{seqspec-dyn-rule}
\frac{ \IB\ts\spec_1\ra\I_1
       \qquad \IB+\I_1\ts\spec_2\ra\I_2 }
     { \IB\ts\seqspec\ra\plusmap{\I_1}{\I_2} }
\end{equation}}

%\ insertion{\thenostrsharing}{
%\begin{equation}
%\frac{\IB\ts\spec\ra\I}    %  type sharing spec
%     {\IB\ts\newsharingspec\ra\I}
%\end{equation}
%}

\deletion{\thedatatyperepl}{
\noindent\comments
\begin{description}
\item{(\ref{localspec-dyn-rule}),(\ref{seqspec-dyn-rule})}
Note that $\of{\vars}{\I_1}$ and $\of{\exns}{\I_1}$ are
not needed for the evaluation of $\spec_2$.
\end{description}}

                         % Descriptions

\replacement{\theidstatus}{\rulesec{Value Descriptions}{\ts\valdesc\ra\vars}}{\rulesec{Value Descriptions}{\ts\valdesc\ra\VI}}
\replacement{\theidstatus}{\begin{equation}         % value description
%\label{valdesc-dyn-rule}
\frac{ \langle\ts\valdesc\ra\vars\rangle }
     { \ts\var\ \langle\AND\ \valdesc\rangle\ra
       \{\var\}\ \langle\cup\ \vars\rangle }\index{62.2}
\end{equation}}{\begin{equation}         % value description
%\label{valdesc-dyn-rule}
\frac{ \langle\ts\valdesc\ra\VI\rangle }
     { \ts\vid\ \langle\AND\ \valdesc\rangle\ra
       \{\vid\mapsto\isv\}\ \langle+\,\VI\rangle }\index{62.2}
\end{equation}}

\insertion{\thedatatyperepl}{
\rulesec{Type Descriptions}{\ts\typdesc\ra\TI}
\begin{equation}
\frac{\langle\ts\typdesc\ra\TI\rangle}
     {\ts\typdescription\ra\{\tycon\mapsto\emptymap\}\langle+\TI\rangle}
\end{equation}
}

\insertion{\thedatatyperepl}{
\rulesec{Datatype Descriptions}{\ts\datdesc\ra\VI, \TI}
\begin{equation}
\frac{\ts\condesc\ra\VI\qquad\langle\ts\datdesc'\ra\VI',\TI'\rangle}
     {\ts\datdescriptiona\ra\VI\,\langle+\,\VI'\rangle, \{\tycon\mapsto\VI\}\langle+\TI'\rangle}
\end{equation}

\rulesec{Constructor Descriptions}{\ts\condesc\ra\VI}
\begin{equation}
\frac{\langle\ts\condesc\ra\VI\rangle}
     {\ts\shortconviddesc\ra\{\vid\mapsto\isc\}\,\langle+\VI\rangle}
\end{equation}
}

\replacement{\theidstatus}{\rulesec{Exception Descriptions}{\ts\exndesc\ra\exns}}{\rulesec{Exception Descriptions}{\ts\exndesc\ra\VI}}
\replacement{\thefixtypos}{
\begin{equation}         % exception description
%\label{exndesc-dyn-rule}
\frac{ \langle\ts\exndesc\ra\exns\rangle }
     { \ts\exn\ \langle\exndesc\rangle\ra\{\exn\}\ \langle\cup\ \exns\rangle }\index{62.3}
\end{equation}}{\begin{equation}         % exception description
%\label{exndesc-dyn-rule}
\frac{ \langle\ts\exndesc\ra\VI\rangle }
     { \ts\vid\ \langle\boxml{and\ }\exndesc\rangle\ra\{\vid\mapsto\ise\}\ \langle+ \VI\rangle }\index{62.3}
\end{equation}}

\replacement{\theidstatus}{\rulesec{Structure Descriptions}{\IB\ts\strdesc\ra\IE}}{\rulesec{Structure Descriptions}{\IB\ts\strdesc\ra\SI}}
\replacement{\theidstatus}{\begin{equation}
%\label{strdesc-dyn-rule}
\frac{ \IB\ts\sigexp\ra\I\qquad\langle\IB\ts\strdesc\ra\IE\rangle }
     { \IB\ts\strdescription\ra\{\strid\mapsto\I\}\ \langle +\ \IE\rangle }\index{62.4}
\end{equation}}{\begin{equation}
%\label{strdesc-dyn-rule}
\frac{ \IB\ts\sigexp\ra\I\qquad\langle\IB\ts\strdesc\ra\SI\rangle }
     { \IB\ts\strdescription\ra\{\strid\mapsto\I\}\ \langle +\ \SI\rangle }\index{62.4}
\end{equation}}

%                       Functor and Program rules
%
\rulesec{Functor Bindings}{\B\ts\funbind\ra\F}
\begin{equation}        % functor binding
%\label{funbind-dyn-rule}
\frac{
      \Inter\B\ts\sigexp\ra\I\qquad
      \langle\REPL{\B\ }{\IB}\ts\funbind\ra\F\rangle
     }
     {
      \begin{array}{c}
       \REPL{\B\ }{\IB}\ts\barefunstrbinder\ \optfunbinda\ra\\
       \qquad\qquad \qquad
              \{\funid\mapsto(\strid:\I,\strexp,\B)\}
              \ \langle +\ \F\rangle
      \end{array}
     }\index{62.5}
\end{equation}
\oldpagebreak
\rulesec{Functor Declarations}{\B\ts\fundec\ra\F}
\begin{equation}        % single functor declaration
%\label{singfundec-dyn-rule}
\frac{ \B\ts\funbind\ra\F }
     { \B\ts\singfundec\ra\F }\index{63.1}
\end{equation}

\rulesec{Top-level Declarations}{\B\ts\topdec\ra\B'/\p}
\begin{equation}        % structure-level declaration
%\label{strdectopdec-dyn-rule}
\frac{\B\ts\strdec\ra\E\quad\B' =\E\ \In\ \Basis\quad\langle \B+\B'\ts\topdec\ra\B''\rangle }
     {\B\ts\strdecintopdec\ra\B'\langle\REPL{+\B\ }{'}\rangle
     }\index{63.2}
\end{equation}

\vspace{6pt}
\begin{equation}        % signature declaration
%\label{sigdectopdec-dyn-rule}
\frac{\Inter\B\ts\sigdec\ra\G\quad B' = \G\ \In\ \Basis\quad
       \langle \B + \B'\ts\topdec\ra\B''\rangle}
     {\B\ts\sigdecintopdec\ra \B'\langle\REPL{+\B\ }{'}\rangle 
     }
\end{equation}

\vspace{6pt}
\begin{equation}        % functor declaration
%\label{fundectopdec-dyn-rule}
\frac{\B\ts\fundec\ra\F\quad \B' = \F\ \In\ \Basis\quad
       \langle \B + \B'\ts\topdec\ra\B''\rangle}
     {\B\ts\fundecintopdec\ra\B'\langle\REPL{+\B\ }{'}\rangle
     }
\end{equation}




\clearpage{}
\thispagestyle{empty}
%!TEX root = root.tex
%

\section{Programs}
\label{prog-sec}
The phrase class Program of programs is defined as follows
\[\program ::= \longprog\]

Hitherto,\index{64.1} the semantic rules have not exposed the interactive
nature of the language.
During an ML session the user can type in a phrase, more
precisely a phrase of the form $\topdec$ as defined in Figure~\ref{prog-syn},
page~\pageref{prog-syn}. Upon the following semicolon,
the machine will then attempt to parse, elaborate and
evaluate the phrase returning either a result or, if any
of the phases fail, an error message.
The outcome is significant for what the user subsequently types,
so we need to answer questions such as: if the elaboration
of a top-level declaration succeeds, but its evaluation
fails, then does the result of the elaboration get
recorded in the static basis?

In practice, ML implementations may provide a directive as
a form of top-level declaration  for including
programs from files rather than directly from the terminal. 
In case a file consists of a sequence of top-level declarations 
(separated by semicolons) and the machine detects an error in one of these,
it is probably sensible to abort the execution of the directive.
Rather than introducing a distinction between,
say, batch programs and interactive programs, we shall tacitly regard
all programs as interactive, and leave to implementers to
clarify how the inclusion of files, if provided,  affects the 
updating of the static and dynamic basis.
Moreover, we shall focus on elaboration and evaluation and leave the
handling of parse errors to implementers (since it naturally depends
on the kind of parser being employed).  Hence, in this section the 
{\sl execution} of a program means the combined elaboration
and evaluation of the program.

So far, for simplicity, we have used the same notation $\B$
to stand for both a static and a dynamic basis, and this has
been possible because we have never needed to discuss static
and dynamic semantics at the same time.
In giving the semantics of programs, however, let us rename as 
\mbox{StaticBasis} the class  Basis defined in the static
semantics of modules, Section~\ref{statmod-sem-obj-sec},
and let us use $\Bstat$ to range over StaticBasis. Similarly, let
us rename as  \mbox{DynamicBasis} the class Basis defined in the
dynamic semantics of modules, Section~\ref{dynmod-comp-obj-sec},
and let us use $\Bdyn$ to range over \mbox{DynamicBasis}.
We now define
\[\B\ {\rm or} \ (\Bstat,\Bdyn)\in{\rm Basis}={\rm Static Basis}
\times {\rm DynamicBasis}.\]
Further, we shall use $\tsstat$ for elaboration as defined in 
Section~\ref{statmod-sec}, and $\tsdyn$ for evaluation
as defined in Section~\ref{dynmod-sec}.
Then $\ts$ will be reserved for the execution of
programs, which thus is expressed by a sentence of the
form
\[\s,\B\ts\program\ra\B',\s'\]
This may be read as follows: 
starting in basis $\B$ with state $\s$ the execution of
$\program$ results in a basis $\B'$ and a state $\s'$.

It\index{64.2} must be understood  that executing a program never results in an
exception. If the evaluation of a $\topdec$ yields an exception 
(for instance because of a $\RAISE$ expression\deletion{\thelibrary}{ or external intervention}) then
the result of executing the program ``$\topdec\ \mbox{\ml{;}}$'' is the
original basis together with the state which is in force when the exception is
generated. In particular, the exception convention\index{65.1} of 
Section~\ref{dyncor-inf-rules-sec}
is not applicable to the ensuing rules.

We represent the non-elaboration of a top-level declaration by
 $\ldots\tsstat\topdec\not\ra$. \deletion{\thelibrary}{(This covers also the case in which a 
user interrupts the elaboration.)}

\rulesec{Programs}{\s,\B\ts\program\ra\B',s'}
\begin{equation}            % failing elaboration
\label{fail-elab-prog-rule}
\frac{\of{\Bstat}{\B}\tsstat\topdec\not\ra\qquad
      \langle\s,\B\ts\program\ra\B',s'\rangle}
     {\s,\B\ts\longprog\ra\B\langle'\rangle,s\langle'\rangle}\index{65.2}
\end{equation}

\begin{equation}            % evaluation raising exception
\label{dyn-exc-prog-rule}
\frac{\begin{array}{lr}
      \of{\Bstat}{\B}\tsstat\topdec\ra\Bstat^{(1)} & \\
      \s,\of{\Bdyn}{\B}\tsdyn\topdec\ra\p,\s' &
                  \langle\s',\B\ts\program\ra\B',\s''\rangle
      \end{array}}
     {\s,\B\ts\longprog\ra\B\langle'\rangle,\s'\langle'\rangle}
\end{equation}

\begin{equation}            % success
\label{success-prog-rule}
\frac{\begin{array}{ll}
       \of{\Bstat}{\B}\tsstat\topdec\ra\Bstat^{(1)} & \\
       \s,\of{\Bdyn}{\B}\tsdyn\topdec\ra\Bdyn^{(1)},s'&
                   \B'=\B\oplus(\Bstat^{(1)},\Bdyn^{(1)})\\
       &           \langle\s',\B'\ts\program\ra\B'',s''\rangle
      \end{array}
      }
      {\s,\B\ts\longprog\ra\B'\langle'\rangle,\s'\langle'\rangle}
\end{equation}
\comments
\begin{description}
\item{(\ref{fail-elab-prog-rule})}
  A failing elaboration has no effect whatever\ADD{, except possibly for
  fixity directives contained in the \topdec}.
\item{(\ref{dyn-exc-prog-rule})}
  An evaluation which yields an exception nullifies
the change in the static basis, but does not
nullify side-effects on the state which may have occurred
before the exception was raised.
\end{description}
\subsection*{Core language Programs}
A\index{65.4} program is called a {\sl core language program\/} if it can be
parsed in the reduced grammar defined as follows:
\begin{enumerate}
\item Replace the definition of top-level declarations by
\[\topdec\ ::=\ \strdec\]
\item Replace the definition of structure-level declarations by
\[\strdec\ ::=\ \dec\]
\deletion{\thelibrary}{
\item Omit\index{65.5} the {\OPEN} declaration from the syntax class of
declarations $\dec$
\item Restrict the long identifier classes to identifiers, i.e.
omit qualified identifiers.}
\end{enumerate}
\deletion{\thelibrary}{
This means that  several components of a basis, for example the
signature and functor environments, are irrelevant to the execution
of a core language program.}
\appendix
\clearpage{}
\thispagestyle{empty}
%!TEX root = root.tex
%

\section{Appendix: Derived Forms}
\label{derived-forms-app}
Several derived\index{66.1} grammatical forms are provided in the Core; they are presented
in Figures~\ref{der-exp}, \ref{der-pat} and \ref{der-dec}. Each derived form is
given with its equivalent form. Thus, each row of the tables should be
considered as a rewriting rule
\[ \mbox{Derived form \ $\Longrightarrow$\  Equivalent form} \]
and these rules may be applied repeatedly to a phrase until it is transformed
into a phrase of the bare language.
See Appendix~\ref{core-gram-app} for the full Core grammar, including all the
derived forms.

In the derived forms for tuples, in terms of records, we use $\overline{n}$ to
mean the ML numeral which stands for the natural number $n$.

Note that a new phrase class ~FvalBind~ of function-value bindings is introduced,
accompanied by a new declaration form ~\FUN\ \tyvarseq\ \fvalbind~. The mixed forms
\REPL{~\VAL\ \REC\ \tyvarseq\ \fvalbind~}{~\VAL\ \tyvarseq\ \REC\ \fvalbind}~,
~\VAL\ \tyvarseq\ \fvalbind~
and ~\FUN\ \tyvarseq\ \valbind~ are not
allowed --- though the first form arises during translation into the bare
language.

\ADD{
In the derived form for record update in Figure~\ref{der-exp}, the \labexps\ may not
contain ellipses.
Furthermore, the term \labpats\ is obtained from \labexps\ by replacing all of the
right-hand sides by wildcards.
The derived form for ellipses in the middle of
expression rows is only valid if it can
be transformed to bare syntax.
This restriction implies that the remaining rows may not again contain
ellipses.
}

\ADD{
Note that the derived forms in Figure~\ref{der-pat} for ellipses in the middle of
pattern and type-expression rows are only valid if they can
be transformed to bare syntax.
This restriction implies that the remaining rows may not again contain
ellipses.
}

The following notes refer to Figure~\ref{der-dec}:
\begin{itemize}
%\item      In the equivalent form for a function-value binding, the
%           variables ~$\var_1$, $\cdots$, $\var_n$~ must be chosen not to
%           occur in the derived form.  The condition $m,n\geq 1$ applies.
\item      There is a version of the derived form for function-value binding
	   which allows the function identifier to be infixed;
	   see Figure~\ref{dec-gram} in Appendix~\ref{core-gram-app}.
\item      In the two forms involving ~\WITHTYPE~, the identifiers bound
           by ~\datbind~ and by ~\typbind~ must be distinct. Then the
           transformed binding ~\datbind$\/'$~ in the equivalent form is
           obtained from ~\datbind~ by expanding out all the definitions
           made by ~\typbind.  More precisely, if ~\typbind~ is
           \[ \tyvarseq_1\ \tycon_1\ \mbox{\ml{=}} \ty_1\ \ \AND
              \ \cdots\ \AND
            \ \ \tyvarseq_n\ \tycon_n\ \mbox{\ml{=}} \ty_n\ \]
           then $\hbox{\datbind}'$ is the result of simultaneous replacement
           (in ~\datbind) of every type expression ~$\tyseq_i\ \tycon_i$~
           ($1\leq i\leq n$)
           by the corresponding defining expression
           \[  \ty_i\{\tyseq_i/\tyvarseq_i\}\]
%\item      The abbreviation of ~\VAL\ {\tt it =} \exp~ to ~\exp~ is only
%           permitted at top-level, i.e. as a ~\program~.
\item[\textcolor{\addcolor}{$\bullet$}]
	\ADD{In the abstype form, $\typbind'$ is obtained from
	\datbind\ by replacing all right-hand sides with the corresponding left-hand side,
	i.e.,\ ``\tyvarseq\ \tycon\ \ml{=} \constrs\ $\langle$\ml{|} \datbind$\rangle$'' becomes
	``\tyvarseq\ \tycon\ \ml{=} \tyvarseq\ \tycon\ $\langle$\ml{|} $\typbind'\ \rangle$''}
\end{itemize}

Figure~\ref{functor-der-forms-fig} shows derived forms for functors.
They allow functors to take, say, a single type or value as a parameter,
in cases where it would seem clumsy to ``wrap up'' the argument as a
structure expression.
\deletion{\thenostrsharing}{
These forms are currently more experimental than the bare syntax of modules, 
but we recommend implementers to include them so that they can be
tested in practice.
In the derived forms for functor bindings and functor signature expressions,
$\strid$ is a new structure identifier and
the form of $\sigexp'$ depends
on the form of $\sigexp$ as follows. 
If $\sigexp$ is simply a signature identifier
$\sigid$, then $\sigexp'$ is also $\sigid$; otherwise $\sigexp$ must take
the form  ~$\SIG\ \spec_1\ \END$~,
and then $\sigexp'$ is
$\mbox{\SIG\ \LOCAL\ \OPEN\ \strid\ \IN\ $\spec_1$\ \END\ \END}$.
%(where $\strid$ is new).
}

Finally, Figure~\ref{spec-der-forms-fig} shows the derived forms for specifications \CUT{and signature expressions}.
\ADD{
In the form involving \WITHTYPE, the identifiers bound by \datdesc\ and by \typbind\ must be distinct.
The transformed description $\datdesc'$ is obtained from \datdesc\ by expanding out all the definitions
made by \typbind, analogous to \datbind\ above.
The phrase ``\TYPE\ \typbind'' can be reinterpreted as a type specification that is subject to further transformation.
}
The last derived form for specifications allows sharing between structure
identifiers as a shorthand for type sharing specifications. 
%Standard ML no 
%longer has a semantic notion of structure sharing; 
%however, for compatability with Standard ML '90, a weaker form
%of structure sharing specification is provided. 
The phrase
\[
\boxml{$\spec$ sharing $\longstrid_1$ = $\cdots$ = $\longstrid_k$}
\]
is a derived form whose equivalent form is
%\pagebreak
\[\begin{array}{l}
  \boxml{$\spec$}\cr
  \boxml{\ \ sharing type $\longtycon_{1}$ =  $\longtycon_{1}'$}\cr
  \boxml{\ \ \ $\cdots$}\cr
  \boxml{\ \ sharing type $\longtycon_{m}$ =  $\longtycon_{m}'$}\cr
\end{array}\]

\noindent
determined as follows. 
First, note that  $\spec$  specifies a set of 
(possibly long) type constructors and structure identifiers, either 
directly or via signature identifiers and $\INCLUDE$ specifications.  
Then the equivalent form contains all type-sharing constraints 
of the form 
\[
\boxml{sharing type $\longstrid_i.\longtycon$ = $\longstrid_j.longtycon$}
\]
$(1\leq i<j\leq k)$,  such that both sides of the equation are long type 
constructors specified by  $\spec$. 

The meaning of the derived form does not depend on the order of the 
type-sharing constraints in the equivalent form.


\begin{figure}
\begin{tabular}{|l|l|l}
\multicolumn{1}{c}{Derived Form} & \multicolumn{1}{c}{Equivalent Form} &
\multicolumn{1}{c}{}\\
\multicolumn{3}{c}{}\\
\multicolumn{2}{l}{{\bf Expressions} \exp}\\
%\multicolumn{2}{l}{EXPRESSIONS \exp}\\
\cline{1-2}
\ml{()}         & \ml{\lttbrace\ \rttbrace} \\
\cline{1-2}
\ml{(}$\exp_1$ \ml{,} $\cdots$ \ml{,} $\exp_\n$\ml{)}
            & \ml{\lttbrace 1=}$\exp_1$\ml{,}\ $\cdots$\ml{,}\
                             $\overline{n}$\ml{=}$\exp_\n$\ml{\rttbrace}
                                                           & $(\n\geq 2)$\\
\cline{1-2}
\ml{\#}\ \lab      & \FN\ \ml{\lttbrace}\lab\ml{=}\vid\ml{,...\rttbrace\  => }\vid
                                                           & (\vid\ new)\\
%\cline{1-2}
%\RAISE\ \longexn    & \RAISE\ \longexn\ \WITH\ \ml{()} \\
\cline{1-2}
\CASE\ \exp\ \OF\ \match
                & \ml{(}\FN\ \match\ml{)(}\exp\ml{)} \\
\cline{1-2}
\IF\ $\exp_1$\ \THEN\ $\exp_2$\ \ELSE\ $\exp_3$
                & \CASE\ $\exp_1$\ \OF\ \TRUE\ \ml{=>}\ \exp$_2$\\
                & \ \ \qquad\qquad\ml{|}\ \FALSE\ \ml{=>}\ \exp$_3$ \\
\cline{1-2}
\ADD{\IF\ $\exp_1$\ \THEN\ $\exp_2$}
		& \ADD{\IF\ $\exp_1$\ \THEN\ $\exp_2$\ \ELSE\ \ml{()}} \\
\cline{1-2}
\exp$_1$\ \ORELSE\ \exp$_2$
                & \IF\ \exp$_1$\ \THEN\ \TRUE\ \ELSE\ \exp$_2$ \\
\cline{1-2}
\exp$_1$\ \ANDALSO\ \exp$_2$
                & \IF\ \exp$_1$\ \THEN\ \exp$_2$\ \ELSE\ \FALSE \\
\cline{1-2}
\ml{(}$\exp_1$ \ml{;} $\cdots$ \ml{;} $\exp_\n$ \ml{;} \exp\ml{)}\
                & \CASE\ \exp$_1$\ \OF\ \ml{(\wildpat) =>}
                                                           & $(\n\geq 1)$ \\
                & \qquad$\cdots$ \\
                & \CASE\ \exp$_n$\ \OF\ \ml{(\wildpat) =>}\ \exp \\
\cline{1-2}
\LET\ \dec\ \IN
                & \LET\ \dec\ \IN                          & $(\n\geq 2)$ \\
\qquad$\exp_1$ \ml{;} $\cdots$ \ml{;} $\exp_\n$ \ADD{$\langle\ml{;}\rangle{}$} \END
                & \ \ \ml{(}$\exp_1$ \ml{;} $\cdots$ \ml{;} $\exp_\n$ \ADD{$\langle\ml{;}\rangle{}$}\ml{)}\
                                                                         \END\\
\cline{1-2}
\WHILE\ \exp$_1$\ \DO\ \exp$_2$
                & \LET\ \VAL\ \REC\ \vid\ \ml{=}\ \FN\ \ml{() =>}
                                                           & (\vid\ new)\\
                & \ \ \IF\ \exp$_1$\ \THEN\
                    \ml{(}\exp$_2$\ml{;}\vid\ml{())}\ \ELSE\ \ml{()} \\
                & \ \ \IN\ \vid\ml{()}\ \END\\
\cline{1-2}
\ml{[}$\exp_1$ \ml{,} $\cdots$ \ml{,} $\exp_\n$\ml{]}
                & \exp$_1$\ \ml{::}\ $\cdots$\ \ml{::}\ \exp$_n$\
                            \ml{::}\ \NIL                 & $(n\geq 0)$ \\
\cline{1-2}
\ADD{\CASE\ \exp\ \OF\ \ml{|}\ \match}
                & \ADD{\CASE\ \exp\ \OF\ \match} \\
\cline{1-2}
\ADD{\exp\ \HANDLE\ \ml{|}\ \match}
                & \ADD{\handlexp} \\
\cline{1-2}
\ADD{\FN\ \ml{|}\ \match}
                & \ADD{\fnexp} \\
\cline{1-2}
\ADD{\ml{(}$\exp_1$ \ml{;} $\cdots$ \ml{;} $\exp_\n$ \ml{;} \ml{)}}
		& \ADD{\ml{(}$\exp_1$ \ml{;} $\cdots$ \ml{;} $\exp_\n$ \ml{)}}
		& $(n\geq 1)$ \\
\cline{1-2}
\ADD{\ttlbrace\ \atexp\ \WHERE\ $\langle$\labexps$\,\rangle$ \ttrbrace}
		& \ADD{\LET} & \ADD{(\vid\ new)} \\
		& \ADD{\VAL\ \ttlbrace $\langle$\labpats\ \ml{,}$\,\rangle$ \ml{...} \ml{=} \vid\ \ttrbrace\ \ml{=}\ \atexp} \\
		& \ADD{\IN\ \ttlbrace $\langle$\labexps\ \ml{,}$\,\rangle$ \ml{...} \ml{=} \vid\ \ttrbrace\ \END} \\
\cline{1-2}
\multicolumn{2}{r}{\ADD{(see note in text concerning \labpats)}}\\
\multicolumn{3}{c}{}\\
\multicolumn{2}{l}{\ADD{{\bf Expression Rows} \labexps}}\\
\cline{1-2}
\ADD{\vid\ $\langle$\ml{:} \ty$\rangle$ $\langle$\ml{,} \labexps$\,\rangle$ }
	& \ADD{\vid\ \ml{=}\ \vid\ $\langle$\ml{:} \ty$\rangle$ $\langle$\ml{,} \labexps$\,\rangle$} \\
\cline{1-2}
\ADD{\ml{...} \ml{=}\ \exp\ml{,} \labexps}
	& \ADD{\ml{...} \ml{=} \LET\ \VAL\ \vid\ \ml{=}\ \exp\ \IN} & \ADD{(\vid\ new)}\\
	& \ADD{\boxml{\ \ \ \ \ \ \ttlbrace}\ \labexps\ml{,}\ \vid\ \ttrbrace\ \END} \\
\cline{1-2}
\multicolumn{2}{r}{\ADD{(see note in text concerning \labexps)}}\\
\multicolumn{3}{c}{}\\
\end{tabular}
\caption{Derived forms of Expressions\index{67.1}}
\label{der-exp}
\end{figure}
\begin{figure}

\begin{tabular}{|l|l|l}
\multicolumn{1}{c}{Derived Form} & \multicolumn{1}{c}{Equivalent Form} &
\multicolumn{1}{c}{}\\
\multicolumn{3}{c}{}\\
\multicolumn{2}{l}{{\bf Patterns} \pat}\\
%\multicolumn{2}{l}{PATTERNS \pat}\\
\cline{1-2}
\ml{()}         & \ml{\lttbrace\ \rttbrace} \\
\cline{1-2}
\ml{(}$\pat_1$ \ml{,} $\cdots$ \ml{,} $\pat_\n$\ml{)}
            & \ml{\lttbrace 1=}$\pat_1$\ml{,}\ $\cdots$ \ml{,}\
                             $\overline{n}$\ml{=}$\pat_\n$\ml{\rttbrace}
                                                           & $(\n\geq 2)$ \\
\cline{1-2}
\ml{[}$\pat_1$ \ml{,} $\cdots$ \ml{,} $\pat_\n$\ml{]}
                & \pat$_1$\ \ml{::}\ $\cdots$\ \ml{::}\ \pat$_n$\
                            \ml{::}\ \NIL                 & $(n\geq 0)$ \\
\cline{1-2}
\ADD{\pat\ \IF\ \exp} & \ADD{\pat\ \WITH\ \ml{true}\ \ml{=}\ \exp} \\
\cline{1-2}
\multicolumn{3}{c}{}\\
\multicolumn{2}{l}{{\bf Pattern Rows} \labpats}\\
%\multicolumn{2}{l}{PATTERN ROWS \labpats}\\
\cline{1-2}
\CUT{\vid$\langle$\ml{:}\ty$\rangle
    \ \langle\AS\ \pat\rangle
    \ \langle$\ml{,} \labpats$\rangle$}
    & \CUT{\vid\ml{ = }\vid$\langle$\ml{:}\ty$\rangle
	\ \langle\AS\ \pat\rangle
	\ \langle$\ml{,} \labpats$\rangle$} \\
\cline{1-2}
\ADD{\ml{...}}
	& \ADD{\ml{...} \ml{=} \ml{\char`\_}} \\
\cline{1-2}
\ADD{\ml{...} $\langle$\ml{=} \pat$\rangle$\ml{,}\ \labpats}
	& \ADD{\labpats\ml{,}\ \ml{...} $\langle$\ml{=} \pat$\rangle$} \\
\cline{1-2}
\multicolumn{2}{r}{\ADD{(see note in text concerning \labpats)}}\\
\multicolumn{3}{c}{}\\
\multicolumn{2}{l}{{\bf Type Expressions} \ty}\\
%\multicolumn{2}{l}{TYPE EXPRESSIONS \ty}\\
\cline{1-2}
$\ty_1$ \ml{*} $\cdots$ \ml{*} $\ty_\n$
            & \ml{\lttbrace 1:}$\ty_1$\ml{,}\ $\cdots$ \ml{,}\
                             $\overline{n}$\ml{:}$\ty_\n$\ml{\rttbrace}
                                                           & $(\n\geq 2)$ \\
\cline{1-2}
\ADD{\ml{...} \ml{:} \ty\ml{,}\ \labtys}
	& \ADD{\labtys\ml{,}\ \ml{...} \ml{:} \ty} \\
\cline{1-2}
\multicolumn{2}{r}{\ADD{(see note in text concerning \labtys)}}\\
\multicolumn{3}{c}{}\\
\end{tabular}
\caption{Derived forms of Patterns and Type Expressions\index{67.2}}
\label{der-pat}
\end{figure}%
\begin{figure}
\begin{tabular}{|l|l|}
\multicolumn{1}{c}{Derived Form} & \multicolumn{1}{c}{Equivalent Form}\\
\multicolumn{2}{c}{}\\
\multicolumn{2}{l}{{\bf Function-value Bindings} \fvalbind}\\
%\multicolumn{2}{l}{FUNCTION-VALUE BINDINGS \fvalbind}\\
\hline
               & $\langle\OP\rangle$\vid\ \ml{=} \FN\ \vid$_1$ \ml{=>} $\cdots$
                              \FN\ \vid$_n$ \ml{=>} \\
               & \CASE\
                 \ml{(}\vid$_1$\ml{,} $\cdots$ \ml{,} \vid$_n$\ml{)} \OF \\
\ \ $\langle\OP\rangle\vid\ \atpat_{11}\cdots\atpat_{1n}$
		& \ \ \ml{(}\atpat$_{11}$\ml{,}$\cdots$\ml{,}\atpat$_{1n}$\ml{)} \ADD{$\langle\IF\ \atexp_1\rangle$} \\
		$\qquad\qquad$ \ADD{$\langle\IF\ \atexp_1\rangle$}
		  $\langle$\ml{:}\ty$_{\ADD{1}}\rangle$
		  \ml{=} \exp$_1$
		&\ \ \ \ \ \ml{=>} \exp$_1\langle$\ml{:}\ty$_{\ADD{1}}\rangle$\\
\ml{|}$\langle\OP\rangle\vid\ \atpat_{21}\cdots\atpat_{2n}$
		& \ml{| (}\atpat$_{21}$\ml{,}$\cdots$\ml{,}\atpat$_{2n}$\ml{)} \ADD{$\langle\IF\ \atexp_2\rangle$} \\
		$\qquad\qquad$ \ADD{$\langle\IF\ \atexp_2\rangle$}
		  $\langle$\ml{:}\ty$_{\ADD{2}}\rangle$
		  \ml{=} \exp$_2$
		&\ \ \ \ \ \ml{=>} \exp$_2\langle$\ml{:}\ty$_{\ADD{2}}\rangle$\\
\ml{|}\qquad$\cdots\qquad\cdots$
               & \ml{|}\qquad$\cdots\qquad\cdots$\\
\ml{|}$\langle\OP\rangle\vid\ \atpat_{m1}\cdots\atpat_{mn}$
		& \ml{| (}\atpat$_{m1}$\ml{,}$\cdots$\ml{,}\atpat$_{mn}$\ml{)} \ADD{$\langle\IF\ \atexp_m\rangle$} \\
		$\qquad\qquad$ \ADD{$\langle\IF\ \atexp_m\rangle$}
		  $\langle$\ml{:}\ty$_{\ADD{m}}\rangle$
		  \ml{=} \exp$_m$
		&\ \ \ \ \ \ml{=>} \exp$_m\langle$\ml{:}\ty$_{\ADD{m}}\rangle$\\
$\langle\AND\ \fvalbind\rangle$
               & $\langle\AND\ \fvalbind\rangle$\\
\hline
\multicolumn{2}{r}{($m,n\geq1$; $\vid_1,\cdots,\vid_n$ distinct and new)}\\
\multicolumn{2}{c}{}\\
\multicolumn{2}{l}{\ADD{{\bf Datatype bindings} \datbind}}\\
\hline
\ADD{\tyvarseq\ \tycon\ \ml{=} \ml{|}\ \constrs\ }
		& \ADD{\tyvarseq\ \tycon\ \ml{=} \constrs\ } \\
\ADD{$\qquad\qquad\qquad\langle\AND\ \datbind\rangle$}
		& \ADD{$\qquad\qquad\qquad\langle\AND\ \datbind\rangle$} \\
\hline
\multicolumn{2}{c}{}\\
\multicolumn{2}{l}{{\bf Declarations} \dec}\\
%\multicolumn{2}{l}{DECLARATIONS \dec}\\
\hline
\FUN\ \tyvarseq\ \ADD{$\langle$\ml{|}$\rangle{}$} \fvalbind
               & \CUT{\VAL\ \tyvarseq\ \REC\ \fvalbind}  \\
               & \ADD{\VAL\ \REC\ \tyvarseq\ \fvalbind}  \\
\hline
\DATATYPE\ \datbind\ \WITHTYPE\ \typbind
               & \DATATYPE\ $\datbind'$\ \ml{;}\ \TYPE\ \typbind \\
\hline
\ADD{\ABSTYPE\ \datbind\ \WITH\ \dec\ \END}
               & \ADD{\LOCAL\ \DATATYPE\ \datbind\ \IN} \\
               & \ADD{\qquad\TYPE\ $\typbind'$\ \ml{;}\ \dec\ \END}\\
\hline
\ABSTYPE\ \datbind\ \WITHTYPE\ \typbind
               & \ABSTYPE\ $\datbind'$ \\
\qquad\qquad\WITH\ \dec\ \END
               & \qquad\WITH\ \TYPE\ \typbind\ \ml{;}\ \dec\ \END\\
\hline
\ADD{\DO\ \exp} & \ADD{\VAL\ \ml{()} \ml{=} \exp} \\
\hline
\multicolumn{2}{r}{(see note in text concerning $\datbind'$ \ADD{and $\typbind'$})}\\
\multicolumn{2}{c}{}\\
\end{tabular}
\caption{Derived forms of \CUT{Function-value} Bindings and Declarations\index{68.1}}
\label{der-dec}
\end{figure}

%               Derived forms of functors:

\begin{figure}\adhocreplacementl{\thenostrsharing}{0mm}{
\begin{tabular}{|l|l|}
\multicolumn{1}{c}{Derived Form} & \multicolumn{1}{c}{Equivalent Form} \\
\multicolumn{2}{c}{}\\
\multicolumn{2}{l}{{\bf Structure  Expressions} \strexp}\\
%\multicolumn{2}{l}{STRUCTURE EXPRESSIONS \strexp}\\
\cline{1-2}
\funappdec & \mbox{\funid\ \ml{(} \STRUCT\ \strdec\ \END\ \ml{)}}\\
\cline{1-2}
\multicolumn{2}{c}{}\\
%\multicolumn{2}{l}{FUNCTOR BINDINGS \funbind}\\
\multicolumn{2}{l}{{\bf Functor Bindings} \funbind}\\
\cline{1-2}        
\mbox{\funid\ \ml{(}\ \spec\ \ml{)}\ $\langle$\ml{:}\ \sigexp$\rangle$\ \ml{=}}&
\mbox{\funid\ \ml{(}\ \strid\ \ml{:} \SIG\ \spec\ \END\ \ml{)} 
              $\langle$\ml{:}\ $\sigexp'\rangle$\ \ml{=}}\\
\mbox{\ \ \strexp\ $\langle$\AND\ \funbind$\rangle$} &
  \mbox{\ \ \LET\ \OPEN\ \strid\ \IN\ \strexp\ \END\ $\langle$\AND\ \funbind$\rangle$} \\
\cline{1-2}
\multicolumn{2}{r}{($\strid$ new; see note in text concerning $\sigexp'$)}\\
\multicolumn{2}{c}{}\\
\multicolumn{2}{l}{{\bf Functor Signature Expressions} \funsigexp}\\
%\multicolumn{2}{l}{FUNCTOR SIGNATURES \funsigexp}\\
\cline{1-2}
\longfunsigexp & \mbox{\ml{(} \strid\ \ml{:}\ \SIG\ \spec\ \END\ \ml{)}
                \ml{:}\ \sigexp$'$} \\
\cline{1-2}
\multicolumn{2}{r}{($\strid$ new; see note in text concerning $\sigexp'$)}\\
\multicolumn{2}{c}{}\\
\multicolumn{2}{l}{{\bf Top-level Declarations} \topdec}\\
\cline{1-2}
\exp           & \VAL\ \ml{it =} \exp  \\
\cline{1-2}
\multicolumn{2}{c}{}\\
\end{tabular}}{
\begin{tabular}{|l|l|}
\multicolumn{1}{c}{Derived Form} & \multicolumn{1}{c}{Equivalent Form} \\
\multicolumn{2}{c}{}\\
\multicolumn{2}{l}{{\bf Structure  Bindings} \strbind}\\
\cline{1-2}
\derivedstrbinder & \equivalentstrbinder\\
\cline{1-2}
\derivedabststrbinder & \equivalentabststrbinder\\
\cline{1-2}
\multicolumn{2}{c}{}\\
\multicolumn{2}{l}{{\bf Structure  Expressions} \strexp}\\
\cline{1-2}
\funappdec & \mbox{\funid\ \ml{(} \STRUCT\ \strdec\ \END\ \ml{)}}\\
\cline{1-2}
\multicolumn{2}{c}{}\\
%\multicolumn{2}{l}{FUNCTOR BINDINGS \funbind}\\
\multicolumn{2}{l}{{\bf Functor Bindings} \funbind}\\
\cline{1-2}        
\mbox{\funid\ \ml{(}\strid\ml{:}\sigexp\ml{)}\ml{:} $\sigexp'$ \ml{=}}&
\mbox{\funid\ \ml{(}\strid\ \ml{:} \sigexp\ml{)} \ \ml{=}}\\
\mbox{\ \ \strexp\ $\langle$\AND\ \funbind$\rangle$} &
  \mbox{\ \ \strexp\ml{:}$\sigexp'$\  $\langle$\AND\ \funbind$\rangle$} \\
\cline{1-2}        
\mbox{\funid\ \ml{(}\strid\ml{:}\sigexp\ml{)}\ABSTRACT $\sigexp'$ \ml{=}}&
\mbox{\funid\ \ml{(}\strid\ \ml{:} \sigexp\ml{)} \ \ml{=}}\\
\mbox{\ \ \strexp\ $\langle$\AND\ \funbind$\rangle$} &
  \mbox{\ \ \strexp\ABSTRACT$\sigexp'$\  $\langle$\AND\ \funbind$\rangle$} \\
\cline{1-2}        
\mbox{\funid\ \ml{(}\ \spec\ \ml{)}\ $\langle$\ml{:}\ \sigexp$\rangle$\ \ml{=}}&
\mbox{\funid\ \ml{(}\ $\strid_\nu$\ \ml{:} \SIG\ \spec\ \END\ \ml{)} 
              \ \ml{=}}\\
\mbox{\ \ \strexp\ $\langle$\AND\ \funbind$\rangle$} &
  \mbox{\ \ \LET\ \OPEN\ $\strid_\nu$ \IN\ \strexp$\langle$\ml{:}\ $\sigexp\rangle$}\\
& \mbox{\ \ \END\ $\langle$\AND\ \funbind$\rangle$} \\
\cline{1-2}        
\mbox{\funid\ \ml{(}\ \spec\ \ml{)}\ $\langle$\ABSTRACT\ \sigexp$\rangle$\ \ml{=}}&
\mbox{\funid\ \ml{(}\ $\strid_\nu$\ \ml{:} \SIG\ \spec\ \END\ \ml{)} 
              \ \ml{=}}\\
\mbox{\ \ \strexp\ $\langle$\AND\ \funbind$\rangle$} &
  \mbox{\ \ \LET\ \OPEN\ $\strid_\nu$ \IN\ \strexp$\langle$\ABSTRACT $\sigexp\rangle$}\\
& \mbox{\ \ \END\ $\langle$\AND\ \funbind$\rangle$} \\
\cline{1-2}
\multicolumn{2}{r}{($\strid_\nu$ new)}\\
\multicolumn{2}{c}{}\\
\multicolumn{2}{l}{{\bf Programs} \program}\\
\cline{1-2}
$\exp\boxml{;}\langle\program\rangle$           & $\VAL\ \boxml{it =}\; \exp\boxml{;}\langle\program\rangle$\\
\cline{1-2}
\multicolumn{2}{c}{}\\
\end{tabular}}
\caption{Derived forms of Functors, Structure Bindings and Programs\index{68.2}}
\label{functor-der-forms-fig}
\end{figure}

\begin{figure}
\adhocinsertion{\thetypabbr}{-10mm}{
\begin{tabular}{|l|l|}
\multicolumn{1}{c}{Derived Form} & \multicolumn{1}{c}{Equivalent Form} \\
\multicolumn{2}{c}{}\\
\multicolumn{2}{l}{{\bf Specifications} \spec}\\ 
\cline{1-2}
\CUT{$\TYPE\;\tyvarseq\;\tycon\,\boxml{=}\,\ty$} & \CUT{\boxml{include}}\\
 &\CUT{\boxml{\ sig $\TYPE\;\tyvarseq\;\tycon$}}\\
 &\CUT{\boxml{\ end where type $\tyvarseq\;\tycon\,\boxml{=}\,\ty$}} \\
\cline{1-2}
\boxml{type $\tyvarseq_1\;\tycon_1$ = $\ty_1$} & \CUT{\boxml{type $\tyvarseq_1\;\tycon_1$ = $\ty_1$}}\\
\boxml{ and $\cdots$} & \CUT{\boxml{type $\cdots$}}\\
\boxml{ $\cdots$} & \CUT{\boxml{ $\cdots$}}\\
\boxml{ and $\tyvarseq_n\;\tycon_n$ = $\ty_n$} & \CUT{\boxml{type $\tyvarseq_n\;\tycon_n$ = $\ty_n$}}\\
	& \ADD{\INCLUDE\ \SIG} \\
	& \ADD{\boxml{\ \ type $\tyvarseq_1\;\tycon_1$}} \\
	& \ADD{\boxml{\ \ type $\cdots$}} \\
	& \ADD{\boxml{\ \ \ $\cdots$}} \\
	& \ADD{\boxml{\ \ type $\tyvarseq_n\;\tycon_n$}} \\
	& \ADD{\boxml{end where type $\tyvarseq_1\;\tycon_1$ = $\ty_1$}} \\
	& \ADD{\boxml{\ \ \ \ where type $\cdots$}} \\
	& \ADD{\boxml{\ \ \ \ \ $\cdots$}} \\
	& \ADD{\boxml{\ \ \ \ where type $\tyvarseq_n\;\tycon_n$ = $\ty_n$}} \\
\cline{1-2}
\ADD{\DATATYPE\ \datdesc\ \WITHTYPE\ \typbind}
	& \ADD{\DATATYPE\ $\datdesc'$\ \ml{;} \TYPE\ \typbind} \\
\cline{1-2}
$\inclspec$  & $\INCLUDE\,\sigid_1; \cdots\, ; \INCLUDE \,sigid_n$\\
\cline{1-2}
\boxml{$\spec$} & \boxml{$\spec$}\\
\boxml{\ sharing $\longstrid_1$ = $\cdots$} & \boxml{\ sharing type $\longtycon_{1}$ = } \\
\boxml{\ \ \qquad\qquad\qquad\qquad = $\longstrid_k$}
  & \boxml{\ \ \ \ \ \qquad\qquad\qquad\qquad $\longtycon_{1}'$} \\
  & \boxml{\ $\cdots$}\\
  & \boxml{\ sharing type $\longtycon_{m}$ = }\\
  & \boxml{\ \ \ \ \ \qquad\qquad\qquad\qquad  $\longtycon_{m}'$}\\
\cline{1-2}
\multicolumn{2}{r}{\vrule height14pt depth0pt width0pt(see notes in text concerning $\longtycon_{1},\ldots,\longtycon_{m}'$ \ADD{and $\datdesc'$})}\\
%\multicolumn{2}{r}{($n\geq 1$)}\\
\multicolumn{2}{c}{}\\
\multicolumn{2}{l}{\ADD{{\bf Datatype Descriptions} \datdesc}}\\ 
\cline{1-2}
\ADD{\tyvarseq\ \tycon\ \ml{=} \ml{|} \condesc\ $\langle\AND\ \datdesc\rangle$}
	& \ADD{\datdescription} \\
\cline{1-2}
\multicolumn{2}{c}{}\\
\multicolumn{2}{l}{\CUT{{\bf Signature Expressions} \sigexp}}\\ 
\cline{1-2}
\CUT{\boxml{$\sigexp$}} & \CUT{\boxml{$\sigexp$}}\\
\CUT{\boxml{where type $\tyvarseq_1\; \longtycon_1$ = $\ty_1$}} & \CUT{\boxml{\ where type $\tyvarseq_1\; \longtycon_1$ = $\ty_1$}}\\
\CUT{\boxml{\ \ and\ type $\cdots$}} & \CUT{\boxml{\ where type $\cdots$}}\\
\CUT{\boxml{\ \ $\cdots$}} & \CUT{\boxml{\ $\cdots$}}\\
\CUT{\boxml{\ \ and\ type $\tyvarseq_n\;\longtycon_n$ = $\ty_n$}} & \CUT{\boxml{\ where type $\tyvarseq_n\;\longtycon_n$ = $\ty_n$}}\\
\cline{1-2}
\end{tabular}}
\caption{Derived forms of Specifications \CUT{and Signature Expressions}}
\label{spec-der-forms-fig}
\end{figure}




\blankPage
\thispagestyle{empty}
%!TEX root = root.tex
%

\section{Appendix: Full Grammar}
\label{core-gram-app}
%In\index{69.1} this Appendix, the full Core 
%grammar is given for reference purposes.
The full grammar of programs is exactly as given at the start of 
Section~\ref{prog-sec}\ADD{, together with the derived form of
Figure~\ref{functor-der-forms-fig} in Appendix~\ref{derived-forms-app}}.

The\index{69.1} full grammar of Modules consists of the grammar of 
Figures \ref{mod-phr}--\ref{prog-syn} in Section~\ref{syn-mod-sec},
together with the derived forms of \replacement{\thenostrsharing}{Figure~\ref{functor-der-forms-fig}}{Figures~\ref{functor-der-forms-fig} and \ref{spec-der-forms-fig}}
in Appendix~\ref{derived-forms-app}.

The remainder of this Appendix is devoted to the full grammar of the
Core. 
Roughly, it consists of the grammar of Section~\ref{syn-core-sec} augmented by
the derived forms of Appendix~\ref{derived-forms-app}.  But there is a further
difference: two additional subclasses of the phrase class ~Exp~ are introduced,
namely ~AppExp~ (application expressions) and ~InfExp~ (infix expressions).
The inclusion relation among the four classes is as follows:
\[ {\rm AtExp}\ \subset\ {\rm AppExp}\
                \subset\ {\rm InfExp}\ \subset\ {\rm Exp} \]
The effect is that certain phrases, such as
``\ml{2 + while $\cdots$ do $\cdots$ }'', are now disallowed.
\ADD{
The same applies to patterns, where the extra classes ~AppPat~ and ~InfPat~
are introduced yielding the following inclusion relation:
\[
  {\rm AtPat}\ \subset\ {\rm AppPat}\ \subset\ {\rm InfPat}\ \subset\ {\rm Pat}
\]
}

The grammatical rules are displayed in Figures~\ref{exp-gram},
\ref{dec-gram}, \ref{pat-gram} and \ref{typ-gram}.
The grammatical conventions are exactly as in
Section~\ref{syn-core-sec}, namely:
\begin{itemize}
  \item The brackets ~$\langle\ \rangle$~ enclose optional phrases.
  \item For\index{69.3} any syntax class X (over which $x$ ranges)
we define the syntax class Xseq (over which {\it xseq} ranges) as follows:
    \begin{quote}
    \begin{tabular}{rcll}
       {\it xseq} & $::=$ & $x$ & (singleton sequence)\\
                  &       &     & (empty sequence)\\
                  &       & \ml{(}$x_1$\ml{,}$\cdots$\ml{,}$x_n$\ml{)}
                                & (sequence,~$n\geq 1$) \\
    \end{tabular}
    \end{quote}
(Note that the ``$\cdots$'' used here, a meta-symbol indicating syntactic
repetition, must not be
confused with ``$\wildrec$'' which is a reserved word of the language.)
  \item Alternative\index{69.4} forms for each phrase class are in order of decreasing
        precedence. This precedence resolves ambiguity in parsing in
the following way. Suppose that a phrase class --- we take $\exp$ as
an example --- has two alternative forms $F_1$ and $F_2$, such that $F_1$ ends
with an $\exp$ and $F_2$ starts with an $\exp$. A specific case is
\begin{tabbing}
\qquad\=$F_1$:\quad\=\IF\ $\exp_1$\ \THEN\ $\exp_2$\ \ELSE\ $\exp_3$\+\\
        $F_2$:     \>\handlexp
\end{tabbing}
It will be enough to see how ambiguity is resolved in this specific case.

Suppose that the lexical sequence
\[\cdots\ \cdots\ \IF\ \cdots\ \THEN\ \cdots\ \ELSE\ \exp\ \HANDLE\ \cdots\ \cdots\]
is to be parsed, where $\exp$ stands for a lexical sequence which 
is already determined as a subphrase (if necessary by applying the 
precedence rule).
Then the higher precedence of $F_2$ (in this case) dictates that $\exp$
associates to the right, i.e. that the correct parse takes the form
\[\cdots\ \cdots\ \IF\ \cdots\ \THEN\ \cdots\ \ELSE\ (\exp\ \HANDLE\ \cdots)\ \cdots\]
not the form
\[\cdots\ (\cdots\ \IF\ \cdots\ \THEN\ \cdots\ \ELSE\ \exp)\ \HANDLE\ \cdots\ \cdots\]
\CUT{Note particularly that the use of precedence does not decrease the class
of admissible phrases; it merely rejects alternative ways of parsing certain
phrases. In particular, the purpose is not to prevent a phrase,
which is an instance of a form with higher precedence, having a constituent
which is an instance of a form with lower precedence. Thus for example}
\[\CUT{\IF\ \cdots\ \THEN\ \WHILE\ \cdots\ \DO\ \cdots\ \ELSE\ \WHILE\ \cdots\ \DO\ \cdots}\]
\CUT{is quite admissible, and will be parsed as}
\[\CUT{\IF\ \cdots\ \THEN\ (\WHILE\ \cdots\ \DO\ \cdots)\ \ELSE\ (\WHILE\ \cdots\ \DO\ \cdots)}\]%
\ADD{%
Note that the use of precedence does not prevent a phrase, which is an instance of a form
with higher precedence, having a constituent which is an instance of a form
with lower precedence, as long as they can be resolved unambiguously.
Thus for example}
\begin{center}
  \ADD{\IF\ $\cdots$ \THEN\ \WHILE\ $\cdots$ \ELSE\ \WHILE\ $\cdots$ \DO\ $\cdots$}
\end{center}%
\ADD{is quite admissible and parses as}
\begin{center}
  \ADD{\IF\ $\cdots$ \THEN\ (\WHILE\ $\cdots$ \DO\ $\cdots$) \ELSE\ (\WHILE\ $\cdots$ \DO\ $\cdots$)}
\end{center}%
\ADD{Note, however, that precedence rules out phrases that cannot be disambiguated without
violating precedence, such as}
\begin{center}
  \ADD{$\cdots$ \ANDALSO\ \IF\ $\cdots$ \THEN\ $\cdots$ \ELSE\ $\cdots$ \ORELSE\ $\cdots$}
\end{center}%

\item L (resp. R)\index{69.5} means left (resp. right) association.

\item The syntax of types binds more tightly than that of expressions.

\item Each\index{69.7} iterated construct (e.g., $\match$, $\cdots$ )
extends as far
right as possible; thus, parentheses may be needed around an expression which
terminates with a match, e.g. ``$\FN\ \match$'', if this occurs within a
larger
match.

\item[\textcolor{\addcolor}{$\bullet$}]
\ADD{Likewise, a conditional ``\IF\ $\exp_1$ \THEN\ $\cdots$'' extends as far right as possible,
which means that optional \ELSE\ branches croup with innermost conditional.}
\end{itemize}

\ADD{%
We impose the following additional restrictions on the syntax:}
\begin{itemize}
  \item
    In the $\fvalbind$ form in Figure~\ref{dec-gram}, if $\vid$ has infix status then either
    ~\OP~ must be present, or $\vid$ must be infixed.  Thus, at the start of
    any clause, ``~\OP\ \vid\ \ml{(}\atpat\ml{,}\atpat$'$\ml{)} $\cdots$'' may be
    written
    ``\ml{(}\atpat\ \vid\ \atpat$'$\ml{)} $\cdots$''; the parentheses may also be
    dropped if \ADD{``\IF\ \exp,''} ``\ml{:}\ty,'' or ``\ml{=}'' follows immediately.
  \item[\textcolor{\addcolor}{$\bullet$}]
    \ADD{In a $\fmatch$ with $m$ rules, the expressions $\exp_1,\,\ldots,\,\exp_{m-1}$
    must not end in a $\match$.}
  \item[\textcolor{\addcolor}{$\bullet$}]
    \ADD{The pattern \pat\ in a \valbind\ may not nested match or guard, unless enclosed
    by parentheses.}
\end{itemize}%

\begin{figure}[h]
\vspace{4pt}
\makeatletter{}
\tabskip\@centering
\halign to\textwidth
{#\hfil\tabskip1em&\hfil$#$\hfil&#\hfil&#\hfil\tabskip\@centering\cr
  \atexp& ::=	& \scon 	& special constant\cr
        & 	& \opp\longvid	& value identifier\cr
	&	& \verb+{+ \ADD{$\langle$\atexp\ \WHERE$\rangle$} $\langle$\labexps$\rangle$ \verb+}+
				& record\cr
        &       & \ml{\#}\ \lab	& record selector\cr
        &       & \ml{()}       & 0-tuple\cr
        &       & \ml{(}$\exp_1$ \ml{,} $\cdots$ \ml{,} $\exp_\n$\ml{)}
                                & $n$-tuple, $n\geq 2$\cr
        &       & \ml{[}$\exp_1$ \ml{,} $\cdots$ \ml{,} $\exp_\n$\ml{]}
                                & list, $n\geq 0$\cr
        &       & \ml{(}$\exp_1$ \ml{;} $\cdots$ \ml{;} $\exp_\n$ \ADD{$\langle$\ml{;}$\rangle$} \ml{)}
                                & sequence, $n\geq \REPL{1\ }{2}$\cr
	&	& \LET\ \dec\ \IN\
                  $\exp_1$ \ml{;} $\cdots$ \ml{;} $\exp_\n$ \ADD{$\langle$\ml{;}$\rangle$} \END
	                        & local declaration, $n\geq 1$\cr
	&	& \CUT{\parexp}	& \cr
\noalign{\vspace{6pt}}
\labexps& ::=	& \longlabexps	& expression row\cr
	&	& \ADD{\vid\ $\langle$\ml{:} \ty$\rangle$ $\langle$ \ml{,} \labexps$\rangle$}
				& \ADD{label as variable}\cr
	&	& \ADD{\ml{...} \ml{=} \exp\ $\langle$ \ml{,} \labexps$\rangle$}
				& \ADD{ellipses}\cr
\noalign{\vspace{6pt}}
 \apexp & ::=	& \atexp	& \cr
        &   	& \apexp\ \atexp& application expression\cr
\noalign{\vspace{6pt}}
\inexp & ::=	& \apexp	& \cr
        &   	& $\inexp_1$\ \vid\ $\inexp_2$
                                & infix expression\cr
\noalign{\vspace{6pt}}
  \exp  & ::=	& \inexp 	& \cr
	&	& \typedexp	& typed (L)\cr
        &       & $\exp_1$\ \ANDALSO\ $\exp_2$
                                & conjunction\cr
        &       & $\exp_1$\ \ORELSE\ $\exp_2$
                                & disjunction\cr
	&	& \exp\ \HANDLE\ \ADD{$\langle$\ml{|}$\rangle$} \match	& handle exception\cr
	&	& \raisexp     	& raise exception\cr
        &       & \IF\ $\exp_1$\ \THEN\ $\exp_2$\ \ADD{$\langle$} \ELSE\ $\exp_3$ \ADD{$\rangle$}
                                & conditional\cr
        &       & \WHILE\ \exp$_1$\ \DO\ \exp$_2$
                                & iteration\cr
        &       & \CASE\ \exp\ \OF\ \ADD{$\langle$\ml{|}$\rangle$} \match
                                & case analysis\cr
	&	& \FN\ \ADD{$\langle$\ml{|}$\rangle$} \match	& function\cr
\noalign{\vspace{6pt}}
\match  & ::=	& \longmatch    & \cr
\noalign{\vspace{6pt}}
\mrule	& ::=	& \longmrule	& \cr
\noalign{\vspace{6pt}}
}
\makeatother
\vspace{3pt}
\caption{Grammar: Expressions and Matches\index{70}}
\label{exp-gram}
\end{figure}

\begin{figure}[h]
\vspace{4pt}
\makeatletter{}
\tabskip\@centering
\halign to\textwidth
{#\hfil\tabskip1em&\hfil$#$\hfil&#\hfil&#\hfil\tabskip\@centering\cr
  \dec  & ::=	& \explicitvaldec	& value declaration\cr
	&	& \FUN\ \tyvarseq\  \fvalbind
	                        & function declaration\cr
	&	& \typedec	& type declaration\cr
	&	& \datatypedeca & datatype declaration\cr
\adhocinsertion{\thedatatyperepl}{2cm}{        &       & \datatyperepldec & datatype replication\cr}
	&	& \abstypedeca  & abstype declaration\cr
        &       & \qquad\WITH\ \dec\ \END
                                & \cr
	&	& \exceptiondec & exception declaration\cr
	&	& \localdec	& local declaration\cr
        &       & \openstrdec   & open declaration, $n\geq 1$\cr
	&	& \emptydec	& empty declaration\cr
	&	& \seqdec	& sequential declaration \ADD{(L)} \cr
        &       & \newlonginfix    & infix (L) directive, $n\geq 1$\cr
        &       & \newlonginfixr   & infix (R) directive, $n\geq 1$\cr
        &       & \newlongnonfix   & nonfix directive, $n\geq 1$\cr
        &	& \ADD{\DO\ \exp} & \ADD{evaluation} \cr
%        &       & \exp          & expression (top-level only)\cr
\noalign{\vspace{6pt}}
\valbind& ::=   & \longvalbind   & \cr
	&	& \CUT{\recvalbind}	& \cr
\noalign{\vspace{6pt}}
\fvalbind& ::=  & \ \ \CUT{$\langle\OP\rangle\vid\ \atpat_{11}\cdots\atpat_{1n}
                  \langle$\ml{:}\ty$\rangle$\ml{=}\exp$_1$} & \CUT{$m,n\geq 1$}\cr
        &       & \CUT{\ml{|}$\langle\OP\rangle\vid\ \atpat_{21}\cdots\atpat_{2n}
                  \langle$\ml{:}\ty$\rangle$\ml{=}\exp$_2$} & \CUT{See also note
                                                                     below}\cr
        &       & \CUT{\ml{|}\qquad$\cdots\qquad\cdots$} &\cr
        &       & \CUT{\ml{|}$\langle\OP\rangle\vid\ \atpat_{m1}\cdots\atpat_{mn}
                  \langle$\ml{:}\ty$\rangle$\ml{=}\exp$_m$} &\cr
        &       & \qquad\qquad\qquad\CUT{$\langle\AND\ \fvalbind\rangle$} &\cr
	&	& \ADD{$\langle$\ml{|}$\rangle$ \fmatch\ $\langle$ \AND\ \fvalbind $\,\rangle$} &\cr
\ADD{\fmatch} & \ADD{::=}
		& \ADD{\fmrule\ $\langle$\ml{|} \fmatch $\,\rangle$} &\cr
\ADD{\fmrule} & \ADD{::=}
		& \ADD{\fpat\ $\langle$\IF\ \atexp$\,\rangle$ $\langle$\ml{:} \ty$\,\rangle$ \ml{=} \exp} &\cr
\ADD{\fpat} & \ADD{::=}
		& \ADD{$\langle\OP\rangle\vid\ \atpat_{1}\,\cdots\,\atpat_{n}$} & \ADD{$n\geq 1$} \cr
	&	& \ADD{\ml{(} $\atpat_1\ \vid\ \atpat_2$ \ml{)}\ $\atpat_{3}\,\cdots\,\atpat_{n}$} & \ADD{$n\geq 2$} \cr
	&	& \ADD{$\atpat_1\ \vid\ \atpat_2$} & \cr
\noalign{\vspace{6pt}}
\typbind& ::=	& \longtypbind	& \cr
\noalign{\vspace{6pt}}
\datbind& ::=	& \tyvarseq\ \tycon\ \ml{=}
		  \ADD{$\langle$\ml{|}$\rangle$} \constrs\ $\langle\AND\ \datbind\rangle$
		& \cr
\noalign{\vspace{6pt}}
\constrs& ::=	& \opp\longvidconstrs & \cr
\noalign{\vspace{6pt}}
\exnbind& ::=	& \generativeexnvidbind	& \cr
        &       & \eqexnvidbind   & \cr
\noalign{\vspace{6pt}}
}
\makeatother
\caption{Grammar: Declarations and Bindings\index{71}}
\label{dec-gram}
\end{figure}

\begin{figure}[h]
\vspace{4pt}
\makeatletter{}
\tabskip\@centering
\halign to\textwidth
{#\hfil\tabskip1em&\hfil$#$\hfil&#\hfil&#\hfil\tabskip\@centering\cr
  \atpat& ::=	& \wildpat	& wildcard\cr
  	&	& \scon  	& special constant\cr\adhocdeletion{\theidstatus}{2cm}{
  	&	& \opp\var  	& variable\cr
	&	& \opp\longcon  & constant\cr
        &       & \opp\longexn  & exception constant\cr}\adhocinsertion{\theidstatus}{4cm}{  	&	& \opp\longvid  	& value identifier\cr}
	&	& \verb+{ +\recpat\verb+ }+       & record\cr
        &       & \ml{()}       & 0-tuple\cr
        &       & \ml{(}$\pat_1$ \ml{,} $\cdots$ \ml{,} $\pat_\n$\ml{)}
                                & $n$-tuple, $n\geq 2$\cr
        &       & \ml{[}$\pat_1$ \ml{,} $\cdots$ \ml{,} $\pat_\n$\ml{]}
                                & list, $n\geq 0$\cr
	&	& \parpat       & \cr
\noalign{\vspace{6pt}}
\labpats& ::=	& \wildrec\ \ADD{$\langle$\ml{,} \labpats$\,\rangle$}
				& \REPL{ellipses }{wildcard}\cr
  	&	& \longlabpats 	& pattern row\cr
        &       & \vid\ $\langle$\ml{:}\ty $\rangle
                  \ \langle\AS\ \pat\rangle
                  \ \langle$\ml{,} \labpats$\rangle$
                                & label as variable\cr
\noalign{\vspace{6pt}}
\CUT{\pat} & \CUT{::=}	& \CUT{\atpat}	& \CUT{atomic}\cr
	&	& \CUT{\opp\vidpat}	& \CUT{constructed value}\cr
	&	& \CUT{\vidinfpat}	& \CUT{constructed value (infix)}\cr
	&	& \CUT{\typedpat}	& \CUT{typed}\cr
	&	& \CUT{\opp\layeredvidpat}	& \CUT{layered}\cr
\noalign{\vspace{6pt}}
\ADD{\appat} & \ADD{::=}
		& \ADD{\atpat}		& \ADD{atomic}\cr
	&	& \ADD{\opp\vidpat}	& \ADD{constructed value}\cr
\noalign{\vspace{6pt}}
\ADD{\inpat} & \ADD{::=}
		& \ADD{\appat}		& \ADD{application}\cr
	&	& \ADD{\vidinfpat}	& \ADD{constructed value (infix)}\cr
\noalign{\vspace{6pt}}
\ADD{\pat} & \ADD{::=}
		& \ADD{\inpat}		& \ADD{infix}\cr
	&	& \ADD{\typedpat}	& \ADD{typed}\cr
	&	& \ADD{\aspat}		& \ADD{conjunctive (R)}\cr
	&	& \ADD{\orpat}		& \ADD{disjunctive (L)}\cr
	&	& \ADD{\nestedpat}	& \ADD{nested match}\cr
	&	& \ADD{\pat\ \IF\ \exp}	& \ADD{guard}\cr
\noalign{\vspace{6pt}}
}
\makeatother
\vspace{3pt}
\caption{Grammar: Patterns\index{72.1}}
\label{pat-gram}
\end{figure}

\begin{figure}[h]
\vspace{4pt}
\makeatletter{}
\tabskip\@centering
\halign to\textwidth
{#\hfil\tabskip1em&\hfil$#$\hfil&#\hfil&#\hfil\tabskip\@centering\cr
  \ty   & ::=	& \tyvar        & type variable\cr
	&	& \verb+{ +\rectype\verb+ }+      & record type expression\cr
	&	& \constype 	& type construction\cr
        &       & $\ty_1$ \ml{*} $\cdots$ \ml{*} $\ty_\n$
                                & tuple type, $\n\geq 2$ \cr
	&	& \funtype      & function type expression \adhocinsertion{\thefixtypos}{-3cm}{(R)}\cr
	&	& \partype      & \cr
\noalign{\vspace{6pt}}
\labtys & ::=	& \longlabtys   & type-expression row\cr
	&	& \ADD{\ml{...} \ml{:} \ty\ $\langle$ \ml{,} \labtys$\rangle$}
				& \ADD{ellipses}\cr
\noalign{\vspace{6pt}}
}
\makeatother
\vspace{3pt}
\caption{Grammar: Type expressions\index{72.2}}
\label{typ-gram}
\end{figure}

\blankPage
\thispagestyle{empty}
%!TEX root = root.tex
%
%\appendix
\section{Appendix: The Initial Static Basis}
\label{init-stat-bas-app}
\insertion{\thelibrary}{
In this appendix (and the next) we define a minimal initial basis for
execution. Richer bases may be provided by libraries.}
\insertion{\thelibrary}{
We\index{73.1} shall indicate components of the initial basis by the subscript 0.
The initial static basis is
$\B_0 = \T_0,\F_0,\G_0,\E_0$,
where $F_0 = \emptymap$, $\G_0 = \emptymap$ and 
$$\T_0\ =\ \{\BOOL,\INT,\REAL,\STRING,\CHAR,\WORD,\LIST,\ADD{,\ARRAY}\REF,\EXCN\}$$
The members of $\T_0$ are type names, not type constructors; for convenience
we have used type-constructor identifiers
to stand also for the type names which are bound to them in the initial
static type environment $\TE_0$.  Of these type names,
\LIST\ADD{, \ARRAY,} and \REF\
have arity 1, the rest have arity 0;  
all except $\EXCN$ \insertion{\thelibrary}{and $\REAL$} admit equality.
Finally, $\E_0 = (\SE_0,\TE_0,\VE_0)$, where $\SE_0 = \emptymap$, 
while $\TE_0$ and $\VE_0$ are shown in Figures~\ref{stat-te} and \ref{stat-ve},
respectively.
}
 
%\vskip-5mm
\begin{figure}[h]
\begin{center}
\begin{tabular}{|rll|}
\hline
$\tycon$   & $\mapsto\ (\ \typefcn$, & $\{\vid_1\mapsto(\tych_1,\is_1),\ldots,\vid_n\mapsto(\tych_n,\is_n)\}\ )\quad (n\geq0)$\\
\hline
\UNIT      & $\mapsto\ (\ \Lambda().\{ \}$,
                                      & $\emptymap\ )$ \\
\BOOL      & $\mapsto\ (\ \BOOL$,    & $\{\TRUE\mapsto(\BOOL,\isc),
                                         \ \FALSE\mapsto(\BOOL,\isc)\}\ )$\\
\INT       & $\mapsto\ (\ \INT$,     & $\{\}\ )$\\
\WORD       & $\mapsto\ (\ \WORD$,     & $\{\}\ )$\\
\REAL      & $\mapsto\ (\ \REAL$,    & $\{\}\ )$\\
\STRING    & $\mapsto\ (\ \STRING$,  & $\{\}\ )$\\
%\UNISTRING    & $\mapsto\ (\ \UNISTRING$,  & $\{\}\ )$\\
\CHAR    & $\mapsto\ (\ \CHAR$,  & $\{\}\ )$\\
%\UNICHAR    & $\mapsto\ (\ \UNICHAR$,  & $\{\}\ )$\\
\LIST      & $\mapsto\ (\ \LIST$,    & $\{\NIL\mapsto(\forall\atyvar\ .\ \atyvar\ \LIST, \isc)$,\\
           &                          & \ml{::}$\mapsto(\forall\atyvar\ .
                                           \ \atyvar\ast\atyvar\ \LIST
                                           \to\atyvar\ \LIST, \isc)\}\ )$\\
\ADD{\ARRAY}     & \ADD{$\mapsto\ (\ \ARRAY$,}    & \ADD{$\emptymap\ )$} \\
\REF       & $\mapsto\ (\ \REF$,     & $\{\REF\mapsto(\forall\ \atyvar\ .\ 
                                           \atyvar\to\atyvar\ \REF,\isc)\}\ )$\\
\EXCN      & $\mapsto\ (\ \EXCN$,     & $\emptymap\ )$\\
\hline
\end{tabular}
\end{center}
\vskip-3mm
\caption{Static $\TE_0$\index{75.2}}
\vskip-3mm
\label{stat-te}
\end{figure}%

\begin{figure}[h]
\begin{tabular}{|rl|rl|}
\multicolumn{2}{c}{NONFIX}&     \multicolumn{2}{c}{INFIX}\\
\hline
$\vid$     & $\mapsto\ (\tych,\is)$    
                          & $\vid$ & $\mapsto\ (\tych,\is)$\\
\hline
\REF & $\mapsto\ (\forall\ \atyvar\ .\ \atyvar\to\atyvar\ \REF$, \isc)
                          &     \multicolumn{2}{l|}{Precedence 5, right associative :} \\
{\tt nil}  & $\mapsto\ (\forall\atyvar.\ \atyvar\ \LIST$, \isc)
                          & \boxml{::}   & $\mapsto\ (\forall\atyvar.
                                          \atyvar\;{\ast}\;\atyvar\;\LIST
                                          \to\atyvar\;\LIST$, \isc)\\
{\tt true}   & $\mapsto\ (\BOOL,\isc)$
                          & \multicolumn{2}{l|}{Precedence 4, left associative :}\\
{\tt false}   & $\mapsto\ (\BOOL,\isc)$
                          & \boxml{=}    & $\mapsto\ (\forall\aetyvar.\
                                          \aetyvar\ \ast\ \aetyvar\to\BOOL, \isv)$\\
{\tt Match}       & $\mapsto\ (\EXCN,\ise)$
                          & \multicolumn{2}{l|}{Precedence 3, left associative :} \\
{\tt Bind} & $\mapsto\ (\EXCN, \ise)$
                          & \boxml{:=}   & $\mapsto\ (\forall\atyvar.\
                                          \atyvar\ \REF\ \ast\ \atyvar\to\{\}, \isv)$\\
\hline
\multicolumn{4}{p{6in}}{
Note: In type schemes we have taken the liberty of writing
$\ty_1\ast\ty_2$ in place of
$\{\mbox{\tt 1}\mapsto\ty_1,\mbox{\tt 2}\mapsto\ty_2\}$.
}
\end{tabular}
%\vskip-4mm
\caption{Static $\VE_0$\index{74}}
\vskip-2mm
\label{stat-ve}
\end{figure}%

\blankPage
\thispagestyle{empty}
%!TEX root = root.tex
%

\section{Appendix: The Initial Dynamic Basis}
\label{init-dyn-bas-app}
We\index{76.1} shall indicate components of the initial basis by the subscript 0.
\insertion{\thelibrary}
The initial dynamic basis is $\B_0 = \F_0,\G_0,\E_0$, 
where $\F_0 = \emptymap$, $\G_0 = \emptymap$ and $\E_0 = (\SE_0, \TE_0,\VE_0)$,
where $\SE_0 = \emptymap$, $\TE_0$ is shown in Figure~\ref{dynTE0.fig} and
\medskip

$\VE_0 = \{\boxml{=}\mapsto(\boxml{=},\isv),\,\boxml{:=}\mapsto(\boxml{:=},\isv),\,\boxml{Match}\mapsto(\boxml{Match},\ise), \,\boxml{Bind}\mapsto(\boxml{Bind},\ise),$\\
       \vrule height0pt width21mm depth 0pt$\boxml{true}\mapsto(\boxml{true},\isc),\,\boxml{false}\mapsto(\boxml{false},\isc),\,$\\
       \vrule height0pt width21mm depth 0pt$\boxml{nil}\mapsto(\boxml{nil},\isc),\,
\hbox{\boxml{::}}\mapsto(\hbox{\boxml{::}},\isc),\,
\hbox{\boxml{ref}}\mapsto(\hbox{\boxml{ref}},\isc)\}$.

\begin{figure}[h]
\begin{center}
\begin{tabular}{|rll|}
\hline
$\tycon$   & $\mapsto$  & $\{\vid_1\mapsto(\V_1,\is_1),\ldots,\vid_n\mapsto(\V_n,\is_n)\}\quad (n\geq0)$\\
\hline
\UNIT      & $\mapsto $ &  $\emptymap$ \\
\BOOL      & $\mapsto $ & $\{\TRUE\mapsto(\TRUE,\isc),
                                         \ \FALSE\mapsto(\FALSE,\isc)\}$\\
\INT       & $\mapsto $ & $\{\}$\\
\WORD      & $\mapsto $ & $\{\}$\\
\REAL      & $\mapsto $ & $\{\}$\\
\STRING    & $\mapsto $ & $\{\}$\\
%\UNISTRING & $\mapsto $ & $\{\}$\\
\CHAR      & $\mapsto $ & $\{\}$\\
%\UNICHAR   & $\mapsto $ & $\{\}$\\
\LIST      & $\mapsto $ & $\{\NIL\mapsto(\NIL,\isc),\ml{::}\mapsto(\ml{::},\isc)\}$\\
\ADD{\ARRAY} & \ADD{$\mapsto$} & \ADD{$\{\}$} \\
\REF       & $\mapsto $ & $\{\REF\mapsto(\REF,\isc)\}$\\
\EXCN      & $\mapsto $ & $\emptymap$\\
\hline
\end{tabular}
\end{center}
\caption{Dynamic $\TE_0$}
\label{dynTE0.fig}
\end{figure}%

\ADD{Furthermore, the initial state $s_0$ is defined to be}
\[
  \ADD{s_0 = (\{\}, \{\boxml{Match},\boxml{Bind}\})}
\]

\blankPage
\thispagestyle{empty}
%!TEX root = root.tex
%

\section{Overloading}
\label{overload.sec}
Two forms of overloading are available:
\begin{itemize}
\item Certain special constants are overloaded.  For example,
{\tt 0w5} may have type $\WORD$ or some other type, depending on
the surrounding program text;
\item Certain operators are overloaded. For example,
{\tt +} may have type $\INT\ast\INT\to\INT$ or
$\REAL\ast\REAL\to\REAL$, depending on
the surrounding program text;
\end{itemize}
Programmers cannot define their own overloaded constants or operators.

Although a formal treatment of overloading is outside the scope
of this document, we do give a complete list of the overloaded operators
and of types with overloaded special constants.
This list is consistent with the Basis Library\cite{sml-basis-lib}.

Every overloaded constant and value identifier has among its types a 
{\em default type},
which is assigned to it, 
when the surrounding text does not resolve the overloading.
For this purpose, the surrounding text is \REPL{%
the smallest enclosing declaration
}{%
no larger than the smallest
enclosing structure-level declaration; an implementation may require
that a smaller context determines the type}.

\subsection{Overloaded special constants}
Libraries may extend the set $\T_0$ of
Appendix~\ref{init-stat-bas-app} with additional type names. Thereafter, certain
subsets of $T_0$ have a special significance;
they are called {\sl overloading classes}
and they are:\medskip

%\halign{\indent#\ \hfil&\ $#$\ &\ $#$\hfil\cr
%\Int &\supseteq&\{\INT\}\cr
%\Real &\supseteq&\{\REAL\}\cr
%\Word &\supseteq&\{\WORD\}\cr
%\String&\supseteq&\{\STRING\}\cr
%\Char&\supseteq&\{\CHAR\}\cr
%\WordInt&=&\Word\cup\INt\cr
%\RealInt&=&\Real\cup\INt\cr
%\Num&=&\Word\cup\Real\cup\INt\cr
%\NumTxt&=&\Word\cup\Real\cup\INt\cup\String\cup\Char\cr}
%\medskip

\begin{displaymath}
  \begin{array}{rcl}
    \Int &\supseteq& \{\INT\}\\
    \Real &\supseteq& \{\REAL\}\\
    \Word &\supseteq& \{\WORD\}\\
    \String &\supseteq& \{\STRING\}\\
    \Char &\supseteq& \{\CHAR\}\\
    \WordInt &=& \Word\cup\INt\\
    \RealInt &=& \Real\cup\INt\\
    \Num &=& \Word\cup\Real\cup\INt\\
    \NumTxt &=& \Word\cup\Real\cup\INt\cup\String\cup\Char\\
  \end{array}%
\end{displaymath}%

\noindent 
Among these, the five first ($\Int$, $\Real$, $\Word$, $\String$ and $\Char$) are said to be
{\sl basic}; the remaining are said to be {\sl composite}.
The reason that the basic classes are specified using
$\supseteq$ rather than $=$ is that libraries may extend 
each of the  basic overloading
classes with further type names.
\ADD{But the class $\Real$ may not contain type names taht admit equality.}
Special constants are overloaded
within each of the basic overloading classes.  However, the basic
overloading classes must be arranged so that every special constant can be
assigned types from at most one of the basic overloading classes.  For
example, to \boxml{0w5} may be assigned type $\WORD$, or
some other member of $\Word$, depending on the surrounding text.  If
the surrounding text does not determine the type of the constant, a
default type is used. The default types for the five sets are $\INT$,
$\REAL$, $\WORD$, $\STRING$ and $\CHAR$ respectively.

       Once overloading resolution has determined the type of a special constant,
       it is a compile-time error if the constant does not make sense or does not 
       denote a value within the machine representation chosen for the type.
       For example, an escape sequence of the form $\uconst$ in a string constant
       of 8-bit characters only makes sense if $xxxx$  denotes
       a number in the range $[0, 255]$. 


\subsection{Overloaded value identifiers}
Overloaded identifiers all have identifier status $\isv$. An
overloaded identifier may be re-bound with any status ($\isv$, $\isc$
and $\ise$) but then it is not overloaded within the scope of
the binding.

\begin{figure}
\begin{center}
\vskip-12pt
\begin{tabular}{|rl|rl|}
\multicolumn{2}{c}{NONFIX}&     \multicolumn{2}{c}{INFIX}\\
\hline
$\var$     & $\mapsto\ \hbox{set of monotypes}$    
                          & $\var$ & $\mapsto\ \hbox{set of monotypes}$\\
\hline
\boxml{abs} & $\mapsto \REALINT\to\REALINT$ 
                       & \multicolumn{2}{l|}{Precedence 7, left associative :} \\
\NEG    & $\mapsto \REALINT\to\REALINT $                      &
                            \boxml{div} & $\mapsto \WORDINT\ \ast\ \WORDINT
                                                                 \to\WORDINT$\\
 &  
                                             &
                            \boxml{mod} & $\mapsto \WORDINT\ \ast\ \WORDINT
                                                                 \to\WORDINT$\\
  &                       &
                            \boxml{*} &$\mapsto \NUM\ \ast\ \NUM
                                                                 \to\NUM$\\
  &                       &
                            \boxml{/} &$\mapsto \RREAL\ \ast\ \RREAL
                                                                 \to\RREAL$\\
  & &
                            \multicolumn{2}{l|}{Precedence 6, left associative :} \\
  &                       &
                            \boxml{+} &$\mapsto \NUM\ \ast\ \NUM
                                                                 \to\NUM$\\
  &                       &
                            \boxml{-} &$\mapsto \NUM\ \ast\ \NUM\to\NUM$\\
  & 
                          & \multicolumn{2}{l|}{Precedence 4, left associative :}\\
              &           &
                            \boxml{<} & $\mapsto\NUMTEXT *\NUMTEXT \to \REPL{\BOOL\ }{\NUMTEXT}$\\
              &           &
                            \boxml{>} & $\mapsto\NUMTEXT *\NUMTEXT \to \REPL{\BOOL\ }{\NUMTEXT}$\\
              &           &
                            \boxml{<=} & $\mapsto\NUMTEXT *\NUMTEXT \to \REPL{\BOOL\ }{\NUMTEXT}$\\
              &           &
                            \boxml{>=} & $\mapsto\NUMTEXT *\NUMTEXT \to \REPL{\BOOL\ }{\NUMTEXT}$\\
\hline
\end{tabular}
\end{center}
\vskip-15pt
\caption{Overloaded identifiers}
\label{overload.fig}
\end{figure}%
The overloaded identifiers are given in Figure~\ref{overload.fig}.
For example, the entry
$$\boxml{abs}\mapsto\REALINT\to\REALINT$$
states that $\boxml{abs}$ may assume one of the types
$\{\t\to\t\,\mid\,\t\in\RealInt\}$.
In general, the same type name must be chosen 
throughout the entire type of the overloaded operator;
thus $\boxml{abs}$ does not have type $\REAL\to\INT$.

The operator \boxml{/} is overloaded on all members of $\Real$,
with default type $\REAL\ast\REAL\to\REAL$.
The default type of any other identifier is that one of its types
which contains the type name {\tt int}.
For example, the program ~~\boxml{fun double(x) = x + x;}~~
declares a function of type $\INT\ \ast\ \INT\to\INT$, while
~~\boxml{fun double(x:real) = x + x;}~~ declares a function of type
$\REAL\ \ast\ \REAL\to\REAL$.

The dynamic semantics of the overloaded operators is defined in
\cite{sml-basis-lib}. 


\thispagestyle{empty}
%!TEX root = root.tex
%
\section{Appendix: The Development of ML}
\label{story-app}

This Appendix records the main stages in the development of ML, and the people
principally involved.  The main emphasis is upon the design of the language;
there is also a section devoted to implementation. On the other hand, no
attempt is made to record work on applications of the language.
\note{\thenewpreface}{This appendix has been revised and extended in 
several ways. A detailed list of changes is not available.} 
\subsection*{Origins}

ML and its semantic description have evolved over a period of about
twenty years.  It is a fusion of many ideas from many people;  in this
appendix we try to record and to acknowledge the important precursors
of its ideas, the important influences upon it, and the important
contributions to its design, implementation and semantic description.

ML, which stands for {\em meta language}, was conceived as a medium for
finding and performing proofs in a formal logical system.  This application
was the focus of the initial design effort, by Robin Milner in collaboration
first with Malcolm Newey and Lockwood Morris, then with Michael Gordon and
Christopher Wadsworth~\cite{GMMNW}. The intended application to proof affected
the design considerably.  Higher order functions in full generality seemed
necessary for programming proof tactics and strategies, and also a robust type
system (see below).  At the same time, imperative features were important for
practical reasons; no-one had experience of large useful programs written in a
pure functional style. In particular, an exception-raising mechanism was
highly desirable for the natural presentation of tactics.

The full definition of this first version of ML was included in a
book~\cite{GMW} which describes LCF, the proof system which ML was designed to
support.  The details of how the proof application exerted an influence on
design is reported by Milner~\cite{Mil2}.  Other early influences were the
applicative languages already in use in Artificial Intelligence, principally
LISP~\cite{McC}, ISWIM~\cite{Lan} and POP2~\cite{BP}.

\subsection*{Polymorphic types}

The polymorphic type discipline and the associated type-assignment algorithm
were\linebreak
 prompted by the need for security;  it is vital to know that when
a program produces an object which it claims to be a theorem, then it
is indeed a theorem.  A type discipline provides the security, but a
polymorphic discipline also permits considerable flexibility.

The key ideas of the type discipline were evolved in combinatory logic by
Haskell Curry and Roger Hindley, who arrived at different but equivalent
algorithms for computing principal type schemes.  Curry's~\cite{Cur} algorithm
was by equation-solving; Hindley~\cite{Hin} used the unification algorithm of
Alan Robinson~\cite{Rob} and also presented the precursor of our type
inference system.  James Morris~\cite{Mor} independently gave an
equation-solving algorithm very similar to Curry's.  The idea of an algorithm
for finding principal type schemes is very natural and may well have been
known earlier. Roger Hindley has pointed out that Carew
Meredith's inference rule for propositional logic called Condensed Detachment,
defined in the early 1950s, clearly suggests that he knew such an algorithm
\cite{Mer}.

Milner~\cite{Mil1}, during the design of ML, rediscovered principal types and
their calculation by unification, for a language (slightly richer than
combinatory logic) containing local declarations.  He and Damas~\cite{DM}
presented the ML type inference systems following Hindley's style.  Damas
\cite{Dam}, using ideas from Michael Gordon, also devised the first
mathematical treatment of polymorphism in the presence of references and
assignment.  Tofte~\cite{Tof-a} produced a different scheme employing
so-called {\em imperative types}, which was adopted in the original version of
the language.  This approach has been superseded in the present language by a
simpler scheme, suggested by Tofte~\cite{Tof-a}, Andrew Wright~\cite{wright95}, and
Xavier Leroy~\cite{Ler1}, according to which polymorphic bindings are
restricted to non-expansive expressions.

\subsection*{Refinement of the Core Language}

Two movements led to the re-design of ML.  One was the work of Rod
Burstall and his group on specifications, crystallised in the specification
language CLEAR~\cite{BG} and in the functional programming language HOPE
\cite{BMS};  the latter was for expressing executable specifications.  The
outcome of this work which is relevant here was twofold.  First, there were
elegant programming features in HOPE, particularly pattern matching and
clausal function definitions; second, there were ideas on modular construction
of specifications, using signatures in the interfaces.  A smaller but
significant movement was by Luca Cardelli, who extended the data-type
repertoire in ML by adding named records and variant types.

In 1983, Milner (prompted by Bernard Sufrin) wrote the first draft of a
standard form of ML attempting to unite these ideas; over the next three years
it evolved into the Standard ML core language.  Notable here was the harmony
found among polymorphism, HOPE patterns and Cardelli records, and the nice
generalisations of ML exceptions due to ideas from Alan Mycroft, Brian Monahan
and Don Sannella.  A simple stream-based I/O mechanism was developed from
ideas of Cardelli by Milner and Harper.  The Standard ML core language is
described in detail in a composite report~\cite{HMM} which also contains a
description of the I/O mechanism and MacQueen's proposal for program modules
(see later for discussion of this). Since then only few changes to the core
language have occurred.  Milner proposed equality types, and these were added,
together with a few minor adjustments~\cite{Mil3}.  The 
last development before the 1990 Definition was in the exception mechanism, by MacQueen using an idea
from Burstall~\cite{AMMT}; it harmonized the ideas of exception and data type
construction.

\subsection*{Modules}

Besides contributory ideas to the core language, HOPE~\cite{BMS} contained a
simple notion of program module.  The most important and original feature of
ML modules, however, stems from the work on parameterised specifications in
CLEAR~\cite{BG}.  MacQueen, who was a member of Burstall's group at the time,
designed~\cite{Mac} a new parametric module feature for HOPE inspired by the
CLEAR work.  He later extended the parameterisation ideas by a novel method of
specifying sharing of components among the structure parameters of a functor,
and produced a draft design which accommodated features already present in ML
-- in particular the polymorphic type system.  This design was discussed in
detail at Edinburgh, leading to MacQueen's first report on modules~\cite{HMM}.

Thereafter, the design came under close scrutiny through a draft operational
static semantics and prototype implementation of it by Harper, through Kevin
Mitchell's implementation of the evaluation, through a denotational semantics
written by Don Sannella, and then through further work on operational
semantics by Harper, Milner, and Tofte.  (More is said about this in the later
section on Semantics.)  In all of this work the central ideas withstood
scrutiny, while it also became clear that there were gaps in the design and
ambiguities in interpretation.  (An example of a gap was the inability to
specify sharing between a functor argument structure and its result structure;
an example of an ambiguity was the question of whether sharing exists in a
structure over and above what is specified in the signature expression which
accompanies its declaration.)

Much discussion ensued; it was possible for a wider group to comment on
modules through using Harper's prototype implementation, while Harper, Milner
and Tofte gained understanding during development of this semantics.  In
parallel, Sannella and Tarlecki explored the implications of modules for the
methodology of program development~\cite{ST}.  Tofte, in his thesis
\cite{tofte88}, proved several technical properties of modules in a skeletal
language, which generated considerable confidence in this design. A key point
in this development was the proof of the existence of principal signatures,
and, in the careful distinction between the notion of {\it enrichment} of
structures, which allows more polymorphism and more components, and {\it
realisation} which allows more sharing.

At a meeting in Edinburgh in 1987 a choice of two designs was presented,
hinging upon whether or not a functor application should coerce its actual
argument to its argument signature.  The meeting chose coercion, and
thereafter the production of Section~\ref{statmod-sec} of this report -- the
static semantics of modules -- was a matter of detailed care.  That section is
undoubtedly the most original and demanding part of this semantics, just as
the ideas of MacQueen upon which it is based are the most far-reaching
extension to the original design of ML.

Considerable experience was gained in implementing, programming with, and
teaching the language during the nearly ten years since the definition was
first published.  Based on this experience a number of design decisions were
revisited at a meeting of the authors in Cambridge at the end of 1995.  At
this meeting it was decided to make several modest, but significant, changes
to the language in order to simplify the semantics and to correct some
shortcomings that had come to light.  The most important of these changes was
the replacement of the imperative type discipline by the so-called value
restriction (discussed above), the elimination of structure sharing as a
separate concept from type sharing, and the introduction of the closely
connected mechanisms of opaque signature matching and type abbreviations in
signatures.  An important impetus for these changes to the modules language
was the work of Leroy~\cite{leroy94}, and Harper and
Lillibridge~\cite{HL} on the type-theoretic interpretation of
modules (described below).

\subsection*{Implementation}

The first implementation of ML was by Malcolm Newey, Lockwood Morris and Robin
Milner in 1974, for the DEC10.  Later Mike Gordon and Chris Wadsworth joined;
their work was mainly in specialising ML towards machine-assisted reasoning.
Around 1980 Luca Cardelli implemented a version on VAX; his work was later
extended by Alan Mycroft, Kevin Mitchell and John Scott.  This version
contained one or two new data-type features, and was based upon the {\em
Functional Abstract Machine (FAM)}, a virtual machine which has been a
considerable stimulus to later implementation.  By providing a reasonably
efficient implementation, this work enabled the language to be taught to
students; this, in turn, prompted the idea that it could become a useful
general purpose language.

In Gothenburg, an implementation was developed by Lennart Augustsson and
Thomas Johnsson in 1982, using lazy evaluation rather than call-by-value; the
result was called {\em Lazy ML} and is described in~\cite{Aug}.  This work is
part of continuing research in many places on implementation of lazy
evaluation in pure functional languages.  But for ML, which includes
exceptions and assignment, the emphasis has been mainly upon strict evaluation
(call-by-value).

In Cambridge, in the early 1980s, Larry Paulson made considerable improvements
to the Edinburgh ML compiler, as part of his wider programme of improving {\em
Edinburgh LCF} to become {\em Cambridge LCF}~\cite{Pau}.  This system has
supported larger proofs than the Edinburgh system, and with greater
convenience; in particular, the compiled ML code ran four to five times
faster.

Around the same time G\'{e}rard Huet at INRIA (Versailles) adapted ML to
Maclisp on Multics, again for use in machine-assisted proof.  There was close
collaboration between INRIA and Cambridge in this period.  ML has undergone a
separate development in the group at INRIA on the {\em CAML}
language~\cite{CCM}.  Work on {\em CAML} included the development of several
extensions to the core language, notably updatable fields in record types,
values with dynamic types, support for lazy evaluation, and handling of
embedded languages with user-defined syntax.  It did not, however, include
modules.

The first implementation of the Standard ML core language was by Mitchell,
Mycroft and Scott at Edinburgh, around 1984.  The prototype implementation of
modules, before that part of the language settled down, was done in 1985-6;
Mitchell dealt with evaluation, while Harper tackled the elaboration (or
`signature checking') which raised problems of a kind not previously
encountered.  Harper's implementation employed a form of unification that was
later adopted in the static semantics of modules.

At around the same time the {\em Poly/ML} implementation began with a
suggestion from Mike Gordon that an interesting application of Matthews' Poly
language would be to implement Standard ML.  Important experience was gained
through Matthews' early implementation of the core language, followed by
several versions of the modules language as they were devised.  {\em Poly/ML}
features arbitrary precision arithmetic, a process package, and a windowing
system.  Considerable experience has been gained with the compiler, notably by
Larry Paulson at Cambridge and by Abstract Hardware Limited (AHL).

  The {\em Standard ML of New Jersey (SML/NJ)} system has been in active development
  since 1986~\cite{am87,am91}.
  Initially started by David MacQueen at Bell Laboratories and Andrew Appel at Princeton University, 
  the project has also benefited from significant contributions
  by Matthias Blume, Emden Gansner, Lal George, John Reppy and Zhong Shao.
  {\em SML/NJ} is a robust and complete environment for Standard ML that supports
  the implementation of large software systems and generates efficient code for a
  number of different hardware and software platforms.  {\em SML/NJ} also serves as
  a laboratory for compiler research: in implementations of module systems for ML;
  code optimization based on continuation-passing style; efficient pattern matching;
  and very fast heap allocation and garbage collection.  Dozens of researchers
  have contributed to the development of the compiler, in such areas as
  efficient closure representations, first-class continuations, type-directed
  compilation, concurrent programming, portable code generators, separate
  compilation, and register allocation.  
  {\em SML/NJ} has also been widely used to explore extending SML with
  concurrency features.

%In 1986 Andrew Appel and David MacQueen began work on the {\em Standard ML of
%New Jersey (SML/NJ)} compiler~\cite{am87}.  {\em SML/NJ} is a robust and
%complete environment for Standard ML that supports the implementation of large
%software systems and generates efficient code for a number of different
%hardware and software platforms.  {\em SML/NJ} also serves as a laboratory for
%compiler research: in implementations of module systems for ML; code
%optimization based on continuation-passing style; efficient pattern matching;
%and very fast heap allocation and garbage collection.  Dozens of researchers
%have contributed to the development of the compiler, in such areas as
%efficient closure representations, first-class continuations, type-directed
%compilation, concurrent programming, portable code generators, separate
%compilation, and register allocation.  


In 1989, Mads Tofte, Nick Rothwell and David N. Turner started work on the
{\em ML Kit Compiler} in Edinburgh. The {\em ML Kit} is a direct translation
of the 1990 Definition into a collection of Standard ML modules, emphasis being
on clarity rather than efficiency. During 1992 and 1993, Version~1 of the {\em
ML Kit} was completed, mostly through the work of Nick Rothwell at Edinburgh
and Lars Birkedal at DIKU\cite{BRTT}. In 1994, region inference was added to
the {\em ML Kit}, by Mads Tofte. Lars Birkedal wrote a region-based C-code
generator and a runtime system in C.  In 1995, Martin Elsman and Niels
Hallenberg extended this work to generate native code for the HP PA-RISC
architecture.

Harlequin Ltd. began the implementation of a commercial compiler in 1990.  The
{\em MLWorks} system is a fully-featured graphical programming environment,
including an interactive debugger, inspector, browser, extensive profiling
facilities, separate compilation and delivery, a foreign-language interface,
and libraries for threads and windowing systems.  
%Harlequin was a major
%partner in the effort to standardize the revised Standard ML basis library.

{\em Caml Light}, a lightweight reimplementation of {\em CAML} released in
1991, added a simple module system in the style of Modula-2, targeted towards
separate compilation of modules: structures and signatures are identified with
files, functors and multiple views of a structure are not supported.  These
were added in the {\em Caml Special Light} implementation in 1995, while
preserving the support for separate compilation. {\em Caml Special Light} and
the present version of Standard ML share several important simplifications,
such as the value restriction on polymorphism, type definitions in signatures,
and the lack of support for structure sharing.  The static semantics for {\em
Caml Special Light} is based on the type-theoretic properties of dependent
function types (functor signatures) and manifest types (type definitions in
signatures)~\cite{leroy94}.

{\em Moscow ML} is an implementation of core Standard ML, created in 1994 by
Sergei Romanenko in Moscow and Peter Sestoft in Copenhagen.  The {\em Caml
Light} system was used to implement the dynamic semantics, and the ML Kit
guided the implementation of the static semantics.  The result is a compact
and robust implementation, suitable for teaching.

The {\em TIL (Typed Intermediate Languages)} compiler developed at Carnegie
Mellon
University by Greg Morrisett, David Tarditi, Perry Cheng, Chris Stone,
Robert Harper,
and Peter Lee demonstrates the use of
types in compilation.  All but the last few stages of {\em TIL} are expressed
as type-directed and type-preserving transforms.  Types are used at run time
to support unboxed, untagged data representations and natural calling
conventions in the presence of variable types and garbage collection.  {\em
TIL} employs a wide variety of conventional functional language optimizations
found in other SML compilers, as well as a set of loop-oriented optimizations.
A description of the compiler and an analysis of its performance appears
in~\cite{Tar}.

Other currently active implementations are by Michael Hedlund at the
Rutherford-Appleton Laboratory, by Robert Duncan, Simon Nichols and Aaron
Sloman at the University of Sussex ({\em POPLOG}) and by Malcolm Newey and his
group at the Australian National University.
 
\subsection*{Semantics}

The description of the first version of ML~\cite{GMW} was informal, and in an
operational style; around the same time a denotational semantics was written,
but never published, by Mike Gordon and Robin Milner.  Meanwhile structured
operational semantics, presented as an inference system, was gaining credence
as a tractable medium.  This originates with the reduction rules of
$\lambda$-calculus, but was developed more widely through the work of Plotkin
\cite{Plo}, and also by Milner.  This was at first only used for dynamic
semantics, but later the benefit of using inference systems for both static
and dynamic semantics became apparent.  This advantage was realised when
Gilles Kahn and his group at INRIA were able to execute early versions of both
forms of semantics for the ML core language using their Typol system
\cite{Des}.  The static and dynamic semantics of the core language reached a
final form mostly through work by Tofte and Milner.

The modules of ML presented little difficulty as far as dynamic semantics is
concerned, but the static semantics of modules was a concerted effort by
several people.  MacQueen's original informal description~\cite{HMM} was the
starting point; Sannella wrote a denotational semantics for several versions,
which showed that several issues had not been settled by the informal
description.  Robert Harper, while writing the first implementation of
modules, made the first draft of the static semantics.  Harper's version made
clear the importance of structure names; work by Milner and Tofte introduced
further ideas including realisation; thereafter a concerted effort by all
three led to several suggestions for modification of the language, and a small
range of alternative interpretations; these were assessed in discussion with
MacQueen, and more widely with the principal users of the language, and an
agreed form was reached.

Concurrently with the formulation of the Definition of Standard ML, Harper and
Mitchell took up the challenge adumbrated by MacQueen~\cite{Mac2} to find
a type-theoretic interpretation of Standard ML~\cite{HM}.  This work led to
the formulation of the XML language, an explicitly-typed $\lambda$-calculus
that captured many aspects of Standard ML.  Although incomplete, their
approach formed the basis for a number of subsequent studies, including the
work of Harper and Lillibridge~\cite{HL} and Leroy~\cite{leroy94} on the
type-theoretic interpretation of modules.  This work influenced the decision
to revise the language, and culminated in a type-theoretic interpretation of
the present language by Harper and Stone~\cite{SH}.  The TIL/ML compiler
(described above) is based directly on this interpretation.

There is no doubt that the interaction between design and semantic description
of modules has been one of the most striking phases in the entire language
development, leading (in the opinion of those involved) to a high degree of
confidence both in the language and in the semantics.


\subsection*{Program Libraries}
%During 1989-1991, Dave Berry produced the first program library for
%Standard ML\cite{mllib91,berry93}. Subsequently, a partnership between
%the originators of {\em SML/NJ}, {\em MLWorks} and 
%{\em Moscow ML} was formed, with the goal of creating an industrial strength
%initial basis for Standard ML. The resulting SML Basis 
%Library\cite{sml-basis-lib} is a much improved and extended 
%replacement of the initial basis defined in the 1990 Definition of 
%Standard ML.
 During 1989-1991, Dave Berry produced the first program library for
 Standard ML\cite{mllib91,berry93}.
 The {\em SML/NJ} system is distributed with a rich library organised by
 Emden Gansner and John Reppy; this library was the starting point for
 the SML Basis Library~.
 The {\em SML Basis Library\/}\cite{sml-basis-lib} has been developed 
 over the past three years in a
 partnership between the SML/NJ effort, {\em MLWorks}, and {\em Moscow ML}.
 The resulting library is a much improved and extended 
 replacement of the initial basis defined in the 1990 Definition of 
 Standard ML.

\subsection*{\protect\color{\addcolor} Successor ML}
{\color{\addcolor}
In 2005, an effort began to ``evolve'' Standard ML; Bob Harper suggested that the resulting language
be called Successor ML.
}

\blankPage
\thispagestyle{empty}
%!TEX root = root.tex
%

\section{Appendix: What is New?}
\label{whatisnew-app}

This appendix describes the differences between this document and
\emph{The Definition of Standard ML (Revised)}.

\subsection{Changes from SML '97}

\subsubsection{Fixes and simplifications}
All of the proposed fixes and simplifications from Appendix~B in the
HaMLet S manual~\cite{hamlet-s} have been integrated into the document,
and are rendered as \FIX{blue text.}
These are as follows (entries are annotated with their corresponding
section in Appendix~B):
\begin{itemize}
\setlength{\itemsep}{0em}
\item Syntax fixes (B.1)
\item Semantic fixes (B.2)
\item Monomorphic non-exhaustive bindings (B.3)
\item Simplified recursive value bindings (B.4)
\item Abstype as derived form (B.5)
\item Fixed manifest type specifications (B.6)
\item Abolish sequenced type realizations (B.7)
\end{itemize}%

In addition to the fixes described by Rossberg, I have also added the $\ARRAY$ type constructor
to the Initial Basis so that its equality property can be properly defined.

\subsubsection{Extensions}
The following is a list of proposed extensions from Appendix~B in the HaMLet S manual
that have been integrated into the document and are rendered as \ADD{magenta text}.
They are marked with the corresponding section of Appendix~B.
\begin{itemize}
\setlength{\itemsep}{0em}
\item Line comments (B.8)
\item Extended literal syntax (B.9)
%\item Vector syntax?
\item Record punning (B.10)
\item Record extension (B.11)
\item Record update (B.12)
\item Conjunctive patterns (B.13)
\item Disjunctive patterns (B.14)
\item Nested matches (B.15)
\item Pattern guards (B.16)
%\item Transformation patterns
%\item Finally
%\item Exceptional syntax
\item Optional bars and semicolons (B.18)
\item Optional else branch (B.19)
%\item Optional op keywords?
%\item Views
\item Do declarations (B.21)
%\item Declarations in sequential expressions?
%\item Monad syntax?
%\item Extensional datatypes?
\item Withtype in signatures (B.22)
%\item Manifest structures ("structure strid : sigexp = longstrid", "sigexp where longstrid = strexp")
%\item Local specifications? ("let dec in sigexp end", "local dec in spec end")
%\item Fixity specifications?
%\item Higher-order functors
%\item Nested signatures
%\item Local modules
%\item First-class modules
\end{itemize}

\subsection{Changed from HaMLet S}
There are a number of differences and omissions in what is described in this document
and the SuccessorML features documented by Rossberg in the HaMLet S manual~\cite{hamlet-s}.
We list these here.
\begin{itemize}
  \item
    The syntax of real literals is specified in a slightly more consistent way.  The consequence of this
    change is that underscores are not permitted immediately following the decimal point.
  \item
    The alternative prefixes ``\ml{0xw}'' and ``\ml{0bw}'' for word literals are not part of
    this specification.
  \item
    Optional bars in matches and semicolons in expression sequences are defined
    in Appendix~\ref{derived-forms-app} as derived forms, instead as part of the core syntax.
\end{itemize}%

\blankPage
\addcontentsline{toc}{section}{\protect\numberline{}{\vrule width0pt height2cm depth0pt References}}
\thispagestyle{empty}
\bibliographystyle{plain}
\bibliography{tofte,bob,succ-ml}
\clearpage
\blankPage
\thispagestyle{empty}

% Disable index printing
%\printindex
\end{document}

 HOW TO REVISE THE INDEX 

The index is made using partly LaTex and partly an ML progam;
the latter is found on the file ``index.ml''. 

When LaTex is run on input ``root.tex'' it produces a file ``root.idx''.
Each line in this file is of the form

               \indexentry{idxkey}{pageref}

where idxkey is a key inserted precisely one place in the document 
(in the form of a LaTex command \index{idxkey})
and pageref is the page number of the page LaTex was printing
by the time it encountered the \index{idxkey} command.

A typical idxkey is 45.1 , which will occur in the TeX file 
somewhere near what produces page 45 in the final document. 
In fact, when the keys were first inserted in the TeX file, 
45.1 would be the first key on page 45, but as the document changes, 
one cannot get the pageref from the idxkey simple by taking the 
prefix of the idxkey up to the full stop.

There are many entries in the index that refer to the same
idxkey. Thus the number of idxkeys has been kept relatively
small, typically 2 or 3 pr page. The basic idea, then, is that
there is an ML program (on file ``index.ml'') which
associates entries in the final index with idxkeys by 
a sequence of expressions

              .......
              item ``handle'' [p``45.4'',``57.3''to``59.1'',p``78.2''];
              item ``{\\it happiness}'' [p''38.1''];
              ......
 
The program will first build a conversion table from idxkeys
(such as ``45.4'') to page references by reading thrugh
``root.idx''. Then it will evaluate all the item and subitem
expressions. The item function produces a line in the
latex file (``index.tex'') using the conversion table.
Thus the above lines may produce

               ...........
               \item ``handle'' 46, 57--58, 78
               \item {\it happiness} 38
               ...........

If insertion of more text in the source files result in 
new page splits, then one should manually check that
the item expressions in ``index.ml'' refer to the
right idxkeys. It may be necessary to change some of the
lists in the item expressions and it may be desirable to
insert new \index commands in the source text. However,
if we for simplicity assume that we simply insert new
material corresponding to a new chapter (not affecting
the page splits in other chapters) then one would proceed
as follows:
First add new item expressions in the index.ml file corre-
sponding to new entries in the index (one will have new
\index commands in the new input, of course). Then one
runs latex on ``root.tex'' to produce the ``root.idx'' file. 
Then one runs index.ml (enter ML and type use ``index.ml'').
Then one runs latex on root again, this time giving the 
correct index.
