%!TEX root = root.tex
%
\section{Dynamic Semantics for Modules}
\label{dynmod-sec}
\subsection{Reduced Syntax}
Since\index{58.1} signature expressions
are mostly dealt with in the static semantics,
the dynamic semantics need only take limited account of them.  
\replacement{\theconstructors}{Unlike types,
it cannot ignore them completely; }{However,
they cannot be ignored completely; }the reason is that an explicit signature
ascription plays the r\^ole of restricting the ``view'' of a structure -- that is,
restricting the domains of its component environments\insertion{\theidstatus}{ and 
imposing identifier status on value identifiers}.  
\replacement{\thenostrsharing}{However, the types
and the sharing properties of structures and signatures are irrelevant to
dynamic evaluation; the syntax is therefore
reduced by the following transformations (in addition to those for the Core),
for the purpose of the dynamic semantics of Modules:}{The syntax is therefore
reduced by the following transformations (in addition to those for the Core),
for the purpose of the dynamic semantics of Modules:}
\begin{itemize}
\item Qualifications ``$\OF\ \ty\,$'' are omitted from \insertion{\theconstructors}{constructor and} exception descriptions.
\deletion{\thedatatyperepl}{\item Any specification of the form ``$\typespec$'', ``$\eqtypespec$'',
``$\DATATYPE$\ $\datdesc$\,'' or
``$\sharingspec$'' is replaced by the empty specification.}
\item \deletion{\thedatatyperepl}{The Modules phrase classes TypDesc, DatDesc, ConDesc and SharEq
      are omitted.}
\insertion{\thenostrsharing}{Any qualification \boxml{sharing type $\cdots$} on
a specification or \boxml{where type $\cdots$} on a signature expression is omitted.}
\end{itemize}

\subsection{Compound Objects}
\label{dynmod-comp-obj-sec}
The\index{58.2} compound objects for the Modules dynamic semantics, extra to those for the
Core dynamic semantics, are shown in Figure~\ref{comp-dynmod-obj}.
\begin{figure}[h]
\vspace{2pt}
\begin{displaymath}
\begin{array}{rcl}
\adhocreplacementl{\thenostrsharing}{1cm}{(\strid:\I,\strexp\langle:\I'\rangle,\B)}{(\strid:\I,\strexp,\B)}
                & \in   & \FunctorClosure\\
                &       & \qquad  = (\StrId\times\Int)\times
                          \adhocreplacementl{\thenostrsharing}{-5cm}{(\StrExp\langle\times\Int\rangle)\times\Basis}{\StrExp\times\Basis}\\
\adhocreplacementl{\theidstatus}{5mm}{(\IE,\vars,\exns)\ {\rm or}\ \I}{\I\ {\rm or}\ (\SI,\TI,\VI)}
                & \in   & \adhocreplacementl{\theidstatus}{-8cm}{\Int = \IntEnv\times\Fin(\Var)\times\Fin(\Exn)}{\Int = \StrInt\times\TyInt\times\ValInt}\\
        \adhocreplacementl{\theidstatus}{3cm}{\IE}{\SI}     & \in   & \adhocreplacementl{\theidstatus}{-9cm}{\IntEnv}{\StrInt} = \finfun{\StrId}{\Int}\\\adhocinsertion{\theidstatus}{2cm}{
        \TI     & \in   & \TyInt =  \finfun{\TyCon}{\ValInt}\\
        \VI     & \in   & \ValInt = \finfun{\VId}{\IdStatus}\\ }
        \G      & \in   & \SigEnv = \finfun{\SigId}{\Int}\\
        \F      & \in   & \FunEnv = \finfun{\FunId}{\FunctorClosure}\\
(\F,\G,\E)\ {\rm or}\ \B
                & \in   & \Basis = \FunEnv\times\SigEnv\times\Env\\
(\G,\adhocreplacementl{\thedatatyperepl}{2cm}{\IE}{\I})\ {\rm or}\ \IB
                & \in   & \IntBasis = \SigEnv\times\adhocreplacementl{\thedatatyperepl}{-4cm}{\IntEnv}{\Int}
\end{array}
\end{displaymath}
\caption{Compound Semantic Objects}
\label{comp-dynmod-obj}
\vspace{3pt}
\end{figure}
%
%
\insertion{\thedatatyperepl}{
An {\sl interface} $\I\in\Int$ represents a ``view'' of a structure.
Specifications and signature expressions will evaluate to interfaces;
moreover, during the evaluation of a specification or signature expression, 
structures (to which a specification or signature expression may
refer via datatype replicating specifications) are represented 
only by their interfaces.  To extract a value interface from
a dynamic value environment we define the operation $\Inter: \ValEnv \to\ValInt$
as follows:
\[\Inter(\VE) = \{\vid\mapsto\is\;;\;\VE(\vid) = (\V,\is)\}\]
In other words, $\Inter(\VE)$ is the value interface obtained from $\VE$ by
removing all values from $\VE$. We then extend $\Inter$ to a function
$\Inter:\Env\to\Int$ as follows:
\[ \Inter(\SE,\TE,\VE)\ =\ (\SI,\TI,\VI)\]
where $\VI$ = $\Inter(\VE)$ and 
\begin{eqnarray*}
\SI & = & \{\strid\mapsto\Inter\E\;;\;\SE(\strid) = \E\}\\
\TI & = & \{\tycon\mapsto\Inter\VE'\;;\;\TE(\tycon) = \VE'\}
\end{eqnarray*}
An {\sl interface basis} $\IB=(\G,\I)$ is a value-free part of a basis, sufficient to
evaluate signature expressions and specifications.
The function $\Inter$ is extended to create an interface basis
from a basis $\B$ as follows:
\[ \Inter(\F,\G,\E)\ =\ (\G, \Inter\E) \]
}

\deletion{\thedatatyperepl}{
An {\sl interface} $\I\in\Int$ represents a ``view'' of a structure.
Specifications and signature expressions will evaluate to interfaces; 
moreover, during the evaluation of a specification or signature expression, 
structures (to which a specification or signature expression may
refer via ``$\OPEN$'') are represented only by their interfaces.  To extract an
interface from a dynamic environment we define the operation
\[ \Inter\ :\ \Env\to\Int \]
as follows:
\[ \Inter(\SE,\VE,\EE)\ =\ (\IE,\Dom\VE,\Dom\EE)\]
where
\[ \IE\ =\ \{\strid\mapsto\Inter\E\ ;\ \SE(\strid)=\E\}\ .\]
An {\sl interface basis}\index{59.1} $\IB=(\G,\IE)$ is that part of a basis needed to
evaluate signature expressions and specifications.
The function $\Inter$ is extended to create an interface basis
from a basis $\B$ as follows:
\[ \Inter(\F,\G,\E)\ =\ (\G, \of{\IE}{(\Inter\E)}) \]}

\insertion{\thedatatyperepl}{
A further operation
\[ \downarrow\ :\ \Env\times\Int\to\Env\]
is required, to cut down an environment $\E$ to a given interface $\I$,
representing the effect of an explicit signature ascription. We first
define $\downarrow: \ValEnv\times\ValInt\to\ValEnv$ by
\[\VE\downarrow\VI = \{\vid\mapsto(\V,\is)\;;\;\VE(\vid) = (\V,\is')\ {\rm and}\ \VI(\vid) = \is\}\]
(Note that $\VE$ and $\VI$ need not have the same
domain and that the identifier status is taken from $\VI$.) 
We then define $\downarrow: \StrEnv \times \StrInt \to \StrEnv$,
$\downarrow: \TyEnv \times\TyInt\to\TyEnv$ and
$\downarrow: \Env\times\Int\to\Env$ simultaneously as follows:
\label{downarrowdef}
\begin{center}
 $\SE\downarrow\SI  =  \{\strid\mapsto\E\downarrow\I\ ;\
          	\SE(\strid)=\E\ {\rm and}\ \SI(\strid)=\I\}$\\[6pt]
 $ \TE\downarrow\TI =  \{\tycon\mapsto \VE'\downarrow\VI'\ ;\ 
               \TE(\tycon) = \VE'\ {\rm and}\ \TI(\tycon) = \VI'\}$ \\[6pt]
 $
 (\SE,\TE,\VE)\downarrow(\SI,\FIXREPL{\TI\ }{\TE},\VI)  = 
               (\SE\downarrow\SI, \TE\downarrow\TI, \VE\downarrow\VI) $
\end{center}}

\deletion{\thedatatyperepl}{
A further operation
\[ \downarrow\ :\ \Env\times\Int\to\Env\]
is required, to cut down an environment $\E$ to a given interface $\I$,
representing the effect of an explicit signature ascription.  It is defined
as follows:
\[ (\SE,\VE,\EE)\downarrow(\IE,\vars,\exns)\ =\ (\SE',\VE',\EE') \]
where
\[ \SE'\ =\ \{\strid\mapsto\E\downarrow\I\ ;\
          \SE(\strid)=\E\ {\rm and}\ \IE(\strid)=\I\} \]
and (taking $\downarrow$ now to mean restriction of a function domain)
\[\VE'=\VE\downarrow\vars,\ \EE'=\EE\downarrow\exns.\]
}

\noindent
It is important to note that an interface \replacement{\thedatatyperepl}{is also a projection of}{can
also be obtained from} the
{\sl static} value $\Sigma$ of a signature expression; 
it is obtained by \replacement{\thedatatyperepl}{omitting structure names $\m$ and type environments
$\TE$,}{first replacing every type
structure $(\typefcn, \VE)$
in the range of every type environment $\TE$ by $\VE$}
and \insertion{\thece}{then} replacing each \replacement{\theidstatus}{variable environment $\VE$ and each 
exception environment $\EE$ by its domain.}{pair $(\sigma,\is)$ in the range
of every value environment $\VE$ by $\is$.}
Thus in an implementation interfaces would naturally be obtained from the
static elaboration; we choose to give separate rules here for obtaining them
in the dynamic semantics since we wish to maintain our separation of the
static and dynamic semantics, for reasons of presentation.

\subsection{Inference Rules}
The\index{59.2} semantic rules allow sentences  of the form
\[ \s,A\ts\phrase\ra A',\s' \]
to be inferred, where $A$ is either a basis\replacement{\theidstatus}{ or an interface basis}{, a signature environment} or empty,
$A'$ is some semantic
object and $\s$,$\s'$ are the states before and after the evaluation
represented by the sentence.  Some hypotheses in rules are not of this form;
they are called {\sl side-conditions}.  The convention for options is
the same as for the Core static semantics.  

The state and exception conventions are adopted as in the Core dynamic
semantics.  However, it may be shown that the only Modules phrases whose 
evaluation
may cause a side-effect or generate an exception packet are of the form
$\strexp$, $\strdec$, $\strbind$ or $\topdec$.
%Also, as will be seen in Section~\ref{prog-sec}, a phrase of the
%form $\program$ can have side-effects, but not generate an
%exception packet.

%               SEMANTICS
%
%                       Structure Expressions
%
\rulesec{Structure Expressions}{\B\ts\strexp\ra \E/\p}
\begin{equation}        % generative strexp
%\label{generative-strexp-dyn-rule}
\frac{\B\ts\strdec\ra\E}
     {\B\ts\encstrexp\ra\E}\index{60.1}
\end{equation}

\begin{equation}        % longstrid
%\label{longstrid-strexp-dyn-rule}
\frac{\B(\longstrid)=\E}
     {\B\ts\longstrid\ra\E}
\end{equation}

\insertion{\theconstructors}{\begin{equation}        % transparent signature constraint
\frac{\B\ts\strexp\ra\E\qquad\Inter\B\ts\sigexp\ra\I}
     {\B\ts\transpconstraint\ra\E\downarrow\I}
\end{equation}

\begin{equation}        % opaque signature constraint
\frac{\B\ts\strexp\ra\E\qquad\Inter\B\ts\sigexp\ra\I}
     {\B\ts\opaqueconstraint\ra\E\downarrow\I}
\end{equation}}

%\vspace{6pt}
\replacement{\thenostrsharing}{\begin{equation}                % functor application
\label{functor-application-dyn-rule}
\frac{ \begin{array}{c}
        \B(\funid)=(\strid:\I,\strexp'\langle:\I'\rangle,\B')\\
        \B\ts\strexp\ra\E\qquad
       \B'+\{\strid\mapsto\E\downarrow\I\}\ts\strexp'\ra\E'\\
       \end{array}
     }
     {\B\ts\funappstr\ra\E'\langle\downarrow\I'\rangle}
\end{equation}}{\begin{equation}                % functor application
\label{functor-application-dyn-rule}
\frac{ \begin{array}{c}
        \B(\funid)=(\strid:\I,\strexp',\B')\\
        \B\ts\strexp\ra\E\qquad
       \B'+\{\strid\mapsto\E\downarrow\I\}\ts\strexp'\ra\E'\\
       \end{array}
     }
     {\B\ts\funappstr\ra\E'}
\end{equation}}

%\vspace{6pt}
\begin{equation}        % let strexp
%\label{letstrexp-dyn-rule}
\frac{\B\ts\strdec\ra\E\qquad\B+\E\ts\strexp\ra\E'}
     {\B\ts\letstrexp\ra\E'}
\end{equation}
\comments
\begin{description}
\item{(\ref{functor-application-dyn-rule})}
Before the evaluation of the functor body $\strexp'$, the
actual argument $\E$ is cut down by the formal parameter
interface $\I$, so that any opening of $\strid$ resulting
from the evaluation of $\strexp'$ will produce no more components
than anticipated during the static elaboration.
\end{description}

\rulesec{Structure-level Declarations}{\B\ts\strdec\ra\E/\p}
                % declarations
\begin{equation}                % core declaration
%\label{dec-dyn-rule}
\frac{ \of{\E}{\B}\ts\dec\ra\E' }
     { \B\ts\dec\ra\E' }\index{60.2}
\end{equation}

\vspace{6pt}
\begin{equation}                % structure declaration
%\label{structure-decl-dyn-rule}
\frac{ \B\ts\strbind\ra\SE }
     { \B\ts\singstrdec\ra\SE\ \In\ \Env }
\end{equation}

\vspace{6pt}
\begin{equation}                % local structure-level declaration
%\label{local structure-level declaration-dyn-rule}
\frac{ \B\ts\strdec_1\ra\E_1\qquad
       \B+\E_1\ts\strdec_2\ra\E_2 }
     { \B\ts\localstrdec\ra\E_2 }
\end{equation}

\vspace{6pt}
\begin{equation}                % empty declaration
%\label{empty-strdec-dyn-rule}
\frac{}
     {\B\ts\emptystrdec\ra \emptymap{\rm\ in}\ \Env}
\end{equation}

\vspace{6pt}
\begin{equation}                % sequential declaration
%\label{sequential-strdec-dyn-rule}
\frac{ \B\ts\strdec_1\ra\E_1\qquad
       \B+\E_1\ts\strdec_2\ra\E_2 }
     { \B\ts\seqstrdec\ra\plusmap{\E_1}{\E_2} }
\end{equation}

\rulesec{Structure Bindings}{\B\ts\strbind\ra\SE/\p}
\replacement{\thenostrsharing}{
\begin{equation}                % structure binding
\frac{ \begin{array}{cl}
       \B\ts\strexp\ra\E\qquad\langle\Inter\B\ts\sigexp\ra\I\rangle\\
       \langle\langle\B\ts\strbind\ra\SE\rangle\rangle
       \end{array}
     }
     {\begin{array}{c}
      \B\ts\strbinder\ra\\
      \qquad\qquad\qquad\{\strid\mapsto\E\langle\downarrow\I\rangle\}
      \ \langle\langle +\ \SE\rangle\rangle
      \end{array}
     }\index{61.1}
\end{equation}
\comment As in the static semantics, when present, $\sigexp$ constrains the
``view'' of the structure. The restriction must be done in the
dynamic semantics to ensure that any dynamic opening of the structure
produces no more components than anticipated during the static
elaboration.}{\begin{equation}                % structure binding
\frac{ 
       \B\ts\strexp\ra\E\qquad
       \langle\B\ts\strbind\ra\SE\rangle
     }
     {
      \B\ts\barestrbindera\ra\{\strid\mapsto\E\}
      \ \langle +\ \SE\rangle
     }\index{61.1}
\end{equation}
}
%
%                   Signature Rules
%

\rulesec{Signature Expressions}{\IB\ts\sigexp\ra\I}
\begin{equation}                % encapsulation sigexp
%\label{encapsulating-sigexp-dyn-rule}
\frac{\IB\ts\spec\ra\I }
     {\IB\ts\encsigexp\ra\I}\index{61.2}
\end{equation}

\begin{equation}                % signature identifier
%\label{signature-identifier-dyn-rule}
\frac{ \IB(\sigid)=\I}
     { \IB\ts\sigid\ra\I }
\end{equation}

\rulesec{Signature Declarations}{\IB\ts\sigdec\ra\G}
\begin{equation}        % single signature declaration
%\label{single-sigdec-dyn-rule}
\frac{ \IB\ts\sigbind\ra\G }
     { \IB\ts\singsigdec\ra\G }\index{61.3}
\end{equation}

\deletion{\thenostrsharing}{
\begin{equation}        % empty signature declaration
%\label{empty-sigdec-dyn-rule}
\frac{}
     { \IB\ts\emptysigdec\ra\emptymap }
\end{equation}

\begin{equation}        % sequential signature declaration
%\label{sequence-sigdec-dyn-rule}
\frac{ \IB\ts\sigdec_1\ra\G_1 \qquad \plusmap{\IB}{\G_1}\ts\sigdec_2\ra\G_2 }
     { \IB\ts\seqsigdec\ra\plusmap{\G_1}{\G_2} }
\end{equation}
}
\rulesec{Signature Bindings}{\IB\ts\sigbind\ra\G}
\begin{equation}        % signature binding
%\label{sigbind-dyn-rule}
\frac{ \IB\ts\sigexp\ra\I
        \qquad\langle\IB\ts\sigbind\ra\G\rangle }
     { \IB\ts\sigbinder\ra\{\sigid\mapsto\I\}
       \ \langle +\ \G\rangle }\index{61.4}
\end{equation}
%
                     % Specifications
%
\rulesec{Specifications}{\IB\ts\spec\ra\I}

\replacement{\theidstatus}{\begin{equation}        % value specification
%\label{valspec-dyn-rule}
\frac{ \ts\valdesc\ra\vars }
     { \IB\ts\valspec\ra\vars\ \In\ \Int }\index{61.5}
\end{equation}}{\begin{equation}        % value specification
%\label{valspec-dyn-rule}
\frac{ \ts\valdesc\ra\VI }
     { \IB\ts\valspec\ra\VI\ \In\ \Int }\index{61.5}
\end{equation}}

\insertion{\thedatatyperepl}{
\begin{equation}
\frac{\ts\typdesc\ra\TI}
     {\IB\ts\typespec\ra\TI\ \In\ \Int}
\end{equation}

\begin{equation}
\frac{\ts\typdesc\ra\TI}
     {\IB\ts\eqtypespec\ra\TI\ \In\ \Int}
\end{equation}}

\insertion{\thedatatyperepl}{
\begin{equation}
\frac{\ts\datdesc\ra\VI,\TI}
     {\IB\ts\datatypespec\ra(\VI,\TI)\ \In\ \Int}
\end{equation}
}

\insertion{\thedatatyperepl}{
\begin{equation}
\frac{\IB(\longtycon) = \VI\qquad \TI = \{\tycon\mapsto\VI\}}
     {\IB\ts\datatypereplspec\ra(\VI,\TI)\ \In\ \Int}
\end{equation}
}

\replacement{\theidstatus}{
\begin{equation}        % exception specification
%\label{exceptionspec-dyn-rule}
\frac{ \ts\exndesc\ra\exns}
     { \IB\ts\exceptionspec\ra\exns\ \In\ \Int }
\end{equation}}{\begin{equation}        % exception specification
%\label{exceptionspec-dyn-rule}
\frac{ \ts\exndesc\ra\VI}
     { \IB\ts\exceptionspec\ra \VI\ \In\ \Int }
\end{equation}}
\oldpagebreak
\replacement{\theidstatus}{\begin{equation}        % structure specification
\label{structurespec-dyn-rule}
\frac{ \IB\ts\strdesc\ra\IE }
     { \IB\ts\structurespec\ra\IE\ \In\ \Int }\index{62.1}
\end{equation}}{\begin{equation}        % structure specification
\label{structurespec-dyn-rule}
\frac{ \IB\ts\strdesc\ra\SI }
     { \IB\ts\structurespec\ra\SI\ \In\ \Int }\index{62.1}
\end{equation}}

\deletion{\thenolocalspec}{
\begin{equation}        % local specification
\label{localspec-dyn-rule}
\frac{ \IB\ts\spec_1\ra\I_1 \qquad
       \plusmap{\IB}{\of{\IE}{\I_1}}\ts\spec_2\ra\I_2 }
     { \IB\ts\localspec\ra\I_2 }
\end{equation}}

\deletion{\thenoopenspec}{
\begin{equation}        % open specification
%\label{openspec-dyn-rule}
\frac{ \IB(\longstrid_1)=\I_1\quad\cdots\quad
       \IB(\longstrid_n)=\I_n }
     { \IB\ts\openspec\ra\I_1 + \cdots +\I_n }
\end{equation}}

\replacement{\thesingleincludespec}{
\begin{equation}        % include signature specification
%\label{inclspec-dyn-rule}
\frac{ \IB(\sigid_1)=\I_1 \quad\cdots\quad
       \IB(\sigid_n)=\I_n }
     { \IB\ts\inclspec\ra\I_1 + \cdots +\I_n }
\end{equation}}{\begin{equation}        % include signature specification
%\label{inclspec-dyn-rule}
\frac{ \IB\ts\sigexp\ra\I}
     { \IB\ts\singleinclspec\ra\I}
\end{equation}}

\begin{equation}        % empty specification
%\label{emptyspec-dyn-rule}
\frac{}
     { \IB\ts\emptyspec\ra\emptymap{\rm\ in}\ \Int }
\end{equation}

\replacement{\theidstatus}{\begin{equation}        % sequential specification
\label{seqspec-dyn-rule}
\frac{ \IB\ts\spec_1\ra\I_1
       \qquad \plusmap{\IB}{\of{\IE}{\I_1}}\ts\spec_2\ra\I_2 }
     { \IB\ts\seqspec\ra\plusmap{\I_1}{\I_2} }
\end{equation}}{\begin{equation}        % sequential specification
\label{seqspec-dyn-rule}
\frac{ \IB\ts\spec_1\ra\I_1
       \qquad \IB+\I_1\ts\spec_2\ra\I_2 }
     { \IB\ts\seqspec\ra\plusmap{\I_1}{\I_2} }
\end{equation}}

%\ insertion{\thenostrsharing}{
%\begin{equation}
%\frac{\IB\ts\spec\ra\I}    %  type sharing spec
%     {\IB\ts\newsharingspec\ra\I}
%\end{equation}
%}

\deletion{\thedatatyperepl}{
\noindent\comments
\begin{description}
\item{(\ref{localspec-dyn-rule}),(\ref{seqspec-dyn-rule})}
Note that $\of{\vars}{\I_1}$ and $\of{\exns}{\I_1}$ are
not needed for the evaluation of $\spec_2$.
\end{description}}

                         % Descriptions

\replacement{\theidstatus}{\rulesec{Value Descriptions}{\ts\valdesc\ra\vars}}{\rulesec{Value Descriptions}{\ts\valdesc\ra\VI}}
\replacement{\theidstatus}{\begin{equation}         % value description
%\label{valdesc-dyn-rule}
\frac{ \langle\ts\valdesc\ra\vars\rangle }
     { \ts\var\ \langle\AND\ \valdesc\rangle\ra
       \{\var\}\ \langle\cup\ \vars\rangle }\index{62.2}
\end{equation}}{\begin{equation}         % value description
%\label{valdesc-dyn-rule}
\frac{ \langle\ts\valdesc\ra\VI\rangle }
     { \ts\vid\ \langle\AND\ \valdesc\rangle\ra
       \{\vid\mapsto\isv\}\ \langle+\,\VI\rangle }\index{62.2}
\end{equation}}

\insertion{\thedatatyperepl}{
\rulesec{Type Descriptions}{\ts\typdesc\ra\TI}
\begin{equation}
\frac{\langle\ts\typdesc\ra\TI\rangle}
     {\ts\typdescription\ra\{\tycon\mapsto\emptymap\}\langle+\TI\rangle}
\end{equation}
}

\insertion{\thedatatyperepl}{
\rulesec{Datatype Descriptions}{\ts\datdesc\ra\VI, \TI}
\begin{equation}
\frac{\ts\condesc\ra\VI\qquad\langle\ts\datdesc'\ra\VI',\TI'\rangle}
     {\ts\datdescriptiona\ra\VI\,\langle+\,\VI'\rangle, \{\tycon\mapsto\VI\}\langle+\TI'\rangle}
\end{equation}

\rulesec{Constructor Descriptions}{\ts\condesc\ra\VI}
\begin{equation}
\frac{\langle\ts\condesc\ra\VI\rangle}
     {\ts\shortconviddesc\ra\{\vid\mapsto\isc\}\,\langle+\VI\rangle}
\end{equation}
}

\replacement{\theidstatus}{\rulesec{Exception Descriptions}{\ts\exndesc\ra\exns}}{\rulesec{Exception Descriptions}{\ts\exndesc\ra\VI}}
\replacement{\thefixtypos}{
\begin{equation}         % exception description
%\label{exndesc-dyn-rule}
\frac{ \langle\ts\exndesc\ra\exns\rangle }
     { \ts\exn\ \langle\exndesc\rangle\ra\{\exn\}\ \langle\cup\ \exns\rangle }\index{62.3}
\end{equation}}{\begin{equation}         % exception description
%\label{exndesc-dyn-rule}
\frac{ \langle\ts\exndesc\ra\VI\rangle }
     { \ts\vid\ \langle\boxml{and\ }\exndesc\rangle\ra\{\vid\mapsto\ise\}\ \langle+ \VI\rangle }\index{62.3}
\end{equation}}

\replacement{\theidstatus}{\rulesec{Structure Descriptions}{\IB\ts\strdesc\ra\IE}}{\rulesec{Structure Descriptions}{\IB\ts\strdesc\ra\SI}}
\replacement{\theidstatus}{\begin{equation}
%\label{strdesc-dyn-rule}
\frac{ \IB\ts\sigexp\ra\I\qquad\langle\IB\ts\strdesc\ra\IE\rangle }
     { \IB\ts\strdescription\ra\{\strid\mapsto\I\}\ \langle +\ \IE\rangle }\index{62.4}
\end{equation}}{\begin{equation}
%\label{strdesc-dyn-rule}
\frac{ \IB\ts\sigexp\ra\I\qquad\langle\IB\ts\strdesc\ra\SI\rangle }
     { \IB\ts\strdescription\ra\{\strid\mapsto\I\}\ \langle +\ \SI\rangle }\index{62.4}
\end{equation}}

%                       Functor and Program rules
%
\rulesec{Functor Bindings}{\B\ts\funbind\ra\F}
\begin{equation}        % functor binding
%\label{funbind-dyn-rule}
\frac{
      \Inter\B\ts\sigexp\ra\I\qquad
      \langle\FIXREPL{\B\ }{\IB}\ts\funbind\ra\F\rangle
     }
     {
      \begin{array}{c}
       \FIXREPL{\B\ }{\IB}\ts\barefunstrbinder\ \optfunbinda\ra\\
       \qquad\qquad \qquad
              \{\funid\mapsto(\strid:\I,\strexp,\B)\}
              \ \langle +\ \F\rangle
      \end{array}
     }\index{62.5}
\end{equation}
\oldpagebreak
\rulesec{Functor Declarations}{\B\ts\fundec\ra\F}
\begin{equation}        % single functor declaration
%\label{singfundec-dyn-rule}
\frac{ \B\ts\funbind\ra\F }
     { \B\ts\singfundec\ra\F }\index{63.1}
\end{equation}

\rulesec{Top-level Declarations}{\B\ts\topdec\ra\B'/\p}
\begin{equation}        % structure-level declaration
%\label{strdectopdec-dyn-rule}
\frac{\B\ts\strdec\ra\E\quad\B' =\E\ \In\ \Basis\quad\langle \B+\B'\ts\topdec\ra\B''\rangle }
     {\B\ts\strdecintopdec\ra\FIXREPL{\B'\langle+\B''\rangle}{\B'\langle{}'\rangle}
     }\index{63.2}
\end{equation}

\vspace{6pt}
\begin{equation}        % signature declaration
%\label{sigdectopdec-dyn-rule}
\frac{\Inter\B\ts\sigdec\ra\G\quad B' = \G\ \In\ \Basis\quad
       \langle \B + \B'\ts\topdec\ra\B''\rangle}
     {\B\ts\sigdecintopdec\ra \FIXREPL{\B'\langle+\B''\rangle}{\B'\langle{}'\rangle}
     }
\end{equation}

\vspace{6pt}
\begin{equation}        % functor declaration
%\label{fundectopdec-dyn-rule}
\frac{\B\ts\fundec\ra\F\quad \B' = \F\ \In\ \Basis\quad
       \langle \B + \B'\ts\topdec\ra\B''\rangle}
     {\B\ts\fundecintopdec\ra\FIXREPL{\B'\langle+\B''\rangle}{\B'\langle{}'\rangle}
     }
\end{equation}



