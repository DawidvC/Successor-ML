%!TEX root = root.tex
%
%

\section*{Preface}
\markboth{}{}
A precise description of a programming language is a prerequisite for its 
implementation and for its use.  The description can take many forms, 
each suited to a different purpose.  A common form is a reference manual,
which is usually a careful narrative description of the meaning of each
construction in the language, often backed up with a formal presentation
of the grammar (for example, in Backus-Naur form). 
This gives the programmer enough 
understanding for many of his purposes.  But it is ill-suited for use
by an implementer, or by someone who wants to formulate laws for
equivalence of programs, or by a programmer who wants to design programs with
mathematical rigour. 

This document is a formal description of both the {\sl grammar} and the
{\sl meaning} of a language which is both designed for large projects and
widely used.  As such, it aims to serve the whole community of people
seriously concerned with the language.  At a time when it is increasingly
understood that programs must withstand rigorous analysis, particularly for 
systems where safety is critical, a rigorous language presentation is even
important for negotiators and contractors; for a robust program
written in an insecure language is like a house built upon sand.


Most people have not looked at a rigorous language presentation before.
To help them particularly, but also to put the present work in perspective
for those more theoretically prepared,  it will be useful here to say something
about three things: the nature of Standard ML, the task of language definition 
in general, and the form of the present Definition.
We also briefly describe the recent revisions to the Definition.
\vspace*{-8pt}

\subsubsection*{Standard ML}
\vspace*{-4pt}
Standard ML is a functional programming language, in the sense that the
full power of mathematical functions is present.  But it grew in response
to a particular programming task, for which it was equipped also
with full imperative power, and a sophisticated exception mechanism.  
It has an advanced form of parametric modules, aimed at organised
development of large programs.  Finally it is strongly typed, and it was
the first language to provide a particular form of polymorphic type which 
makes the strong typing remarkably flexible.  This combination of
ingredients has not made it unduly large, but their novelty has been
a fascinating challenge to semantic method (of which we say more below).
 
ML has evolved over \replacement{\thenewpreface}{fourteen}{twenty} years as a fusion of many ideas 
from many people. This evolution is
described in some detail in Appendix~\ref{story-app} of the book, where also we
acknowledge all those who have contributed to it, both in design
and in implementation.

`ML' stands for 
{\sl meta language}; this is the term logicians use for a language in which
other (formal or informal) languages are discussed and analysed.  
Originally ML was conceived as a medium for finding and performing 
proofs in a logical language.   Conducting rigorous argument as dialogue 
between person and machine has been a \replacement{\thenostrsharing}{strong research interest at Edinburgh 
and elsewhere, throughout these fourteen years.}{growing research topic 
throughout these twenty years.} The difficulties are enormous, 
and make stern demands
upon the programming language which is used for this dialogue.   Those who are
not familiar with computer-assisted reasoning may be surprised that a 
programming language, which was designed for this rather esoteric activity, 
should ever lay claim to being {\sl generally} useful.
On reflection, they should not be surprised.  LISP is a prime example of
a language invented for esoteric purposes and becoming widely used.  LISP
was invented for use in artificial intelligence (AI); the important thing
about AI here is not that it is esoteric,  but that
it is difficult and varied; so much so, that anything which works well for
it must work well for many other applications too.

The same can be said about the initial purpose of ML, but with a different
emphasis.  Rigorous proofs are complex things, which need varied
and sophisticated presentation -- particularly on the screen in interactive
mode. Furthermore the proof methods, or
strategies, involved are some of the most complex algorithms which we know.   
This all applies equally to AI, but one demand is made more strongly 
by proof than perhaps by any other application: the demand for rigour.  

This demand established the character of ML.  In order to be sure that,
when the user and the computer claim to have together performed a rigorous
argument, their claim is justified, it was seen 
that the language must be strongly typed. On the other hand, to be
useful in a difficult  application, the type system had to be rather
flexible, and permit the machine to guide the user rather than impose
a burden upon him.   A reasonable solution was found, in which the machine
helps the user significantly  by
inferring his types for him. Thereby the machine also confers complete
reliability on his programs, in this sense: 
If a program claims that a certain result 
follows from the rules of reasoning which the user has supplied, then the 
claim may be fully trusted.

The principle of inferring useful structural information about programs
is also represented, at the level of program modules, by the inference of 
{\sl signatures}.
Signatures describe the interfaces between modules, and are vital for robust
large-scale programs.  When the user combines modules, the signature
discipline prevents him from mismatching their interfaces.  By programming
with interfaces and parametric modules, it becomes possible to focus on the
structure of a large system, and to compile parts of it in isolation from
one another -- even when the system is incomplete.

This emphasis on types and signatures has had a profound effect on the 
language Definition. Over half this document is devoted to inferring types 
and signatures for programs.  But the method used is exactly the same as for 
inferring what {\sl values} a program delivers; indeed, a type or signature is 
the result of a kind of abstract evaluation of a program phrase. 

In designing ML, 
the interplay among three activities -- language design, definition and 
implementation -- was extremely close. This was particularly true for the
newest part, the parametric modules. This part of the language grew from an 
initial proposal by David MacQueen, itself highly developed; but both 
formal definition and implementation had a strong influence on the detailed
design.  In general, those who took part in the three activities cannot now 
imagine how they could have been properly done separately.  

\subsubsection*{Language Definition}
Every programming language presents  its own conceptual view of 
computation.  This view is usually indicated by the names used for the phrase 
classes of the language, or by its keywords: terms like package, module, 
structure, exception, channel, type, procedure, reference, sharing, \ldots. 
These terms also have their abstract counterparts,  which may be called
{\sl semantic objects}; these are what people really have in mind when they use
the language, or discuss it, or think in it.  Also, it is these objects,
not the syntax, which represent the particular conceptual view of each
language; they are the character of the language.  Therefore a definition
of the language must be in terms of these objects.  

As is commonly done in programming language semantics, we shall loosely
talk of these semantic objects as {\sl meanings}. Of course, it is 
perfectly possible to understand the semantic theory of a language, and
yet be unable to understand the meaning of a particular program, in the
sense of its {\sl intention} or {\sl purpose}.  The aim of a language
definition is not to formalise everything which could possibly be called the
meaning of a program, but to establish a theory of semantic objects  
upon which the understanding of particular programs may rest.

The job of a language-definer is twofold.  First -- as we have already 
suggested -- he must create a world of meanings appropriate for the language, 
and must find a way of saying
what these meanings precisely are.  Here, he meets a problem; notation
of {\sl some} kind must be used to denote and describe these meanings --
but not a {\sl programming language} notation, unless he is passing the
buck and defining one programming language in terms of another. Given
a concern for rigour, mathematical notation is an obvious choice.  Moreover, 
it is not enough just to
write down mathematical definitions. The world of meanings only becomes
meaningful if the objects possess nice properties, which make them tractable.
So the language-definer really has to develop a small {\sl theory} of his 
meanings, in the same way that a mathematician develops a theory.  
Typically, after initially defining some objects, the mathematician goes on to 
verify properties which indicate that they are objects worth studying. 
It is this part, a kind of scene-setting, which the language-definer shares 
with the 
mathematician.  Of course he can take many objects and their theories 
directly from mathematics,  such as functions, relations,
trees, sequences, \ldots. But he must also give some special theory for the
objects which make his language particular, as we do for types, structures and 
signatures in this book; otherwise his language definition may be 
formal but will give no insight.

The second part of the definer's job is to define {\sl evaluation} precisely.
This means that he must define at least {\sl what} meaning, $M$, results
from evaluating any phrase $P$ of his language (though he need not explain
exactly {\sl how} the meaning results; that is he need not give the full 
detail of every computation).  This part of his job must be formal
to some extent, if only because the phrases $P$ of his language are indeed
formal objects.  But there is another reason for formality.  The task is
complex and error-prone, and therefore demands a high level of explicit
organisation (which is, largely, the meaning of `formality'); moreover,
it will be used to specify an equally complex, error-prone and formal
construction: an implementation.

We shall now explain the keystone of our semantic method.  First, we need a
slight but important refinement. A phrase $P$ is never evaluated {\sl in
vacuo} to a meaning $M$, but always {\sl against a background}; this 
background -- call it $B$ -- is itself a semantic object, 
being a distillation of the meanings
preserved from evaluation of earlier phrases (typically variable declarations,
procedure declarations, etc.).  In fact evaluation is background-dependent
-- $M$ depends upon $B$ as well as upon $P$.   

The keystone of the method, then, is a certain kind of assertion about
evaluation; it takes the form
\[ B\vdash P\Rightarrow M\]
and may be pronounced: `Against the background $B$, the phrase $P$ evaluates
to the meaning $M$'.  {\sl The formal purpose of this Definition is no more, 
and no less, than to decree exactly which assertions of this form are true.}
This could be achieved in many ways. We have chosen to do it in a structured
way, as others have, by giving rules which allow assertions about a 
{\sl compound} phrase $P$ to be inferred from assertions about its
{\sl constituent} phrases $P_1,\ldots,P_n$.

We have written the Definition in a form suggested by the previous remarks.
That is, we have defined our semantic objects in mathematical
notation which is completely independent of Standard ML, and we have
developed just enough of their theory to give sense to our rules of
evaluation. 

   Following another suggestion above, we have factored our task
by describing {\sl abstract} evaluation -- the inference and checking of
types and signatures (which can be done at compile-time) -- completely 
separately from {\sl concrete} evaluation.  
It really is a factorisation, because a {\sl full} value in all its glory --
you can think of it as a concrete object with a type
attached -- never has to be presented.\note{\thecommentary}{Paragraph moved}

\deletion{\thenewpreface}{
\subsubsection*{The form of the Definition
\footnote{
   The Definition has evolved through a sequence of three previous versions,
   circulated as Technical Reports.  For those who have followed the
   sequence, we should point out that the treatment of {\sl equality types}
   and of {\sl admissibility} has been slightly modified in this publication 
   to meet the claim for principal signatures.  The changes are mainly
   in Sections~4.9, 5.5 and 5.13 and in the inference rules~19, 20,
   29 and 65.}}}

\subsubsection*{The Revision of Standard ML}
{\it The Definition of Standard ML} was published in 1990.  
Since then the implementation technology of
the language has advanced enormously, and its users have 
multiplied.  The language and its Definition 
have therefore incited close scrutiny, evaluation, much approval, 
sometimes strong criticism.\note{\thenostrsharing}{Section added.}

The originators of the language have sifted this response, and
found that there are inadequacies in the original language and its 
formal Definition.  They are of three kinds: missing features which
many users want;  complex and little-used features which most
users can do without; and mistakes of definition.  What is
remarkable is that these inadequacies are rather few, and that they
are rather uncontroversial.  

This new version of the Definition addresses the three kinds
of inadequacy respectively by additions, subtractions and corrections.
But we have only made such amendments when one or more aspects of
SML -- the language itself, its usage, its implementation, its formal 
Definition -- have thus become simpler, without complicating the
other aspects.  It is worth noting that even the additions meet this 
criterion; for example we have introduced type abbreviations
in signatures to simplify the use of the language, but the way we
have done it has even simplified the Definition too.   In fact,
after our changes the formal Definition has fewer rules.

In this exercise we have consulted the major implementers and several
users, and have found broad agreement.  In the 1990 Definition it was
predicted that further versions of the Definition would be produced as
the language develops, with the intention to minimise the number of
versions.  This is the first revised version, and we foresee no
others. The changes that have been made to the 1990~Definition are
enumerated in Appendix~\ref{whatisnew-app}.

The resulting document is, we hope, valuable as the essential point of 
reference for Standard ML.  If it is to play this role well, it must
be supplemented by other literature.  \replacement{\thenewpreface}{Some}{Many}
expository books have already been 
written, and this Definition will be useful as a 
background reference for their readers. 
We \replacement{\thenewpreface}{have also become}{became}
convinced, while
writing the \insertion{\thenewpreface}{1990} Definition, that we could 
not discuss many questions without 
making it far too long.  Such questions are: Why were certain design choices
made?  What are their implications for programming?  
Was there a good alternative meaning for some constructs, or was our
hand forced?  What different forms of phrase are equivalent? What is the proof
of certain claims?  Many of these
questions \replacement{\thenewpreface}{will not be}{are not} 
answered by pedagogic texts either. 
\replacement{\thenewpreface}{So we are writing
a Commentary\cite{mt91} on the Definition which will assist people in reading it, and
which will serve as a bridge between the Definition and other texts.}
{We therefore wrote a 
Commentary on the 1990
Definition to
assist people in reading it, and to
serve as a bridge between the Definition and other texts. Though in
part outdated by the present revision, the Commentary still largely fulfils its 
purpose.}

\insertion{\thenewpreface}{There exist several textbooks on programming
with Standard ML\cite{paulson96,mcp93,Ullman94,stansifer92}. The second
edition of Paulson's book\cite{paulson96} conforms with the present revision.}


We wish to thank Dave Berry, Lars Birkedal, Martin Elsman, Stefan Kahrs
and John Reppy for many detailed comments and suggestions which have assisted
the revision.

\begin{flushright}
Robin Milner\quad Mads Tofte\quad Robert Harper\quad David MacQueen\\[1cm]
November 1996
\end{flushright}

%\newpage
\deletion{\thenewpreface}{\begin{flushright} Edinburgh\\August 1989 \end{flushright}}

\subsection*{Successor ML}
{\it The Definition of Standard ML (Revised)} was published in 1997~\cite{sml97-definition}, and
{\it The Standard ML Basis Library} was published in 2004~\cite{sml-basis-lib}.
Since that time, while SML implementations have matured, the language that they
implement has remained static.
Successor ML is a collection of proposed changes and extensions to SML that both
address problems in the Definition and improve and grow the language in natural ways.
This document merges the formal description of Successor ML features developed
by Andreas Rossberg~\cite{hamlet-s} into the Standard ML Revised Definition.
It is hoped that the resulting document will serve as basis for the future development
of Standard ML.

\begin{flushright}
John Reppy\\[1cm]
June 2015
\end{flushright}

