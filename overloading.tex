%!TEX root = root.tex
%

\section{Overloading}
\label{overload.sec}
Two forms of overloading are available:
\begin{itemize}
\item Certain special constants are overloaded.  For example,
{\tt 0w5} may have type $\WORD$ or some other type, depending on
the surrounding program text;
\item Certain operators are overloaded. For example,
{\tt +} may have type $\INT\ast\INT\to\INT$ or
$\REAL\ast\REAL\to\REAL$, depending on
the surrounding program text;
\end{itemize}
Programmers cannot define their own overloaded constants or operators.

Although a formal treatment of overloading is outside the scope
of this document, we do give a complete list of the overloaded operators
and of types with overloaded special constants.
This list is consistent with the Basis Library\cite{sml-basis-lib}.

Every overloaded constant and value identifier has among its types a 
{\em default type},
which is assigned to it, 
when the surrounding text does not resolve the overloading.
For this purpose, the surrounding text is \REPL{%
the smallest enclosing declaration
}{%
no larger than the smallest
enclosing structure-level declaration; an implementation may require
that a smaller context determines the type}.

\subsection{Overloaded special constants}
Libraries may extend the set $\T_0$ of
Appendix~\ref{init-stat-bas-app} with additional type names. Thereafter, certain
subsets of $T_0$ have a special significance;
they are called {\sl overloading classes}
and they are:\medskip

%\halign{\indent#\ \hfil&\ $#$\ &\ $#$\hfil\cr
%\Int &\supseteq&\{\INT\}\cr
%\Real &\supseteq&\{\REAL\}\cr
%\Word &\supseteq&\{\WORD\}\cr
%\String&\supseteq&\{\STRING\}\cr
%\Char&\supseteq&\{\CHAR\}\cr
%\WordInt&=&\Word\cup\INt\cr
%\RealInt&=&\Real\cup\INt\cr
%\Num&=&\Word\cup\Real\cup\INt\cr
%\NumTxt&=&\Word\cup\Real\cup\INt\cup\String\cup\Char\cr}
%\medskip

\begin{displaymath}
  \begin{array}{rcl}
    \Int &\supseteq& \{\INT\}\\
    \Real &\supseteq& \{\REAL\}\\
    \Word &\supseteq& \{\WORD\}\\
    \String &\supseteq& \{\STRING\}\\
    \Char &\supseteq& \{\CHAR\}\\
    \WordInt &=& \Word\cup\INt\\
    \RealInt &=& \Real\cup\INt\\
    \Num &=& \Word\cup\Real\cup\INt\\
    \NumTxt &=& \Word\cup\Real\cup\INt\cup\String\cup\Char\\
  \end{array}%
\end{displaymath}%

\noindent 
Among these, the five first ($\Int$, $\Real$, $\Word$, $\String$ and $\Char$) are said to be
{\sl basic}; the remaining are said to be {\sl composite}.
The reason that the basic classes are specified using
$\supseteq$ rather than $=$ is that libraries may extend 
each of the  basic overloading
classes with further type names.
\ADD{But the class $\Real$ may not contain type names taht admit equality.}
Special constants are overloaded
within each of the basic overloading classes.  However, the basic
overloading classes must be arranged so that every special constant can be
assigned types from at most one of the basic overloading classes.  For
example, to \boxml{0w5} may be assigned type $\WORD$, or
some other member of $\Word$, depending on the surrounding text.  If
the surrounding text does not determine the type of the constant, a
default type is used. The default types for the five sets are $\INT$,
$\REAL$, $\WORD$, $\STRING$ and $\CHAR$ respectively.

       Once overloading resolution has determined the type of a special constant,
       it is a compile-time error if the constant does not make sense or does not 
       denote a value within the machine representation chosen for the type.
       For example, an escape sequence of the form $\uconst$ in a string constant
       of 8-bit characters only makes sense if $xxxx$  denotes
       a number in the range $[0, 255]$. 


\subsection{Overloaded value identifiers}
Overloaded identifiers all have identifier status $\isv$. An
overloaded identifier may be re-bound with any status ($\isv$, $\isc$
and $\ise$) but then it is not overloaded within the scope of
the binding.

\begin{figure}
\begin{center}
\vskip-12pt
\begin{tabular}{|rl|rl|}
\multicolumn{2}{c}{NONFIX}&     \multicolumn{2}{c}{INFIX}\\
\hline
$\var$     & $\mapsto\ \hbox{set of monotypes}$    
                          & $\var$ & $\mapsto\ \hbox{set of monotypes}$\\
\hline
\boxml{abs} & $\mapsto \REALINT\to\REALINT$ 
                       & \multicolumn{2}{l|}{Precedence 7, left associative :} \\
\NEG    & $\mapsto \REALINT\to\REALINT $                      &
                            \boxml{div} & $\mapsto \WORDINT\ \ast\ \WORDINT
                                                                 \to\WORDINT$\\
 &  
                                             &
                            \boxml{mod} & $\mapsto \WORDINT\ \ast\ \WORDINT
                                                                 \to\WORDINT$\\
  &                       &
                            \boxml{*} &$\mapsto \NUM\ \ast\ \NUM
                                                                 \to\NUM$\\
  &                       &
                            \boxml{/} &$\mapsto \RREAL\ \ast\ \RREAL
                                                                 \to\RREAL$\\
  & &
                            \multicolumn{2}{l|}{Precedence 6, left associative :} \\
  &                       &
                            \boxml{+} &$\mapsto \NUM\ \ast\ \NUM
                                                                 \to\NUM$\\
  &                       &
                            \boxml{-} &$\mapsto \NUM\ \ast\ \NUM\to\NUM$\\
  & 
                          & \multicolumn{2}{l|}{Precedence 4, left associative :}\\
              &           &
                            \boxml{<} & $\mapsto\NUMTEXT *\NUMTEXT \to \REPL{\BOOL\ }{\NUMTEXT}$\\
              &           &
                            \boxml{>} & $\mapsto\NUMTEXT *\NUMTEXT \to \REPL{\BOOL\ }{\NUMTEXT}$\\
              &           &
                            \boxml{<=} & $\mapsto\NUMTEXT *\NUMTEXT \to \REPL{\BOOL\ }{\NUMTEXT}$\\
              &           &
                            \boxml{>=} & $\mapsto\NUMTEXT *\NUMTEXT \to \REPL{\BOOL\ }{\NUMTEXT}$\\
\hline
\end{tabular}
\end{center}
\vskip-15pt
\caption{Overloaded identifiers}
\label{overload.fig}
\end{figure}%
The overloaded identifiers are given in Figure~\ref{overload.fig}.
For example, the entry
$$\boxml{abs}\mapsto\REALINT\to\REALINT$$
states that $\boxml{abs}$ may assume one of the types
$\{\t\to\t\,\mid\,\t\in\RealInt\}$.
In general, the same type name must be chosen 
throughout the entire type of the overloaded operator;
thus $\boxml{abs}$ does not have type $\REAL\to\INT$.

The operator \boxml{/} is overloaded on all members of $\Real$,
with default type $\REAL\ast\REAL\to\REAL$.
The default type of any other identifier is that one of its types
which contains the type name {\tt int}.
For example, the program ~~\boxml{fun double(x) = x + x;}~~
declares a function of type $\INT\ \ast\ \INT\to\INT$, while
~~\boxml{fun double(x:real) = x + x;}~~ declares a function of type
$\REAL\ \ast\ \REAL\to\REAL$.

The dynamic semantics of the overloaded operators is defined in
\cite{sml-basis-lib}. 

